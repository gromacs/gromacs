\chapter{Interfacing with other software.}
\section{GROMOS}
Some of the files that are used by the GROMOS87~\cite{biomos} software
package can be read by {\gromacs} software. Actually, we have adapted
to GROMOS programs, \myindex{PROGMT} and \myindex{PRORMT} so that they
read GROMOS files, and write out {\gromacs} files. The programs were
renamed to avoid confusion, and they will be described below.
It should be stressed here, that these programs are supplied to
facilitate conversion, but no guarantee is given that the
programs are 100\% correct.

\subsection{stripifp}
Reads an interaction-function-parameter file usually named
\myindex{IFP37C4.DAT} from tape11
and writes out {\gromacs} forcefield files:\\
\myindex{nonbond.itp}: contains atomtypes and lennard-jones parameters\\
\myindex{bonded.itp}: contains bonded force parameters.

These files have the same format and function as the standard {\gromacs}
force field files: \myindex{ffgmxnb.itp} and \myindex{ffgmxbon.itp}.

\subsection{striprt}
Reads an residue topology building block file usually named
\myindex{RT37C4.DAT} from tape12
and writes two {\gromacs} database files:\\
\myindex{residue.rtp}: contains residue building blocks that can be used to build proteins etc.\\
\myindex{atomtype.atp}: contains the names of the different atomtypes used in the force field.

A consistent force field can be created by renaming all the four files
nonbond.itp, bonded.itp, residue.rtp and atomtype.atp, to have the same
prefix, e.g.:
myffnb.itp, myffbon.itp, myff.rtp and myff.atp.

For a (somewhat more) complete description of the force field files please
refer to chapter~\ref{ch:ff}.

\subsection{rmt2top}
Reads a GROMOS87 formatted topology file from stdin and writes out
the solute part of the topology to {\tt rmt2top.top}. Typical use
would be: 
\type{rmt2top < topol.fmt > log}
This would print a lot of info to the {\tt log} file and create your
new {\gromacs} topology {\tt rmt2top.top}.

