\section{Longe Range Electrostatics}
\label{sec:lr_elstat}
\subsection{Ewald summation}
\label{sec:ewald}
The total electrostatic energy of $N$ particles and the periodic
images are given by
\begin{equation}
V=\frac{f}{2}\sum_{n_x}\sum_{n_y}
\sum_{n_{z}*} \sum_{i}^{N} \sum_{j}^{N} 
\frac{q_i q_j}{{\bf r}_{ij,{\bf n}}}.
\label{totalcoulomb}
\end{equation}
$(n_x,n_y,n_z)={\bf n}$ is the box index vector, and the star indicates that
terms with $i=j$ should be omitted when $(n_x,n_y,n_z)=(0,0,0)$. The
distance ${\bf r}_{ij,{\bf n}}$ is the real distance between the charges and
not the minimum-image. This sum is conditionally convergent, but 
very slow.

Ewald summation was first introduced as a method to calculate
long-range interactions of the periodic images in
crystals~\cite{Ewald21}. The idea is to convert the single slowly converging
sum \ref{totalcoulomb} into two fast converging terms and a constant
term:
\begin{eqnarray}
V &=& V_{dir} + V_{rec} + V_{0} \\[0.5ex]
V_{dir} &=& \frac{f}{2} \sum_{i,j}^{N} 
\sum_{n_x}\sum_{n_y}
\sum_{n_{z}*} q_i q_j \frac{{\mathrm erfc}(\beta {r}_{ij,{\bf n}} )}{{r}_{ij,{\bf n}}} \\[0.5ex]
V_{rec} &=& \frac{f}{2 \pi V} \sum_{i,j}^{N} q_i q_j 
\sum_{m_x}\sum_{m_y}
\sum_{m_{z}*} \frac{\exp{\left( -(\pi {\bf m}/\beta)^2 + 2 \pi i
      {\bf m} \cdot ({\bf r}_i - {\bf r}_j)\right)}}{{\bf m}^2} \\[0.5ex]
V_{0} &=& -\frac{f \beta}{\sqrt{\pi}}\sum_{i}^{N} q_i^2,
\end{eqnarray}
where $\beta$ is a parameter that determines the relative weight of the
direct and reciprocal sums and ${\bf m}=(m_x,m_y,m_z)$.
In this way we can use a short cut-off (of the order of $1$~nm) in the direct space sum and a
short cut-off in the reciprocal space sum (e.g. 10 wavevectors in each 
direction). Unfortunately, the computational cost of the reciprocal
part of the sum increases as $N^2$
(or $N^{3/2}$ with a slightly better algorithm) and it is therefore not 
realistic to use for any large systems.

\subsubsection{Using Ewald}
Don't use Ewald unless you are absolutely sure this is what you want - 
for almost all cases the PME method below will perform much better. 
If you still want to employ classical Ewald summation enter this in
your {\tt .mdp} file, if the side of your box is about $3$~nm:
\begin{verbatim}
eeltype         = Ewald
rvdw            = 0.9
rlist           = 0.9
rcoulomb        = 0.9
fourierspacing  = 0.6
ewald_rtol      = 1e-5
\end{verbatim}
The fourierspacing parameter times the box dimensions 
determines the highest magnitude of wavevectors
$m_x,m_y,m_z$ to use in each direction. With a 3~nm cubic box this example
would use $11$ wavectors (from $-5$ to $5$) in each direction.
The ewald\_rtol parameter is the relative
strength of the electrostatic interaction at the cut-off. Decreasing
this gives you a more accurate direct sum, but a less accurate
reciprocal sum. 
 
\subsection{PME}
\label{sec:pme}
Particle-mesh Ewald is a method proposed by Tom
Darden~\cite{Darden93,Essmann95} to improve the performance of the
reciprocal sum. Instead of directly summing wavevectors, the charges
are assigned to a grid using cardinal B-spline interpolation. This
grid is then fourier transformed with a 3D FFT algorithm and 
the reciprocal energy term obtained by a single sum over the grid in k-space.

The potential at the grid points is calculated by inverse
transformation, and by using the interpolation factors we get the
forces on each atom. 

The PME algorithm scales as $N \log(N)$, and is substantially faster
than ordinary Ewald summation on medium to large systems. On very
small systems it might still be better to use Ewald to avoid the
overhead in setting up grids and transforms.

\subsubsection{Using PME}
To use Particle-mesh Ewald summation in {\gromacs}, specify the
following lines in your {\tt .mdp} file:
\begin{verbatim}
eeltype         = PME
rvdw            = 0.9
rlist           = 0.9
rcoulomb        = 0.9
fourierspacing  = 0.12
pme_order       = 4
ewald_rtol      = 1e-5
\end{verbatim}
In this case the fourierspacing parameter determines the maximum spacing for the
FFT grid and pme\_order controls the interpolation
order. Using 4th order (cubic) interpolation and this spacing 
should give electrostatic energies accurate to about $5\cdot10^{-3}$. Since
the Lennard-Jones energies are not this accurate it might even be possible 
to increase this spacing slightly.

Pressure scaling works with PME, but be aware of the fact that
anisotropic scaling can introduce artificial ordering in some systems.

\subsection{PPPM}
\label{sec:pppm}
The Particle-Particle Particle-Mesh methods of Hockney \& Eastwood
can also be applied in {\gromacs} for the treatment of longe range 
electrostatic interactions~\cite{Hockney81,Darden93,Luty95a}. 
With this algorithm the charges of all particles are spread over a grid of dimensions
($n_x$,$n_y$,$n_z$) using a weighting function called the
triangle-shaped charged distribution:
\beq
\begin{array}{lcl}
W(\ve{r}) &=&   W(x)~W(y)~W(z)  \\[1ex]
W(\xi)  &=& \left\{
\begin{array}{ll}
\frac{3}{4} - \left(\frac{\xi}{h}\right)^2 
        & |\xi| \leq \frac{h}{2}                                \\[0.5ex]
\frac{1}{2}\left(\frac{3}{2} - \frac{|\xi|}{h}\right)^2 
        & \frac{h}{2} < |\xi| < \frac{3h}{2}                    \\[0.5ex]
0       & \frac{3h}{2} \leq |\xi|                               \\[0.5ex]
\end{array}
\right.
\end{array}
\eeq
where $\xi$ (is x, y or z) is the distance to a grid point in the corresponding
dimension. Only the 27 closest grid points need to be taken into account for each charge.

Then, this charge distribution is fourier transformed using a 3D inverse FFT 
routine.
In fourier space a convolution with function $\hat{G}$ is performed:
\beq
\hat{G}(k)      ~=~     \frac{\hat{g}(k)}{\epsilon_0 k^2}
\eeq
where $\hat{g}$ is the fourier transform of the charge spread function
g(r). This yield the long range potential $\hat{\phi}(k)$ on the mesh, which
can be transformed using a forward FFT routine into the real space potential.
Finally the potential and forces are retrieved using interpolation~\cite{Luty95a}.
%
% note - this accuracy is just a rough estimate...
%
It is not easy to calculate the full long-range virial tensor with
PPPM, but it is possible to obtain the trace. This means that the sum
of the pressure components is correct (and therefore the isotropic
pressure) but not necessarily the individual pressure components! 

\subsubsection{Using PPPM}
To use the PPPM algorithm in {\gromacs}, specify the
following lines in your {\tt .mdp} file:
\begin{verbatim}
eeltype         = PPPM
rlist           = 1.0
rcoulomb        = 0.85
rcoulomb_switch = 0.0
rvdw            = 1.0
fourierspacing  = 0.075
\end{verbatim}
For details on the switch parameters see the section on modified
long-range interactions in this manual. When using pppm we 
recommend to take at most 0.075 nm per gridpoint (e.g. 20 gridpoints for 1.5 nm).
PPPM does not provide the same accuracy as PME but is faster in most
cases. PPPM can not be used with pressure coupling.

\subsection{Optimizing fourier transforms}
To get the best possible performance you should try to avoid large
prime numbers for grid dimensions.
The FFT code used in {\gromacs} is
optimized for grid sizes of the form $2^a 3^b 5^c 7^d 11^e 13^f$,
where $e+f$ is $0$ or $1$ and the other exponents arbitrary. (See
further the documentation of the FFT algorithms at {\tt http://www.fftw.org}.)

It is also possible to optimize the transforms for the current problem
by performing some calculations at the start of the run. This is not
done per default since it takes a couple of minutes, but for large
runs it will save time. Turn it on by specifying

\begin{verbatim}
optimize_fft      = yes
\end{verbatim}
in your {\tt .mdp} file.

When running in parallel the grid must be communicated several times
and thus hurting scaling performance. With PME you can improve this
by increasing grid spacing while simultaneously increasing the
interpolation to e.g. 6th order. 
Since the interpolation is entirely local a this will
improve the scaling in most cases.

%
% Temporarily removed since I'm not sure about the state of the testlr 
% program...
%
%It is possible to test the accuracy of your settings using the program 
%{\tt\myindex{testlr}} in the {\tt src/gmxlib} dir. This program computes
%forces and potentials using PPPM and an Ewald implementation and gives the
%absolute and RMS errors in both:
%\begin{verbatim}
%ERROR ANALYSIS
%Error:         Max Abs         RMS            
%Force            1.132       0.251
%Potential        0.113       0.035
%\end{verbatim}
%{\bf Note:} these numbers were generated using a grid spacing of
%0.058 nm and $r_c$ = 1.0 nm.
%
%You can see what the accuracy is without optimizing the
%$\hat{G}(k)$ function, if you pass the {\tt -ghat} option to {\tt
%testlr}. Try it if you think the {\tt mk\_ghat} procedure is a waste
%of time.
