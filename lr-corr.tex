%
% $Id$
% 
%       This source code is part of
% 
%        G   R   O   M   A   C   S
% 
% GROningen MAchine for Chemical Simulations
% 
%               VERSION 2.0
% 
% Copyright (c) 1991-1999
% BIOSON Research Institute, Dept. of Biophysical Chemistry
% University of Groningen, The Netherlands
% 
% Please refer to:
% GROMACS: A message-passing parallel molecular dynamics implementation
% H.J.C. Berendsen, D. van der Spoel and R. van Drunen
% Comp. Phys. Comm. 91, 43-56 (1995)
% 
% Also check out our WWW page:
% http://md.chem.rug.nl/~gmx
% or e-mail to:
% gromacs@chem.rug.nl
% 
% And Hey:
% Gnomes, ROck Monsters And Chili Sauce
%

%\newcommand{\dr}{{\rm d}r}
%\newcommand{\avcsix}{\left< C_6 \right>}
\chapter{Long range corrections}
\label{ch:lrcorr}
\section{Dispersion}
\index{dispersion correction}
In this section we derive long range corrections due to the use of a
cut-off for Lennard Jones or Buckingham interactions.
We assume that the cut-off is
so long that the repulsion term can safely be neglected, and therefore
only the dispersion term is taken into account. Due to the nature of
the dispersion interaction, energy and pressure corrections both are
negative. While the energy correction is usually small, it may be
important for free energy calculations. The pressure correction in
contrast is very large and can not be neglected. Although it is in
principle possible to parameterize a force field such that the pressure
is close to 1 bar even without correction, such a method makes the
parameterization dependent on the cut-off and is therefore
undesirable. Please note that it is not consistent to use the long
range correction to the dispersion without using either a
\normindex{reaction field} method or a proper long range
electrostatics method such as Ewald summation or PPPM.

\subsection{Energy}
\label{sec:ecorr}
The long range contribution of the dispersion interaction to the
virial can be derived analytically, if we assume a homogeneous
system beyond the cut-off distance $r_c$. The dispersion energy
between two particles is written as:
\beq
V(\rij)	~=~	- C_6 \rij^{-6}
\eeq
and the corresponding force is
\beq
\Fvij	~=~	- 6\,C_6 \rij^{-8}\rvij
\eeq
In a periodic system it is not easy to calculate the full potentials,
so usually a cut-off is applied, which can be abrupt or smooth.
We will call the potential and force with cut-off $V_c$ and $\ve{F}_c$/
The long range contribution to the dispersion energy
in a system with $N$ particles and particle density $\rho$ = $N/V$ is:
\beq
V_{lr}  ~=~ \half N \rho\int_0^{\infty}   4\pi r^2 g(r) \left( V(r) -V_c(r) \right) {\dr}
\eeq
which we can integrate assuming that the radial distribution function $g(r)$ 
is 1 beyond the cut-off:
\beq
V_{lr}  ~=~ \half N \rho\int_0^{\infty}   4\pi r^2 \left( V(r) -V_c(r) \right) {\dr}
\eeq
For a plain cut-off at $r_c$ the integration can be performed analytically~\cite{Allen87}:
\bea
V_{lr} & = & \half N \rho\int_{r_c}^{\infty}   4\pi r^2 C_6 r^{-6} {\dr} \\
       & = & -\frac{2}{3} \pi N \rho\, C_6 r_c^{-3}
\eea
If we consider for example a box of pure water, simulated with a cut-off
of 0.9 nm and a density of 1 g cm$^{-3}$ this correction is 
-0.25 kJ mol$^{-1}$.

For a homogeneous mixture we need to define 
n {\em average dispersion constant}:
\beq
\label{eqn:avcsix}
\avcsix	= \frac{2}{N(N-1)}\sum_i^N\sum_{j>i}^N C_6(i,j)\\
\eeq
A special form of a non-homogeneous system in this respect,
is a pure liquid in which the atoms have different $C_6$ values.
In practice this definition encompasses almost every molecule,
except mono-atomic molecules and symmetric molecules like $N_2$ or $O_2$.
Therefore we always have to determine the average dispersion constant
$\avcsix$ in simulations.

In the case of inhomogeneous simulation systems, {\eg} a system with a
lipid interface, the energy correction can be applied if 
$\avcsix$ for both components is comparable.

\subsection{Virial and pressure}
The scalar virial of the system due to the dispersion interaction between
two particles $i$ and $j$ is given by:
\beq
\Xi	~=~	-\rvij \cdot \Fvij ~=~	6\,C_6 \rij^{-6}
\eeq
The pressure is given by:
\beq
P	~=~	\frac{2}{3\,V}\left(E_{kin} - \Xi\right)
\eeq
The long-range correction to the virial is given by:
\beq
\Xi_{lr} ~=~ \half N \rho \int_0^{\infty} 4\pi r^2 \, (\Xi -\Xi_c) \dr
\eeq
We can again integrate the long range contribution to the 
virial for a plain cut-off~\cite{Allen87}:
\bea
\Xi_{lr}&=&	\half N \rho \int_{r_c}^{\infty} 4\pi r^2 6\,C_6 \rij^{-6}\,  \dr	\nonumber\\
	&=&	12 \pi N \rho\,C_6  \int_{r_c}^{\infty} \rij^{-4}\dr \nonumber\\
	&=&	4 \pi N C_6 \rho r_c^{-3}
\eea
The corresponding correction to the pressure is
\beq
P_{lr}	~=~	-\frac{4}{3} \pi C_6 \rho^2 r_c^{-3}
\eeq
Using the same example of a water box, the correction to the virial is
3 kJ mol$^{-1}$ the corresponding correction to the pressure for 
SPC water at liquid density is approx. -280 bar.

For homogeneous mixtures we can again use the average dispersion constant
$\avcsix$ (\eqnref{avcsix}):
\beq
P_{lr}	~=~	-\frac{4}{3} \pi \avcsix \rho^2 r_c^{-3}
\label{eqn:pcorr}
\eeq
For inhomogeneous systems \eqnref{pcorr} can be applied under the same
restriction as holds for the energy (see \secref{ecorr}).
