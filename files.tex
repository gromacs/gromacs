\begin{table}[p]
\begin{tabularx}{\linewidth}{rlccX}
\hline
    Def. &           &      & Default &  \\
    Name & Extension & Type &  Option & Description \\
\hline
\tt   grompp & \tt .mdp & Asc & \tt -f & grompp input file with MD parameters \\
\tt     traj & \tt .??? &     & \tt -f & Generic trajectory: xtc trr trj gro g96 pdb \\
\tt     traj & \tt .??? &     & \tt    & Full precision trajectory: trr trj \\
\tt     traj & \tt .trr & XDR & \tt    & Trajectory in portable xdr format \\
\tt     traj & \tt .trj & Bin & \tt    & Trajectory file (cpu specific) \\
\tt     traj & \tt .xtc & XDR & \tt    & Compressed trajectory (portable xdr format) \\
\tt    gtraj & \tt .g87 & Asc & \tt    & Gromos-87 ASCII trajectory format \\
\tt     ener & \tt .??? &     & \tt    & Generic energy: edr ene \\
\tt     ener & \tt .edr & XDR & \tt    & Energy file in portable XDR format \\
\tt     ener & \tt .ene & Bin & \tt    & Energy file \\
\tt     conf & \tt .??? &     & \tt -c & Generic structure: gro g96 pdb tpr tpb tpa \\
\tt      out & \tt .??? &     & \tt -o & Generic structure: gro g96 pdb \\
\tt     conf & \tt .gro & Asc & \tt -c & Coordinate file in Gromos-87 format \\
\tt     conf & \tt .g96 & Asc & \tt -c & Coordinate file in Gromos-96 format \\
\tt    eiwit & \tt .pdb & Asc & \tt -f & Protein data bank file \\
\tt    eiwit & \tt .brk & Asc & \tt -f & Brookhaven data bank file \\
\tt    eiwit & \tt .ent & Asc & \tt -f & Entry in the protein date bank \\
\tt      run & \tt .log & Asc & \tt -l & Log file \\
\tt    graph & \tt .xvg & Asc & \tt -o & xvgr/xmgr file \\
\tt    hello & \tt .out & Asc & \tt -o & Generic output file \\
\tt    index & \tt .ndx & Asc & \tt -n & Index file \\
\tt    topol & \tt .top & Asc & \tt -p & Topology file \\
\tt   topinc & \tt .itp & Asc & \tt    & Include file for topology \\
\tt    topol & \tt .??? &     & \tt -s & Generic run input: tpr tpb tpa \\
\tt    topol & \tt .??? &     & \tt -s & Structure+mass(db): tpr tpb tpa gro g96 pdb \\
\tt    topol & \tt .tpr & XDR & \tt -s & Portable xdr run input file \\
\tt    topol & \tt .tpa & Asc & \tt -s & Ascii run input file \\
\tt    topol & \tt .tpb & Bin & \tt -s & Binary run input file \\
\tt      doc & \tt .tex & Asc & \tt -o & LaTeX file \\
\tt  residue & \tt .rtp & Asc & \tt    & Residue Type file used by pdb2gmx \\
\tt   atomtp & \tt .atp & Asc & \tt    & Atomtype file used by pdb2gmx \\
\tt    polar & \tt .hdb & Asc & \tt    & Hydrogen data base \\
\tt   nnnice & \tt .dat & Asc & \tt    & Generic data file \\
\tt     user & \tt .dlg & Asc & \tt    & Dialog Box data for ngmx \\
\tt       ss & \tt .map & Asc & \tt    & File that maps matrix data to colors \\
\tt     plot & \tt .eps & Asc & \tt    & Encapsulated PostScript (tm) file \\
\tt       ss & \tt .mat & Asc & \tt    & Matrix Data file \\
\tt       ps & \tt .m2p & Asc & \tt    & Input file for mat2ps \\
\tt  hessian & \tt .mtx & Bin & \tt -m & Hessian matrix \\
\tt      sam & \tt .edi & Asc & \tt    & ED sampling input \\
\tt      sam & \tt .edo & Asc & \tt    & ED sampling output \\
\tt     pull & \tt .ppa & Asc & \tt    & Pull parameters \\
\tt     pull & \tt .pdo & Asc & \tt    & Pull data output \\
\tt     root & \tt .xpm & Asc & \tt    & X PixMap compatible matrix file \\
\hline
\end{tabularx}
\caption{File types: A = ascii, B = binary, P = XDR portable.}
\label{Tab:form}
\end{table}
