\begin{description}
\item {\large\bf Generating topologies and coordinates}
\vspace{-2ex}\begin{tabbing}
{\bf g_helixorient} \= \kill
{\bf pdb2gmx} \> converts pdb files to topology and coordinate files \\
{\bf g_x2top} \> generates a primitive topology from coordinates  \\
{\bf editconf} \> edits the box and writes subgroups  \\
{\bf genbox} \> solvates a system \\
{\bf genion} \> generates mono atomic ions on energetically favorable positions \\
{\bf genconf} \> multiplies a conformation in 'random' orientations \\
{\bf genrestr} \> generates position restraints or distance restraints for index groups \\
{\bf g_protonate} \> protonates structures \\
\end{tabbing}\vspace{-2ex}

\item {\large\bf Running a simulation}
\vspace{-2ex}\begin{tabbing}
{\bf g_helixorient} \= \kill
{\bf grompp} \> makes a run input file \\
{\bf tpbconv} \> makes a run input file for restarting a crashed run \\
{\bf mdrun} \> performs a simulation, do a normal mode analysis or an energy minimization \\
\end{tabbing}\vspace{-2ex}

\item {\large\bf Viewing trajectories}
\vspace{-2ex}\begin{tabbing}
{\bf g_helixorient} \= \kill
{\bf ngmx} \> displays a trajectory \\
{\bf g_highway} \> X Window System gadget for highway simulations \\
{\bf g_nmtraj} \> generate a virtual trajectory from an eigenvector \\
\end{tabbing}\vspace{-2ex}

\item {\large\bf Processing energies}
\vspace{-2ex}\begin{tabbing}
{\bf g_helixorient} \= \kill
{\bf g_energy} \> writes energies to xvg files and displays averages \\
{\bf g_enemat} \> extracts an energy matrix from an energy file \\
{\bf mdrun} \> with -rerun (re)calculates energies for trajectory frames \\
\end{tabbing}\vspace{-2ex}

\item {\large\bf Converting files}
\vspace{-2ex}\begin{tabbing}
{\bf g_helixorient} \= \kill
{\bf editconf} \> converts and manipulates structure files \\
{\bf trjconv} \> converts and manipulates trajectory files \\
{\bf trjcat} \> concatenates trajectory files \\
{\bf eneconv} \> converts energy files \\
{\bf xpm2ps} \> converts XPM matrices to encapsulated postscript (or XPM) \\
{\bf g_sigeps} \> convert c6/12 or c6/cn combinations to and from sigma/epsilon \\
\end{tabbing}\vspace{-2ex}

\item {\large\bf Tools}
\vspace{-2ex}\begin{tabbing}
{\bf g_helixorient} \= \kill
{\bf make_ndx} \> makes index files \\
{\bf mk_angndx} \> generates index files for g_angle \\
{\bf gmxcheck} \> checks and compares files \\
{\bf gmxdump} \> makes binary files human readable \\
{\bf g_traj} \> plots x, v and f of selected atoms/groups (and more) from a trajectory \\
{\bf g_analyze} \> analyzes data sets \\
{\bf trjorder} \> orders molecules according to their distance to a group \\
{\bf g_filter} \> frequency filters trajectories, useful for making smooth movies \\
{\bf g_lie} \> free energy estimate from linear combinations \\
{\bf g_dyndom} \> interpolate and extrapolate structure rotations \\
{\bf g_morph} \> linear interpolation of conformations  \\
{\bf g_wham} \> weighted histogram analysis after umbrella sampling \\
{\bf xpm2ps} \> convert XPM (XPixelMap) file to postscript \\
{\bf g_sham} \> read/write xmgr and xvgr data sets \\
{\bf g_spatial} \> calculates the spatial distribution function (more control than g_sdf) \\
{\bf g_sdf} \> calculates the spatial distribution function (faster than g_spatial) \\
{\bf g_select} \> selects groups of atoms based on flexible textual selections \\
{\bf g_tune_pme} \> time mdrun as a function of PME nodes to optimize settings \\
\end{tabbing}\vspace{-2ex}

\item {\large\bf Distances between structures}
\vspace{-2ex}\begin{tabbing}
{\bf g_helixorient} \= \kill
{\bf g_rms} \> calculates rmsd's with a reference structure and rmsd matrices \\
{\bf g_confrms} \> fits two structures and calculates the rmsd  \\
{\bf g_cluster} \> clusters structures \\
{\bf g_rmsf} \> calculates atomic fluctuations \\
\end{tabbing}\vspace{-2ex}

\item {\large\bf Distances in structures over time}
\vspace{-2ex}\begin{tabbing}
{\bf g_helixorient} \= \kill
{\bf g_mindist} \> calculates the minimum distance between two groups \\
{\bf g_dist} \> calculates the distances between the centers of mass of two groups \\
{\bf g_bond} \> calculates distances between atoms \\
{\bf g_mdmat} \> calculates residue contact maps \\
{\bf g_polystat} \> calculates static properties of polymers \\
{\bf g_rmsdist} \> calculates atom pair distances averaged with power -2, -3 or -6 \\
\end{tabbing}\vspace{-2ex}

\item {\large\bf Mass distribution properties over time}
\vspace{-2ex}\begin{tabbing}
{\bf g_helixorient} \= \kill
{\bf g_traj} \> plots x, v, f, box, temperature and rotational energy \\
{\bf g_gyrate} \> calculates the radius of gyration \\
{\bf g_msd} \> calculates mean square displacements \\
{\bf g_polystat} \> calculates static properties of polymers \\
{\bf g_rotacf} \> calculates the rotational correlation function for molecules \\
{\bf g_rdf} \> calculates radial distribution functions \\
{\bf g_rotmat} \> plots the rotation matrix for fitting to a reference structure \\
{\bf g_vanhove} \> calculates Van Hove displacement functions \\
\end{tabbing}\vspace{-2ex}

\item {\large\bf Analyzing bonded interactions}
\vspace{-2ex}\begin{tabbing}
{\bf g_helixorient} \= \kill
{\bf g_bond} \> calculates bond length distributions \\
{\bf mk_angndx} \> generates index files for g_angle \\
{\bf g_angle} \> calculates distributions and correlations for angles and dihedrals \\
{\bf g_dih} \> analyzes dihedral transitions \\
\end{tabbing}\vspace{-2ex}

\item {\large\bf Structural properties}
\vspace{-2ex}\begin{tabbing}
{\bf g_helixorient} \= \kill
{\bf g_hbond} \> computes and analyzes hydrogen bonds \\
{\bf g_saltbr} \> computes salt bridges \\
{\bf g_sas} \> computes solvent accessible surface area \\
{\bf g_order} \> computes the order parameter per atom for carbon tails \\
{\bf g_principal} \> calculates axes of inertia for a group of atoms \\
{\bf g_rdf} \> calculates radial distribution functions \\
{\bf g_sdf} \> calculates solvent distribution functions \\
{\bf g_sgangle} \> computes the angle and distance between two groups \\
{\bf g_sorient} \> analyzes solvent orientation around solutes \\
{\bf g_spol} \> analyzes solvent dipole orientation and polarization around solutes \\
{\bf g_bundle} \> analyzes bundles of axes, {\eg} helices \\
{\bf g_disre} \> analyzes distance restraints \\
{\bf g_clustsize} \> calculate size distributions of atomic clusters \\
{\bf g_anadock} \> cluster structures from Autodock runs \\
\end{tabbing}\vspace{-2ex}

\item {\large\bf Kinetic properties}
\vspace{-2ex}\begin{tabbing}
{\bf g_helixorient} \= \kill
{\bf g_traj} \> plots x, v, f, box, temperature and rotational energy \\
{\bf g_velacc} \> calculates velocity autocorrelation functions \\
{\bf g_tcaf} \> calculates viscosities of liquids \\
{\bf g_bar} \> calculates free energy difference estimates through Bennett's acceptance ratio \\
{\bf g_current} \> calculate current autocorrelation function of system \\
{\bf g_vanhove} \> compute Van Hove correlation function \\
{\bf g_principal} \> calculate principal axes of inertion for a group of atoms \\
\end{tabbing}\vspace{-2ex}

\item {\large\bf Electrostatic properties}
\vspace{-2ex}\begin{tabbing}
{\bf g_helixorient} \= \kill
{\bf genion} \> generates mono atomic ions on energetically favorable positions \\
{\bf g_potential} \> calculates the electrostatic potential across the box \\
{\bf g_dipoles} \> computes the total dipole plus fluctuations \\
{\bf g_dielectric} \> calculates frequency dependent dielectric constants \\
{\bf g_current} \> calculates dielectric constants for charged systems \\
{\bf g_spol} \> analyze dipoles around a solute \\
\end{tabbing}\vspace{-2ex}

\item {\large\bf Protein specific analysis}
\vspace{-2ex}\begin{tabbing}
{\bf g_helixorient} \= \kill
{\bf do_dssp} \> assigns secondary structure and calculates solvent accessible surface area \\
{\bf g_chi} \> calculates everything you want to know about chi and other dihedrals \\
{\bf g_helix} \> calculates basic properties of alpha helices \\
{\bf g_helixorient} \> calculates local pitch/bending/rotation/orientation inside helices \\
{\bf g_rama} \> computes Ramachandran plots \\
{\bf g_xrama} \> shows animated Ramachandran plots \\
{\bf g_wheel} \> plots helical wheels \\
\end{tabbing}\vspace{-2ex}

\item {\large\bf Interfaces}
\vspace{-2ex}\begin{tabbing}
{\bf g_helixorient} \= \kill
{\bf g_potential} \> calculates the electrostatic potential across the box \\
{\bf g_density} \> calculates the density of the system \\
{\bf g_densmap} \> calculates 2D planar or axial-radial density maps \\
{\bf g_order} \> computes the order parameter per atom for carbon tails \\
{\bf g_h2order} \> computes the orientation of water molecules \\
{\bf g_bundle} \> analyzes bundles of axes, {\eg} transmembrane helices \\
{\bf g_membed} \> embeds a protein into a lipid bilayer \\
\end{tabbing}\vspace{-2ex}

\item {\large\bf Covariance analysis}
\vspace{-2ex}\begin{tabbing}
{\bf g_helixorient} \= \kill
{\bf g_covar} \> calculates and diagonalizes the covariance matrix \\
{\bf g_anaeig} \> analyzes the eigenvectors \\
{\bf make_edi} \> generate input files for essential dynamics sampling \\
\end{tabbing}\vspace{-2ex}

\item {\large\bf Normal modes}
\vspace{-2ex}\begin{tabbing}
{\bf g_helixorient} \= \kill
{\bf grompp} \> makes a run input file \\
{\bf mdrun} \> finds a potential energy minimum \\
{\bf mdrun} \> calculates the Hessian \\
{\bf g_nmeig} \> diagonalizes the Hessian  \\
{\bf g_nmtraj} \> generate oscillating trajectory of an eigenmode \\
{\bf g_anaeig} \> analyzes the normal modes \\
{\bf g_nmens} \> generates an ensemble of structures from the normal modes \\
\end{tabbing}\vspace{-2ex}

\end{description}

