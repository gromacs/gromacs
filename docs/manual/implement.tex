%
% This file is part of the GROMACS molecular simulation package.
%
% Copyright (c) 2013,2014,2015, by the GROMACS development team, led by
% Mark Abraham, David van der Spoel, Berk Hess, and Erik Lindahl,
% and including many others, as listed in the AUTHORS file in the
% top-level source directory and at http://www.gromacs.org.
%
% GROMACS is free software; you can redistribute it and/or
% modify it under the terms of the GNU Lesser General Public License
% as published by the Free Software Foundation; either version 2.1
% of the License, or (at your option) any later version.
%
% GROMACS is distributed in the hope that it will be useful,
% but WITHOUT ANY WARRANTY; without even the implied warranty of
% MERCHANTABILITY or FITNESS FOR A PARTICULAR PURPOSE.  See the GNU
% Lesser General Public License for more details.
%
% You should have received a copy of the GNU Lesser General Public
% License along with GROMACS; if not, see
% http://www.gnu.org/licenses, or write to the Free Software Foundation,
% Inc., 51 Franklin Street, Fifth Floor, Boston, MA  02110-1301  USA.
%
% If you want to redistribute modifications to GROMACS, please
% consider that scientific software is very special. Version
% control is crucial - bugs must be traceable. We will be happy to
% consider code for inclusion in the official distribution, but
% derived work must not be called official GROMACS. Details are found
% in the README & COPYING files - if they are missing, get the
% official version at http://www.gromacs.org.
%
% To help us fund GROMACS development, we humbly ask that you cite
% the research papers on the package. Check out http://www.gromacs.org.

\chapter{Some implementation details}
In this chapter we will present some implementation details. This is
far from complete, but we deemed it necessary to clarify some things
that would otherwise be hard to understand.

\section{Single Sum Virial in {\gromacs}}
\label{sec:virial}
The \normindex{virial} $\Xi$ can be written in full tensor form as:
\beq
\Xi~=~-\half~\sum_{i < j}^N~\rvij\otimes\Fvij
\eeq
where $\otimes$ denotes the {\em direct product} of two vectors.\footnote
{$({\bf u}\otimes{\bf v})^{\ab}~=~{\bf u}_{\al}{\bf v}_{\be}$} When this is 
computed in the inner loop of an MD program 9 multiplications and 9
additions are needed.\footnote{The calculation of 
Lennard-Jones and Coulomb forces is about 50 floating point operations.}

Here it is shown how it is possible to extract the virial calculation
from the inner loop~\cite{Bekker93b}.

\subsection{Virial}
In a system with periodic boundary conditions\index{periodic boundary 
conditions}, the
periodicity must be taken into account for the virial:
\beq
\Xi~=~-\half~\sum_{i < j}^{N}~\rnij\otimes\Fvij
\eeq
where $\rnij$ denotes the distance vector of the
{\em nearest image} of atom $i$ from atom $j$. In this definition we add
a {\em shift vector} $\delta_i$ to the position vector $\rvi$ 
of atom $i$. The difference vector $\rnij$ is thus equal to:
\beq
\rnij~=~\rvi+\delta_i-\rvj
\eeq
or in shorthand:
\beq
\rnij~=~\rni-\rvj
\eeq
In a triclinic system, there are 27 possible images of $i$; when a truncated 
octahedron is used, there are 15 possible images.

\subsection{Virial from non-bonded forces}
Here the derivation for the single sum virial in the {\em non-bonded force} 
routine is given. $i \neq j$ in all formulae below.
\newcommand{\di}{\delta_{i}}
\newcommand{\qrt}{\frac{1}{4}}
\bea
\Xi	
&~=~&-\half~\sum_{i < j}^{N}~\rnij\otimes\Fvij				\\
&~=~&-\qrt\sum_{i=1}^N~\sum_{j=1}^N ~(\rvi+\di-\rvj)\otimes\Fvij	\\
&~=~&-\qrt\sum_{i=1}^N~\sum_{j=1}^N ~(\rvi+\di)\otimes\Fvij-\rvj\otimes\Fvij	\\
&~=~&-\qrt\left(\sum_{i=1}^N~\sum_{j=1}^N ~(\rvi+\di)\otimes\Fvij~-~\sum_{i=1}^N~\sum_{j=1}^N ~\rvj\otimes\Fvij\right)	\\
&~=~&-\qrt\left(\sum_{i=1}^N~(\rvi+\di)\otimes\sum_{j=1}^N~\Fvij~-~\sum_{j=1}^N ~\rvj\otimes\sum_{i=1}^N~\Fvij\right)	\\
&~=~&-\qrt\left(\sum_{i=1}^N~(\rvi+\di)\otimes\Fvi~+~\sum_{j=1}^N ~\rvj\otimes\Fvj\right)	\\
&~=~&-\qrt\left(2~\sum_{i=1}^N~\rvi\otimes\Fvi+\sum_{i=1}^N~\di\otimes\Fvi\right)
\eea
In these formulae we introduced:
\bea
\Fvi&~=~&\sum_{j=1}^N~\Fvij					\\
\Fvj&~=~&\sum_{i=1}^N~\Fvji
\eea
which is the total force on $i$ with respect to $j$. Because we use Newton's Third Law:
\beq
\Fvij~=~-\Fvji
\eeq
we must, in the implementation, double the term containing the shift $\delta_i$.

\subsection{The intra-molecular shift (mol-shift)}
For the bonded forces and SHAKE it is possible to make a {\em mol-shift}
list, in which the periodicity is stored. We simple have an array {\tt mshift}
in which for each atom an index in the {\tt shiftvec} array is stored.

The algorithm to generate such a list can be derived from graph theory,
considering each particle in a molecule as a bead in a graph, the bonds 
as edges.
\begin{enumerate}
\item[1]	Represent the bonds and atoms as bidirectional graph
\item[2]	Make all atoms white
\item[3]	Make one of the white atoms black (atom $i$) and put it in the
		central box
\item[4]	Make all of the neighbors of $i$ that are currently 
		white, gray 
\item[5]	Pick one of the gray atoms (atom $j$), give it the
		correct periodicity with respect to any of 
		its black neighbors
		and make it black
\item[6]	Make all of the neighbors of $j$ that are currently 
		white, gray
\item[7]	If any gray atom remains, go to [5]
\item[8]	If any white atom remains, go to [3]
\end{enumerate}
Using this algorithm we can 
\begin{itemize}
\item	optimize the bonded force calculation as well as SHAKE 
\item	calculate the virial from the bonded forces
	in the single sum method again
\end{itemize}

Find a representation of the bonds as a bidirectional graph.

\subsection{Virial from Covalent Bonds}
Since the covalent bond force gives a contribution to the virial, we have:
\bea
b	&~=~&	\|\rnij\|					\\
V_b	&~=~&	\half k_b(b-b_0)^2				\\
\Fvi	&~=~&	-\nabla V_b					\\
	&~=~&	k_b(b-b_0)\frac{\rnij}{b}			\\
\Fvj	&~=~&	-\Fvi
\eea
The virial contribution from the bonds then is:
\bea
\Xi_b	&~=~&	-\half(\rni\otimes\Fvi~+~\rvj\otimes\Fvj)	\\
	&~=~&	-\half\rnij\otimes\Fvi
\eea

\subsection{Virial from SHAKE}
An important contribution to the virial comes from shake. Satisfying 
the constraints a force {\bf G} that is exerted on the particles ``shaken.'' If this
force does not come out of the algorithm (as in standard SHAKE) it can be
calculated afterward (when using {\em leap-frog}) by:
\bea
\Delta\rvi&~=~&\rvi(t+\Dt)-
[\rvi(t)+{\bf v}_i(t-\frac{\Dt}{2})\Dt+\frac{\Fvi}{m_i}\Dt^2]	\\
{\bf G}_i&~=~&\frac{m_i\Delta\rvi}{\Dt^2}
\eea
This does not help us in the general case. Only when no periodicity
is needed (like in rigid water) this can be used, otherwise
we must add the virial calculation in the inner loop of SHAKE.

When it {\em is} applicable the virial can be calculated in the single sum way:
\beq
\Xi~=~-\half\sum_i^{N_c}~\rvi\otimes\Fvi
\eeq
where $N_c$ is the number of constrained atoms.

%Another method is the Non-Iterative shake as proposed (and implemented)
%by Yoneya. In this algorithm the Lagrangian multipliers are solved in a 
%matrix equation, and given these multipliers it is easy to get the periodicity
%in the virial afterwards. 

%More...


\section{Optimizations}
Here we describe some of the algorithmic optimizations used 
in {\gromacs}, apart from parallelism. 

\subsection{Inner Loops for Water}
\label{sec:waterloops}
{\gromacs} uses special inner loops to calculate non-bonded
interactions for water molecules with other atoms, and yet
another set of loops for interactions between pairs of
water molecules. There highly optimized loops for two types of water models.
For three site models similar to
SPC~\cite{Berendsen81}, {\ie}:
\begin{enumerate}
\item   There are three atoms in the molecule.
\item   The whole molecule is a single charge group.
\item   The first atom has Lennard-Jones (\secref{lj}) and 
        Coulomb (\secref{coul}) interactions.
\item   Atoms two and three have only Coulomb interactions, 
        and equal charges.
\end{enumerate}
These loops also works for the SPC/E~\cite{Berendsen87} and 
TIP3P~\cite{Jorgensen83} water models.
And for four site water models similar to TIP4P~\cite{Jorgensen83}:
\begin{enumerate}
\item   There are four atoms in the molecule.
\item   The whole molecule is a single charge group.
\item   The first atom has only Lennard-Jones (\secref{lj}) interactions.
\item   Atoms two and three have only Coulomb (\secref{coul}) interactions, 
        and equal charges.
\item   Atom four has only Coulomb interactions.
\end{enumerate}

The benefit of these implementations is that there are more floating-point
operations in a single loop, which implies that some compilers
can schedule the code better. However, it turns out that even
some of the most advanced compilers have problems with scheduling, 
implying that manual tweaking is necessary to get optimum 
\normindex{performance}.
This may include common-sub-expression elimination, or moving
code around. 

% LocalWords:  Virial virial triclinic intra mol mshift shiftvec sqrt SPC lj yf
% LocalWords:  coul
