Algorithms
==========

Introduction
------------

In this chapter we first give describe some general concepts used in
|Gromacs|: *periodic boundary conditions*
(sec. 
) and the *group concept*
(sec. 
). The MD algorithm is
described in sec. 
first a global form of the
algorithm is given, which is refined in subsequent subsections. The
(simple) EM (Energy Minimization) algorithm is described in
sec. 
Some other algorithms for special
purpose dynamics are described after this.

A few issues are of general interest. In all cases the *system* must be
defined, consisting of molecules. Molecules again consist of particles
with defined interaction functions. The detailed description of the
*topology* of the molecules and of the *force field* and the calculation
of forces is given in chapter 
In the present
chapter we describe other aspects of the algorithm, such as pair list
generation, update of velocities and positions, coupling to external
temperature and pressure, conservation of constraints.

The *analysis* of the data generated by an MD simulation is treated in 

Periodic boundary conditions
----------------------------

.. _fig-pbc:

.. figure:: plots/pbctric.eps

   Periodic boundary conditions in two dimensions.

The classical way to minimize edge effects in a finite system is to
apply *periodic boundary conditions*. The atoms of the system to be
simulated are put into a space-filling box, which is surrounded by
translated copies of itself (:numref:`Fig. (%s) <fig-pbc>`). Thus
there are no boundaries of the system; the artifact caused by unwanted
boundaries in an isolated cluster is now replaced by the artifact of
periodic conditions. If the system is crystalline, such boundary
conditions are desired (although motions are naturally restricted to
periodic motions with wavelengths fitting into the box). If one wishes
to simulate non-periodic systems, such as liquids or solutions, the
periodicity by itself causes errors. The errors can be evaluated by
comparing various system sizes; they are expected to be less severe than
the errors resulting from an unnatural boundary with vacuum.

There are several possible shapes for space-filling unit cells. Some,
like the *rhombic dodecahedron*
and the *truncated
octahedron*
Adams, Adams, and
Hills (1979) are closer to being a sphere than a cube is, and are
therefore better suited to the study of an approximately spherical
macromolecule in solution, since fewer solvent molecules are required to
fill the box given a minimum distance between macromolecular images. At
the same time, rhombic dodecahedra and truncated octahedra are special
cases of *triclinic* unit cells
the most general space-filling unit cells that comprise all possible
space-filling shapes Bekker et al. (1995). For this reason, |Gromacs| is
based on the triclinic unit cell.

|Gromacs| uses periodic boundary conditions, combined with the *minimum
image convention*
only one
– the nearest – image of each particle is considered for short-range
non-bonded interaction terms. For long-range electrostatic interactions
this is not always accurate enough, and |Gromacs| therefore also
incorporates lattice sum methods such as Ewald Sum, PME and PPPM.

|Gromacs| supports triclinic boxes of any shape. The simulation box (unit
cell) is defined by the 3 box vectors :math:`{\bf a}`,\ :math:`{\bf b}`
and :math:`{\bf c}`. The box vectors must satisfy the following
conditions:

.. math:: a_y = a_z = b_z = 0
          :label: eqnboxrot

.. math:: a_x>0,~~~~b_y>0,~~~~c_z>0
          :label: eqnboxshift

.. math:: |b_x| \leq \frac{1}{2} \, a_x,~~~~
          |c_x| \leq \frac{1}{2} \, a_x,~~~~
          |c_y| \leq \frac{1}{2} \, b_y
          :label: eqnboxshift2

Equations :eq:`(%s) <eqnboxrot>` can always be satisfied by
rotating the box. Inequalities (:eq:`(%s) <eqnboxshift>`) and
(:eq:`(%s) <eqnboxshift2>`) can always be satisfied by adding
and subtracting box vectors.

Even when simulating using a triclinic box, |Gromacs| always keeps the
particles in a brick-shaped volume for efficiency, as illustrated in
:numref:`Fig. (%s) <fig-pbc>` for a 2-dimensional system. Therefore,
from the output trajectory it might seem that the simulation was done in
a rectangular box. The program :ref:`trjconv <gmx trjconv>` can be used to
convert the trajectory to a different unit-cell representation.

It is also possible to simulate without periodic boundary conditions,
but it is usually more efficient to simulate an isolated cluster of
molecules in a large periodic box, since fast grid searching can only be
used in a periodic system.

.. _fig-boxshapes:

.. figure:: plots/rhododec.eps

     A rhombic dodecahedron and truncated octahedron (arbitrary
     orientations).

Some useful box types
~~~~~~~~~~~~~~~~~~~~~

.. |mathd| replace:: :math:`d`
.. |mathd3| replace:: :math:`d^{3}`
.. |mathd23| replace:: :math:`\frac{1}{2}\sqrt{2}~d^{3}`
.. |mathd70| replace:: :math:`0.707~d^{3}`
.. |mathd43| replace:: :math:`\frac{4}{9}\sqrt{3}~d^{3}`
.. |mathd77| replace:: :math:`0.770~d^{3}`
.. |math12d| replace:: :math:`\frac{1}{2}~d`
.. |math13d| replace:: :math:`\frac{1}{3}~d`
.. |math13dn| replace:: :math:`-\frac{1}{3}~d`
.. |math12s2| replace:: :math:`\frac{1}{2}\sqrt{2}~d`
.. |math12s3| replace:: :math:`\frac{1}{2}\sqrt{3}~d`
.. |math16s3| replace:: :math:`\frac{1}{6}\sqrt{3}~d`
.. |math13s6| replace:: :math:`\frac{1}{3}\sqrt{6}~d`
.. |math23s2| replace:: :math:`\frac{2}{3}\sqrt{2}~d`
.. |math13s2| replace:: :math:`\frac{1}{3}\sqrt{2}~d`
.. |angbc| replace:: :math:`\angle` **bc** 
.. |angac| replace:: :math:`\angle` **ac** 
.. |angab| replace:: :math:`\angle` **ab** 
.. |90deg| replace:: :math:`90^\circ`
.. |60deg| replace:: :math:`60^\circ`
.. |71deg| replace:: :math:`71.53^\circ`
.. |109deg| replace:: :math:`109.47^\circ`

.. _table-boxtypes:

.. table:: Overview over different box types
    :align: center
    :widths: auto

    +-------------+-----------+-----------+-----------------------------------+------------------------------+
    | box type    | image     | box       | box vectors                       | box vector angles            | 
    |             |           |           +---------+------------+------------+---------+----------+---------+
    |             | distance  | volume    | **a**   | **b**      | **c**      | |angbc| | |angac|  | |angab| |
    +=============+===========+===========+=========+============+============+=========+==========+=========+
    |             |           |           | |mathd| |   0        |   0        |         |          |         |
    |             |           |           +---------+------------+------------+         |          |         |
    | cubic       | |mathd|   | |mathd3|  |   0     | |mathd|    |   0        | |90deg| | |90deg|  | |90deg| |
    |             |           |           +---------+------------+------------+         |          |         |
    |             |           |           |   0     |   0        | |mathd|    |         |          |         |
    +-------------+-----------+-----------+---------+------------+------------+---------+----------+---------+
    | rhombic     |           | |mathd23| | |mathd| | 0          | |math12d|  |         |          |         |
    |             |           |           +---------+------------+------------+         |          |         |
    | dodcahdron  | |mathd|   | |mathd70| | 0       | |mathd|    | |math12d|  | |60deg| | |60deg|  | |60deg| |
    |             |           |           +---------+------------+------------+         |          |         |
    | (xy-square) |           |           | 0       | 0          | |math12s2| |         |          |         |
    +-------------+-----------+-----------+---------+------------+------------+---------+----------+---------+
    | rhombic     |           | |mathd23| | |mathd| | |math12d|  | |math12d|  |         |          |         |
    |             |           |           +---------+------------+------------+         |          |         |
    | dodcahdron  | |mathd|   | |mathd70| | 0       | |math12s3| | |math16s3| | |60deg| | |60deg|  | |60deg| |
    |             |           |           +---------+------------+------------+         |          |         |
    | (xy-        |           |           | 0       | 0          | |math13s6| |         |          |         |
    | hexagon)    |           |           |         |            |            |         |          |         |
    +-------------+-----------+-----------+---------+------------+------------+---------+----------+---------+
    | truncated   |           | |mathd43| | |mathd| | |math13d|  | |math13dn| |         |          |         |
    |             |           |           +---------+------------+------------+         |          |         |
    | octahedron  | |mathd|   | |mathd77| | 0       | |math23s2| | |math13s2| | |71deg| | |109deg| | |71deg| |
    |             |           |           +---------+------------+------------+         |          |         |
    |             |           |           | 0       | 0          | |math13s6| |         |          |         |
    +-------------+-----------+-----------+---------+------------+------------+---------+----------+---------+

The three most useful box types for simulations of solvated systems are
described in :numref:`Table %s <table-boxtypes>`. The rhombic
dodecahedron (:numref:`Fig. (%s) <fig-boxshapes>`) is the smallest and
most regular space-filling unit cell. Each of the 12 image cells is at
the same distance. The volume is 71% of the volume of a cube having the
same image distance. This saves about 29% of CPU-time when simulating a
spherical or flexible molecule in solvent. There are two different
orientations of a rhombic dodecahedron that satisfy equations
:eq:`(%s) <eqnboxrot>`, :eq:`(%s) <eqnboxshift>` and
:eq:`(%s) <eqnboxshift2>`. The program :ref:`editconf <gmx editconf>`
produces the orientation which has a square intersection with the
xy-plane. This orientation was chosen because the first two box vectors
coincide with the x and y-axis, which is easier to comprehend. The other
orientation can be useful for simulations of membrane proteins. In this
case the cross-section with the xy-plane is a hexagon, which has an area
which is 14% smaller than the area of a square with the same image
distance. The height of the box (:math:`c_z`) should be changed to
obtain an optimal spacing. This box shape not only saves CPU time, it
also results in a more uniform arrangement of the proteins.

Cut-off restrictions
~~~~~~~~~~~~~~~~~~~~

The minimum image
convention
implies that the
cut-off radius used to truncate non-bonded interactions may not exceed
half the shortest box vector:

.. math:: R_c < {\frac{1}{2}}\min(\|{\bf a}\|,\|{\bf b}\|,\|{\bf c}\|),
          :label: eqnphysicalrc

because otherwise more than one image would be within the cut-off
distance of the force. When a macromolecule, such as a protein, is
studied in solution, this restriction alone is not sufficient: in
principle, a single solvent molecule should not be able to ‘see’ both
sides of the macromolecule. This means that the length of each box
vector must exceed the length of the macromolecule in the direction of
that edge *plus* two times the cut-off radius :math:`R_c`. It is,
however, common to compromise in this respect, and make the solvent
layer somewhat smaller in order to reduce the computational cost. For
efficiency reasons the cut-off with triclinic boxes is more restricted.
For grid search the extra restriction is weak:

.. math:: R_c < \min(a_x,b_y,c_z)
         :label: eqngridrc
   

For simple search the extra restriction is stronger:

.. math:: R_c < {\frac{1}{2}}\min(a_x,b_y,c_z)
          :label: eqnsimplerc

Each unit cell (cubic, rectangular or triclinic) is surrounded by 26
translated images. A particular image can therefore always be identified
by an index pointing to one of 27 *translation vectors* and constructed
by applying a translation with the indexed vector (see
). Restriction
:eq:`(%s) <eqngridrc>` ensures that only 26 images need to be
considered.

The group concept
-----------------

The |Gromacs| MD and analysis programs use
user-defined *groups* of atoms to perform certain actions on. The
maximum number of groups is 256, but each atom can only belong to six
different groups, one each of the following:

temperature-coupling group
    The temperature coupling
    parameters (reference temperature, time constant, number of degrees
    of freedom, see 
    can be defined for
    each T-coupling group separately. For example, in a solvated
    macromolecule the solvent (that tends to generate more heating by
    force and integration errors) can be coupled with a shorter time
    constant to a bath than is a macromolecule, or a surface can be kept
    cooler than an adsorbing molecule. Many different T-coupling groups
    may be defined. See also center of mass groups below.

freeze group

    Atoms that belong to a freeze group are kept stationary in the
    dynamics. This is useful during equilibration, *e.g.* to avoid badly
    placed solvent molecules giving unreasonable kicks to protein atoms,
    although the same effect can also be obtained by putting a
    restraining potential on the atoms that must be protected. The
    freeze option can be used, if desired, on just one or two
    coordinates of an atom, thereby freezing the atoms in a plane or on
    a line. When an atom is partially frozen, constraints will still be
    able to move it, even in a frozen direction. A fully frozen atom can
    not be moved by constraints. Many freeze groups can be defined.
    Frozen coordinates are unaffected by pressure scaling; in some cases
    this can produce unwanted results, particularly when constraints are
    also used (in this case you will get very large pressures).
    Accordingly, it is recommended to avoid combining freeze groups with
    constraints and pressure coupling. For the sake of equilibration it
    could suffice to start with freezing in a constant volume
    simulation, and afterward use position restraints in conjunction
    with constant pressure.

accelerate group

    On each atom in an “accelerate group” an acceleration
    :math:`{\mbox{\boldmath ${a}$}}^g` is imposed. This is equivalent to
    an external force. This feature makes it possible to drive the
    system into a non-equilibrium state and enables the performance of
    non-equilibrium
    MD
    and hence to obtain transport properties.

energy-monitor group

    Mutual interactions between all energy-monitor groups are compiled
    during the simulation. This is done separately for Lennard-Jones and
    Coulomb terms. In principle up to 256 groups could be defined, but
    that would lead to 256\ :math:`\times`\ 256 items! Better use this
    concept sparingly.

    All non-bonded interactions between pairs of energy-monitor groups
    can be excluded
    (see details in the User Guide).

    Pairs of particles from excluded
    pairs of energy-monitor groups are not put into the pair list. This
    can result in a significant speedup for simulations where
    interactions within or between parts of the system are not required.

center of mass group

    In |Gromacs|, the center of mass (COM) motion can be removed, for
    either the complete system or for groups of atoms. The latter is
    useful, *e.g.* for systems where there is limited friction (*e.g.*
    gas systems) to prevent center of mass motion to occur. It makes
    sense to use the same groups for temperature coupling and center of
    mass motion removal.

Compressed position output group

    In order to further reduce the size of the compressed trajectory
    file (:ref:`xtc` or
    :ref:`tng`), it is possible to
    store only a subset of all particles. All x-compression groups that
    are specified are saved, the rest are not. If no such groups are
    specified, than all atoms are saved to the compressed trajectory
    file.

The use of groups in |Gromacs| tools is described in
sec. 

Molecular Dynamics
------------------

.. _gmx-md-scheme:

**THE GLOBAL MD ALGORITHM**

--------------

| 
| **1. Input initial conditions**
| Potential interaction :math:`V` as a function of atom positions
| Positions :math:`{\mbox{\boldmath ${r}$}}` of all atoms in the system
| Velocities :math:`{\mbox{\boldmath ${v}$}}` of all atoms in the system
| :math:`\Downarrow`

--------------

| 
| **repeat 2,3,4** for the required number of steps:

--------------

| 
| **2. Compute forces**
| The force on any atom
| :math:`{\mbox{\boldmath ${F}$}}_i = - \displaystyle\frac{\partial V}{\partial {\mbox{\boldmath ${r}$}}_i}`
| is computed by calculating the force between non-bonded atom pairs:
| :math:`{\mbox{\boldmath ${F}$}}_i = \sum_j {\mbox{\boldmath ${F}$}}_{ij}`
| plus the forces due to bonded interactions (which may depend on 1, 2,
  3, or 4 atoms), plus restraining and/or external forces.
| The potential and kinetic energies and the pressure tensor may be
  computed.
| :math:`\Downarrow`
| **3. Update configuration**
| The movement of the atoms is simulated by numerically solving Newton’s
  equations of motion
| :math:`\displaystyle \frac {{\mbox{d}}^2{\mbox{\boldmath ${r}$}}_i}{{\mbox{d}}t^2} = \frac{{\mbox{\boldmath ${F}$}}_i}{m_i}`
| or
| :math:`\displaystyle   \frac{{\mbox{d}}{\mbox{\boldmath ${r}$}}_i}{{\mbox{d}}t} = {\mbox{\boldmath ${v}$}}_i ; \;\;   \frac{{\mbox{d}}{\mbox{\boldmath ${v}$}}_i}{{\mbox{d}}t} = \frac{{\mbox{\boldmath ${F}$}}_i}{m_i}` 
| :math:`\Downarrow`
| **4.** if required: **Output step**
| write positions, velocities, energies, temperature, pressure, etc.

A global flow scheme for MD is given above.
Each MD or EM run requires as input
a set of initial coordinates and – optionally – initial velocities of
all particles involved. This chapter does not describe how these are
obtained; for the setup of an actual MD run check the online manual at
`www.gromacs.org <http://www.gromacs.org>`__.

Initial conditions
~~~~~~~~~~~~~~~~~~

Topology and force field
^^^^^^^^^^^^^^^^^^^^^^^^

The system topology, including a description of the force field, must be
read in. Force fields and topologies are described in
chapter 
and 
respectively.

All this information is static; it is never modified during the run.

Coordinates and velocities
^^^^^^^^^^^^^^^^^^^^^^^^^^

.. _fig-maxwell:

.. figure:: plots/maxwell.eps

   A Maxwell-Boltzmann velocity distribution, generated from
   random numbers.

Then, before a run starts, the box size and the coordinates and
velocities of all particles are required. The box size and shape is
determined by three vectors (nine numbers)
:math:`{\mbox{\boldmath ${b}$}}_1, {\mbox{\boldmath ${b}$}}_2, {\mbox{\boldmath ${b}$}}_3`,
which represent the three basis vectors of the periodic box.

If the run starts at :math:`t=t_0`, the coordinates at :math:`t=t_0`
must be known. The *leap-frog algorithm*, the default algorithm used to
update the time step with :math:`{{\Delta t}}` (see
), also requires that the velocities at
:math:`t=t_0 - {{\frac{1}{2}}{{\Delta t}}}` are known. If velocities are
not available, the program can generate initial atomic velocities
:math:`v_i, i=1\ldots 3N` with a Maxwell-Boltzmann distribution
(:numref:`Fig. (%s) <fig-maxwell>`) at a given absolute temperature
:math:`T`:

.. math:: p(v_i) = \sqrt{\frac{m_i}{2 \pi kT}}\exp\left(-\frac{m_i v_i^2}{2kT}\right)

where :math:`k` is Boltzmann’s constant (see
chapter 
). To accomplish this, normally
distributed random numbers are generated by adding twelve random numbers
:math:`R_k` in the range :math:`0 \le R_k < 1` and subtracting 6.0 from
their sum. The result is then multiplied by the standard deviation of
the velocity distribution :math:`\sqrt{kT/m_i}`. Since the resulting
total energy will not correspond exactly to the required temperature
:math:`T`, a correction is made: first the center-of-mass motion is
removed and then all velocities are scaled so that the total energy
corresponds exactly to :math:`T` (see :eq:`eqn. (%s) <eqn-ET>`).

Center-of-mass motion
^^^^^^^^^^^^^^^^^^^^^

The center-of-mass
velocity
is normally set to zero at every step; there is (usually) no net
external force acting on the system and the center-of-mass velocity
should remain constant. In practice, however, the update algorithm
introduces a very slow change in the center-of-mass velocity, and
therefore in the total kinetic energy of the system – especially when
temperature coupling is used. If such changes are not quenched, an
appreciable center-of-mass motion can develop in long runs, and the
temperature will be significantly misinterpreted. Something similar may
happen due to overall rotational motion, but only when an isolated
cluster is simulated. In periodic systems with filled boxes, the overall
rotational motion is coupled to other degrees of freedom and does not
cause such problems.

Neighbor searching
~~~~~~~~~~~~~~~~~~

As mentioned in chapter 
internal forces are
either generated from fixed (static) lists, or from dynamic lists. The
latter consist of non-bonded interactions between any pair of particles.
When calculating the non-bonded forces, it is convenient to have all
particles in a rectangular box. As shown in
:numref:`Fig. (%s) <fig-pbc>`, it is possible to transform a triclinic
box into a rectangular box. The output coordinates are always in a
rectangular box, even when a dodecahedron or triclinic box was used for
the simulation. :eq:`Equation (%s) <eqnboxrot>` ensures that we
can reset particles in a rectangular box by first shifting them with box
vector :math:`{\bf c}`, then with :math:`{\bf b}` and finally with
:math:`{\bf a}`. Equations :eq:`(%s) <eqnboxshift2>`,
:eq:`(%s) <eqnphysicalrc>` and :eq:`(%s) <eqngridrc>`
ensure that we can find the 14 nearest triclinic images within a linear
combination that does not involve multiples of box vectors.

Pair lists generation
^^^^^^^^^^^^^^^^^^^^^

The non-bonded pair forces need to be calculated only for those pairs
:math:`i,j` for which the distance :math:`r_{ij}` between :math:`i` and
the nearest image
of :math:`j` is less than a given cut-off radius :math:`R_c`. Some of
the particle pairs that fulfill this criterion are excluded, when their
interaction is already fully accounted for by bonded interactions.
|Gromacs| employs a *pair list* that contains those particle pairs for
which non-bonded forces must be calculated. The pair list contains
particles :math:`i`, a displacement vector for particle :math:`i`, and
all particles :math:`j` that are within ``rlist`` of this particular
image of particle :math:`i`. The list is updated every ``nstlist``
steps.

To make the neighbor list
all
particles that are close (*i.e.* within the neighbor list cut-off) to a
given particle must be found. This searching, usually called neighbor
search (NS) or pair search, involves periodic boundary conditions and
determining the *image* (see sec. 
The
search algorithm is :math:`O(N)`, although a simpler :math:`O(N^2)`
algorithm is still available under some conditions.

Cut-off schemes
^^^^^^^^^^^^^^^

From version 4.6, |Gromacs| supports two different cut-off scheme setups:
the original one based on particle groups and one using a Verlet buffer.
There are some important differences that affect results, performance
and feature support. The group scheme can be made to work (almost) like
the Verlet scheme, but this will lead to a decrease in performance. The
group scheme is especially fast for water molecules, which are abundant
in many simulations, but on the most recent x86 processors, this
advantage is negated by the better instruction-level parallelism
available in the Verlet-scheme implementation. The group scheme is
deprecated in version 5.0, and will be removed in a future version. For
practical details of choosing and setting up cut-off schemes, please see
the User Guide.

In the group scheme, a neighbor list is generated consisting of pairs of
groups of at least one particle. These groups were originally charge
groups
(see sec. ), but with a proper
treatment of long-range electrostatics, performance in unbuffered
simulations is their only advantage. A pair of groups is put into the
neighbor list when their center of geometry is within the cut-off
distance. Interactions between all particle pairs (one from each charge
group) are calculated for a certain number of MD steps, until the
neighbor list is updated. This setup is efficient, as the neighbor
search only checks distance between charge-group pair, not particle
pairs (saves a factor of :math:`3 \times 3 = 9` with a three-particle
water model) and the non-bonded force kernels can be optimized for, say,
a water molecule “group”. Without explicit buffering, this setup leads
to energy drift as some particle pairs which are within the cut-off
don’t interact and some outside the cut-off do interact. This can be
caused by

-  particles moving across the cut-off between neighbor search steps,
   and/or

-  for charge groups consisting of more than one particle, particle
   pairs moving in/out of the cut-off when their charge group center of
   geometry distance is outside/inside of the cut-off.

Explicitly adding a buffer to the neighbor list will remove such
artifacts, but this comes at a high computational cost. How severe the
artifacts are depends on the system, the properties in which you are
interested, and the cut-off setup.

The Verlet cut-off scheme uses a buffered pair list by default. It also
uses clusters of particles, but these are not static as in the group
scheme. Rather, the clusters are defined spatially and consist of 4 or 8
particles, which is convenient for stream computing, using e.g. SSE, AVX
or CUDA on GPUs. At neighbor search steps, a pair list is created with a
Verlet buffer, ie. the pair-list cut-off is larger than the interaction
cut-off. In the non-bonded kernels, interactions are only computed when
a particle pair is within the cut-off distance at that particular time
step. This ensures that as particles move between pair search steps,
forces between nearly all particles within the cut-off distance are
calculated. We say *nearly* all particles, because |Gromacs| uses a fixed
pair list update frequency for efficiency. A particle-pair, whose
distance was outside the cut-off, could possibly move enough during this
fixed number of steps that its distance is now within the cut-off. This
small chance results in a small energy drift, and the size of the chance
depends on the temperature. When temperature coupling is used, the
buffer size can be determined automatically, given a certain tolerance
on the energy drift.

The Verlet cut-off scheme is implemented in a very efficient fashion
based on clusters of particles. The simplest example is a cluster size
of 4 particles. The pair list is then constructed based on cluster
pairs. The cluster-pair search is much faster searching based on
particle pairs, because :math:`4 \times 4 = 16` particle pairs are put
in the list at once. The non-bonded force calculation kernel can then
calculate many particle-pair interactions at once, which maps nicely to
SIMD or SIMT units on modern hardware, which can perform multiple
floating operations at once. These non-bonded kernels are much faster
than the kernels used in the group scheme for most types of systems,
particularly on newer hardware.

Additionally, when the list buffer is determined automatically as
described below, we also apply dynamic pair list pruning. The pair list
can be constructed infrequently, but that can lead to a lot of pairs in
the list that are outside the cut-off range for all or most of the life
time of this pair list. Such pairs can be pruned out by applying a
cluster-pair kernel that only determines which clusters are in range.
Because of the way the non-bonded data is regularized in |Gromacs|, this
kernel is an order of magnitude faster than the search and the
interaction kernel. On the GPU this pruning is overlapped with the
integration on the CPU, so it is free in most cases. Therefore we can
prune every 4-10 integration steps with little overhead and
significantly reduce the number of cluster pairs in the interaction
kernel. This procedure is applied automatically, unless the user set the
pair-list buffer size manually.

Energy drift and pair-list buffering
^^^^^^^^^^^^^^^^^^^^^^^^^^^^^^^^^^^^

For a canonical (NVT) ensemble, the average energy error caused by
diffusion of :math:`j` particles from outside the pair-list cut-off
:math:`r_\ell` to inside the interaction cut-off :math:`r_c` over the
lifetime of the list can be determined from the atomic displacements and
the shape of the potential at the cut-off. The displacement distribution
along one dimension for a freely moving particle with mass :math:`m`
over time :math:`t` at temperature :math:`T` is a Gaussian :math:`G(x)`
of zero mean and variance :math:`\sigma^2 = t^2 k_B T/m`. For the
distance between two particles, the variance changes to
:math:`\sigma^2 = \sigma_{12}^2 =
t^2 k_B T(1/m_1+1/m_2)`. Note that in practice particles usually
interact with (bump into) other particles over time :math:`t` and
therefore the real displacement distribution is much narrower. Given a
non-bonded interaction cut-off distance of :math:`r_c` and a pair-list
cut-off :math:`r_\ell=r_c+r_b` for :math:`r_b` the Verlet buffer size,
we can then write the average energy error after time :math:`t` for all
missing pair interactions between a single :math:`i` particle of type 1
surrounded by all :math:`j` particles that are of type 2 with number
density :math:`\rho_2`, when the inter-particle distance changes from
:math:`r_0` to :math:`r_t`, as:

.. math::

   \langle \Delta V \rangle =
   \int_{0}^{r_c} \int_{r_\ell}^\infty 4 \pi r_0^2 \rho_2 V(r_t) G\!\left(\frac{r_t-r_0}{\sigma}\right) d r_0\, d r_t

To evaluate this analytically, we need to make some approximations.
First we replace :math:`V(r_t)` by a Taylor expansion around
:math:`r_c`, then we can move the lower bound of the integral over
:math:`r_0` to :math:`-\infty` which will simplify the result:

.. math::

   \begin{aligned}
   \langle \Delta V \rangle &\approx&
   \int_{-\infty}^{r_c} \int_{r_\ell}^\infty 4 \pi r_0^2 \rho_2 \Big[ V'(r_c) (r_t - r_c) +
   \nonumber\\
   & &
   \phantom{\int_{-\infty}^{r_c} \int_{r_\ell}^\infty 4 \pi r_0^2 \rho_2 \Big[}
   V''(r_c)\frac{1}{2}(r_t - r_c)^2 +
   \nonumber\\
   & &
   \phantom{\int_{-\infty}^{r_c} \int_{r_\ell}^\infty 4 \pi r_0^2 \rho_2 \Big[}
     V'''(r_c)\frac{1}{6}(r_t - r_c)^3 +
     \nonumber\\
   & &
   \phantom{\int_{-\infty}^{r_c} \int_{r_\ell}^\infty 4 \pi r_0^2 \rho_2 \Big[}
     O \! \left((r_t - r_c)^4 \right)\Big] G\!\left(\frac{r_t-r_0}{\sigma}\right) d r_0 \, d r_t\end{aligned}

Replacing the factor :math:`r_0^2` by :math:`(r_\ell + \sigma)^2`,
which results in a slight overestimate, allows us to calculate the
integrals analytically:

.. math::

   \begin{aligned}
   \langle \Delta V \rangle \!
   &\approx&
   4 \pi (r_\ell+\sigma)^2 \rho_2
   \int_{-\infty}^{r_c} \int_{r_\ell}^\infty \Big[ V'(r_c) (r_t - r_c) +
   \nonumber\\
   & &
   \phantom{4 \pi (r_\ell+\sigma)^2 \rho_2 \int_{-\infty}^{r_c} \int_{r_\ell}^\infty \Big[}
   V''(r_c)\frac{1}{2}(r_t - r_c)^2 +
   \nonumber\\
   & &
   \phantom{4 \pi (r_\ell+\sigma)^2 \rho_2 \int_{-\infty}^{r_c} \int_{r_\ell}^\infty \Big[}
   V'''(r_c)\frac{1}{6}(r_t - r_c)^3 \Big] G\!\left(\frac{r_t-r_0}{\sigma}\right)
   d r_0 \, d r_t\\
   &=&
   4 \pi (r_\ell+\sigma)^2 \rho_2 \bigg\{
   \frac{1}{2}V'(r_c)\left[r_b \sigma G\!\left(\frac{r_b}{\sigma}\right) - (r_b^2+\sigma^2)E\!\left(\frac{r_b}{\sigma}\right) \right] +
   \nonumber\\
   & &
   \phantom{4 \pi (r_\ell+\sigma)^2 \rho_2 \bigg\{ }
   \frac{1}{6}V''(r_c)\left[ \sigma(r_b^2+2\sigma^2) G\!\left(\frac{r_b}{\sigma}\right) - r_b(r_b^2+3\sigma^2 ) E\!\left(\frac{r_b}{\sigma}\right) \right] +
   \nonumber\\
   & &
   \phantom{4 \pi (r_\ell+\sigma)^2 \rho_2 \bigg\{ }
   \frac{1}{24}V'''(r_c)\bigg[ r_b\sigma(r_b^2+5\sigma^2) G\!\left(\frac{r_b}{\sigma}\right)
   \nonumber\\
   & &
   \phantom{4 \pi (r_\ell+\sigma)^2 \rho_2 \bigg\{ \frac{1}{24}V'''(r_c)\bigg[ }
    - (r_b^4+6r_b^2\sigma^2+3\sigma^4 ) E\!\left(\frac{r_b}{\sigma}\right) \bigg]
   \bigg\}\end{aligned}

where :math:`G(x)` is a Gaussian distribution with 0 mean and unit
variance and :math:`E(x)=\frac{1}{2}\mathrm{erfc}(x/\sqrt{2})`. We
always want to achieve small energy error, so :math:`\sigma` will be
small compared to both :math:`r_c` and :math:`r_\ell`, thus the
approximations in the equations above are good, since the Gaussian
distribution decays rapidly. The energy error needs to be averaged over
all particle pair types and weighted with the particle counts. In
GROMACS we don’t allow cancellation of error between pair types, so we
average the absolute values. To obtain the average energy error per unit
time, it needs to be divided by the neighbor-list life time
:math:`t = ({\tt nstlist} - 1)\times{\tt dt}`. The function can not be
inverted analytically, so we use bisection to obtain the buffer size
:math:`r_b` for a target drift. Again we note that in practice the error
we usually be much smaller than this estimate, as in the condensed phase
particle displacements will be much smaller than for freely moving
particles, which is the assumption used here.

When (bond) constraints are present, some particles will have fewer
degrees of freedom. This will reduce the energy errors. For simplicity,
we only consider one constraint per particle, the heaviest particle in
case a particle is involved in multiple constraints. This simplification
overestimates the displacement. The motion of a constrained particle is
a superposition of the 3D motion of the center of mass of both particles
and a 2D rotation around the center of mass. The displacement in an
arbitrary direction of a particle with 2 degrees of freedom is not
Gaussian, but rather follows the complementary error function:

.. math:: \frac{\sqrt{\pi}}{2\sqrt{2}\sigma}\,\mathrm{erfc}\left(\frac{|r|}{\sqrt{2}\,\sigma}\right)
          :label: eqn2Ddisp

where :math:`\sigma^2` is again :math:`t^2 k_B T/m`. This distribution
can no longer be integrated analytically to obtain the energy error. But
we can generate a tight upper bound using a scaled and shifted Gaussian
distribution (not shown). This Gaussian distribution can then be used to
calculate the energy error as described above. The rotation displacement
around the center of mass can not be more than the length of the arm. To
take this into account, we scale :math:`\sigma` in
:eq:`eqn. (%s) <eqn2Ddisp>` (details not presented here) to
obtain an overestimate of the real displacement. This latter effect
significantly reduces the buffer size for longer neighborlist lifetimes
in e.g. water, as constrained hydrogens are by far the fastest
particles, but they can not move further than 0.1 nm from the heavy atom
they are connected to.

There is one important implementation detail that reduces the energy
errors caused by the finite Verlet buffer list size. The derivation
above assumes a particle pair-list. However, the GROMACS implementation
uses a cluster pair-list for efficiency. The pair list consists of pairs
of clusters of 4 particles in most cases, also called a
:math:`4 \times 4` list, but the list can also be :math:`4 \times 8`
(GPU CUDA kernels and AVX 256-bit single precision kernels) or
:math:`4 \times 2` (SSE double-precision kernels). This means that the
pair-list is effectively much larger than the corresponding
:math:`1 \times 1` list. Thus slightly beyond the pair-list cut-off
there will still be a large fraction of particle pairs present in the
list. This fraction can be determined in a simulation and accurately
estimated under some reasonable assumptions. The fraction decreases with
increasing pair-list range, meaning that a smaller buffer can be used.
For typical all-atom simulations with a cut-off of 0.9 nm this fraction
is around 0.9, which gives a reduction in the energy errors of a factor
of 10. This reduction is taken into account during the automatic Verlet
buffer calculation and results in a smaller buffer size.

.. _fig-verletdrift:

.. figure:: plots/verlet-drift.eps

   Energy drift per atom for an SPC/E water system at 300K with a
   time step of 2 fs and a pair-list update period of 10 steps
   (pair-list life time: 18 fs). PME was used with
   ``ewald-rtol`` set to 10\ :math:`^{-5}`; this parameter
   affects the shape of the potential at the cut-off. Error estimates
   due to finite Verlet buffer size are shown for a :math:`1 \times 1`
   atom pair list and :math:`4 \times 4` atom pair list without and with
   (dashed line) cancellation of positive and negative errors. Real
   energy drift is shown for simulations using double- and
   mixed-precision settings. Rounding errors in the SETTLE constraint
   algorithm from the use of single precision causes the drift to become
   negative at large buffer size. Note that at zero buffer size, the
   real drift is small because positive (H-H) and negative (O-H) energy
   errors cancel.

In :numref:`Fig. (%s) <fig-verletdrift>` one can see that for small
buffer sizes the drift of the total energy is much smaller than the pair
energy error tolerance, due to cancellation of errors. For larger buffer
size, the error estimate is a factor of 6 higher than drift of the total
energy, or alternatively the buffer estimate is 0.024 nm too large. This
is because the protons don’t move freely over 18 fs, but rather vibrate.

Cut-off artifacts and switched interactions
^^^^^^^^^^^^^^^^^^^^^^^^^^^^^^^^^^^^^^^^^^^

With the Verlet scheme, the pair potentials are shifted to be zero at
the cut-off, which makes the potential the integral of the force. This
is only possible in the group scheme if the shape of the potential is
such that its value is zero at the cut-off distance. However, there can
still be energy drift when the forces are non-zero at the cut-off. This
effect is extremely small and often not noticeable, as other integration
errors (e.g. from constraints) may dominate. To completely avoid cut-off
artifacts, the non-bonded forces can be switched exactly to zero at some
distance smaller than the neighbor list cut-off (there are several ways
to do this in GROMACS, see sec. 
). One
then has a buffer with the size equal to the neighbor list cut-off less
the longest interaction cut-off.

Simple search
^^^^^^^^^^^^^

Due to :eq:`eqns. (%s) <eqnboxrot>` and
:eq:`(%s) <eqnsimplerc>`, the vector
:math:`{{\mbox{\boldmath ${r}$}}_{ij}}` connecting images within the
cut-off :math:`R_c` can be found by constructing:

.. math::

   \begin{aligned}
   {\mbox{\boldmath ${r}$}}'''   & = & {\mbox{\boldmath ${r}$}}_j-{\mbox{\boldmath ${r}$}}_i \\
   {\mbox{\boldmath ${r}$}}''    & = & {\mbox{\boldmath ${r}$}}''' - {\mbox{\boldmath ${c}$}}*\mathrm{round}(r'''_z/c_z) \\
   {\mbox{\boldmath ${r}$}}'     & = & {\mbox{\boldmath ${r}$}}'' - {\mbox{\boldmath ${b}$}}*\mathrm{round}(r''_y/b_y) \\
   {\mbox{\boldmath ${r}$}}_{ij} & = & {\mbox{\boldmath ${r}$}}' - {\mbox{\boldmath ${a}$}}*\mathrm{round}(r'_x/a_x)
   \end{aligned}

When distances between two particles in a triclinic box are needed that
do not obey :eq:`eqn. (%s) <eqnboxrot>`, many shifts of
combinations of box vectors need to be considered to find the nearest
image.

.. _fig-grid:

.. figure:: plots/nstric.eps

   Grid search in two dimensions. The arrows are the box vectors.

Grid search
^^^^^^^^^^^

The grid search is schematically depicted in
:numref:`Fig. (%s) <fig-grid>`. All particles are put on the NS grid,
with the smallest spacing :math:`\ge` :math:`R_c/2` in each of the
directions. In the direction of each box vector, a particle :math:`i`
has three images. For each direction the image may be -1,0 or 1,
corresponding to a translation over -1, 0 or +1 box vector. We do not
search the surrounding NS grid cells for neighbors of :math:`i` and then
calculate the image, but rather construct the images first and then
search neighbors corresponding to that image of :math:`i`. As
:numref:`Fig. (%s) <fig-grid>` shows, some grid cells may be searched
more than once for different images of :math:`i`. This is not a problem,
since, due to the minimum image convention, at most one image will “see”
the :math:`j`-particle. For every particle, fewer than 125 (5:math:`^3`)
neighboring cells are searched. Therefore, the algorithm scales linearly
with the number of particles. Although the prefactor is large, the
scaling behavior makes the algorithm far superior over the standard
:math:`O(N^2)` algorithm when there are more than a few hundred
particles. The grid search is equally fast for rectangular and triclinic
boxes. Thus for most protein and peptide simulations the rhombic
dodecahedron will be the preferred box shape.

Charge groups
^^^^^^^^^^^^^
Charge
groups were originally introduced to reduce cut-off artifacts of Coulomb
interactions. When a plain cut-off is used, significant jumps in the
potential and forces arise when atoms with (partial) charges move in and
out of the cut-off radius. When all chemical moieties have a net charge
of zero, these jumps can be reduced by moving groups of atoms with net
charge zero, called charge groups, in and out of the neighbor list. This
reduces the cut-off effects from the charge-charge level to the
dipole-dipole level, which decay much faster. With the advent of full
range electrostatics methods, such as particle-mesh Ewald
(sec. 
), the use of charge groups is no longer
required for accuracy. It might even have a slight negative effect on
the accuracy or efficiency, depending on how the neighbor list is made
and the interactions are calculated.

But there is still an important reason for using *charge groups*:
efficiency with the group cut-off scheme. Where applicable, neighbor
searching is carried out on the basis of charge groups which are defined
in the molecular topology. If the nearest image distance between the
*geometrical centers* of the atoms of two charge groups is less than the
cut-off radius, all atom pairs between the charge groups are included in
the pair list. The neighbor searching for a water system, for instance,
is :math:`3^2=9` times faster when each molecule is treated as a charge
group. Also the highly optimized water force loops (see
sec. 
) only work when all atoms in a
water molecule form a single charge group. Currently the name
*neighbor-search group* would be more appropriate, but the name charge
group is retained for historical reasons. When developing a new force
field, the advice is to use charge groups of 3 to 4 atoms for optimal
performance. For all-atom force fields this is relatively easy, as one
can simply put hydrogen atoms, and in some case oxygen atoms, in the
same charge group as the heavy atom they are connected to; for example:
CH\ :math:`_3`, CH\ :math:`_2`, CH, NH\ :math:`_2`, NH, OH,
CO\ :math:`_2`, CO.

With the Verlet cut-off scheme, charge groups are ignored.

Compute forces
~~~~~~~~~~~~~~

Potential energy
^^^^^^^^^^^^^^^^

When forces are computed, the potential
energy
of each interaction term is computed as well. The total potential energy
is summed for various contributions, such as Lennard-Jones, Coulomb, and
bonded terms. It is also possible to compute these contributions for
*energy-monitor groups* of atoms that are separately defined (see
sec. 
).

Kinetic energy and temperature
^^^^^^^^^^^^^^^^^^^^^^^^^^^^^^

The temperature
is given by the total
kinetic
energy
of the :math:`N`-particle system:

.. math:: E_{kin} = {\frac{1}{2}}\sum_{i=1}^N m_i v_i^2

From this the absolute temperature :math:`T` can be computed using:

.. math::  {\frac{1}{2}}N_{\mathrm{df}} kT = E_{\mathrm{kin}}
           :label: eqnET

where :math:`k` is Boltzmann’s constant and :math:`N_{df}` is the
number of degrees of freedom which can be computed from:

.. math:: N_{\mathrm{df}}  ~=~     3 N - N_c - N_{\mathrm{com}}

Here :math:`N_c` is the number of
*constraints*
imposed on the system.
When performing molecular dynamics :math:`N_{\mathrm{com}}=3` additional
degrees of freedom must be removed, because the three center-of-mass
velocities are constants of the motion, which are usually set to zero.
When simulating in vacuo, the rotation around the center of mass can
also be removed, in this case :math:`N_{\mathrm{com}}=6`. When more than
one temperature-coupling
group
is used, the number
of degrees of freedom for group :math:`i` is:

.. math:: N^i_{\mathrm{df}}  ~=~  (3 N^i - N^i_c) \frac{3 N - N_c - N_{\mathrm{com}}}{3 N - N_c}

The kinetic energy can also be written as a tensor, which is necessary
for pressure calculation in a triclinic system, or systems where shear
forces are imposed:

.. math:: {\bf E}_{\mathrm{kin}} = {\frac{1}{2}}\sum_i^N m_i {{\mbox{\boldmath ${v}$}}_i}\otimes {{\mbox{\boldmath ${v}$}}_i}

Pressure and virial
^^^^^^^^^^^^^^^^^^^

The pressure
tensor **P** is calculated
from the difference between kinetic energy :math:`E_{\mathrm{kin}}` and
the virial
:math:`{\bf \Xi}`:

.. math:: {\bf P} = \frac{2}{V} ({\bf E}_{\mathrm{kin}}-{\bf \Xi})
          :label: eqnP

where :math:`V` is the volume of the computational box. The scalar
pressure :math:`P`, which can be used for pressure coupling in the case
of isotropic systems, is computed as:

.. math:: P       = {\rm trace}({\bf P})/3

The virial :math:`{\bf \Xi}` tensor is defined as:

.. math:: {\bf \Xi} = -{\frac{1}{2}}\sum_{i<j} {\mbox{\boldmath ${r}$}}_ij \otimes {\mbox{\boldmath ${F}$}}_ij 
          :label: eqnXi

The GROMACS implementation of the virial computation is described in
sec. 
The leap-frog integrator
~~~~~~~~~~~~~~~~~~~~~~~~

.. _fig-leapfrog:

.. figure:: plots/leapfrog.eps

   The Leap-Frog integration method. The algorithm is called
   Leap-Frog because :math:`{\mbox{\boldmath ${r}$}}` and
   :math:`{\mbox{\boldmath ${v}$}}` are leaping like frogs over each
   other’s backs.

The default MD integrator in |Gromacs| is the so-called *leap-frog*
algorithm Hockney, Goel, and Eastwood (1974) for the integration of the
equations of motion. When extremely accurate integration with
temperature and/or pressure coupling is required, the velocity Verlet
integrators are also present and may be preferable (see
). The leap-frog algorithm uses
positions :math:`{\mbox{\boldmath ${r}$}}` at time :math:`t` and
velocities :math:`{\mbox{\boldmath ${v}$}}` at time
:math:`t-{{\frac{1}{2}}{{\Delta t}}}`; it updates positions and
velocities using the forces :math:`{\mbox{\boldmath ${F}$}}(t)`
determined by the positions at time :math:`t` using these relations:

.. math:: \begin{aligned}
          {\mbox{\boldmath ${v}$}}(t+{{\frac{1}{2}}{{\Delta t}}})  &~=~&   {\mbox{\boldmath ${v}$}}(t-{{\frac{1}{2}}{{\Delta t}}})+\frac{{{\Delta t}}}{m}{\mbox{\boldmath ${F}$}}(t)   \\
          {\mbox{\boldmath ${r}$}}(t+{{\Delta t}})   &~=~&   {\mbox{\boldmath ${r}$}}(t)+{{\Delta t}}{\mbox{\boldmath ${v}$}}(t+{{\frac{1}{2}}{{\Delta t}}})\end{aligned}
          :label: eqnleapfrogv

The algorithm is visualized in :numref:`Fig. (%s) <fig-leapfrog>`. It
produces trajectories that are identical to the Verlet Verlet. (1967)
algorithm, whose position-update relation is

.. math:: {\mbox{\boldmath ${r}$}}(t+{{\Delta t}})~=~2{\mbox{\boldmath ${r}$}}(t) - {\mbox{\boldmath ${r}$}}(t-{{\Delta t}}) + \frac{1}{m}{\mbox{\boldmath ${F}$}}(t){{\Delta t}}^2+O({{\Delta t}}^4)

The algorithm is of third order in :math:`{\mbox{\boldmath ${r}$}}` and
is time-reversible. See ref. Berendsen and Gunsteren (1986) for the
merits of this algorithm and comparison with other time integration
algorithms.

The equations of
motion
are modified for temperature coupling and pressure coupling, and
extended to include the conservation of constraints, all of which are
described below.

The velocity Verlet integrator
~~~~~~~~~~~~~~~~~~~~~~~~~~~~~~

The velocity Verlet algorithm Swope et al. (1982) is also implemented in
|Gromacs|, though it is not yet fully integrated with all sets of options.
In velocity Verlet, positions :math:`{\mbox{\boldmath ${r}$}}` and
velocities :math:`{\mbox{\boldmath ${v}$}}` at time :math:`t` are used
to integrate the equations of motion; velocities at the previous half
step are not required.

.. math:: \begin{aligned}
          {\mbox{\boldmath ${v}$}}(t+{{\frac{1}{2}}{{\Delta t}}})  &~=~&   {\mbox{\boldmath ${v}$}}(t)+\frac{{{\Delta t}}}{2m}{\mbox{\boldmath ${F}$}}(t)   \\
          {\mbox{\boldmath ${r}$}}(t+{{\Delta t}})   &~=~&   {\mbox{\boldmath ${r}$}}(t)+{{\Delta t}}\,{\mbox{\boldmath ${v}$}}(t+{{\frac{1}{2}}{{\Delta t}}}) \\
          {\mbox{\boldmath ${v}$}}(t+{{\Delta t}})   &~=~&   {\mbox{\boldmath ${v}$}}(t+{{\frac{1}{2}}{{\Delta t}}})+\frac{{{\Delta t}}}{2m}{\mbox{\boldmath ${F}$}}(t+{{\Delta t}})\end{aligned}
          :label: eqnvelocityverlet1

or, equivalently,

.. math:: \begin{aligned}
          {\mbox{\boldmath ${r}$}}(t+{{\Delta t}})   &~=~&   {\mbox{\boldmath ${r}$}}(t)+ {{\Delta t}}\,{\mbox{\boldmath ${v}$}} + \frac{{{\Delta t}}^2}{2m}{\mbox{\boldmath ${F}$}}(t) \\
          {\mbox{\boldmath ${v}$}}(t+{{\Delta t}})   &~=~&   {\mbox{\boldmath ${v}$}}(t)+ \frac{{{\Delta t}}}{2m}\left[{\mbox{\boldmath ${F}$}}(t) + {\mbox{\boldmath ${F}$}}(t+{{\Delta t}})\right]\end{aligned}
          :label: eqnvelocityverlet2

With no temperature or pressure coupling, and with *corresponding*
starting points, leap-frog and velocity Verlet will generate identical
trajectories, as can easily be verified by hand from the equations
above. Given a single starting file with the *same* starting point
:math:`{\mbox{\boldmath ${x}$}}(0)` and
:math:`{\mbox{\boldmath ${v}$}}(0)`, leap-frog and velocity Verlet will
*not* give identical trajectories, as leap-frog will interpret the
velocities as corresponding to :math:`t=-{{\frac{1}{2}}{{\Delta t}}}`,
while velocity Verlet will interpret them as corresponding to the
timepoint :math:`t=0`.

Understanding reversible integrators: The Trotter decomposition
~~~~~~~~~~~~~~~~~~~~~~~~~~~~~~~~~~~~~~~~~~~~~~~~~~~~~~~~~~~~~~~

To further understand the relationship between velocity Verlet and
leap-frog integration, we introduce the reversible Trotter formulation
of dynamics, which is also useful to understanding implementations of
thermostats and barostats in |Gromacs|.

A system of coupled, first-order differential equations can be evolved
from time :math:`t = 0` to time :math:`t` by applying the evolution
operator

.. math::

   \begin{aligned}
   \Gamma(t) &=& \exp(iLt) \Gamma(0) \nonumber \\
   iL &=& \dot{\Gamma}\cdot \nabla_{\Gamma},\end{aligned}

where :math:`L` is the Liouville operator, and :math:`\Gamma` is the
multidimensional vector of independent variables (positions and
velocities). A short-time approximation to the true operator, accurate
at time :math:`{{\Delta t}}= t/P`, is applied :math:`P` times in
succession to evolve the system as

.. math:: \Gamma(t) = \prod_{i=1}^P \exp(iL{{\Delta t}}) \Gamma(0)

For NVE dynamics, the Liouville operator is

.. math::

   \begin{aligned}
   iL = \sum_{i=1}^{N} {{{\mbox{\boldmath{$v$}}}}}_i \cdot \nabla_{{{{\mbox{\boldmath{$r$}}}}}_i} + \sum_{i=1}^N \frac{1}{m_i}{{{\mbox{\boldmath{$F$}}}}}(r_i) \cdot \nabla_{{{{\mbox{\boldmath{$v$}}}}}_i}.\end{aligned}

This can be split into two additive operators

.. math::

   \begin{aligned}
   iL_1 &=& \sum_{i=1}^N \frac{1}{m_i}{{{\mbox{\boldmath{$F$}}}}}(r_i) \cdot \nabla_{{{{\mbox{\boldmath{$v$}}}}}_i} \nonumber \\
   iL_2 &=& \sum_{i=1}^{N} {{{\mbox{\boldmath{$v$}}}}}_i \cdot \nabla_{{{{\mbox{\boldmath{$r$}}}}}_i} \end{aligned}

Then a short-time, symmetric, and thus reversible approximation of the
true dynamics will be

.. math:: \begin{aligned}
          \exp(iL{{\Delta t}}) = \exp(iL_2{{\frac{1}{2}}{{\Delta t}}}) \exp(iL_1{{\Delta t}}) \exp(iL_2{{\frac{1}{2}}{{\Delta t}}}) + \mathcal{O}({{\Delta t}}^3).
          \end{aligned}
          :label: eqNVETrotter

This corresponds to velocity Verlet integration. The first exponential
term over :math:`{{\frac{1}{2}}{{\Delta t}}}` corresponds to a velocity
half-step, the second exponential term over :math:`{{\Delta t}}`
corresponds to a full velocity step, and the last exponential term over
:math:`{{\frac{1}{2}}{{\Delta t}}}` is the final velocity half step. For
future times :math:`t = n{{\Delta t}}`, this becomes

.. math::

   \begin{aligned}
   \exp(iLn{{\Delta t}}) &\approx&  \left(\exp(iL_2{{\frac{1}{2}}{{\Delta t}}}) \exp(iL_1{{\Delta t}}) \exp(iL_2{{\frac{1}{2}}{{\Delta t}}})\right)^n \nonumber \\
                &\approx&  \exp(iL_2{{\frac{1}{2}}{{\Delta t}}}) \bigg(\exp(iL_1{{\Delta t}}) \exp(iL_2{{\Delta t}})\bigg)^{n-1} \nonumber \\
                &       &  \;\;\;\; \exp(iL_1{{\Delta t}}) \exp(iL_2{{\frac{1}{2}}{{\Delta t}}}) \end{aligned}

This formalism allows us to easily see the difference between the
different flavors of Verlet integrators. The leap-frog integrator can be
seen as starting with :eq:`Eq. (%s) <eqNVETrotter>` with the
:math:`\exp\left(iL_1 {\Delta t}\right)` term, instead of the half-step
velocity term, yielding

.. math::

   \begin{aligned}
   \exp(iLn{\Delta t}) &=& \exp\left(iL_1 {\Delta t}\right) \exp\left(iL_2 {{\Delta t}}\right) + \mathcal{O}({{\Delta t}}^3).\end{aligned}

Here, the full step in velocity is between
:math:`t-{{\frac{1}{2}}{{\Delta t}}}` and
:math:`t+{{\frac{1}{2}}{{\Delta t}}}`, since it is a combination of the
velocity half steps in velocity Verlet. For future times
:math:`t = n{{\Delta t}}`, this becomes

.. math::

   \begin{aligned}
   \exp(iLn{\Delta t}) &\approx& \bigg(\exp\left(iL_1 {\Delta t}\right) \exp\left(iL_2 {{\Delta t}}\right)  \bigg)^{n}.\end{aligned}

Although at first this does not appear symmetric, as long as the full
velocity step is between :math:`t-{{\frac{1}{2}}{{\Delta t}}}` and
:math:`t+{{\frac{1}{2}}{{\Delta t}}}`, then this is simply a way of
starting velocity Verlet at a different place in the cycle.

Even though the trajectory and thus potential energies are identical
between leap-frog and velocity Verlet, the kinetic energy and
temperature will not necessarily be the same. Standard velocity Verlet
uses the velocities at the :math:`t` to calculate the kinetic energy and
thus the temperature only at time :math:`t`; the kinetic energy is then
a sum over all particles

.. math::

   \begin{aligned}
   KE_{\mathrm{full}}(t) &=& \sum_i \left(\frac{1}{2m_i}{\mbox{\boldmath ${v}$}}_i(t)\right)^2 \nonumber\\ 
         &=& \sum_i \frac{1}{2m_i}\left(\frac{1}{2}{\mbox{\boldmath ${v}$}}_i(t-{{\frac{1}{2}}{{\Delta t}}})+\frac{1}{2}{\mbox{\boldmath ${v}$}}_i(t+{{\frac{1}{2}}{{\Delta t}}})\right)^2,\end{aligned}

with the square on the *outside* of the average. Standard leap-frog
calculates the kinetic energy at time :math:`t` based on the average
kinetic energies at the timesteps :math:`t+{{\frac{1}{2}}{{\Delta t}}}`
and :math:`t-{{\frac{1}{2}}{{\Delta t}}}`, or the sum over all particles

.. math::

   \begin{aligned}
   KE_{\mathrm{average}}(t) &=& \sum_i \frac{1}{2m_i}\left(\frac{1}{2}{\mbox{\boldmath ${v}$}}_i(t-{{\frac{1}{2}}{{\Delta t}}})^2+\frac{1}{2}{\mbox{\boldmath ${v}$}}_i(t+{{\frac{1}{2}}{{\Delta t}}})^2\right),\end{aligned}

where the square is *inside* the average.

A non-standard variant of velocity Verlet which averages the kinetic
energies :math:`KE(t+{{\frac{1}{2}}{{\Delta t}}})` and
:math:`KE(t-{{\frac{1}{2}}{{\Delta t}}})`, exactly like leap-frog, is
also now implemented in |Gromacs| (as :ref:`mdp` file option
``md-vv-avek``). Without temperature and pressure coupling,
velocity Verlet with half-step-averaged kinetic energies and leap-frog
will be identical up to numerical precision. For temperature- and
pressure-control schemes, however, velocity Verlet with
half-step-averaged kinetic energies and leap-frog will be different, as
will be discussed in the section in thermostats and barostats.

The half-step-averaged kinetic energy and temperature are slightly more
accurate for a given step size; the difference in average kinetic
energies using the half-step-averaged kinetic energies (*md* and
*md-vv-avek*) will be closer to the kinetic energy obtained in the limit
of small step size than will the full-step kinetic energy (using
``md-vv``). For NVE simulations, this difference is usually not
significant, since the positions and velocities of the particles are
still identical; it makes a difference in the way the the temperature of
the simulations are **interpreted**, but **not** in the trajectories that
are produced. Although the kinetic energy is more accurate with the
half-step-averaged method, meaning that it changes less as the timestep
gets large, it is also more noisy. The RMS deviation of the total energy
of the system (sum of kinetic plus potential) in the half-step-averaged
kinetic energy case will be higher (about twice as high in most cases)
than the full-step kinetic energy. The drift will still be the same,
however, as again, the trajectories are identical.

For NVT simulations, however, there **will** be a difference, as discussed
in the section on temperature control, since the velocities of the
particles are adjusted such that kinetic energies of the simulations,
which can be calculated either way, reach the distribution corresponding
to the set temperature. In this case, the three methods will not give
identical results.

Because the velocity and position are both defined at the same time
:math:`t` the velocity Verlet integrator can be used for some methods,
especially rigorously correct pressure control methods, that are not
actually possible with leap-frog. The integration itself takes
negligibly more time than leap-frog, but twice as many communication
calls are currently required. In most cases, and especially for large
systems where communication speed is important for parallelization and
differences between thermodynamic ensembles vanish in the :math:`1/N`
limit, and when only NVT ensembles are required, leap-frog will likely
be the preferred integrator. For pressure control simulations where the
fine details of the thermodynamics are important, only velocity Verlet
allows the true ensemble to be calculated. In either case, simulation
with double precision may be required to get fine details of
thermodynamics correct.

Multiple time stepping
~~~~~~~~~~~~~~~~~~~~~~

Several other simulation packages uses multiple time stepping for bonds
and/or the PME mesh forces. In |Gromacs| we have not implemented this
(yet), since we use a different philosophy. Bonds can be constrained
(which is also a more sound approximation of a physical quantum
oscillator), which allows the smallest time step to be increased to the
larger one. This not only halves the number of force calculations, but
also the update calculations. For even larger time steps, angle
vibrations involving hydrogen atoms can be removed using virtual
interaction sites (see sec. 
), which brings the
shortest time step up to PME mesh update frequency of a multiple time
stepping scheme.

Temperature coupling
~~~~~~~~~~~~~~~~~~~~

While direct use of molecular dynamics gives rise to the NVE (constant
number, constant volume, constant energy ensemble), most quantities that
we wish to calculate are actually from a constant temperature (NVT)
ensemble, also called the canonical ensemble. |Gromacs| can use the
*weak-coupling* scheme of Berendsen Berendsen et al. (1984), stochastic
randomization through the Andersen thermostat Andersen (1980), the
extended ensemble Nosé-Hoover scheme Nosé (1984; Hoover 1985), or a
velocity-rescaling scheme Bussi, Donadio, and Parrinello (2007) to
simulate constant temperature, with advantages of each of the schemes
laid out below.

There are several other reasons why it might be necessary to control the
temperature of the system (drift during equilibration, drift as a result
of force truncation and integration errors, heating due to external or
frictional forces), but this is not entirely correct to do from a
thermodynamic standpoint, and in some cases only masks the symptoms
(increase in temperature of the system) rather than the underlying
problem (deviations from correct physics in the dynamics). For larger
systems, errors in ensemble averages and structural properties incurred
by using temperature control to remove slow drifts in temperature appear
to be negligible, but no completely comprehensive comparisons have been
carried out, and some caution must be taking in interpreting the
results.

When using temperature and/or pressure coupling the total energy is no
longer conserved. Instead there is a conserved energy
quantity
the formula of
which will depend on the combination or temperature and pressure
coupling algorithm used. For all coupling algorithms, except for
Andersen temperature coupling and Parrinello-Rahman pressure coupling
combined with shear stress, the conserved energy quantity is computed
and stored in the energy and log file. Note that this quantity will not
be conserved when external forces are applied to the system, such as
pulling on group with a changing distance or an electric field.
Furthermore, how well the energy is conserved depends on the accuracy of
all algorithms involved in the simulation. Usually the algorithms that
cause most drift are constraints and the pair-list buffer, depending on
the parameters used.

Berendsen temperature coupling
^^^^^^^^^^^^^^^^^^^^^^^^^^^^^^

The Berendsen algorithm mimics weak coupling with first-order kinetics
to an external heat bath with given temperature :math:`T_0`. See
ref. Berendsen (1991) for a comparison with the Nosé-Hoover scheme. The
effect of this algorithm is that a deviation of the system temperature
from :math:`T_0` is slowly corrected according to:

.. math::  \frac{{\mbox{d}}T}{{\mbox{d}}t} = \frac{T_0-T}{\tau}
           :label: eqnTcoupling

which means that a temperature deviation decays exponentially with a
time constant :math:`\tau`. This method of coupling has the advantage
that the strength of the coupling can be varied and adapted to the user
requirement: for equilibration purposes the coupling time can be taken
quite short (*e.g.* 0.01 ps), but for reliable equilibrium runs it can
be taken much longer (*e.g.* 0.5 ps) in which case it hardly influences
the conservative dynamics.

The Berendsen thermostat suppresses the fluctuations of the kinetic
energy. This means that one does not generate a proper canonical
ensemble, so rigorously, the sampling will be incorrect. This error
scales with :math:`1/N`, so for very large systems most ensemble
averages will not be affected significantly, except for the distribution
of the kinetic energy itself. However, fluctuation properties, such as
the heat capacity, will be affected. A similar thermostat which does
produce a correct ensemble is the velocity rescaling thermostat Bussi,
Donadio, and Parrinello (2007) described below.

The heat flow into or out of the system is affected by scaling the
velocities of each particle every step, or every :math:`n_\mathrm{TC}`
steps, with a time-dependent factor :math:`\lambda`, given by:

.. math::  \lambda = \left[ 1 + \frac{n_\mathrm{TC} \Delta t}{\tau_T}
           \left\{\frac{T_0}{T(t -  {{\frac{1}{2}}{{\Delta t}}})} - 1 \right\} \right]^{1/2}
           :label: eqnlambda

The parameter :math:`\tau_T` is close, but not exactly equal, to the
time constant :math:`\tau` of the temperature coupling
(:eq:`eqn. (%s) <eqnTcoupling>`):

.. math:: \tau = 2 C_V \tau_T / N_{df} k

where :math:`C_V` is the total heat capacity of the system, :math:`k`
is Boltzmann’s constant, and :math:`N_{df}` is the total number of
degrees of freedom. The reason that :math:`\tau \neq \tau_T` is that the
kinetic energy change caused by scaling the velocities is partly
redistributed between kinetic and potential energy and hence the change
in temperature is less than the scaling energy. In practice, the ratio
:math:`\tau / \tau_T` ranges from 1 (gas) to 2 (harmonic solid) to 3
(water). When we use the term *temperature coupling time constant,* we
mean the parameter :math:`\tau_T`\.
**Note** that in practice the scaling factor :math:`\lambda` is limited
to the range of 0.8 :math:`<= \lambda <=` 1.25, to avoid scaling by very
large numbers which may crash the simulation. In normal use,
:math:`\lambda` will always be much closer to 1.0.

The thermostat modifies the kinetic energy at each scaling step by:

.. math:: \Delta E_k = (\lambda - 1)^2 E_k

The sum of these changes over the run needs to subtracted from the
total energy to obtain the conserved energy quantity.

Velocity-rescaling temperature coupling
^^^^^^^^^^^^^^^^^^^^^^^^^^^^^^^^^^^^^^^

The velocity-rescaling thermostat Bussi, Donadio, and Parrinello (2007)
is essentially a Berendsen thermostat (see above) with an additional
stochastic term that ensures a correct kinetic energy distribution by
modifying it according to

.. math::  {\mbox{d}}K = (K_0 - K) \frac{{\mbox{d}}t}{\tau_T} + 2 \sqrt{\frac{K K_0}{N_f}} \frac{{\mbox{d}}W}{\sqrt{\tau_T}},
           :label: eqnvrescale

where :math:`K` is the kinetic energy, :math:`N_f` the number of
degrees of freedom and :math:`{\mbox{d}}W` a Wiener process. There are
no additional parameters, except for a random seed. This thermostat
produces a correct canonical ensemble and still has the advantage of the
Berendsen thermostat: first order decay of temperature deviations and no
oscillations.

Andersen thermostat
^^^^^^^^^^^^^^^^^^^

One simple way to maintain a thermostatted ensemble is to take an
:math:`NVE` integrator and periodically re-select the velocities of the
particles from a Maxwell-Boltzmann distribution. Andersen (1980) This
can either be done by randomizing all the velocities simultaneously
(massive collision) every :math:`\tau_T/{{\Delta t}}` steps
(``andersen-massive``), or by randomizing every particle
with some small probability every timestep (``andersen``),
equal to :math:`{{\Delta t}}/\tau`, where in both cases
:math:`{{\Delta t}}` is the timestep and :math:`\tau_T` is a
characteristic coupling time scale. Because of the way constraints
operate, all particles in the same constraint group must be randomized
simultaneously. Because of parallelization issues, the
``andersen`` version cannot currently (5.0) be used in
systems with constraints. ``andersen-massive`` can be used
regardless of constraints. This thermostat is also currently only
possible with velocity Verlet algorithms, because it operates directly
on the velocities at each timestep.

This algorithm completely avoids some of the ergodicity issues of other
thermostatting algorithms, as energy cannot flow back and forth between
energetically decoupled components of the system as in velocity scaling
motions. However, it can slow down the kinetics of system by randomizing
correlated motions of the system, including slowing sampling when
:math:`\tau_T` is at moderate levels (less than 10 ps). This algorithm
should therefore generally not be used when examining kinetics or
transport properties of the system. Basconi and Shirts (2013)

Nosé-Hoover temperature coupling
^^^^^^^^^^^^^^^^^^^^^^^^^^^^^^^^

The Berendsen weak-coupling algorithm is extremely efficient for
relaxing a system to the target temperature, but once the system has
reached equilibrium it might be more important to probe a correct
canonical ensemble. This is unfortunately not the case for the
weak-coupling scheme.

To enable canonical ensemble simulations, |Gromacs| also supports the
extended-ensemble approach first proposed by Nosé Nosé (1984) and later
modified by Hoover Hoover (1985). The system Hamiltonian is extended by
introducing a thermal reservoir and a friction term in the equations of
motion. The friction force is proportional to the product of each
particle’s velocity and a friction parameter, :math:`\xi`. This friction
parameter (or *heat bath* variable) is a fully dynamic quantity with its
own momentum (:math:`p_{\xi}`) and equation of motion; the time
derivative is calculated from the difference between the current kinetic
energy and the reference temperature.

In this formulation, the particles´ equations of motion in
the global :ref:`MD scheme <gmx-md-scheme>` are replaced by:

.. math:: \frac {{\mbox{d}}^2{\mbox{\boldmath ${r}$}}_i}{{\mbox{d}}t^2} = \frac{{\mbox{\boldmath ${F}$}}_i}{m_i} - 
          \frac{p_{\xi}}{Q}\frac{{\mbox{d}}{\mbox{\boldmath ${r}$}}_i}{{\mbox{d}}t} ,
          :label: eqnNHeqnofmotion

where the equation of motion for the heat bath parameter :math:`\xi` is:

.. math:: \frac {{\mbox{d}}p_{\xi}}{{\mbox{d}}t} = \left( T - T_0 \right).

The reference temperature is denoted :math:`T_0`, while :math:`T` is
the current instantaneous temperature of the system. The strength of the
coupling is determined by the constant :math:`Q` (usually called the
*mass parameter* of the reservoir) in combination with the reference
temperature

Note that in some
derivations, an alternative notation
:math:`\xi_{\mathrm{alt}} = v_{\xi} = p_{\xi}/Q` is used.

The conserved quantity for the Nosé-Hoover equations of motion is not
the total energy, but rather

.. math::

   \begin{aligned}
   H = \sum_{i=1}^{N} \frac{{{{\mbox{\boldmath{$p$}}}}}_i}{2m_i} + U\left({{{\mbox{\boldmath{$r$}}}}}_1,{{{\mbox{\boldmath{$r$}}}}}_2,\ldots,{{{\mbox{\boldmath{$r$}}}}}_N\right) +\frac{p_{\xi}^2}{2Q} + N_fkT\xi,\end{aligned}

where :math:`N_f` is the total number of degrees of freedom.

In our opinion, the mass parameter is a somewhat awkward way of
describing coupling strength, especially due to its dependence on
reference temperature (and some implementations even include the number
of degrees of freedom in your system when defining :math:`Q`). To
maintain the coupling strength, one would have to change :math:`Q` in
proportion to the change in reference temperature. For this reason, we
prefer to let the |Gromacs| user work instead with the period
:math:`\tau_T` of the oscillations of kinetic energy between the system
and the reservoir instead. It is directly related to :math:`Q` and
:math:`T_0` via:

.. math:: Q = \frac {\tau_T^2 T_0}{4 \pi^2}.

This provides a much more intuitive way of selecting the Nosé-Hoover
coupling strength (similar to the weak-coupling relaxation), and in
addition :math:`\tau_T` is independent of system size and reference
temperature.

It is however important to keep the difference between the weak-coupling
scheme and the Nosé-Hoover algorithm in mind: Using weak coupling you
get a strongly damped *exponential relaxation*, while the Nosé-Hoover
approach produces an *oscillatory relaxation*. The actual time it takes
to relax with Nosé-Hoover coupling is several times larger than the
period of the oscillations that you select. These oscillations (in
contrast to exponential relaxation) also means that the time constant
normally should be 4–5 times larger than the relaxation time used with
weak coupling, but your mileage may vary.

Nosé-Hoover dynamics in simple systems such as collections of harmonic
oscillators, can be *nonergodic*, meaning that only a subsection of
phase space is ever sampled, even if the simulations were to run for
infinitely long. For this reason, the Nosé-Hoover chain approach was
developed, where each of the Nosé-Hoover thermostats has its own
Nosé-Hoover thermostat controlling its temperature. In the limit of an
infinite chain of thermostats, the dynamics are guaranteed to be
ergodic. Using just a few chains can greatly improve the ergodicity, but
recent research has shown that the system will still be nonergodic, and
it is still not entirely clear what the practical effect of this Cooke
and Schmidler (2008). Currently, the default number of chains is 10, but
this can be controlled by the user. In the case of chains, the equations
are modified in the following way to include a chain of thermostatting
particles Martyna, Klein, and Tuckerman (1992):

.. math::  \begin{aligned}
           \frac {{\mbox{d}}^2{\mbox{\boldmath ${r}$}}_i}{{\mbox{d}}t^2} &~=~& \frac{{\mbox{\boldmath ${F}$}}_i}{m_i} - \frac{p_{{\xi}_1}}{Q_1} \frac{{\mbox{d}}{\mbox{\boldmath ${r}$}}_i}{{\mbox{d}}t} \nonumber \\
           \frac {{\mbox{d}}p_{{\xi}_1}}{{\mbox{d}}t} &~=~& \left( T - T_0 \right) - p_{{\xi}_1} \frac{p_{{\xi}_2}}{Q_2} \nonumber \\
           \frac {{\mbox{d}}p_{{\xi}_{i=2\ldots N}}}{{\mbox{d}}t} &~=~& \left(\frac{p_{\xi_{i-1}}^2}{Q_{i-1}} -kT\right) - p_{\xi_i} \frac{p_{\xi_{i+1}}}{Q_{i+1}} \nonumber \\
           \frac {{\mbox{d}}p_{\xi_N}}{{\mbox{d}}t} &~=~& \left(\frac{p_{\xi_{N-1}}^2}{Q_{N-1}}-kT\right)
           \end{aligned}
           :label: eqnNHchaineqnofmotion

The conserved quantity for Nosé-Hoover chains is

.. math::

   \begin{aligned}
   H = \sum_{i=1}^{N} \frac{{{{\mbox{\boldmath{$p$}}}}}_i}{2m_i} + U\left({{{\mbox{\boldmath{$r$}}}}}_1,{{{\mbox{\boldmath{$r$}}}}}_2,\ldots,{{{\mbox{\boldmath{$r$}}}}}_N\right) +\sum_{k=1}^M\frac{p^2_{\xi_k}}{2Q^{\prime}_k} + N_fkT\xi_1 + kT\sum_{k=2}^M \xi_k \end{aligned}

The values and velocities of the Nosé-Hoover thermostat variables are
generally not included in the output, as they take up a fair amount of
space and are generally not important for analysis of simulations, but
this can be overridden by defining the environment variable
``GMX_NOSEHOOVER_CHAINS``, which will print the values of
all the positions and velocities of all Nosé-Hoover particles in the
chain to the :ref:`edr` file. Leap-frog simulations currently
can only have Nosé-Hoover chain lengths of 1, but this will likely be
updated in later version.

As described in the integrator section, for temperature coupling, the
temperature that the algorithm attempts to match to the reference
temperature is calculated differently in velocity Verlet and leap-frog
dynamics. Velocity Verlet (*md-vv*) uses the full-step kinetic energy,
while leap-frog and *md-vv-avek* use the half-step-averaged kinetic
energy.

We can examine the Trotter decomposition again to better understand the
differences between these constant-temperature integrators. In the case
of Nosé-Hoover dynamics (for simplicity, using a chain with :math:`N=1`,
with more details in Ref. Martyna et al. (1996)), we split the Liouville
operator as

.. math:: iL = iL_1 + iL_2 + iL_{\mathrm{NHC}},

where

.. math::

   \begin{aligned}
   iL_1 &=& \sum_{i=1}^N \left[\frac{{{{\mbox{\boldmath{$p$}}}}}_i}{m_i}\right]\cdot \frac{\partial}{\partial {{{\mbox{\boldmath{$r$}}}}}_i} \nonumber \\
   iL_2 &=& \sum_{i=1}^N {{{\mbox{\boldmath{$F$}}}}}_i\cdot \frac{\partial}{\partial {{{\mbox{\boldmath{$p$}}}}}_i} \nonumber \\
   iL_{\mathrm{NHC}} &=& \sum_{i=1}^N-\frac{p_{\xi}}{Q}{{{\mbox{\boldmath{$v$}}}}}_i\cdot \nabla_{{{{\mbox{\boldmath{$v$}}}}}_i} +\frac{p_{\xi}}{Q}\frac{\partial }{\partial \xi} + \left( T - T_0 \right)\frac{\partial }{\partial p_{\xi}}\end{aligned}

For standard velocity Verlet with Nosé-Hoover temperature control, this
becomes

.. math::

   \begin{aligned}
   \exp(iL{\Delta t}) &=& \exp\left(iL_{\mathrm{NHC}}{\Delta t}/2\right) \exp\left(iL_2 {\Delta t}/2\right) \nonumber \\
   &&\exp\left(iL_1 {\Delta t}\right) \exp\left(iL_2 {\Delta t}/2\right) \exp\left(iL_{\mathrm{NHC}}{\Delta t}/2\right) + \mathcal{O}({{\Delta t}}^3).\end{aligned}

For half-step-averaged temperature control using *md-vv-avek*, this
decomposition will not work, since we do not have the full step
temperature until after the second velocity step. However, we can
construct an alternate decomposition that is still reversible, by
switching the place of the NHC and velocity portions of the
decomposition:

.. math::  \begin{aligned}
           \exp(iL{\Delta t}) &=& \exp\left(iL_2 {\Delta t}/2\right) \exp\left(iL_{\mathrm{NHC}}{\Delta t}/2\right)\exp\left(iL_1 {\Delta t}\right)\nonumber \\
           &&\exp\left(iL_{\mathrm{NHC}}{\Delta t}/2\right) \exp\left(iL_2 {\Delta t}/2\right)+ \mathcal{O}({{\Delta t}}^3)
           \end{aligned}
           :label: eqhalfstepNHCintegrator

This formalism allows us to easily see the difference between the
different flavors of velocity Verlet integrator. The leap-frog
integrator can be seen as starting with
:eq:`Eq. (%s) <eqhalfstepNHCintegrator>` just before the
:math:`\exp\left(iL_1
{\Delta t}\right)` term, yielding:

.. math::

   \begin{aligned}
   \exp(iL{\Delta t}) &=&  \exp\left(iL_1 {\Delta t}\right) \exp\left(iL_{\mathrm{NHC}}{\Delta t}/2\right) \nonumber \\
   &&\exp\left(iL_2 {\Delta t}\right) \exp\left(iL_{\mathrm{NHC}}{\Delta t}/2\right) + \mathcal{O}({{\Delta t}}^3)\end{aligned}

and then using some algebra tricks to solve for some quantities are
required before they are actually calculated Holian, Voter, and Ravelo
(1995).

Group temperature coupling
^^^^^^^^^^^^^^^^^^^^^^^^^^

In |Gromacs| temperature
coupling can be performed on groups of atoms, typically a protein and
solvent. The reason such algorithms were introduced is that energy
exchange between different components is not perfect, due to different
effects including cut-offs etc. If now the whole system is coupled to
one heat bath, water (which experiences the largest cut-off noise) will
tend to heat up and the protein will cool down. Typically 100 K
differences can be obtained. With the use of proper electrostatic
methods (PME) these difference are much smaller but still not
negligible. The parameters for temperature coupling in groups are given
in the :ref:`mdp` file. Recent investigation has shown that
small temperature differences between protein and water may actually be
an artifact of the way temperature is calculated when there are finite
timesteps, and very large differences in temperature are likely a sign
of something else seriously going wrong with the system, and should be
investigated carefully Eastwood et al. (2010).

One special case should be mentioned: it is possible to
temperature-couple only part of the system, leaving other parts without
temperature coupling. This is done by specifying :math:`{-1}` for the
time constant :math:`\tau_T` for the group that should not be
thermostatted. If only part of the system is thermostatted, the system
will still eventually converge to an NVT system. In fact, one suggestion
for minimizing errors in the temperature caused by discretized timesteps
is that if constraints on the water are used, then only the water
degrees of freedom should be thermostatted, not protein degrees of
freedom, as the higher frequency modes in the protein can cause larger
deviations from the *true* temperature, the temperature obtained with
small timesteps Eastwood et al. (2010).

Pressure coupling
~~~~~~~~~~~~~~~~~

In the same spirit as the temperature coupling, the system can also be
coupled to a *pressure bath.* |Gromacs| supports both the Berendsen
algorithm Berendsen et al. (1984) that scales coordinates and box
vectors every step, the extended-ensemble Parrinello-Rahman
approach Parrinello and Rahman (1981; Nosé and Klein 1983), and for the
velocity Verlet variants, the Martyna-Tuckerman-Tobias-Klein (MTTK)
implementation of pressure control Martyna et al. (1996).
Parrinello-Rahman and Berendsen can be combined with any of the
temperature coupling methods above. MTTK can only be used with
Nosé-Hoover temperature control. From 5.1 afterwards, it can only used
when the system does not have constraints.

Berendsen pressure coupling
^^^^^^^^^^^^^^^^^^^^^^^^^^^

The Berendsen algorithm rescales the coordinates and box vectors every
step, or every :math:`n_\mathrm{PC}` steps, with a matrix
:math:`\mu`, which has the effect of a
first-order kinetic relaxation of the pressure towards a given reference
pressure :math:`{\bf P}_0` according to

.. math:: \frac{{\mbox{d}}{\bf P}}{{\mbox{d}}t} = \frac{{\bf P}_0-{\bf P}}{\tau_p}.

The scaling matrix 
:math:`\mu` is given by

.. math::  \mu_{ij}
           = \delta_{ij} - \frac{n_\mathrm{PC}\Delta t}{3\, \tau_p} \beta_{ij} \{P_{0ij} - P_{ij}(t) \}.
           :label: eqnmu

Here,
:math:`\beta` is the isothermal compressibility
of the system. In most cases this will be a diagonal matrix, with equal
elements on the diagonal, the value of which is generally not known. It
suffices to take a rough estimate because the value of
:math:`\beta` only influences the non-critical
time constant of the pressure relaxation without affecting the average
pressure itself. For water at 1 atm and 300 K
:math:`\beta = 4.6 \times 10^{-10}`
Pa\ :math:`^{-1} = 4.6 \times 10^{-5}` bar\ :math:`^{-1}`, which is
:math:`7.6 \times 10^{-4}` MD units (see
chapter 
). Most other liquids have similar
values. When scaling completely anisotropically, the system has to be
rotated in order to obey eqn.
This
rotation is approximated in first order in the scaling, which is usually
less than :math:`10^{-4}`. The actual scaling matrix
:math:`\mu'` is

.. math::

   \mbox{\boldmath $\mu'$} = 
   \left(\begin{array}{ccc}
   \mu_{xx} & \mu_{xy} + \mu_{yx} & \mu_{xz} + \mu_{zx} \\
   0        & \mu_{yy}            & \mu_{yz} + \mu_{zy} \\
   0        & 0                   & \mu_{zz}
   \end{array}\right).

The velocities are neither scaled nor rotated. Since the equations of
motion are modified by pressure coupling, the conserved energy quantity
also needs to be modified. For first order pressure coupling, the work
the barostat applies to the system every step needs to be subtracted
from the total energy to obtain the conserved energy quantity:

.. math::

   - \sum_{i,j} (\mu_{ij} -\delta_{ij}) P_{ij} V =
   \sum_{i,j} 2(\mu_{ij} -\delta_{ij}) \Xi_{ij}

where :math:`\delta_{ij}` is the Kronecker delta and :math:`{\bf \Xi}`
is the virial. Note that the factor 2 originates from the factor
:math:`\frac{1}{2}` in the virial definition
(:eq:`eqn. (%s) <eqnXi>`).

In |Gromacs|, the Berendsen scaling can also be done isotropically, which
means that instead of :math:`{\mbox{\boldmath ${P}$}}` a diagonal matrix
with elements of size trace\ :math:`({\mbox{\boldmath ${P}$}})/3` is
used. For systems with interfaces, semi-isotropic scaling can be useful.
In this case, the :math:`x/y`-directions are scaled isotropically and
the :math:`z` direction is scaled independently. The compressibility in
the :math:`x/y` or :math:`z`-direction can be set to zero, to scale only
in the other direction(s).

If you allow full anisotropic deformations and use constraints you might
have to scale more slowly or decrease your timestep to avoid errors from
the constraint algorithms. It is important to note that although the
Berendsen pressure control algorithm yields a simulation with the
correct average pressure, it does not yield the exact NPT ensemble, and
it is not yet clear exactly what errors this approximation may yield.

Parrinello-Rahman pressure coupling
^^^^^^^^^^^^^^^^^^^^^^^^^^^^^^^^^^^

In cases where the fluctuations in pressure or volume are important *per
se* (*e.g.* to calculate thermodynamic properties), especially for small
systems, it may be a problem that the exact ensemble is not well defined
for the weak-coupling scheme, and that it does not simulate the true NPT
ensemble.

|Gromacs| also supports constant-pressure simulations using the
Parrinello-Rahman approach Parrinello and Rahman (1981; Nosé and Klein
1983), which is similar to the Nosé-Hoover temperature coupling, and in
theory gives the true NPT ensemble. With the Parrinello-Rahman barostat,
the box vectors as represented by the matrix
:math:`{\mbox{\boldmath ${b}$}}` obey the matrix equation of

motion

.. math:: \frac{{\mbox{d}}{\mbox{\boldmath ${b}$}}^2}{{\mbox{d}}t^2}= V {\mbox{\boldmath ${W}$}}^{-1} {\mbox{\boldmath ${b}$}}'^{-1} \left( {\mbox{\boldmath ${P}$}} - {\mbox{\boldmath ${P}$}}_{ref}\right).

The volume of the box is denoted :math:`V`, and
:math:`{\mbox{\boldmath ${W}$}}` is a matrix parameter that determines
the strength of the coupling. The matrices
:math:`{\mbox{\boldmath ${P}$}}` and
:math:`{\mbox{\boldmath ${P}$}}`\ :math:`_{ref}` are the current and
reference pressures, respectively.

The equations of motion for the particles are also changed, just as for
the Nosé-Hoover coupling. In most cases you would combine the
Parrinello-Rahman barostat with the Nosé-Hoover thermostat, but to keep
it simple we only show the Parrinello-Rahman modification here. The
modified Hamiltonian, which will be conserved, is:

.. math::

   E_\mathrm{pot} + E_\mathrm{kin} +  \sum_i P_{ii} V +
   \sum_{i,j} \frac{1}{2} W_{ij}  \left( \frac{{\mbox{d}}b_{ij}}{{\mbox{d}}t} \right)^2

The equations of motion for the atoms, obtained from the Hamiltonian
are:

.. math::

   \begin{aligned}
    \frac {{\mbox{d}}^2{\mbox{\boldmath ${r}$}}_i}{{\mbox{d}}t^2} & = & \frac{{\mbox{\boldmath ${F}$}}_i}{m_i} -
   {\mbox{\boldmath ${M}$}} \frac{{\mbox{d}}{\mbox{\boldmath ${r}$}}_i}{{\mbox{d}}t} , \\ {\mbox{\boldmath ${M}$}} & = & {\mbox{\boldmath ${b}$}}^{-1} \left[
     {\mbox{\boldmath ${b}$}} \frac{{\mbox{d}}{\mbox{\boldmath ${b}$}}'}{{\mbox{d}}t} + \frac{{\mbox{d}}{\mbox{\boldmath ${b}$}}}{{\mbox{d}}t} {\mbox{\boldmath ${b}$}}'
     \right] {\mbox{\boldmath ${b}$}}'^{-1}.\end{aligned}

The (inverse) mass parameter matrix
:math:`{\mbox{\boldmath ${W}$}}^{-1}` determines the strength of the
coupling, and how the box can be deformed. The box restriction
(:eq:`(%s) <eqnboxrot>`) will be fulfilled automatically if the
corresponding elements of :math:`{\mbox{\boldmath ${W}$}}^{-1}` are
zero. Since the coupling strength also depends on the size of your box,
we prefer to calculate it automatically in |Gromacs|. You only have to
provide the approximate isothermal compressibilities
:math:`\beta` and the pressure time constant
:math:`\tau_p` in the input file (:math:`L` is the largest box matrix
element):

.. math::

   \left(
   {\mbox{\boldmath ${W}$}}^{-1} \right)_{ij} = \frac{4 \pi^2 \beta_{ij}}{3 \tau_p^2 L}.

Just as for the Nosé-Hoover thermostat, you should realize that the
Parrinello-Rahman time constant is *not* equivalent to the relaxation
time used in the Berendsen pressure coupling algorithm. In most cases
you will need to use a 4–5 times larger time constant with
Parrinello-Rahman coupling. If your pressure is very far from
equilibrium, the Parrinello-Rahman coupling may result in very large box
oscillations that could even crash your run. In that case you would have
to increase the time constant, or (better) use the weak-coupling scheme
to reach the target pressure, and then switch to Parrinello-Rahman
coupling once the system is in equilibrium. Additionally, using the
leap-frog algorithm, the pressure at time :math:`t` is not available
until after the time step has completed, and so the pressure from the
previous step must be used, which makes the algorithm not directly
reversible, and may not be appropriate for high precision thermodynamic
calculations.

Surface-tension coupling
^^^^^^^^^^^^^^^^^^^^^^^^

When a periodic system consists of more than one phase, separated by
surfaces which are parallel to the :math:`xy`-plane, the surface tension
and the :math:`z`-component of the pressure can be coupled to a pressure
bath. Presently, this only works with the Berendsen pressure coupling
algorithm in |Gromacs|. The average surface tension :math:`\gamma(t)` can
be calculated from the difference between the normal and the lateral
pressure

.. math::

   \begin{aligned}
   \gamma(t) & = & 
   \frac{1}{n} \int_0^{L_z}
   \left\{ P_{zz}(z,t) - \frac{P_{xx}(z,t) + P_{yy}(z,t)}{2} \right\} \mbox{d}z \\
   & = &
   \frac{L_z}{n} \left\{ P_{zz}(t) - \frac{P_{xx}(t) + P_{yy}(t)}{2} \right\},\end{aligned}

where :math:`L_z` is the height of the box and :math:`n` is the number
of surfaces. The pressure in the z-direction is corrected by scaling the
height of the box with :math:`\mu_{zz}`

.. math:: \Delta P_{zz} = \frac{\Delta t}{\tau_p} \{ P_{0zz} - P_{zz}(t) \}

.. math:: \mu_{zz} = 1 + \beta_{zz} \Delta P_{zz}

This is similar to normal pressure coupling, except that the factor of
:math:`1/3` is missing. The pressure correction in the
:math:`z`-direction is then used to get the correct convergence for the
surface tension to the reference value :math:`\gamma_0`. The correction
factor for the box length in the :math:`x`/:math:`y`-direction is

.. math::

   \mu_{x/y} = 1 + \frac{\Delta t}{2\,\tau_p} \beta_{x/y}
           \left( \frac{n \gamma_0}{\mu_{zz} L_z}
           - \left\{ P_{zz}(t)+\Delta P_{zz} - \frac{P_{xx}(t) + P_{yy}(t)}{2} \right\} 
           \right)

The value of :math:`\beta_{zz}` is more critical than with normal
pressure coupling. Normally an incorrect compressibility will just scale
:math:`\tau_p`, but with surface tension coupling it affects the
convergence of the surface tension. When :math:`\beta_{zz}` is set to
zero (constant box height), :math:`\Delta P_{zz}` is also set to zero,
which is necessary for obtaining the correct surface tension.

MTTK pressure control algorithms
^^^^^^^^^^^^^^^^^^^^^^^^^^^^^^^^

As mentioned in the previous section, one weakness of leap-frog
integration is in constant pressure simulations, since the pressure
requires a calculation of both the virial and the kinetic energy at the
full time step; for leap-frog, this information is not available until
*after* the full timestep. Velocity Verlet does allow the calculation,
at the cost of an extra round of global communication, and can compute,
mod any integration errors, the true NPT ensemble.

The full equations, combining both pressure coupling and temperature
coupling, are taken from Martyna *et al.* Martyna et al. (1996) and
Tuckerman Tuckerman et al. (2006) and are referred to here as MTTK
equations (Martyna-Tuckerman-Tobias-Klein). We introduce for convenience
:math:`\epsilon = (1/3)\ln (V/V_0)`, where :math:`V_0` is a reference
volume. The momentum of :math:`\epsilon` is
:math:`{v_{\epsilon}}= p_{\epsilon}/W =
\dot{\epsilon} = \dot{V}/3V`, and define :math:`\alpha = 1 + 3/N_{dof}`
(see Ref Tuckerman et al. (2006))

The isobaric equations are

.. math::

   \begin{aligned}
   \dot{{{{\mbox{\boldmath{$r$}}}}}}_i &=& \frac{{{{\mbox{\boldmath{$p$}}}}}_i}{m_i} + \frac{{p_{\epsilon}}}{W} {{{\mbox{\boldmath{$r$}}}}}_i \nonumber \\
   \frac{\dot{{{{\mbox{\boldmath{$p$}}}}}}_i}{m_i} &=& \frac{1}{m_i}{{{\mbox{\boldmath{$F$}}}}}_i - \alpha\frac{{p_{\epsilon}}}{W} \frac{{{{\mbox{\boldmath{$p$}}}}}_i}{m_i} \nonumber \\
   \dot{\epsilon} &=& \frac{{p_{\epsilon}}}{W} \nonumber \\
   \frac{\dot{{p_{\epsilon}}}}{W} &=& \frac{3V}{W}(P_{\mathrm{int}} - P) + (\alpha-1)\left(\sum_{n=1}^N\frac{{{{\mbox{\boldmath{$p$}}}}}_i^2}{m_i}\right),\\\end{aligned}

where

.. math::

   \begin{aligned}
   P_{\mathrm{int}} &=& P_{\mathrm{kin}} -P_{\mathrm{vir}} = \frac{1}{3V}\left[\sum_{i=1}^N \left(\frac{{{{\mbox{\boldmath{$p$}}}}}_i^2}{2m_i} - {{{\mbox{\boldmath{$r$}}}}}_i \cdot {{{\mbox{\boldmath{$F$}}}}}_i\
   \right)\right].\end{aligned}

The terms including :math:`\alpha` are required to make phase space
incompressible Tuckerman et al. (2006). The :math:`\epsilon`
acceleration term can be rewritten as

.. math::

   \begin{aligned}
   \frac{\dot{{p_{\epsilon}}}}{W} &=& \frac{3V}{W}\left(\alpha P_{\mathrm{kin}} - P_{\mathrm{vir}} - P\right)\end{aligned}

In terms of velocities, these equations become

.. math::

   \begin{aligned}
   \dot{{{{\mbox{\boldmath{$r$}}}}}}_i &=& {{{\mbox{\boldmath{$v$}}}}}_i + {v_{\epsilon}}{{{\mbox{\boldmath{$r$}}}}}_i \nonumber \\
   \dot{{{{\mbox{\boldmath{$v$}}}}}}_i &=& \frac{1}{m_i}{{{\mbox{\boldmath{$F$}}}}}_i - \alpha{v_{\epsilon}}{{{\mbox{\boldmath{$v$}}}}}_i \nonumber \\
   \dot{\epsilon} &=& {v_{\epsilon}}\nonumber \\
   \dot{{v_{\epsilon}}} &=& \frac{3V}{W}(P_{\mathrm{int}} - P) + (\alpha-1)\left( \sum_{n=1}^N \frac{1}{2} m_i {{{\mbox{\boldmath{$v$}}}}}_i^2\right)\nonumber \\
   P_{\mathrm{int}} &=& P_{\mathrm{kin}} - P_{\mathrm{vir}} = \frac{1}{3V}\left[\sum_{i=1}^N \left(\frac{1}{2} m_i{{{\mbox{\boldmath{$v$}}}}}_i^2 - {{{\mbox{\boldmath{$r$}}}}}_i \cdot {{{\mbox{\boldmath{$F$}}}}}_i\right)\right]\end{aligned}

For these equations, the conserved quantity is

 .. math::

   \begin{aligned}
   H = \sum_{i=1}^{N} \frac{{{{\mbox{\boldmath{$p$}}}}}_i^2}{2m_i} + U\left({{{\mbox{\boldmath{$r$}}}}}_1,{{{\mbox{\boldmath{$r$}}}}}_2,\ldots,{{{\mbox{\boldmath{$r$}}}}}_N\right) + \frac{p_\epsilon}{2W} + PV\end{aligned}

The next step is to add temperature control. Adding Nosé-Hoover chains,
including to the barostat degree of freedom, where we use :math:`\eta`
for the barostat Nosé-Hoover variables, and :math:`Q^{\prime}` for the
coupling constants of the thermostats of the barostats, we get

.. math::

   \begin{aligned}
   \dot{{{{\mbox{\boldmath{$r$}}}}}}_i &=& \frac{{{{\mbox{\boldmath{$p$}}}}}_i}{m_i} + \frac{{p_{\epsilon}}}{W} {{{\mbox{\boldmath{$r$}}}}}_i \nonumber \\
   \frac{\dot{{{{\mbox{\boldmath{$p$}}}}}}_i}{m_i} &=& \frac{1}{m_i}{{{\mbox{\boldmath{$F$}}}}}_i - \alpha\frac{{p_{\epsilon}}}{W} \frac{{{{\mbox{\boldmath{$p$}}}}}_i}{m_i} - \frac{p_{\xi_1}}{Q_1}\frac{{{{\mbox{\boldmath{$p$}}}}}_i}{m_i}\nonumber \\
   \dot{\epsilon} &=& \frac{{p_{\epsilon}}}{W} \nonumber \\
   \frac{\dot{{p_{\epsilon}}}}{W} &=& \frac{3V}{W}(\alpha P_{\mathrm{kin}} - P_{\mathrm{vir}} - P) -\frac{p_{\eta_1}}{Q^{\prime}_1}{p_{\epsilon}}\nonumber \\
   \dot{\xi}_k &=& \frac{p_{\xi_k}}{Q_k} \nonumber \\ 
   \dot{\eta}_k &=& \frac{p_{\eta_k}}{Q^{\prime}_k} \nonumber \\
   \dot{p}_{\xi_k} &=& G_k - \frac{p_{\xi_{k+1}}}{Q_{k+1}} \;\;\;\; k=1,\ldots, M-1 \nonumber \\ 
   \dot{p}_{\eta_k} &=& G^\prime_k - \frac{p_{\eta_{k+1}}}{Q^\prime_{k+1}} \;\;\;\; k=1,\ldots, M-1 \nonumber \\
   \dot{p}_{\xi_M} &=& G_M \nonumber \\
   \dot{p}_{\eta_M} &=& G^\prime_M, \nonumber \\\end{aligned}

where

.. math::

   \begin{aligned}
   P_{\mathrm{int}} &=& P_{\mathrm{kin}} - P_{\mathrm{vir}} = \frac{1}{3V}\left[\sum_{i=1}^N \left(\frac{{{{\mbox{\boldmath{$p$}}}}}_i^2}{2m_i} - {{{\mbox{\boldmath{$r$}}}}}_i \cdot {{{\mbox{\boldmath{$F$}}}}}_i\right)\right] \nonumber \\
   G_1  &=& \sum_{i=1}^N \frac{{{{\mbox{\boldmath{$p$}}}}}^2_i}{m_i} - N_f kT \nonumber \\
   G_k  &=&  \frac{p^2_{\xi_{k-1}}}{2Q_{k-1}} - kT \;\; k = 2,\ldots,M \nonumber \\
   G^\prime_1 &=& \frac{{p_{\epsilon}}^2}{2W} - kT \nonumber \\
   G^\prime_k &=& \frac{p^2_{\eta_{k-1}}}{2Q^\prime_{k-1}} - kT \;\; k = 2,\ldots,M\end{aligned}

The conserved quantity is now

.. math::

   \begin{aligned}
   H = \sum_{i=1}^{N} \frac{{{{\mbox{\boldmath{$p$}}}}}_i}{2m_i} + U\left({{{\mbox{\boldmath{$r$}}}}}_1,{{{\mbox{\boldmath{$r$}}}}}_2,\ldots,{{{\mbox{\boldmath{$r$}}}}}_N\right) + \frac{p^2_\epsilon}{2W} + PV + \nonumber \\
   \sum_{k=1}^M\frac{p^2_{\xi_k}}{2Q_k} +\sum_{k=1}^M\frac{p^2_{\eta_k}}{2Q^{\prime}_k} + N_fkT\xi_1 +  kT\sum_{i=2}^M \xi_k + kT\sum_{k=1}^M \eta_k\end{aligned}

Returning to the Trotter decomposition formalism, for pressure control
and temperature control Martyna et al. (1996) we get:

.. math::

   \begin{aligned}
   iL = iL_1 + iL_2 + iL_{\epsilon,1} + iL_{\epsilon,2} + iL_{\mathrm{NHC-baro}} + iL_{\mathrm{NHC}}\end{aligned}

where “NHC-baro” corresponds to the Nosè-Hoover chain of the barostat,
and NHC corresponds to the NHC of the particles,

.. math::

   \begin{aligned}
   iL_1 &=& \sum_{i=1}^N \left[\frac{{{{\mbox{\boldmath{$p$}}}}}_i}{m_i} + \frac{{p_{\epsilon}}}{W}{{{\mbox{\boldmath{$r$}}}}}_i\right]\cdot \frac{\partial}{\partial {{{\mbox{\boldmath{$r$}}}}}_i} \\
   iL_2 &=& \sum_{i=1}^N {{{\mbox{\boldmath{$F$}}}}}_i - \alpha \frac{{p_{\epsilon}}}{W}{{{\mbox{\boldmath{$p$}}}}}_i \cdot \frac{\partial}{\partial {{{\mbox{\boldmath{$p$}}}}}_i} \\
   iL_{\epsilon,1} &=& \frac{p_\epsilon}{W} \frac{\partial}{\partial \epsilon}\\
   iL_{\epsilon,2} &=& G_{\epsilon} \frac{\partial}{\partial p_\epsilon}\end{aligned}

and where

.. math::

   \begin{aligned}
   G_{\epsilon} = 3V\left(\alpha P_{\mathrm{kin}} - P_{\mathrm{vir}} - P\right)\end{aligned}

Using the Trotter decomposition, we get

.. math::

   \begin{aligned}
   \exp(iL{\Delta t}) &=& \exp\left(iL_{\mathrm{NHC-baro}}{\Delta t}/2\right)\exp\left(iL_{\mathrm{NHC}}{\Delta t}/2\right) \nonumber \nonumber \\
   &&\exp\left(iL_{\epsilon,2}{\Delta t}/2\right) \exp\left(iL_2 {\Delta t}/2\right) \nonumber \nonumber \\
   &&\exp\left(iL_{\epsilon,1}{\Delta t}\right) \exp\left(iL_1 {\Delta t}\right) \nonumber \nonumber \\
   &&\exp\left(iL_2 {\Delta t}/2\right) \exp\left(iL_{\epsilon,2}{\Delta t}/2\right) \nonumber \nonumber \\
   &&\exp\left(iL_{\mathrm{NHC}}{\Delta t}/2\right)\exp\left(iL_{\mathrm{NHC-baro}}{\Delta t}/2\right) + \mathcal{O}({\Delta t}^3)\end{aligned}

The action of :math:`\exp\left(iL_1 {\Delta t}\right)` comes from the
solution of the the differential equation
:math:`\dot{{{{\mbox{\boldmath{$r$}}}}}}_i = {{{\mbox{\boldmath{$v$}}}}}_i + {v_{\epsilon}}{{{\mbox{\boldmath{$r$}}}}}_i`
with
:math:`{{{\mbox{\boldmath{$v$}}}}}_i = {{{\mbox{\boldmath{$p$}}}}}_i/m_i`
and :math:`{v_{\epsilon}}` constant with initial condition
:math:`{{{\mbox{\boldmath{$r$}}}}}_i(0)`, evaluate at
:math:`t=\Delta t`. This yields the evolution

.. math:: {{{\mbox{\boldmath{$r$}}}}}_i({\Delta t}) = {{{\mbox{\boldmath{$r$}}}}}_i(0)e^{{v_{\epsilon}}{\Delta t}} + \Delta t {{{\mbox{\boldmath{$v$}}}}}_i(0) e^{{v_{\epsilon}}{\Delta t}/2} {\frac{\sinh{\left( {v_{\epsilon}}{\Delta t}/2\right)}}{{v_{\epsilon}}{\Delta t}/2}}.

The action of :math:`\exp\left(iL_2 {\Delta t}/2\right)` comes from the
solution of the differential equation
:math:`\dot{{{{\mbox{\boldmath{$v$}}}}}}_i = \frac{{{{\mbox{\boldmath{$F$}}}}}_i}{m_i} -
\alpha{v_{\epsilon}}{{{\mbox{\boldmath{$v$}}}}}_i`, yielding

.. math:: {{{\mbox{\boldmath{$v$}}}}}_i({\Delta t}/2) = {{{\mbox{\boldmath{$v$}}}}}_i(0)e^{-\alpha{v_{\epsilon}}{\Delta t}/2} + \frac{\Delta t}{2m_i}{{{\mbox{\boldmath{$F$}}}}}_i(0) e^{-\alpha{v_{\epsilon}}{\Delta t}/4}{\frac{\sinh{\left( \alpha{v_{\epsilon}}{\Delta t}/4\right)}}{\alpha{v_{\epsilon}}{\Delta t}/4}}.

*md-vv-avek* uses the full step kinetic energies for determining the
pressure with the pressure control, but the half-step-averaged kinetic
energy for the temperatures, which can be written as a Trotter
decomposition as

.. math::

   \begin{aligned}
   \exp(iL{\Delta t}) &=& \exp\left(iL_{\mathrm{NHC-baro}}{\Delta t}/2\right)\nonumber \exp\left(iL_{\epsilon,2}{\Delta t}/2\right) \exp\left(iL_2 {\Delta t}/2\right) \nonumber \\
   &&\exp\left(iL_{\mathrm{NHC}}{\Delta t}/2\right) \exp\left(iL_{\epsilon,1}{\Delta t}\right) \exp\left(iL_1 {\Delta t}\right) \exp\left(iL_{\mathrm{NHC}}{\Delta t}/2\right) \nonumber \\
   &&\exp\left(iL_2 {\Delta t}/2\right) \exp\left(iL_{\epsilon,2}{\Delta t}/2\right) \exp\left(iL_{\mathrm{NHC-baro}}{\Delta t}/2\right) + \mathcal{O}({\Delta t}^3)\end{aligned}

With constraints, the equations become significantly more complicated,
in that each of these equations need to be solved iteratively for the
constraint forces. Before |Gromacs| 5.1, these iterative constraints were
solved as described in T.-Q. Yu et al. (2010). From |Gromacs| 5.1 onward,
MTTK with constraints has been removed because of numerical stability
issues with the iterations.

Infrequent evaluation of temperature and pressure coupling
^^^^^^^^^^^^^^^^^^^^^^^^^^^^^^^^^^^^^^^^^^^^^^^^^^^^^^^^^^

Temperature and pressure control require global communication to compute
the kinetic energy and virial, which can become costly if performed
every step for large systems. We can rearrange the Trotter decomposition
to give alternate symplectic, reversible integrator with the coupling
steps every :math:`n` steps instead of every steps. These new
integrators will diverge if the coupling time step is too large, as the
auxiliary variable integrations will not converge. However, in most
cases, long coupling times are more appropriate, as they disturb the
dynamics less Martyna et al. (1996).

Standard velocity Verlet with Nosé-Hoover temperature control has a
Trotter expansion

.. math::

   \begin{aligned}
   \exp(iL{\Delta t}) &\approx& \exp\left(iL_{\mathrm{NHC}}{\Delta t}/2\right) \exp\left(iL_2 {\Delta t}/2\right) \nonumber \\
   &&\exp\left(iL_1 {\Delta t}\right) \exp\left(iL_2 {\Delta t}/2\right) \exp\left(iL_{\mathrm{NHC}}{\Delta t}/2\right).\end{aligned}

If the Nosé-Hoover chain is sufficiently slow with respect to the
motions of the system, we can write an alternate integrator over
:math:`n` steps for velocity Verlet as

.. math::

   \begin{aligned}
   \exp(iL{\Delta t}) &\approx& (\exp\left(iL_{\mathrm{NHC}}(n{\Delta t}/2)\right)\left[\exp\left(iL_2 {\Delta t}/2\right)\right. \nonumber \\
   &&\left.\exp\left(iL_1 {\Delta t}\right) \exp\left(iL_2 {\Delta t}/2\right)\right]^n \exp\left(iL_{\mathrm{NHC}}(n{\Delta t}/2)\right).\end{aligned}

For pressure control, this becomes

.. math::

   \begin{aligned}
   \exp(iL{\Delta t}) &\approx& \exp\left(iL_{\mathrm{NHC-baro}}(n{\Delta t}/2)\right)\exp\left(iL_{\mathrm{NHC}}(n{\Delta t}/2)\right) \nonumber \nonumber \\
   &&\exp\left(iL_{\epsilon,2}(n{\Delta t}/2)\right) \left[\exp\left(iL_2 {\Delta t}/2\right)\right. \nonumber \nonumber \\
   &&\exp\left(iL_{\epsilon,1}{\Delta t}\right) \exp\left(iL_1 {\Delta t}\right) \nonumber \nonumber \\
   &&\left.\exp\left(iL_2 {\Delta t}/2\right)\right]^n \exp\left(iL_{\epsilon,2}(n{\Delta t}/2)\right) \nonumber \nonumber \\
   &&\exp\left(iL_{\mathrm{NHC}}(n{\Delta t}/2)\right)\exp\left(iL_{\mathrm{NHC-baro}}(n{\Delta t}/2)\right),\end{aligned}

where the box volume integration occurs every step, but the auxiliary
variable integrations happen every :math:`n` steps.

The complete update algorithm
~~~~~~~~~~~~~~~~~~~~~~~~~~~~~
.. _gmx_md_update:

**THE UPDATE ALGORITHM**

--------------

 
 Given:
 Positions :math:`{\mbox{\boldmath ${r}$}}` of all atoms at time
 :math:`t`
 Velocities :math:`{\mbox{\boldmath ${v}$}}` of all atoms at time
 :math:`t-{{\frac{1}{2}}{{\Delta t}}}`
 Accelerations :math:`{\mbox{\boldmath ${F}$}}/m` on all atoms at time
 :math:`t`.
 (Forces are computed disregarding any constraints)
 Total kinetic energy and virial at :math:`t-{{\Delta t}}`
 :math:`\Downarrow`
 **1.** Compute the scaling factors :math:`\lambda` and :math:`\mu`
 according to eqns.
 and
 :math:`\Downarrow`
 **2.** Update and scale velocities:
 :math:`{\mbox{\boldmath ${v}$}}' =  \lambda ({\mbox{\boldmath ${v}$}} +
 {\mbox{\boldmath ${a}$}} \Delta t)`
 :math:`\Downarrow`
 **3.** Compute new unconstrained coordinates:
 :math:`{\mbox{\boldmath ${r}$}}' = {\mbox{\boldmath ${r}$}} + {\mbox{\boldmath ${v}$}}'
 \Delta t`
 :math:`\Downarrow`
 **4.** Apply constraint algorithm to coordinates:
 constrain(\ :math:`{\mbox{\boldmath ${r}$}}^{'} \rightarrow  {\mbox{\boldmath ${r}$}}'';
 \,  {\mbox{\boldmath ${r}$}}`)
 :math:`\Downarrow`
 **5.** Correct velocities for constraints:
 :math:`{\mbox{\boldmath ${v}$}} = ({\mbox{\boldmath ${r}$}}'' -
 {\mbox{\boldmath ${r}$}}) / \Delta t`
 :math:`\Downarrow`
 **6.** Scale coordinates and box:
 :math:`{\mbox{\boldmath ${r}$}} = \mu {\mbox{\boldmath ${r}$}}''; {\mbox{\boldmath ${b}$}} =
 \mu  {\mbox{\boldmath ${b}$}}`

The complete algorithm for the update of velocities and coordinates is
given using leap-frog in :ref:`the outline above <gmx_md_update>`
The SHAKE algorithm of step 4 is explained below.

|Gromacs| has a provision to *freeze* (prevent motion of) selected
particles
which must be defined as a
*freeze group*
This is implemented using a *freeze factor*
:math:`{\mbox{\boldmath ${f}$}}_g`, which is a vector, and differs for
each freeze group (see sec. 
). This
vector contains only zero (freeze) or one (don’t freeze). When we take
this freeze factor and the external acceleration
:math:`{\mbox{\boldmath ${a}$}}_h` into account the update algorithm for
the velocities becomes

.. math:: {\mbox{\boldmath ${v}$}}(t+{\frac{\Delta t}{2}})~=~{\mbox{\boldmath ${f}$}}_g * \lambda * \left[ {\mbox{\boldmath ${v}$}}(t-{\frac{\Delta t}{2}}) +\frac{{\mbox{\boldmath ${F}$}}(t)}{m}\Delta t + {\mbox{\boldmath ${a}$}}_h \Delta t \right],

where :math:`g` and :math:`h` are group indices which differ per atom.

Output step
~~~~~~~~~~~

The most important output of the MD run is the *trajectory
file* 
which contains particle coordinates and (optionally) velocities at
regular intervals. The trajectory file contains frames that could
include positions, velocities and/or forces, as well as information
about the dimensions of the simulation volume, integration step,
integration time, etc. The interpretation of the time varies with the
integrator chosen, as described above. For Velocity Verlet integrators,
velocities labeled at time :math:`t` are for that time. For other
integrators (e.g. leap-frog, stochastic dynamics), the velocities
labeled at time :math:`t` are for time
:math:`t - {{\frac{1}{2}}{{\Delta t}}}`.

Since the trajectory files are lengthy, one should not save every step!
To retain all information it suffices to write a frame every 15 steps,
since at least 30 steps are made per period of the highest frequency in
the system, and Shannon’s sampling
theorem
states that two samples per period of the highest frequency in a
band-limited signal contain all available information. But that still
gives very long files! So, if the highest frequencies are not of
interest, 10 or 20 samples per ps may suffice. Be aware of the
distortion of high-frequency motions by the *stroboscopic effect*,
called *aliasing*: higher frequencies are mirrored with respect to the
sampling frequency and appear as lower frequencies.

|Gromacs| can also write reduced-precision coordinates for a subset of the
simulation system to a special compressed trajectory file format. All
the other tools can read and write this format. See the User Guide for
details on how to set up your :ref:`mdp` file to have
:ref:`mdrun <gmx mdrun>` use this feature.

Shell molecular dynamics
------------------------

|Gromacs| can simulate polarizability
using the shell model
of Dick and
Overhauser Dick and Overhauser (1958). In such models a shell particle
representing the electronic degrees of freedom is attached to a nucleus
by a spring. The potential energy is minimized with respect to the shell
position at every step of the simulation (see below). Successful
applications of shell models in |Gromacs| have been published for
:math:`N_2` Jordan et al. (1995) and water Maaren and Spoel (2001).

Optimization of the shell positions
~~~~~~~~~~~~~~~~~~~~~~~~~~~~~~~~~~~

The force 
:math:`{\mbox{\boldmath ${F}$}}`\ :math:`_S` on a shell
particle :math:`S` can be decomposed into two components

.. math:: {\mbox{\boldmath ${F}$}}_S ~=~ {\mbox{\boldmath ${F}$}}_{bond} + {\mbox{\boldmath ${F}$}}_{nb}

where 
:math:`{\mbox{\boldmath ${F}$}}`\ :math:`_{bond}` denotes the
component representing the polarization energy, usually represented by a
harmonic potential and
:math:`{\mbox{\boldmath ${F}$}}`\ :math:`_{nb}` is the sum of Coulomb
and van der Waals interactions. If we assume that
:math:`{\mbox{\boldmath ${F}$}}`\ :math:`_{nb}` is almost constant we
can analytically derive the optimal position of the shell, i.e. where
:math:`{\mbox{\boldmath ${F}$}}`\ :math:`_S` = 0. If we have the
shell S connected to atom A we have

.. math:: {\mbox{\boldmath ${F}$}}_{bond} ~=~ k_b \left( {\mbox{\boldmath ${x}$}}_S - {\mbox{\boldmath ${x}$}}_A\right).

In an iterative solver, we have positions
:math:`{\mbox{\boldmath ${x}$}}`\ :math:`_S(n)` where :math:`n` is
the iteration count. We now have at iteration :math:`n`

.. math:: {\mbox{\boldmath ${F}$}}_{nb} ~=~ {\mbox{\boldmath ${F}$}}_S - k_b \left( {\mbox{\boldmath ${x}$}}_S(n) - {\mbox{\boldmath ${x}$}}_A\right)

and the optimal position for the shells :math:`x_S(n+1)` thus follows
from

.. math:: {\mbox{\boldmath ${F}$}}_S - k_b \left( {\mbox{\boldmath ${x}$}}_S(n) - {\mbox{\boldmath ${x}$}}_A\right) + k_b \left( {\mbox{\boldmath ${x}$}}_S(n+1) - {\mbox{\boldmath ${x}$}}_A\right) = 0

if we write

.. math:: \Delta {\mbox{\boldmath ${x}$}}_S = {\mbox{\boldmath ${x}$}}_S(n+1) - {\mbox{\boldmath ${x}$}}_S(n)

we finally obtain

.. math:: \Delta {\mbox{\boldmath ${x}$}}_S = {\mbox{\boldmath ${F}$}}_S/k_b

which then yields the algorithm to compute the next trial in the
optimization of shell positions

.. math:: {\mbox{\boldmath ${x}$}}_S(n+1) ~=~ {\mbox{\boldmath ${x}$}}_S(n) + {\mbox{\boldmath ${F}$}}_S/k_b.

Constraint algorithms
---------------------

Constraints can be imposed in |Gromacs| using LINCS (default) or the
traditional SHAKE method.

SHAKE
~~~~~

The SHAKE Ryckaert, Ciccotti, and Berendsen (1977) algorithm changes a
set of unconstrained coordinates :math:`{\mbox{\boldmath ${r}$}}^{'}` to
a set of coordinates :math:`{\mbox{\boldmath ${r}$}}''` that fulfill a
list of distance constraints, using a set
:math:`{\mbox{\boldmath ${r}$}}` reference, as

.. math:: {\rm SHAKE}({\mbox{\boldmath ${r}$}}^{'} \rightarrow {\mbox{\boldmath ${r}$}}'';\, {\mbox{\boldmath ${r}$}})

This action is consistent with solving a set of Lagrange multipliers in
the constrained equations of motion. SHAKE needs a *relative tolerance*;
it will continue until all constraints are satisfied within that
relative tolerance. An error message is given if SHAKE cannot reset the
coordinates because the deviation is too large, or if a given number of
iterations is surpassed.

Assume the equations of motion must fulfill :math:`K` holonomic
constraints, expressed as

.. math:: \sigma_k({\mbox{\boldmath ${r}$}}_1 \ldots {\mbox{\boldmath ${r}$}}_N) = 0; \;\; k=1 \ldots K.

For example,
:math:`({\mbox{\boldmath ${r}$}}_1 - {\mbox{\boldmath ${r}$}}_2)^2 - b^2 = 0`.
Then the forces are defined as

.. math::

   - \frac{\partial}{\partial {\mbox{\boldmath ${r}$}}_i} \left( V + \sum_{k=1}^K \lambda_k
   \sigma_k \right),

where :math:`\lambda_k` are Lagrange multipliers which must be solved
to fulfill the constraint equations. The second part of this sum
determines the *constraint forces* :math:`{\mbox{\boldmath ${G}$}}_i`,
defined by

.. math::

   {\mbox{\boldmath ${G}$}}_i = -\sum_{k=1}^K \lambda_k \frac{\partial \sigma_k}{\partial
   {\mbox{\boldmath ${r}$}}_i}

The displacement due to the constraint forces in the leap-frog or
Verlet algorithm is equal to
:math:`({\mbox{\boldmath ${G}$}}_i/m_i)({{\Delta t}})^2`. Solving the
Lagrange multipliers (and hence the displacements) requires the solution
of a set of coupled equations of the second degree. These are solved
iteratively by SHAKE.
For the special
case of rigid water molecules, that often make up more than 80% of the
simulation system we have implemented the
SETTLE
algorithm Miyamoto and Kollman (1992)
(sec. 
).

For velocity Verlet, an additional round of constraining must be done,
to constrain the velocities of the second velocity half step, removing
any component of the velocity parallel to the bond vector. This step is
called RATTLE, and is covered in more detail in the original Andersen
paper Andersen (1983).

LINCS
~~~~~

The LINCS algorithm
^^^^^^^^^^^^^^^^^^^

LINCS is an algorithm that resets bonds to their correct lengths after
an unconstrained update Hess et al. (1997). The method is non-iterative,
as it always uses two steps. Although LINCS is based on matrices, no
matrix-matrix multiplications are needed. The method is more stable and
faster than SHAKE, but it can only be used with bond constraints and
isolated angle constraints, such as the proton angle in OH. Because of
its stability, LINCS is especially useful for Brownian dynamics. LINCS
has two parameters, which are explained in the subsection parameters.
The parallel version of LINCS, P-LINCS, is described in subsection

The LINCS formulas
^^^^^^^^^^^^^^^^^^

We consider a system of :math:`N` particles, with positions given by a
:math:`3N` vector :math:`{\mbox{\boldmath ${r}$}}(t)`. For molecular
dynamics the equations of motion are given by Newton’s Law

.. math:: {{\mbox{d}}^2 {\mbox{\boldmath ${r}$}} \over {\mbox{d}}t^2} = {{{\mbox{\boldmath ${M}$}}}^{-1}}{\mbox{\boldmath ${F}$}},
          :label: eqnc1

where :math:`{\mbox{\boldmath ${F}$}}` is the :math:`3N` force vector
and :math:`{{\mbox{\boldmath ${M}$}}}` is a :math:`3N \times 3N`
diagonal matrix, containing the masses of the particles. The system is
constrained by :math:`K` time-independent constraint equations

.. math:: g_i({\mbox{\boldmath ${r}$}}) = | {\mbox{\boldmath ${r}$}}_{i_1}-{\mbox{\boldmath ${r}$}}_{i_2} | - d_i = 0 ~~~~~~i=1,\ldots,K.
          :label: eqnc2

In a numerical integration scheme, LINCS is applied after an
unconstrained update, just like SHAKE. The algorithm works in two steps
(see figure :numref:`Fig. (%s) <fig-lincs>`). In the first step, the
projections of the new bonds on the old bonds are set to zero. In the
second step, a correction is applied for the lengthening of the bonds
due to rotation. The numerics for the first step and the second step are
very similar. A complete derivation of the algorithm can be found in
Hess et al. (1997). Only a short description of the first step is given
here.

.. _fig-lincs:

.. figure:: plots/lincs.eps

   The three position updates needed for one time step. The dashed
   line is the old bond of length :math:`d`, the solid lines are the new
   bonds. :math:`l=d \cos \theta` and :math:`p=(2 d^2 - l^2)^{1 \over 2}`.

A new notation is introduced for the gradient matrix of the constraint
equations which appears on the right hand side of this equation:

.. math::  B_{hi} = {{\partial}g_h \over {\partial}r_i}
           :label: eqnc3

Notice that :math:`{{\mbox{\boldmath ${B}$}}}` is a :math:`K \times 3N`
matrix, it contains the directions of the constraints. The following
equation shows how the new constrained coordinates
:math:`{\mbox{\boldmath ${r}$}}_{n+1}` are related to the unconstrained
coordinates :math:`{\mbox{\boldmath ${r}$}}_{n+1}^{unc}` by

.. math::  \begin{array}{c}
           {\mbox{\boldmath ${r}$}}_{n+1}=({\mbox{\boldmath ${I}$}}-{{\mbox{\boldmath ${T}$}}}_n {\mbox{\boldmath ${B}$}}_n) {\mbox{\boldmath ${r}$}}_{n+1}^{unc} + {{\mbox{\boldmath ${T}$}}}_n {{\mbox{\boldmath ${d}$}}}=  
           \\[2mm]
           {\mbox{\boldmath ${r}$}}_{n+1}^{unc} - 
           {{{\mbox{\boldmath ${M}$}}}^{-1}}{{\mbox{\boldmath ${B}$}}}_n ({{\mbox{\boldmath ${B}$}}}_n {{{\mbox{\boldmath ${M}$}}}^{-1}}{{\mbox{\boldmath ${B}$}}}_n^T)^{-1} ({{\mbox{\boldmath ${B}$}}}_n {\mbox{\boldmath ${r}$}}_{n+1}^{unc} - {{\mbox{\boldmath ${d}$}}}) 
           \end{array}
           :label: eqnm0

where
:math:`{{\mbox{\boldmath ${T}$}}}= {{{\mbox{\boldmath ${M}$}}}^{-1}}{{\mbox{\boldmath ${B}$}}}^T ({{\mbox{\boldmath ${B}$}}}{{{\mbox{\boldmath ${M}$}}}^{-1}}{{\mbox{\boldmath ${B}$}}}^T)^{-1}`.
The derivation of this equation from :eq:`eqns. (%s) <eqnc1>` and
:eq:`(%s) <eqnc2>` can be found in Hess et al. (1997).

This first step does not set the real bond lengths to the prescribed
lengths, but the projection of the new bonds onto the old directions of
the bonds. To correct for the rotation of bond :math:`i`, the projection
of the bond, :math:`p_i`, on the old direction is set to

.. math::  p_i=\sqrt{2 d_i^2 - l_i^2},
           :label: eqnm1a

where :math:`l_i` is the bond length after the first projection. The
corrected positions are

.. math::  {\mbox{\boldmath ${r}$}}_{n+1}^*=({\mbox{\boldmath ${I}$}}-{{\mbox{\boldmath ${T}$}}}_n {{\mbox{\boldmath ${B}$}}}_n){\mbox{\boldmath ${r}$}}_{n+1} + {{\mbox{\boldmath ${T}$}}}_n {\mbox{\boldmath ${p}$}}.
           :label: eqnm1b

This correction for rotational effects is actually an iterative
process, but during MD only one iteration is applied. The relative
constraint deviation after this procedure will be less than 0.0001 for
every constraint. In energy minimization, this might not be accurate
enough, so the number of iterations is equal to the order of the
expansion (see below).

Half of the CPU time goes to inverting the constraint coupling matrix
:math:`{{\mbox{\boldmath ${B}$}}}_n {{{\mbox{\boldmath ${M}$}}}^{-1}}{{\mbox{\boldmath ${B}$}}}_n^T`,
which has to be done every time step. This :math:`K \times K` matrix has
:math:`1/m_{i_1} + 1/m_{i_2}` on the diagonal. The off-diagonal elements
are only non-zero when two bonds are connected, then the element is
:math:`\cos \phi /m_c`, where :math:`m_c` is the mass of the atom
connecting the two bonds and :math:`\phi` is the angle between the
bonds.

The matrix :math:`{\mbox{\boldmath ${T}$}}` is inverted through a power
expansion. A :math:`K \times K` matrix :math:`{\mbox{\boldmath ${S}$}}`
is introduced which is the inverse square root of the diagonal of
:math:`{\mbox{\boldmath ${B}$}}_n {{{\mbox{\boldmath ${M}$}}}^{-1}}{{\mbox{\boldmath ${B}$}}}_n^T`.
This matrix is used to convert the diagonal elements of the coupling
matrix to one:

.. math:: \begin{array}{c}
          ({{\mbox{\boldmath ${B}$}}}_n {{{\mbox{\boldmath ${M}$}}}^{-1}}{{\mbox{\boldmath ${B}$}}}_n^T)^{-1}
          = {{\mbox{\boldmath ${S}$}}}{{\mbox{\boldmath ${S}$}}}^{-1} ({{\mbox{\boldmath ${B}$}}}_n {{{\mbox{\boldmath ${M}$}}}^{-1}}{{\mbox{\boldmath ${B}$}}}_n^T)^{-1} {{\mbox{\boldmath ${S}$}}}^{-1} {{\mbox{\boldmath ${S}$}}}\\[2mm]
          = {{\mbox{\boldmath ${S}$}}}({{\mbox{\boldmath ${S}$}}}{{\mbox{\boldmath ${B}$}}}_n {{{\mbox{\boldmath ${M}$}}}^{-1}}{{\mbox{\boldmath ${B}$}}}_n^T {{\mbox{\boldmath ${S}$}}})^{-1} {{\mbox{\boldmath ${S}$}}}=
          {{\mbox{\boldmath ${S}$}}}({\mbox{\boldmath ${I}$}} - {\mbox{\boldmath ${A}$}}_n)^{-1} {{\mbox{\boldmath ${S}$}}}\end{array}
          :label: eqnm2

The matrix :math:`{\mbox{\boldmath ${A}$}}_n` is symmetric and sparse
and has zeros on the diagonal. Thus a simple trick can be used to
calculate the inverse:

.. math:: ({\mbox{\boldmath ${I}$}}-{\mbox{\boldmath ${A}$}}_n)^{-1}= 
          {\mbox{\boldmath ${I}$}} + {\mbox{\boldmath ${A}$}}_n + {\mbox{\boldmath ${A}$}}_n^2 + {\mbox{\boldmath ${A}$}}_n^3 + \ldots
          :label: eqnm3

This inversion method is only valid if the absolute values of all the
eigenvalues of :math:`{\mbox{\boldmath ${A}$}}_n` are smaller than one.
In molecules with only bond constraints, the connectivity is so low that
this will always be true, even if ring structures are present. Problems
can arise in angle-constrained molecules. By constraining angles with
additional distance constraints, multiple small ring structures are
introduced. This gives a high connectivity, leading to large
eigenvalues. Therefore LINCS should NOT be used with coupled
angle-constraints.

For molecules with all bonds constrained the eigenvalues of :math:`A`
are around 0.4. This means that with each additional order in the
expansion :eq:`eqn. (%s) <eqnm3>` the deviations decrease by a
factor 0.4. But for relatively isolated triangles of constraints the
largest eigenvalue is around 0.7. Such triangles can occur when removing
hydrogen angle vibrations with an additional angle constraint in alcohol
groups or when constraining water molecules with LINCS, for instance
with flexible constraints. The constraints in such triangles converge
twice as slow as the other constraints. Therefore, starting with |Gromacs|
4, additional terms are added to the expansion for such triangles

.. math:: ({\mbox{\boldmath ${I}$}}-{\mbox{\boldmath ${A}$}}_n)^{-1} \approx
          {\mbox{\boldmath ${I}$}} + {\mbox{\boldmath ${A}$}}_n + \ldots + {\mbox{\boldmath ${A}$}}_n^{N_i} +
          \left({\mbox{\boldmath ${A}$}}^*_n + \ldots + {{\mbox{\boldmath ${A}$}}_n^*}^{N_i} \right) {\mbox{\boldmath ${A}$}}_n^{N_i}
          :label: eqnm3ang

where :math:`N_i` is the normal order of the expansion and
:math:`{\mbox{\boldmath ${A}$}}^*` only contains the elements of
:math:`{\mbox{\boldmath ${A}$}}` that couple constraints within rigid
triangles, all other elements are zero. In this manner, the accuracy of
angle constraints comes close to that of the other constraints, while
the series of matrix vector multiplications required for determining the
expansion only needs to be extended for a few constraint couplings. This
procedure is described in the P-LINCS paperHess (2007).

The LINCS Parameters
^^^^^^^^^^^^^^^^^^^^

The accuracy of LINCS depends on the number of matrices used in the
expansion :eq:`eqn. (%s) <eqnm3>`. For MD calculations a fourth
order expansion is enough. For Brownian dynamics with large time steps
an eighth order expansion may be necessary. The order is a parameter in
the :ref:`mdp` file. The implementation of LINCS is done in
such a way that the algorithm will never crash. Even when it is
impossible to to reset the constraints LINCS will generate a
conformation which fulfills the constraints as well as possible.
However, LINCS will generate a warning when in one step a bond rotates
over more than a predefined angle. This angle is set by the user in the
:ref:`mdp` file.

Simulated Annealing
-------------------

The well known simulated
annealing
(SA) protocol is supported in |Gromacs|, and you can even couple multiple
groups of atoms separately with an arbitrary number of reference
temperatures that change during the simulation. The annealing is
implemented by simply changing the current reference temperature for
each group in the temperature coupling, so the actual relaxation and
coupling properties depends on the type of thermostat you use and how
hard you are coupling it. Since we are changing the reference
temperature it is important to remember that the system will NOT
instantaneously reach this value - you need to allow for the inherent
relaxation time in the coupling algorithm too. If you are changing the
annealing reference temperature faster than the temperature relaxation
you will probably end up with a crash when the difference becomes too
large.

The annealing protocol is specified as a series of corresponding times
and reference temperatures for each group, and you can also choose
whether you only want a single sequence (after which the temperature
will be coupled to the last reference value), or if the annealing should
be periodic and restart at the first reference point once the sequence
is completed. You can mix and match both types of annealing and
non-annealed groups in your simulation.

Stochastic Dynamics
-------------------

Stochastic or velocity Langevin
dynamics
adds a friction and a noise term to Newton’s equations of motion, as

.. math::  m_i {{\mbox{d}}^2 {\mbox{\boldmath ${r}$}}_i \over {\mbox{d}}t^2} =
           - m_i \gamma_i {{\mbox{d}}{\mbox{\boldmath ${r}$}}_i \over {\mbox{d}}t} + {\mbox{\boldmath ${F}$}}_i({\mbox{\boldmath ${r}$}}) + {\stackrel{\circ}{{\mbox{\boldmath ${r}$}}}}_i
           :label: eqnSDeq

where :math:`\gamma_i` is the friction constant :math:`[1/\mbox{ps}]`
and :math:`{\stackrel{\circ}{{\mbox{\boldmath ${r}$}}}}_i\!\!(t)` is a
noise process with
:math:`\langle {\stackrel{\circ}{r}}_i\!\!(t) {\stackrel{\circ}{r}}_j\!\!(t+s) \rangle = 2 m_i \gamma_i k_B T \delta(s) \delta_{ij}`. When :math:`1/\gamma_i`
is large compared to the time scales present in the system, one could
see stochastic dynamics as molecular dynamics with stochastic
temperature-coupling. The advantage compared to MD with Berendsen
temperature-coupling is that in case of SD the generated ensemble is
known. For simulating a system in vacuum there is the additional
advantage that there is no accumulation of errors for the overall
translational and rotational degrees of freedom. When :math:`1/\gamma_i`
is small compared to the time scales present in the system, the dynamics
will be completely different from MD, but the sampling is still correct.

In |Gromacs| there is one simple and efficient implementation. Its
accuracy is equivalent to the normal MD leap-frog and Velocity Verlet
integrator. It is nearly identical to the common way of discretizing the
Langevin equation, but the friction and velocity term are applied in an
impulse fashion Goga et al. (2012). It can be described as:

.. math::  \begin{aligned}
           {\mbox{\boldmath ${v}$}}'  &~=~&   {\mbox{\boldmath ${v}$}}(t-{{\frac{1}{2}}{{\Delta t}}}) + \frac{1}{m}{\mbox{\boldmath ${F}$}}(t){{\Delta t}}\\
           \Delta{\mbox{\boldmath ${v}$}}     &~=~&   -\alpha \, {\mbox{\boldmath ${v}$}}'(t+{{\frac{1}{2}}{{\Delta t}}}) + \sqrt{\frac{k_B T}{m}(1 - \alpha^2)} \, {{\mbox{\boldmath ${r}$}}^G}_i \\
           {\mbox{\boldmath ${r}$}}(t+{{\Delta t}})   &~=~&   {\mbox{\boldmath ${r}$}}(t)+\left({\mbox{\boldmath ${v}$}}' +\frac{1}{2}\Delta {\mbox{\boldmath ${v}$}}\right){{\Delta t}}\\
           {\mbox{\boldmath ${v}$}}(t+{{\frac{1}{2}}{{\Delta t}}})  &~=~&   {\mbox{\boldmath ${v}$}}' + \Delta {\mbox{\boldmath ${v}$}} \\
           \alpha &~=~& 1 - e^{-\gamma {{\Delta t}}}\end{aligned}
           :label: eqnsd1xupd

where :math:`{{\mbox{\boldmath ${r}$}}^G}_i` is Gaussian distributed
noise with :math:`\mu = 0`, :math:`\sigma = 1`. The velocity is first
updated a full time step without friction and noise to get
:math:`{\mbox{\boldmath ${v}$}}'`, identical to the normal update in
leap-frog. The friction and noise are then applied as an impulse at step
:math:`t+{{\Delta t}}`. The advantage of this scheme is that the
velocity-dependent terms act at the full time step, which makes the
correct integration of forces that depend on both coordinates and
velocities, such as constraints and dissipative particle dynamics (DPD,
not implented yet), straightforward. With constraints, the coordinate
update :eq:`eqn. (%s) <eqnsd1xupd>` is split into a normal
leap-frog update and a :math:`\Delta {\mbox{\boldmath ${v}$}}`. After
both of these updates the constraints are applied to coordinates and
velocities.

When using SD as a thermostat, an appropriate value for :math:`\gamma`
is e.g. 0.5 ps\ :math:`^{-1}`, since this results in a friction that is
lower than the internal friction of water, while it still provides
efficient thermostatting.

Brownian Dynamics
-----------------

In the limit of high friction, stochastic dynamics reduces to Brownian
dynamics, also called position Langevin dynamics. This applies to
over-damped systems, *i.e.* systems in which the inertia effects are
negligible. The equation is

.. math:: {{\mbox{d}}{\mbox{\boldmath ${r}$}}_i \over {\mbox{d}}t} = \frac{1}{\gamma_i} {\mbox{\boldmath ${F}$}}_i({\mbox{\boldmath ${r}$}}) + {\stackrel{\circ}{{\mbox{\boldmath ${r}$}}}}_i

where :math:`\gamma_i` is the friction coefficient
:math:`[\mbox{amu/ps}]` and
:math:`{\stackrel{\circ}{{\mbox{\boldmath ${r}$}}}}_i\!\!(t)` is a noise
process with
:math:`\langle {\stackrel{\circ}{r}}_i\!\!(t) {\stackrel{\circ}{r}}_j\!\!(t+s) \rangle = 2 \delta(s) \delta_{ij} k_B T / \gamma_i`. 
In |Gromacs| the equations
are integrated with a simple, explicit scheme

.. math::

   {\mbox{\boldmath ${r}$}}_i(t+\Delta t) = {\mbox{\boldmath ${r}$}}_i(t) +
           {\Delta t \over \gamma_i} {\mbox{\boldmath ${F}$}}_i({\mbox{\boldmath ${r}$}}(t)) 
           + \sqrt{2 k_B T {\Delta t \over \gamma_i}}\, {{\mbox{\boldmath ${r}$}}^G}_i,

where :math:`{{\mbox{\boldmath ${r}$}}^G}_i` is Gaussian distributed
noise with :math:`\mu = 0`, :math:`\sigma = 1`. The friction
coefficients :math:`\gamma_i` can be chosen the same for all particles
or as :math:`\gamma_i = m_i\,\gamma_i`, where the friction constants
:math:`\gamma_i` can be different for different groups of atoms. Because
the system is assumed to be over-damped, large timesteps can be used.
LINCS should be used for the constraints since SHAKE will not converge
for large atomic displacements. BD is an option of the
:ref:`mdrun <gmx mdrun>` program.

Energy Minimization
-------------------

Energy minimization in |Gromacs|
can be done using steepest descent, conjugate gradients, or l-bfgs
(limited-memory Broyden-Fletcher-Goldfarb-Shanno quasi-Newtonian
minimizer...we prefer the abbreviation). EM is just an option of the
:ref:`mdrun <gmx mdrun>` program.

Steepest Descent
~~~~~~~~~~~~~~~~

Although steepest descent is certainly not the most efficient algorithm
for searching, it is robust and easy to implement.

We define the vector :math:`{\mbox{\boldmath ${r}$}}` as the vector of
all :math:`3N` coordinates. Initially a maximum displacement :math:`h_0`
(*e.g.* 0.01 nm) must be given.

First the forces :math:`{\mbox{\boldmath ${F}$}}` and potential energy
are calculated. New positions are calculated by

  .. math:: {\mbox{\boldmath ${r}$}}_{n+1} =  {\mbox{\boldmath ${r}$}}_n + \frac{{\mbox{\boldmath ${F}$}}_n}{\max (|{\mbox{\boldmath ${F}$}}_n|)} h_n,

where :math:`h_n` is the maximum displacement and
:math:`{\mbox{\boldmath ${F}$}}_n` is the force, or the negative
gradient of the potential :math:`V`. The notation :math:`\max
(|{\mbox{\boldmath ${F}$}}_n|)` means the largest of the absolute
values of the force components. The forces and energy are again
computed for the new positions

| If (:math:`V_{n+1} < V_n`) the new positions are accepted and
  :math:`h_{n+1} = 1.2
  h_n`.
| If (:math:`V_{n+1} \geq V_n`) the new positions are rejected and
  :math:`h_n = 0.2 h_n`.

The algorithm stops when either a user-specified number of force
evaluations has been performed (*e.g.* 100), or when the maximum of the
absolute values of the force (gradient) components is smaller than a
specified value :math:`\epsilon`. Since force truncation produces some
noise in the energy evaluation, the stopping criterion should not be
made too tight to avoid endless iterations. A reasonable value for
:math:`\epsilon` can be estimated from the root mean square force
:math:`f` a harmonic oscillator would exhibit at a temperature
:math:`T`. This value is

.. math:: f = 2 \pi \nu \sqrt{ 2mkT},

where :math:`\nu` is the oscillator frequency, :math:`m` the (reduced)
mass, and :math:`k` Boltzmann’s constant. For a weak oscillator with a
wave number of 100 cm\ :math:`^{-1}` and a mass of 10 atomic units, at a
temperature of 1 K, :math:`f=7.7` kJ mol\ :math:`^{-1}` nm:math:`^{-1}`.
A value for :math:`\epsilon` between 1 and 10 is acceptable.

Conjugate Gradient
~~~~~~~~~~~~~~~~~~

Conjugate gradient is slower than steepest descent in the early stages
of the minimization, but becomes more efficient closer to the energy
minimum. The parameters and stop criterion are the same as for steepest
descent. In |Gromacs| conjugate gradient can not be used with constraints,
including the SETTLE algorithm for water Miyamoto and Kollman (1992), as
this has not been implemented. If water is present it must be of a
flexible model, which can be specified in the :ref:`mdp` file
by ``define = -DFLEXIBLE``.

This is not really a restriction, since the accuracy of conjugate
gradient is only required for minimization prior to a normal-mode
analysis, which cannot be performed with constraints. For most other
purposes steepest descent is efficient enough.

L-BFGS
~~~~~~

The original BFGS algorithm works by successively creating better
approximations of the inverse Hessian matrix, and moving the system to
the currently estimated minimum. The memory requirements for this are
proportional to the square of the number of particles, so it is not
practical for large systems like biomolecules. Instead, we use the
L-BFGS algorithm of Nocedal Byrd, Lu, and Nocedal (1995; Zhu, Byrd, and
Nocedal 1997), which approximates the inverse Hessian by a fixed number
of corrections from previous steps. This sliding-window technique is
almost as efficient as the original method, but the memory requirements
are much lower - proportional to the number of particles multiplied with
the correction steps. In practice we have found it to converge faster
than conjugate gradients, but due to the correction steps it is not yet
parallelized. It is also noteworthy that switched or shifted
interactions usually improve the convergence, since sharp cut-offs mean
the potential function at the current coordinates is slightly different
from the previous steps used to build the inverse Hessian approximation.

Normal-Mode Analysis
--------------------

Normal-mode analysis Levitt, Sander, and Stern (1983;
G\ :math:`\bar{\rm o}`, Noguti, and Nishikawa 1983; Brooks and Karplus
1983) can be performed using |Gromacs|, by diagonalization of the
mass-weighted Hessian
:math:`H`:

.. math::

   \begin{aligned}
   R^T M^{-1/2} H M^{-1/2} R   &=& \mbox{diag}(\lambda_1,\ldots,\lambda_{3N})
   \\
   \lambda_i &=& (2 \pi \omega_i)^2\end{aligned}

where :math:`M` contains the atomic masses, :math:`R` is a matrix that
contains the eigenvectors as columns, :math:`\lambda_i` are the
eigenvalues and :math:`\omega_i` are the corresponding frequencies.

First the Hessian matrix, which is a :math:`3N \times 3N` matrix where
:math:`N` is the number of atoms, needs to be calculated:

.. math::

   \begin{aligned}
   H_{ij}  &=&     \frac{\partial^2 V}{\partial x_i \partial x_j}\end{aligned}

where :math:`x_i` and :math:`x_j` denote the atomic x, y or z
coordinates. In practice, this equation is not used, but the Hessian is
calculated numerically from the force as:

.. math::

   \begin{aligned}
   H_{ij} &=& -
     \frac{f_i({\bf x}+h{\bf e}_j) - f_i({\bf x}-h{\bf e}_j)}{2h}
   \\
   f_i     &=& - \frac{\partial V}{\partial x_i}\end{aligned}

where :math:`{\bf e}_j` is the unit vector in direction :math:`j`. It
should be noted that for a usual normal-mode calculation, it is
necessary to completely minimize the energy prior to computation of the
Hessian. The tolerance required depends on the type of system, but a
rough indication is 0.001 kJ mol\ :math:`^{-1}`. Minimization should be
done with conjugate gradients or L-BFGS in double precision.

A number of |Gromacs| programs are involved in these calculations. First,
the energy should be minimized using :ref:`mdrun <gmx mdrun>`. Then,
:ref:`mdrun <gmx mdrun>` computes the Hessian. **Note** that for generating
the run input file, one should use the minimized conformation from the
full precision trajectory file, as the structure file is not accurate
enough. :ref:`gmx nmeig` does the
diagonalization and the sorting of the normal modes according to their
frequencies. Both :ref:`mdrun <gmx mdrun>` and :ref:`gmx nmeig`
should be run in double precision. The normal modes can be analyzed with
the program :ref:`gmx anaeig`. Ensembles of structures at any
temperature and for any subset of normal modes can be generated with
:ref:`gmx nmens`. An overview of
normal-mode analysis and the related principal component analysis (see
sec. 
) can be found in Hayward and
G\ :math:`\bar{\rm o}` (1995).

Free energy calculations
------------------------

Slow-growth methods
~~~~~~~~~~~~~~~~~~~

Free energy calculations can be performed in |Gromacs| using a number of
methods, including “slow-growth.” An example problem might be
calculating the difference in free energy of binding of an inhibitor
**I** to an enzyme **E** and to a mutated enzyme
**E**\ :math:`^{\prime}`. It is not feasible with computer simulations
to perform a docking calculation for such a large complex, or even
releasing the inhibitor from the enzyme in a reasonable amount of
computer time with reasonable accuracy. However, if we consider the free
energy cycle in :numref:`Fig. (%s) A<fig-free1>` we can write:

.. math:: \Delta G_1 - \Delta G_2 =       \Delta G_3 - \Delta G_4
          :label: eqnddg

If we are interested in the left-hand term we can equally well compute
the right-hand term.

.. _fig-free1:

.. figure:: plots/free1.eps

            Free energy cycles. **A:** to calculate :math:`\Delta G_{12}`, the free
            energy difference between the binding of inhibitor **I** to enzymes
            **E** respectively **E**\ :math:`^{\prime}`. 

.. _fig-free2:

.. figure:: plots/free2.eps

            Free energy cycles. **B:** to calculate
            :math:`\Delta G_{12}`, the free energy difference for binding of
            inhibitors **I** respectively **I**\ :math:`^{\prime}` to enzyme
            **E**.

If we want to compute the difference in free energy of binding of two
inhibitors **I** and **I**\ :math:`^{\prime}` to an enzyme **E**
(:numref:`Fig. (%s) <fig-free2>`) we can again use
:eq:`eqn. (%s) eqnddg>` to compute the desired property.

Free energy differences between two molecular species can be calculated
in |Gromacs| using the “slow-growth” method. Such free energy differences
between different molecular species are physically meaningless, but they
can be used to obtain meaningful quantities employing a thermodynamic
cycle. The method requires a simulation during which the Hamiltonian of
the system changes slowly from that describing one system (A) to that
describing the other system (B). The change must be so slow that the
system remains in equilibrium during the process; if that requirement is
fulfilled, the change is reversible and a slow-growth simulation from B
to A will yield the same results (but with a different sign) as a
slow-growth simulation from A to B. This is a useful check, but the user
should be aware of the danger that equality of forward and backward
growth results does not guarantee correctness of the results.

The required modification of the Hamiltonian :math:`H` is realized by
making :math:`H` a function of a *coupling parameter* :math:`\lambda:
H=H(p,q;\lambda)` in such a way that :math:`\lambda=0` describes system
A and :math:`\lambda=1` describes system B:

.. math:: H(p,q;0)=H{^{\mathrm{A}}}(p,q);~~~~ H(p,q;1)=H{^{\mathrm{B}}}(p,q).

In |Gromacs|, the functional form of the :math:`\lambda`-dependence is
different for the various force-field contributions and is described in
section sec. 
The Helmholtz free energy :math:`A` is related to the partition function
:math:`Q` of an :math:`N,V,T` ensemble, which is assumed to be the
equilibrium ensemble generated by a MD simulation at constant volume and
temperature. The generally more useful Gibbs free energy :math:`G` is
related to the partition function :math:`\Delta` of an :math:`N,p,T`
ensemble, which is assumed to be the equilibrium ensemble generated by a
MD simulation at constant pressure and temperature:

.. math::

   \begin{aligned}
    A(\lambda) &=&  -k_BT \ln Q \\
    Q &=& c \int\!\!\int \exp[-\beta H(p,q;\lambda)]\,dp\,dq \\
    G(\lambda) &=&  -k_BT \ln \Delta \\
    \Delta &=& c \int\!\!\int\!\!\int \exp[-\beta H(p,q;\lambda) -\beta
   pV]\,dp\,dq\,dV \\
   G &=& A + pV, \end{aligned}

where :math:`\beta = 1/(k_BT)` and :math:`c = (N! h^{3N})^{-1}`. These
integrals over phase space cannot be evaluated from a simulation, but it
is possible to evaluate the derivative with respect to :math:`\lambda`
as an ensemble average:

.. math::

   \frac{dA}{d\lambda} =  \frac{\int\!\!\int (\partial H/ \partial
   \lambda) \exp[-\beta H(p,q;\lambda)]\,dp\,dq}{\int\!\!\int \exp[-\beta
   H(p,q;\lambda)]\,dp\,dq} = 
   \left\langle \frac{\partial H}{\partial \lambda} \right\rangle_{NVT;\lambda},

with a similar relation for :math:`dG/d\lambda` in the :math:`N,p,T`
ensemble. The difference in free energy between A and B can be found by
integrating the derivative over :math:`\lambda`:

.. math::  \begin{aligned}
           A{^{\mathrm{B}}}(V,T)-A{^{\mathrm{A}}}(V,T) &=& \int_0^1 \left\langle \frac{\partial
           H}{\partial \lambda} \right\rangle_{NVT;\lambda} \,d\lambda 
           \end{aligned}
           :label: eqdelA

.. math:: \begin{aligned}
          G{^{\mathrm{B}}}(p,T)-G{^{\mathrm{A}}}(p,T) &=& \int_0^1 \left\langle \frac{\partial
          H}{\partial \lambda} \right\rangle_{NpT;\lambda} \,d\lambda.
          \end{aligned}
          :label: eqdelG

If one wishes to evaluate
:math:`G{^{\mathrm{B}}}(p,T)-G{^{\mathrm{A}}}(p,T)`, the natural choice
is a constant-pressure simulation. However, this quantity can also be
obtained from a slow-growth simulation at constant volume, starting with
system A at pressure :math:`p` and volume :math:`V` and ending with
system B at pressure :math:`p_B`, by applying the following small (but,
in principle, exact) correction:

.. math::

   G{^{\mathrm{B}}}(p)-G{^{\mathrm{A}}}(p) =
   A{^{\mathrm{B}}}(V)-A{^{\mathrm{A}}}(V) - \int_p^{p{^{\mathrm{B}}}}[V{^{\mathrm{B}}}(p')-V]\,dp'

Here we omitted the constant :math:`T` from the notation. This
correction is roughly equal to
:math:`-\frac{1}{2} (p{^{\mathrm{B}}}-p)\Delta V=(\Delta V)^2/(2
\kappa V)`, where :math:`\Delta V` is the volume change at :math:`p` and
:math:`\kappa` is the isothermal compressibility. This is usually small;
for example, the growth of a water molecule from nothing in a bath of
1000 water molecules at constant volume would produce an additional
pressure of as much as 22 bar, but a correction to the Helmholtz free
energy of just -1 kJ mol\ :math:`^{-1}`. In Cartesian coordinates, the
kinetic energy term in the Hamiltonian depends only on the momenta, and
can be separately integrated and, in fact, removed from the equations.
When masses do not change, there is no contribution from the kinetic
energy at all; otherwise the integrated contribution to the free energy
is :math:`-\frac{3}{2} k_BT \ln
(m{^{\mathrm{B}}}/m{^{\mathrm{A}}})`. **Note** that this is only true in
the absence of constraints.

Thermodynamic integration
~~~~~~~~~~~~~~~~~~~~~~~~~

|Gromacs| offers the possibility to integrate
eq. 
or eq. 
in one
simulation over the full range from A to B. However, if the change is
large and insufficient sampling can be expected, the user may prefer to
determine the value of :math:`\langle
dG/d\lambda \rangle` accurately at a number of well-chosen intermediate
values of :math:`\lambda`. This can easily be done by setting the
stepsize ``delta_lambda`` to zero. Each simulation can be
equilibrated first, and a proper error estimate can be made for each
value of :math:`dG/d\lambda` from the fluctuation of
:math:`\partial H/\partial
\lambda`. The total free energy change is then determined afterward by
an appropriate numerical integration procedure.

|Gromacs| now also supports the use of Bennett’s Acceptance Ratio Bennett
(1976) for calculating values of :math:`\Delta`\ G for transformations
from state A to state B using the program :ref:`gmx bar`. The same data can also be used to
calculate free energies using MBAR Shirts and Chodera (2008), though the
analysis currently requires external tools from the external
`pymbar package <https://SimTK.org/home/pymbar>`__.

The :math:`\lambda`-dependence for the force-field contributions is
described in detail in section sec. 
Replica exchange
----------------

Replica exchange molecular dynamics (REMD
) is a
method that can be used to speed up the sampling of any type of
simulation, especially if conformations are separated by relatively high
energy barriers. It involves simulating multiple replicas of the same
system at different temperatures and randomly exchanging the complete
state of two replicas at regular intervals with the probability:

.. math::

   P(1 \leftrightarrow 2)=\min\left(1,\exp\left[
   \left(\frac{1}{k_B T_1} - \frac{1}{k_B T_2}\right)(U_1 - U_2)
    \right] \right)

where :math:`T_1` and :math:`T_2` are the reference temperatures and
:math:`U_1` and :math:`U_2` are the instantaneous potential energies of
replicas 1 and 2 respectively. After exchange the velocities are scaled
by :math:`(T_1/T_2)^{\pm0.5}` and a neighbour search is performed the
next step. This combines the fast sampling and frequent barrier-crossing
of the highest temperature with correct Boltzmann sampling at all the
different temperatures Hukushima and Nemoto (1996; Sugita and Okamoto
1999). We only attempt exchanges for neighbouring temperatures as the
probability decreases very rapidly with the temperature difference. One
should not attempt exchanges for all possible pairs in one step. If, for
instance, replicas 1 and 2 would exchange, the chance of exchange for
replicas 2 and 3 not only depends on the energies of replicas 2 and 3,
but also on the energy of replica 1. In |Gromacs| this is solved by
attempting exchange for all *odd* pairs on *odd* attempts and for all
*even* pairs on *even* attempts. If we have four replicas: 0, 1, 2 and
3, ordered in temperature and we attempt exchange every 1000 steps,
pairs 0-1 and 2-3 will be tried at steps 1000, 3000 etc. and pair 1-2 at
steps 2000, 4000 etc.

How should one choose the temperatures? The energy difference can be
written as:

.. math:: U_1 - U_2 =  N_{df} \frac{c}{2} k_B (T_1 - T_2)

where :math:`N_{df}` is the total number of degrees of freedom of one
replica and :math:`c` is 1 for harmonic potentials and around 2 for
protein/water systems. If :math:`T_2 = (1+\epsilon) T_1` the probability
becomes:

.. math::

   P(1 \leftrightarrow 2)
     = \exp\left( -\frac{\epsilon^2 c\,N_{df}}{2 (1+\epsilon)} \right)
   \approx \exp\left(-\epsilon^2 \frac{c}{2} N_{df} \right)

Thus for a probability of :math:`e^{-2}\approx 0.135` one obtains
:math:`\epsilon \approx 2/\sqrt{c\,N_{df}}`. With all bonds constrained
one has :math:`N_{df} \approx 2\, N_{atoms}` and thus for :math:`c` = 2
one should choose :math:`\epsilon` as :math:`1/\sqrt{N_{atoms}}`.
However there is one problem when using pressure coupling. The density
at higher temperatures will decrease, leading to higher energy Seibert
*et al.* (2005), which should be taken into account. The |Gromacs| website
features a so-called ``REMD calculator``, that lets you type in the
temperature range and the number of atoms, and based on that proposes a
set of temperatures.

An extension to the REMD for the isobaric-isothermal ensemble was
proposed by Okabe et al. (2001). In this work the
exchange probability is modified to:

.. math::

   P(1 \leftrightarrow 2)=\min\left(1,\exp\left[
   \left(\frac{1}{k_B T_1} - \frac{1}{k_B T_2}\right)(U_1 - U_2) +
   \left(\frac{P_1}{k_B T_1} - \frac{P_2}{k_B T_2}\right)\left(V_1-V_2\right)
    \right] \right)

where :math:`P_1` and :math:`P_2` are the respective reference
pressures and :math:`V_1` and :math:`V_2` are the respective
instantaneous volumes in the simulations. In most cases the differences
in volume are so small that the second term is negligible. It only plays
a role when the difference between :math:`P_1` and :math:`P_2` is large
or in phase transitions.

Hamiltonian replica exchange is also supported in |Gromacs|. In
Hamiltonian replica exchange, each replica has a different Hamiltonian,
defined by the free energy pathway specified for the simulation. The
exchange probability to maintain the correct ensemble probabilities is:

.. math::

   P(1 \leftrightarrow 2)=\min\left(1,\exp\left[
       \left(\frac{1}{k_B T} - \frac{1}{k_B T}\right)((U_1(x_2) - U_1(x_1)) + (U_2(x_1) - U_2(x_2)))
   \right]
   \right)

The separate Hamiltonians are defined by the free energy functionality
of |Gromacs|, with swaps made between the different values of
:math:`\lambda` defined in the mdp file.

Hamiltonian and temperature replica exchange can also be performed
simultaneously, using the acceptance criteria:

.. math::

   P(1 \leftrightarrow 2)=\min\left(1,\exp\left[
   \left(\frac{1}{k_B T} - \right)(\frac{U_1(x_2) - U_1(x_1)}{k_B T_1} + \frac{U_2(x_1) - U_2(x_2)}{k_B T_2})
    \right] \right)

Gibbs sampling replica exchange has also been implemented in
|Gromacs| Chodera and Shirts (2011). In Gibbs sampling replica exchange,
all possible pairs are tested for exchange, allowing swaps between
replicas that are not neighbors.

Gibbs sampling replica exchange requires no additional potential energy
calculations. However there is an additional communication cost in Gibbs
sampling replica exchange, as for some permutations, more than one round
of swaps must take place. In some cases, this extra communication cost
might affect the efficiency.

All replica exchange variants are options of the :ref:`mdrun <gmx mdrun>`
program. It will only work when MPI is installed, due to the inherent
parallelism in the algorithm. For efficiency each replica can run on a
separate rank. See the manual page of :ref:`mdrun <gmx mdrun>` on how to
use these multinode features.

Essential Dynamics sampling
---------------------------

The results from Essential Dynamics (see
sec. 
) of a protein can be used to guide
MD simulations. The idea is that from an initial MD simulation (or from
other sources) a definition of the collective fluctuations with largest
amplitude is obtained. The position along one or more of these
collective modes can be constrained in a (second) MD simulation in a
number of ways for several purposes. For example, the position along a
certain mode may be kept fixed to monitor the average force (free-energy
gradient) on that coordinate in that position. Another application is to
enhance sampling efficiency with respect to usual MD Groot, Amadei,
Aalten, et al. (1996; Groot, Amadei, Scheek, et al. 1996). In this case,
the system is encouraged to sample its available configuration space
more systematically than in a diffusion-like path that proteins usually
take.

Another possibility to enhance sampling is
flooding
Here a flooding potential is
added to certain (collective) degrees of freedom to expel the system out
of a region of phase space Lange, Schafer, and Grubmuller (2006).

The procedure for essential dynamics sampling or flooding is as follows.
First, the eigenvectors and eigenvalues need to be determined using
covariance analysis (:ref:`gmx covar`) or normal-mode analysis
(:ref:`gmx nmeig`). Then, this information is fed into
:ref:`make_edi <gmx make_edi>`, which has many options for selecting vectors
and setting parameters, see ``gmx make_edi -h``. The
generated :ref:`edi` input file is then passed to
:ref:`mdrun <gmx mdrun>`.

Expanded Ensemble
-----------------

In an expanded ensemble simulation Lyubartsev et al. (1992), both the
coordinates and the thermodynamic ensemble are treated as configuration
variables that can be sampled over. The probability of any given state
can be written as:

.. math:: P(\vec{x},k) \propto \exp\left(-\beta_k U_k + g_k\right),

where :math:`\beta_k = \frac{1}{k_B T_k}` is the :math:`\beta`
corresponding to the :math:`k`\ th thermodynamic state, and :math:`g_k`
is a user-specified weight factor corresponding to the :math:`k`\ th
state. This space is therefore a *mixed*, *generalized*, or *expanded*
ensemble which samples from multiple thermodynamic ensembles
simultaneously. :math:`g_k` is chosen to give a specific weighting of
each subensemble in the expanded ensemble, and can either be fixed, or
determined by an iterative procedure. The set of :math:`g_k` is
frequently chosen to give each thermodynamic ensemble equal probability,
in which case :math:`g_k` is equal to the free energy in non-dimensional
units, but they can be set to arbitrary values as desired. Several
different algorithms can be used to equilibrate these weights, described
in the mdp option listings.

In |Gromacs|, this space is sampled by
alternating sampling in the :math:`k` and :math:`\vec{x}` directions.
Sampling in the :math:`\vec{x}` direction is done by standard molecular
dynamics sampling; sampling between the different thermodynamics states
is done by Monte Carlo, with several different Monte Carlo moves
supported. The :math:`k` states can be defined by different
temperatures, or choices of the free energy :math:`\lambda` variable, or
both. Expanded ensemble simulations thus represent a serialization of
the replica exchange formalism, allowing a single simulation to explore
many thermodynamic states.

Parallelization
---------------

The CPU time required for a simulation can be reduced by running the
simulation in parallel over more than one core. Ideally, one would want
to have linear scaling: running on :math:`N` cores makes the simulation
:math:`N` times faster. In practice this can only be achieved for a
small number of cores. The scaling will depend a lot on the algorithms
used. Also, different algorithms can have different restrictions on the
interaction ranges between atoms.

Domain decomposition
--------------------

Since most interactions in molecular simulations are local, domain
decomposition is a natural way to decompose the system. In domain
decomposition, a spatial domain is assigned to each rank, which will
then integrate the equations of motion for the particles that currently
reside in its local domain. With domain decomposition, there are two
choices that have to be made: the division of the unit cell into domains
and the assignment of the forces to domains. Most molecular simulation
packages use the half-shell method for assigning the forces. But there
are two methods that always require less communication: the eighth
shell Liem, Brown, and Clarke (1991) and the midpoint Bowers, Dror, and
Shaw (2006) method. |Gromacs| currently uses the eighth shell method, but
for certain systems or hardware architectures it might be advantageous
to use the midpoint method. Therefore, we might implement the midpoint
method in the future. Most of the details of the domain decomposition
can be found in the |Gromacs| 4 paper Hess et al. (2008).

Coordinate and force communication
~~~~~~~~~~~~~~~~~~~~~~~~~~~~~~~~~~

In the most general case of a triclinic unit cell, the space in divided
with a 1-, 2-, or 3-D grid in parallelepipeds that we call domain
decomposition cells. Each cell is assigned to a particle-particle rank.
The system is partitioned over the ranks at the beginning of each MD
step in which neighbour searching is performed. Since the neighbour
searching is based on charge groups, charge groups are also the units
for the domain decomposition. Charge groups are assigned to the cell
where their center of geometry resides. Before the forces can be
calculated, the coordinates from some neighbouring cells need to be
communicated, and after the forces are calculated, the forces need to be
communicated in the other direction. The communication and force
assignment is based on zones that can cover one or multiple cells. An
example of a zone setup is shown in Fig. 

.. _fig-ddcells:

.. figure:: plots/dd-cells.eps

   A non-staggered domain decomposition grid of
   3\ :math:`\times`\ 2\ :math:`\times`\ 2 cells. Coordinates in zones 1
   to 7 are communicated to the corner cell that has its home particles
   in zone 0. :math:`r_c` is the cut-off radius.

The coordinates are communicated by moving data along the “negative”
direction in :math:`x`, :math:`y` or :math:`z` to the next neighbour.
This can be done in one or multiple pulses. In
:numref:`Fig. (%s) <fig-ddcells>` two pulses in :math:`x` are
required, then one in :math:`y` and then one in :math:`z`. The forces
are communicated by reversing this procedure. See the |Gromacs| 4
paper Hess et al. (2008) for details on determining which non-bonded and
bonded forces should be calculated on which rank.

Dynamic load balancing
~~~~~~~~~~~~~~~~~~~~~~

When different ranks have a different computational load (load
imbalance), all ranks will have to wait for the one that takes the most
time. One would like to avoid such a situation. Load imbalance can occur
due to four reasons:

-  inhomogeneous particle distribution

-  inhomogeneous interaction cost distribution (charged/uncharged,
   water/non-water due to |Gromacs| water innerloops)

-  statistical fluctuation (only with small particle numbers)

-  differences in communication time, due to network topology and/or
   other jobs on the machine interfering with our communication

So we need a dynamic load balancing algorithm where the volume of each
domain decomposition cell can be adjusted *independently*. To achieve
this, the 2- or 3-D domain decomposition grids need to be staggered.
:numref:`Fig. (%s) <figddtric>` shows the most general case in 2-D.
Due to the staggering, one might require two distance checks for
deciding if a charge group needs to be communicated: a non-bonded
distance and a bonded distance check.

.. _fig-ddtric:

.. figure:: plots/dd-tric.eps

   The zones to communicate to the rank of zone 0, see the text
   for details. :math:`r_c` and :math:`r_b` are the non-bonded and
   bonded cut-off radii respectively, :math:`d` is an example of a
   distance between following, staggered boundaries of cells.

By default, :ref:`mdrun <gmx mdrun>` automatically turns on the dynamic
load balancing during a simulation when the total performance loss due
to the force calculation imbalance is 2% or more. **Note** that the
reported force load imbalance numbers might be higher, since the force
calculation is only part of work that needs to be done during an
integration step. The load imbalance is reported in the log file at log
output steps and when the ``-v`` option is used also on
screen. The average load imbalance and the total performance loss due to
load imbalance are reported at the end of the log file.

There is one important parameter for the dynamic load balancing, which
is the minimum allowed scaling. By default, each dimension of the domain
decomposition cell can scale down by at least a factor of 0.8. For 3-D
domain decomposition this allows cells to change their volume by about a
factor of 0.5, which should allow for compensation of a load imbalance
of 100%. The minimum allowed scaling can be changed with the
``-dds`` option of :ref:`mdrun <gmx mdrun>`.

The load imbalance is measured by timing a single region of the MD step
on each MPI rank. This region can not include MPI communication, as
timing of MPI calls does not allow separating wait due to imbalance from
actual communication. The domain volumes are then scaled, with
under-relaxation, inversely proportional with the measured time. This
procedure will decrease the load imbalance when the change in load in
the measured region correlates with the change in domain volume and the
load outside the measured region does not depend strongly on the domain
volume. In CPU-only simulations, the load is measured between the
coordinate and the force communication. In simulations with non-bonded
work on GPUs, we overlap communication and work on the CPU with
calculation on the GPU. Therefore we measure from the last communication
before the force calculation to when the CPU or GPU is finished,
whichever is last. When not using PME ranks, we subtract the time in PME
from the CPU time, as this includes MPI calls and the PME load is
independent of domain size. This generally works well, unless the
non-bonded load is low and there is imbalance in the bonded
interactions. Then two issues can arise. Dynamic load balancing can
increase the imbalance in update and constraints and with PME the
coordinate and force redistribution time can go up significantly.
Although dynamic load balancing can significantly improve performance in
cases where there is imbalance in the bonded interactions on the CPU,
there are many situations in which some domains continue decreasing in
size and the load imbalance increases and/or PME coordinate and force
redistribution cost increases significantly. As of version 2016.1,
:ref:`mdrun <gmx mdrun>` disables the dynamic load balancing when
measurement indicates that it deteriorates performance. This means that
in most cases the user will get good performance with the default,
automated dynamic load balancing setting.

Constraints in parallel
~~~~~~~~~~~~~~~~~~~~~~~

Since with domain decomposition parts of molecules can reside on
different ranks, bond constraints can cross cell boundaries. Therefore a
parallel constraint algorithm is required. |Gromacs| uses the
P-LINCS
algorithm Hess (2007), which is the
parallel version of the LINCS
algorithm Hess
et al. (1997) . (see 
). The P-LINCS
procedure is illustrated in :numref:`Fig. (%s) <fig-plincs>`. When
molecules cross the cell boundaries, atoms in such molecules up to
(``lincs_order + 1``) bonds away are communicated over the
cell boundaries. Then, the normal LINCS algorithm can be applied to the
local bonds plus the communicated ones. After this procedure, the local
bonds are correctly constrained, even though the extra communicated ones
are not. One coordinate communication step is required for the initial
LINCS step and one for each iteration. Forces do not need to be
communicated.

.. _fig-plincs:

.. figure:: plots/par-lincs2.eps

   Example of the parallel setup of P-LINCS with one molecule
   split over three domain decomposition cells, using a matrix expansion
   order of 3. The top part shows which atom coordinates need to be
   communicated to which cells. The bottom parts show the local
   constraints (solid) and the non-local constraints (dashed) for each
   of the three cells.

Interaction ranges
~~~~~~~~~~~~~~~~~~

Domain decomposition takes advantage of the locality of interactions.
This means that there will be limitations on the range of interactions.
By default, :ref:`mdrun <gmx mdrun>` tries to find the optimal balance
between interaction range and efficiency. But it can happen that a
simulation stops with an error message about missing interactions, or
that a simulation might run slightly faster with shorter interaction
ranges. A list of interaction ranges and their default values is given
in in :numref:`Table %s <table-ddranges>`

.. |nbrange| replace:: :math:`r_c`\ =\ max(\ :math:`r_{\mathrm{list}}`\ ,\ :math:`r_{\mathrm{VdW}}`\ ,\ :math:`r_{\mathrm{Coul}}`\ )
.. |tbrange| replace:: max(:math:`r_{\mathrm{mb}}`\ ,\ :math:`r_c`) 
.. |mbrange| replace:: :math:`r_{\mathrm{mb}}` 
.. |csrange| replace:: :math:`r_{\mathrm{con}}`
.. |vsrange| replace:: :math:`r_{\mathrm{con}}` 
.. |mdrunr| replace:: :ref:`mdrun <gmx mdrun>` ``-rdd``
.. |mdrunc| replace:: :ref:`mdrun <gmx mdrun>` ``-rcon``

.. _table-ddranges:

.. table:: The interaction ranges with domain decomposition.
    :widths: auto
    :align: center

    +-------------------+-----------+-----------------+------------------------+
    | interaction       | range     | option          | default                |
    +===================+===========+=================+========================+
    | non-bonded        | |nbrange| | :ref:`mdp` file |                        |
    +-------------------+-----------+-----------------+------------------------+
    | two-body bonded   | |tbrange| | |mdrunr|        | starting conf. + 10%   |
    +-------------------+-----------+-----------------+------------------------+
    | multi-body bonded | |mbrange| | |mdrunr|        | starting conf. + 10%   |
    +-------------------+-----------+-----------------+------------------------+
    | constraints       | |csrange| | |mdrunc|        | est. from bond lengths |
    +-------------------+-----------+-----------------+------------------------+
    | virtual sites     | |vsrange| | |mdrunc|        | 0                      |
    +-------------------+-----------+-----------------+------------------------+


In most cases the defaults of :ref:`mdrun <gmx mdrun>` should not cause the
simulation to stop with an error message of missing interactions. The
range for the bonded interactions is determined from the distance
between bonded charge-groups in the starting configuration, with 10%
added for headroom. For the constraints, the value of
:math:`r_{\mathrm{con}}` is determined by taking the maximum distance
that (``lincs_order + 1``) bonds can cover when they all
connect at angles of 120 degrees. The actual constraint communication is
not limited by :math:`r_{\mathrm{con}}`, but by the minimum cell size
:math:`L_C`, which has the following lower limit:

.. math:: L_C \geq \max(r_{\mathrm{mb}},r_{\mathrm{con}})

Without dynamic load balancing the system is actually allowed to scale
beyond this limit when pressure scaling is used. **Note** that for
triclinic boxes, :math:`L_C` is not simply the box diagonal component
divided by the number of cells in that direction, rather it is the
shortest distance between the triclinic cells borders. For rhombic
dodecahedra this is a factor of :math:`\sqrt{3/2}` shorter along
:math:`x` and :math:`y`.

When :math:`r_{\mathrm{mb}} > r_c`, :ref:`mdrun <gmx mdrun>` employs a
smart algorithm to reduce the communication. Simply communicating all
charge groups within :math:`r_{\mathrm{mb}}` would increase the amount
of communication enormously. Therefore only charge-groups that are
connected by bonded interactions to charge groups which are not locally
present are communicated. This leads to little extra communication, but
also to a slightly increased cost for the domain decomposition setup. In
some cases, *e.g.* coarse-grained simulations with a very short cut-off,
one might want to set :math:`r_{\mathrm{mb}}` by hand to reduce this
cost.

Multiple-Program, Multiple-Data PME parallelization
~~~~~~~~~~~~~~~~~~~~~~~~~~~~~~~~~~~~~~~~~~~~~~~~~~~

Electrostatics interactions are long-range, therefore special algorithms
are used to avoid summation over many atom pairs. In |Gromacs| this is
usually . PME (sec. 
). Since with PME all
particles interact with each other, global communication is required.
This will usually be the limiting factor for scaling with domain
decomposition. To reduce the effect of this problem, we have come up
with a Multiple-Program, Multiple-Data approach Hess et al. (2008).
Here, some ranks are selected to do only the PME mesh calculation, while
the other ranks, called particle-particle (PP) ranks, do all the rest of
the work. For rectangular boxes the optimal PP to PME rank ratio is
usually 3:1, for rhombic dodecahedra usually 2:1. When the number of PME
ranks is reduced by a factor of 4, the number of communication calls is
reduced by about a factor of 16. Or put differently, we can now scale to
4 times more ranks. In addition, for modern 4 or 8 core machines in a
network, the effective network bandwidth for PME is quadrupled, since
only a quarter of the cores will be using the network connection on each
machine during the PME calculations.

.. _fig-mpmdpme:

.. figure:: plots/mpmd-pme.eps

   Example of 8 ranks without (left) and with (right) MPMD. The
   PME communication (red arrows) is much higher on the left than on the
   right. For MPMD additional PP - PME coordinate and force
   communication (blue arrows) is required, but the total communication
   complexity is lower.

:ref:`mdrun <gmx mdrun>` will by default interleave the PP and PME ranks.
If the ranks are not number consecutively inside the machines, one might
want to use :ref:`mdrun <gmx mdrun>` ``-ddorder pp_pme``. For machines with a
real 3-D torus and proper communication software that assigns the ranks
accordingly one should use :ref:`mdrun <gmx mdrun>` ``-ddorder cartesian``.

To optimize the performance one should usually set up the cut-offs and
the PME grid such that the PME load is 25 to 33% of the total
calculation load. :ref:`grompp <gmx grompp>` will print an estimate for this
load at the end and also :ref:`mdrun <gmx mdrun>` calculates the same
estimate to determine the optimal number of PME ranks to use. For high
parallelization it might be worthwhile to optimize the PME load with the
`mdp` settings and/or the number of PME ranks with the
``-npme`` option of :ref:`mdrun <gmx mdrun>`. For changing the
electrostatics settings it is useful to know the accuracy of the
electrostatics remains nearly constant when the Coulomb cut-off and the
PME grid spacing are scaled by the same factor. **Note** that it is
usually better to overestimate than to underestimate the number of PME
ranks, since the number of PME ranks is smaller than the number of PP
ranks, which leads to less total waiting time.

The PME domain decomposition can be 1-D or 2-D along the :math:`x`
and/or :math:`y` axis. 2-D decomposition is also known as pencil
decomposition 
because of the
shape of the domains at high parallelization. 1-D decomposition along
the :math:`y` axis can only be used when the PP decomposition has only 1
domain along :math:`x`. 2-D PME decomposition has to have the number of
domains along :math:`x` equal to the number of the PP decomposition.
:ref:`mdrun <gmx mdrun>` automatically chooses 1-D or 2-D PME decomposition
(when possible with the total given number of ranks), based on the
minimum amount of communication for the coordinate redistribution in PME
plus the communication for the grid overlap and transposes. To avoid
superfluous communication of coordinates and forces between the PP and
PME ranks, the number of DD cells in the :math:`x` direction should
ideally be the same or a multiple of the number of PME ranks. By
default, :ref:`mdrun <gmx mdrun>` takes care of this issue.

Domain decomposition flow chart
~~~~~~~~~~~~~~~~~~~~~~~~~~~~~~~

In :numref:`Fig. (%s) <fig-ddflow>` a flow chart is shown for domain
decomposition with all possible communication for different algorithms.
For simpler simulations, the same flow chart applies, without the
algorithms and communication for the algorithms that are not used.

.. _fig-ddflow:

.. figure:: plots/flowchart.eps

   Flow chart showing the algorithms and communication (arrows)
   for a standard MD simulation with virtual sites, constraints and
   separate PME-mesh ranks.

Implicit solvation
------------------

Implicit solvent models provide an efficient way of representing the
electrostatic effects of solvent molecules, while saving a large piece
of the computations involved in an accurate, aqueous description of the
surrounding water in molecular dynamics simulations. Implicit solvation
models offer several advantages compared with explicit solvation,
including eliminating the need for the equilibration of water around the
solute, and the absence of viscosity, which allows the protein to more
quickly explore conformational space.

Implicit solvent calculations in |Gromacs| can be done using the
generalized Born-formalism, and the Still Qiu et al. (1997),
HCT Hawkins, Cramer, and Truhlar (1996), and OBC Onufriev, Bashford, and
Case (2004) models are available for calculating the Born radii.

Here, the free energy :math:`G_{\mathrm{solv}}` of solvation is the sum
of three terms, a solvent-solvent cavity term
(:math:`G_{\mathrm{cav}}`), a solute-solvent van der Waals term
(:math:`G_{\mathrm{vdw}}`), and finally a solvent-solute electrostatics
polarization term (:math:`G_{\mathrm{pol}}`).

The sum of :math:`G_{\mathrm{cav}}` and :math:`G_{\mathrm{vdw}}`
corresponds to the (non-polar) free energy of solvation for a molecule
from which all charges have been removed, and is commonly called
:math:`G_{\mathrm{np}}`, calculated from the total solvent accessible
surface area multiplied with a surface tension. The total expression for
the solvation free energy then becomes:

.. math::  G_{\mathrm{solv}} = G_{\mathrm{np}}  + G_{\mathrm{pol}}
           :label: eqngbsolv

Under the generalized Born model, :math:`G_{\mathrm{pol}}` is calculated
from the generalized Born equation Qiu et al. (1997):

.. math::  G_{\mathrm{pol}} = \left(1-\frac{1}{\epsilon}\right) \sum_{i=1}^n \sum_{j>i}^n \frac {q_i q_j}{\sqrt{r^2_{ij} + b_i b_j \exp\left(\frac{-r^2_{ij}}{4 b_i b_j}\right)}}
           :label: eqngbstill

In |Gromacs|, we have introduced the substitution Larsson and Lindahl
(2010):

.. math::  c_i=\frac{1}{\sqrt{b_i}}
           :label: eqngbsubst

which makes it possible to introduce a cheap transformation to a new
variable :math:`x` when evaluating each interaction, such that:

.. math::  x=\frac{r_{ij}}{\sqrt{b_i b_j }} = r_{ij} c_i c_j
           :label: eqngbsubst2

In the end, the full re-formulation of 
becomes:

.. math:: G_{\mathrm{pol}} = \left(1-\frac{1}{\epsilon}\right) \sum_{i=1}^n \sum_{j>i}^n \frac{q_i q_j}{\sqrt{b_i  b_j}} ~\xi (x) = \left(1-\frac{1}{\epsilon}\right) \sum_{i=1}^n q_i c_i \sum_{j>i}^n q_j c_j~\xi (x)
          :label: eqngbfinal

The non-polar part (:math:`G_{\mathrm{np}}`) of
:eq:`Equation (%s) <eqngbsolv>` is calculated directly from the
Born radius of each atom using a simple ACE type approximation by
Schaefer *et al.* Schaefer, Bartels, and Karplus (1998), including a
simple loop over all atoms. This requires only one extra solvation
parameter, independent of atom type, but differing slightly between the
three Born radii models.

