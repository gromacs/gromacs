%
% This file is part of the GROMACS molecular simulation package.
%
% Copyright (c) 2013,2014, by the GROMACS development team, led by
% Mark Abraham, David van der Spoel, Berk Hess, and Erik Lindahl,
% and including many others, as listed in the AUTHORS file in the
% top-level source directory and at http://www.gromacs.org.
%
% GROMACS is free software; you can redistribute it and/or
% modify it under the terms of the GNU Lesser General Public License
% as published by the Free Software Foundation; either version 2.1
% of the License, or (at your option) any later version.
%
% GROMACS is distributed in the hope that it will be useful,
% but WITHOUT ANY WARRANTY; without even the implied warranty of
% MERCHANTABILITY or FITNESS FOR A PARTICULAR PURPOSE.  See the GNU
% Lesser General Public License for more details.
%
% You should have received a copy of the GNU Lesser General Public
% License along with GROMACS; if not, see
% http://www.gnu.org/licenses, or write to the Free Software Foundation,
% Inc., 51 Franklin Street, Fifth Floor, Boston, MA  02110-1301  USA.
%
% If you want to redistribute modifications to GROMACS, please
% consider that scientific software is very special. Version
% control is crucial - bugs must be traceable. We will be happy to
% consider code for inclusion in the official distribution, but
% derived work must not be called official GROMACS. Details are found
% in the README & COPYING files - if they are missing, get the
% official version at http://www.gromacs.org.
%
% To help us fund GROMACS development, we humbly ask that you cite
% the research papers on the package. Check out http://www.gromacs.org.

\chapter{Definitions and Units}
\label{ch:defunits}
\section{Notation}
The following conventions for mathematical typesetting 
are used throughout this document:

\centerline{
\begin{tabular}{l|l|c}
Item		&	Notation	& Example	\\
\hline	
Vector		&	Bold italic	& $\rvi$	\\
Vector Length	&	Italic		& $r_i$		\\
\end{tabular}
}

We define the {\em lowercase} subscripts 
$i$, $j$, $k$ and $l$ to denote particles:
$\rvi$ is the {\em position vector} of particle $i$, and using this 
notation:
\bea
\rvij	=	\rvj-\rvi	\\
\rij	=	| \rvij |
\eea
The force on particle $i$ is denoted by $\ve{F}_i$ and 
\beq
\ve{F}_{ij} = \mbox{force on $i$ exerted by $j$}
\eeq
Please note that we changed notation as of version 2.0 to $\rvij=\rvj-\rvi$ since this
is the notation commonly used. If you encounter an error, let us know.

\section{\normindex{MD units}\index{units}}
{\gromacs} uses a consistent set of units that produce values in the
vicinity of unity for most relevant molecular quantities. Let us call
them {\em MD units}. The basic units in this system are nm, ps, K,
electron charge (e) and atomic mass unit (u), see
\tabref{basicunits}.
\begin{table}
\centerline{
\begin{tabular}{|l|c|l|}
\dline
Quantity	& Symbol&  Unit						\\
\hline			
length		&  r	&  nm $= 10^{-9}$ m				\\
mass		&  m	&  u (atomic mass unit)	$=$ 
				1.6605402(10)$\times 10^{-27}$ kg	\\
		&	&  ($1/12$ the mass of a $^{12}$C atom)		\\
		&	&  $1.6605402(10)\times 10^{-27}$ kg		\\
time		&  t	&  ps $= 10^{-12}$ s				\\
charge		&  q	&  {\it e} $=$ electronic charge $=
				1.60217733(49)\times 10^{-19}$ C	\\
temperature	&  T	&  K  						\\
\dline
\end{tabular}
}
\caption[Basic units used in {\gromacs}.]{Basic units used in
{\gromacs}. Numbers in parentheses give accuracy.}
\label{tab:basicunits}
\end{table}

Consistent with these units are a set of derived units, given in
\tabref{derivedunits}.
\begin{table}
\centerline{
\begin{tabular}{|l|c|l|}
\dline
Quantity	& Symbol   & Unit				\\
\hline
energy		& $E,V$	   & kJ~mol$^{-1}$			\\
Force		& $\ve{F}$ & kJ~mol$^{-1}$~nm$^{-1}$		\\
pressure	& $p$	   & kJ~mol$^{-1}$~nm$^{-3} =
				10^{30}/N_{AV}$~Pa		\\
        	&	   & $1.660\,54\times 10^6$~Pa $= 
				16.6054$~bar         		\\
velocity	& $v$	   & nm~ps$^{-1} = 1000$ m s$^{-1}$		\\
dipole moment   & $\mu$	   & \emph{e}~nm                		\\ 
electric potential& $\Phi$ & kJ~mol$^{-1}$~\emph{e}$^{-1} = 
				0.010\,364\,272(3)$ Volt   	\\
electric field	& $E$	   & kJ~mol$^{-1}$~nm$^{-1}$~\emph{e}$^{-1} =
				1.036\,427\,2(3) \times 10^7$~V m$^{-1}$	\\
\dline
\end{tabular}
}
\caption{Derived units}
\label{tab:derivedunits}
\end{table}

The {\bf electric conversion factor} $f=\frac{1}{4 \pi
\varepsilon_o}=138.935\,485(9)$ kJ~mol$^{-1}$~nm~e$^{-2}$. It relates
the mechanical quantities to the electrical quantities as in
\beq
 V = f \frac{q^2}{r} \mbox{\ \ or\ \ } F = f \frac{q^2}{r^2}
\eeq

Electric potentials $\Phi$ and electric fields $\ve{E}$ are
intermediate quantities in the calculation of energies and
forces. They do not occur inside {\gromacs}. If they are used in
evaluations, there is a choice of equations and related units. We
strongly recommend following the usual practice of including the factor
$f$ in expressions that evaluate $\Phi$ and $\ve{E}$:
\bea
\Phi(\ve{r}) = f \sum_j \frac{q_j}{|\ve{r}-\ve{r}_j|} 	\\
\ve{E}(\ve{r}) = f \sum_j q_j \frac{(\ve{r}-\ve{r}_j)}{|\ve{r}-\ve{r}_j|^3}
\eea
With these definitions, $q\Phi$ is an energy and $q\ve{E}$ is a
force. The units are those given in \tabref{derivedunits}:
about 10 mV for potential. Thus, the potential of an electronic charge
at a distance of 1 nm equals $f \approx 140$ units $\approx
1.4$~V. (exact value: 1.439965 V)

{\bf Note} that these units are mutually consistent; changing any of the
units is likely to produce inconsistencies and is therefore {\em
strongly discouraged\/}! In particular: if \AA \ are used instead of
nm, the unit of time changes to 0.1 ps. If kcal mol$^{-1}$ (= 4.184
kJ mol$^{-1}$) is used instead of kJ mol$^{-1}$ for energy, the unit of time becomes
0.488882 ps and the unit of temperature changes to 4.184 K. But in
both cases all electrical energies go wrong, because they will still
be computed in kJ mol$^{-1}$, expecting nm as the unit of length. Although
careful rescaling of charges may still yield consistency, it is clear
that such confusions must be rigidly avoided.
  
In terms of the MD units, the usual physical constants take on
different values (see \tabref{consts}). All quantities are per mol rather than per
molecule. There is no distinction between Boltzmann's constant $k$ and
the gas constant $R$: their value is
$0.008\,314\,51$~kJ~mol$^{-1}$~K$^{-1}$.
\begin{table}
\centerline{
\begin{tabular}{|c|l|l|}
\dline
Symbol	& Name			& Value					\\
\hline
$N_{AV}$& Avogadro's number 	& $6.022\,136\,7(36)\times 10^{23}$  mol$^{-1}$	\\
$R$   	& gas constant 	& $8.314\,510(70)\times 10^{-3}$~kJ~mol$^{-1}$~K$^{-1}$	\\
$k_B$  	& Boltzmann's constant  & \emph{idem}	\\
$h$	& Planck's constant	& $0.399\,031\,32(24)$~kJ~mol$^{-1}$~ps \\
$\hbar$	& Dirac's constant	& $0.063\,507\,807(38)$~kJ~mol$^{-1}$~ps \\
$c$	& velocity of light	& $299\,792.458$~nm~ps$^{-1}$ \\
\dline
\end{tabular}
}
\caption{Some Physical Constants}
\label{tab:consts}
\end{table}

\section{Reduced units\index{reduced units}}
When simulating Lennard-Jones (LJ) systems, it might be advantageous to
use reduced units ({\ie}, setting
$\epsilon_{ii}=\sigma_{ii}=m_i=k_B=1$ for one type of atoms). This is
possible. When specifying the input in reduced units, the output will
also be in reduced units. The one exception is the {\em
temperature}, which is expressed in $0.008\,314\,51$ reduced
units. This is a consequence of using Boltzmann's constant in the
evaluation of temperature in the code. Thus not $T$, but $k_BT$, is the
reduced temperature. A {\gromacs} temperature $T=1$ means a reduced
temperature of $0.008\ldots$ units; if a reduced temperature of 1 is
required, the {\gromacs} temperature should be 120.2717.

In \tabref{reduced} quantities are given for LJ potentials:
\beq
V_{LJ} = 4\epsilon \left[ \left(\frac{\sigma}{r}\right)^{12} - \left(\frac{\sigma}{r}\right)^{6} \right]
\eeq

\begin{table}
\centerline{
\begin{tabular}{|l|c|l|}
\dline
Quantity	& Symbol	& Relation to SI			\\
\hline
Length		& r$^*$  	& r $\sigma^{-1}$			\\
Mass		& m$^*$  	& m M$^{-1}$				\\
Time		& t$^*$  	& t $\sigma^{-1}$ $\sqrt{\epsilon/M}$ \\
Temperature  	& T$^*$  	& k$_B$T $\epsilon^{-1}$ 		\\
Energy		& E$^*$  	& E $\epsilon^{-1}$           	\\
Force		& F$^*$  	& F $\sigma~\epsilon^{-1}$		\\
Pressure	& P$^*$  	& P $\sigma ^3 \epsilon^{-1}$		\\
Velocity	& v$^*$  	& v $\sqrt{M/\epsilon}$		\\
Density		& $\rho^*$  	& N $\sigma ^3~V^{-1}$		\\
\dline
\end{tabular}
}
\caption{Reduced Lennard-Jones quantities}
\label{tab:reduced}
\end{table}


% LocalWords:  ij basicunits derivedunits kJ mol mV kcal consts LJ BT
% LocalWords:  nm ps
