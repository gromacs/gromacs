%
% This file is part of the GROMACS molecular simulation package.
%
% Copyright (c) 2013,2014, by the GROMACS development team, led by
% Mark Abraham, David van der Spoel, Berk Hess, and Erik Lindahl,
% and including many others, as listed in the AUTHORS file in the
% top-level source directory and at http://www.gromacs.org.
%
% GROMACS is free software; you can redistribute it and/or
% modify it under the terms of the GNU Lesser General Public License
% as published by the Free Software Foundation; either version 2.1
% of the License, or (at your option) any later version.
%
% GROMACS is distributed in the hope that it will be useful,
% but WITHOUT ANY WARRANTY; without even the implied warranty of
% MERCHANTABILITY or FITNESS FOR A PARTICULAR PURPOSE.  See the GNU
% Lesser General Public License for more details.
%
% You should have received a copy of the GNU Lesser General Public
% License along with GROMACS; if not, see
% http://www.gnu.org/licenses, or write to the Free Software Foundation,
% Inc., 51 Franklin Street, Fifth Floor, Boston, MA  02110-1301  USA.
%
% If you want to redistribute modifications to GROMACS, please
% consider that scientific software is very special. Version
% control is crucial - bugs must be traceable. We will be happy to
% consider code for inclusion in the official distribution, but
% derived work must not be called official GROMACS. Details are found
% in the README & COPYING files - if they are missing, get the
% official version at http://www.gromacs.org.
%
% To help us fund GROMACS development, we humbly ask that you cite
% the research papers on the package. Check out http://www.gromacs.org.

\setlength {\parindent}{0.0cm}
\setlength {\parskip}{1ex}
\newcommand{\ve}[1]{\mbox{\boldmath ${#1}$}} 
  % defines bold italic vectors. To be used in text or math mode.
  % Example: \ve{F}
\newcommand{\de}{\mbox{d}} 
  % defines a straight d for derivatives.
\newcommand{\intel}{Intel {\em i\/}860}
\newcommand{\gromacs}{GROMACS}
\newcommand{\gromosv}[1]{GROMOS-#1}
\newcommand{\gromos}{GROMOS}
\newcommand{\dline}{\hline\hline}
\newcommand{\etal}{{\em et al.}}
\newcommand{\beq}{\begin{equation}}
\newcommand{\eeq}{\end{equation}}
\newcommand{\bea}{\begin{eqnarray}}
\newcommand{\eea}{\end{eqnarray}}
\newcommand{\Dt}{{\Delta t}}
\newcommand{\half}{\frac{1}{2}}
\newcommand{\hDt}{\half \Dt}
\newcommand{\rvi}{\ve{r}_i}
\newcommand{\rvj}{\ve{r}_j}
\newcommand{\rvk}{\ve{r}_k}
\newcommand{\rij}{r_{ij}}
\newcommand{\rvij}{\ve{r}_{ij}}
\newcommand{\rnorm}{\frac{\rvij}{\rij}}
\newcommand{\Fvi}{\ve{F}_i}
\newcommand{\Fvj}{\ve{F}_j}
\newcommand{\Fvk}{\ve{F}_k}
\newcommand{\Fvij}{\ve{F}_{ij}}
\newcommand{\Fvji}{\ve{F}_{ji}}
\newcommand{\vvi}{\ve{v}_i}
\newcommand{\al}{\alpha}
\newcommand{\be}{\beta}
\newcommand{\ab}{\alpha\beta}
\newcommand{\rnij}{\ve{r}_{ij}^n}
\newcommand{\rni}{\ve{r}_i^n}
\newcommand{\hdt}{\frac{\Delta t}{2}}
%\newcommand{\type}[1]{\\ {\tt \% #1}\\}
\newcommand{\normindex}[1]{#1{\index{#1}}}
\newcommand{\seeindexquiet}[2]{\index{#1|see{#2}}}
\newcommand{\seeindex}[2]{#1{\seeindexquiet{#1}{#2}}}
\newcommand{\swapindexquiet}[2]{{\index{#1 #2}}{\seeindexquiet{#2, #1}{#1 #2}}}
\newcommand{\swapindex}[2]{{\normindex{#1 #2}}{\seeindexquiet{#2, #1}{#1 #2}}}
\newcommand{\swapindexthreequiet}[3]{{\index{#1 #2 #3}}{\seeindexquiet{#2!#3!#1}{#1 #2 #3}}}
\newcommand{\swapindexthree}[3]{{\normindex{#1 #2 #3}}{\seeindexquiet{#2!#3!#1}{#1 #2 #3}}}
\newcommand{\pawsindexquiet}[2]{{\seeindexquiet{#1 #2}{#2, #1}}{\index{#2, #1}}}
\newcommand{\pawsindex}[2]{#1 #2{\pawsindexquiet{#1}{#2}}}
\newcommand{\boldindex}[1]{#1\index{#1@\textbf{#1}}}
\newcommand{\eg}{\em e.g.\@}
\newcommand{\ie}{\em i.e.\@}
\newcommand{\mc}[3]{\multicolumn{#1}{#2}{#3}}

% Commands for correct spacing in tables
%       TeX and TUG News, Vol.2, No.3, p10, 1993. 
%
\newcommand\T{\rule{0pt}{2.6ex}}         % Top strut
\newcommand\B{\rule[-1.2ex]{0pt}{0pt}}   % Bottom strut
\newcommand{\Ts}{\rule{0pt}{2.4ex}}      % Smaller top strut (To be used in 
                                         % math mode in \frac, together with 
                                         % \displaystyle

\newcommand{\captspace}{\vspace{2mm}}

\newif\ifpdfman
\ifx\pdfoutput\undefined
  \pdfmanfalse % not using pdflatex  
\else
  \pdfoutput=1 % using pdflatex
  \pdfmantrue
\fi

\ifpdfman
  \usepackage[pdftex]{graphicx}
  \usepackage[pdftex,plainpages=false,pdfpagelabels,colorlinks=true,urlcolor=blue,citecolor=blue,pdfstartview=FitV]{hyperref}
\else
  \usepackage{graphicx}
  \usepackage[plainpages=false,colorlinks=true,urlcolor=blue,citecolor=blue]{hyperref}
\fi

% Work-arounds to allow unescaped underscores in text.
% These \usepackage commands should come late in the
% list of packages, and in this order.
\newcommand{\UnderscoreCommands}{\do\index}
\usepackage[strings]{underscore}
\usepackage[english]{babel}

% Slightly darker colors print better on b/w printers
\usepackage{color}
\definecolor{blue}{rgb}{0,0,0.5}
\definecolor{red}{rgb}{0.5,0,0}

\makeindex
% The total a4 paper width is 210 mm, and the margins are defined
% relative to the standard margin of 1 inch.
\setlength{\textwidth}{15cm}
\setlength{\oddsidemargin}{9.6mm}  %35mm-1inch
\setlength{\evensidemargin}{-0.4mm} %25mm-1inch
\setlength{\topmargin}{-4.4mm} % 21mm top margin
\setlength{\textheight}{22cm}

\renewcommand{\textfraction}{0.0}
\newcommand{\qtw}{3.75cm}  % 1/4 of the textwidth
\newcommand{\ttw}{4.33cm}  % 1/3 of the textwidth
\newcommand{\htw}{7.5cm}  % 1/2 of the textwidth
\newcommand{\ntw}{13cm} % 0.9 of the textwidth
\newcommand{\gmxver}{@PROJECT_VERSION@}
\newcommand{\gmxyear}{\the\year}
\newcommand{\figref}[1]{Fig.~\ref{fig:#1}}
\newcommand{\figsref}[2]{Figs.~\ref{fig:#1} and ~\ref{fig:#2}}
\newcommand{\tabref}[1]{Table~\ref{tab:#1}}
\newcommand{\eqnref}[1]{eqn.~\ref{eqn:#1}}
\newcommand{\eqnsref}[2]{eqns.~\ref{eqn:#1} and \ref{eqn:#2}}
\newcommand{\secref}[1]{sec.~\ref{sec:#1}}
\newcommand{\tsecref}[1]{\ref{sec:#1}}
\newcommand{\ssecref}[1]{\ref{subsec:#1}}
\newcommand{\sssecref}[1]{\ref{subsubsec:#1}}
\newcommand{\chref}[1]{chapter~\ref{ch:#1}}
\newcommand{\appref}[1]{Appendix~\ref{app:#1}}
\newcommand{\wwwpage}{\href{http://www.gromacs.org}{www.gromacs.org}}
\newcommand{\gerritpage}{\href{http://gerrit.gromacs.org}{gerrit.gromacs.org}}
\newcommand{\redminepage}{\href{http://redmine.gromacs.org}{redmine.gromacs.org}}
% NOTE: \wwwpage is explicitly included in the 'verbatim' citing instructions!
\newcommand{\mcc}[2]{\multicolumn{#1}{c|}{#2}}
\newcommand{\mcl}[2]{\multicolumn{#1}{l}{#2}}
\setlength{\headwidth}{\textwidth}
\setlength{\headheight}{1.0cm}

% LocalWords:  GROMOS et al ij ji rgb eqn eqns GROMACS
