%
% This file is part of the GROMACS molecular simulation package.
%
% Copyright (c) 2013,2014, by the GROMACS development team, led by
% Mark Abraham, David van der Spoel, Berk Hess, and Erik Lindahl,
% and including many others, as listed in the AUTHORS file in the
% top-level source directory and at http://www.gromacs.org.
%
% GROMACS is free software; you can redistribute it and/or
% modify it under the terms of the GNU Lesser General Public License
% as published by the Free Software Foundation; either version 2.1
% of the License, or (at your option) any later version.
%
% GROMACS is distributed in the hope that it will be useful,
% but WITHOUT ANY WARRANTY; without even the implied warranty of
% MERCHANTABILITY or FITNESS FOR A PARTICULAR PURPOSE.  See the GNU
% Lesser General Public License for more details.
%
% You should have received a copy of the GNU Lesser General Public
% License along with GROMACS; if not, see
% http://www.gnu.org/licenses, or write to the Free Software Foundation,
% Inc., 51 Franklin Street, Fifth Floor, Boston, MA  02110-1301  USA.
%
% If you want to redistribute modifications to GROMACS, please
% consider that scientific software is very special. Version
% control is crucial - bugs must be traceable. We will be happy to
% consider code for inclusion in the official distribution, but
% derived work must not be called official GROMACS. Details are found
% in the README & COPYING files - if they are missing, get the
% official version at http://www.gromacs.org.
%
% To help us fund GROMACS development, we humbly ask that you cite
% the research papers on the package. Check out http://www.gromacs.org.

\chapter{Technical Details}

\section{Mixed or Double precision}
{\gromacs} can be compiled in either mixed\index{mixed
precision|see{precision, mixed}}\index{precision, mixed} or
\pawsindex{double}{precision}. Documentation of previous {\gromacs}
versions referred to ``single precision'', but the implementation
has made selective use of double precision for many years.
Using single precision
for all variables would lead to a significant reduction in accuracy.
Although in ``mixed precision'' all state vectors, i.e. particle coordinates,
velocities and forces, are stored in single precision, critical variables
are double precision. A typical example of the latter is the virial,
which is a sum over all forces in the system, which have varying signs.
In addition, in many parts of the code we managed to avoid double precision
for arithmetic, by paying attention to summation order or reorganization
of mathematical expressions. The default configuration uses mixed precision,
but it is easy to turn on double precision by adding the option
{\tt -DGMX_DOUBLE=on} to {\tt cmake}. Double precision
will be 20 to 100\% slower than mixed precision depending on the
architecture you are running on. Double precision will use somewhat
more memory and run input, energy and full-precision trajectory files
will be almost twice as large.

The energies in mixed precision are accurate up to the last decimal,
the last one or two decimals of the forces are non-significant.
The virial is less accurate than the forces, since the virial is only one
order of magnitude larger than the size of each element in the sum over
all atoms (\secref{virial}).
In most cases this is not really a problem, since the fluctuations in the
virial can be two orders of magnitude larger than the average.
Using cut-offs for the Coulomb interactions cause large errors
in the energies, forces, and virial.
Even when using a reaction-field or lattice sum method, the errors
are larger than, or comparable to, the errors due to the partial use of
single precision.
Since MD is chaotic, trajectories with very similar starting conditions will
diverge rapidly, the divergence is faster in mixed precision than in double
precision.

For most simulations, mixed precision is accurate enough.
In some cases double precision is required to get reasonable results:
\begin{itemize}
\item normal mode analysis,
for the conjugate gradient or l-bfgs minimization and the calculation and
diagonalization of the Hessian
\item long-term energy conservation, especially for large systems
\end{itemize}

% LocalWords:  Opteron Itanium PowerPC Altivec Athlon Fortran virial bfgs Nasm
% LocalWords:  diagonalization Cygwin MPI Multi GMXHOME extern gmx tx pid buf
% LocalWords:  bufsize txs rx rxs init nprocs fp msg GMXRC DUMPNL BUFS GMXNPRI
% LocalWords:  unbuffered SGI npri mdrun covar nmeig setenv XPM XVG EPS
% LocalWords:  PDB xvg xpm eps pdb xmgrace ghostview rasmol GMXTIMEUNIT fs dssp
% LocalWords:  mpi distclean ing mpirun goofus doofus fred topol np
% LocalWords:  internet gromacs DGMX cmake SIMD intrinsics AVX PME XN
% LocalWords:  Verlet pre config CONSTRAINTVIR MAXBACKUP TPI ngmx mdp
% LocalWords:  LONGFORMAT DISTGCT CPT tpr cpt CUDA EWALD TWINCUT rvdw
% LocalWords:  rcoulomb STREAMSYNC cudaStreamSynchronized ECC GPUs sc
% LocalWords:  ZYX PERF GPU PINHT hyperthreading DISRE NONBONDED ENX
% LocalWords:  edr ENER gpu FSYNC ENV LJCOMB TOL MAXCONSTRWARN LINCS
% LocalWords:  SOLV NBLISTCG NBNXN XNN ALLVSALL cudaStreamSynchronize
% LocalWords:  USR SIGINT SIGTERM SIGUSR NODECOMM intra PULLVIR multi
% LocalWords:  NOCHARGEGROUPS NOPREDICT NSCELL NCG NTHREADS OpenMP CP
% LocalWords:  PMEONEDD Coulombic der Waals SCSIGMA TPIC GMXNPRIALL
% LocalWords:  GOMP KMP pme NSTLIST ENVVAR nstlist startup OMP NUM ps
% LocalWords:  ACC SCF BASENAME Orca CPMCSCF MCSCF DEVEL EXE GKRWIDTH
% LocalWords:  MAXRESRENUM grompp FFRTP TER NXXX CXXX rtp GZIP gunzip
% LocalWords:  GMXFONT ns MEM MULTIPROT multiprot NCPUS CPUs OPENMM
% LocalWords:  PLUGIN OpenMM plugins SASTEP TESTMC eneconv VMD VMDDIR
% LocalWords:  GMX_USE_XMGR xmgr parallelization nt online Nvidia nb cpu
% LocalWords:  testverlet grommp
