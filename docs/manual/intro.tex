%
% This file is part of the GROMACS molecular simulation package.
%
% Copyright (c) 2013,2014, by the GROMACS development team, led by
% Mark Abraham, David van der Spoel, Berk Hess, and Erik Lindahl,
% and including many others, as listed in the AUTHORS file in the
% top-level source directory and at http://www.gromacs.org.
%
% GROMACS is free software; you can redistribute it and/or
% modify it under the terms of the GNU Lesser General Public License
% as published by the Free Software Foundation; either version 2.1
% of the License, or (at your option) any later version.
%
% GROMACS is distributed in the hope that it will be useful,
% but WITHOUT ANY WARRANTY; without even the implied warranty of
% MERCHANTABILITY or FITNESS FOR A PARTICULAR PURPOSE.  See the GNU
% Lesser General Public License for more details.
%
% You should have received a copy of the GNU Lesser General Public
% License along with GROMACS; if not, see
% http://www.gnu.org/licenses, or write to the Free Software Foundation,
% Inc., 51 Franklin Street, Fifth Floor, Boston, MA  02110-1301  USA.
%
% If you want to redistribute modifications to GROMACS, please
% consider that scientific software is very special. Version
% control is crucial - bugs must be traceable. We will be happy to
% consider code for inclusion in the official distribution, but
% derived work must not be called official GROMACS. Details are found
% in the README & COPYING files - if they are missing, get the
% official version at http://www.gromacs.org.
%
% To help us fund GROMACS development, we humbly ask that you cite
% the research papers on the package. Check out http://www.gromacs.org.

\chapter{Introduction}

\section{Computational Chemistry and Molecular Modeling}

\label{sec:Compchem}

{\gromacs} is an engine to perform molecular dynamics simulations and 
energy minimization. These are two of the many techniques that belong 
to the realm of \swapindex{computational}{chemistry} and 
\swapindex{molecular}{modeling}. 
{\em Computational chemistry} is just a name to indicate the use of 
computational techniques in chemistry, ranging from quantum mechanics 
of molecules to dynamics of large complex molecular aggregates. {\em 
Molecular modeling} indicates the general process of describing 
complex chemical systems in terms of a realistic atomic model, with the 
goal being to understand and predict macroscopic properties based on detailed 
knowledge on an atomic scale. Often, molecular modeling is used to 
design new materials, for which the accurate prediction of physical 
properties of realistic systems is required. 

Macroscopic physical properties can be distinguished by ($a$) {\em static 
equilibrium properties}, such as the binding constant  of an inhibitor to an 
enzyme, the average potential energy of a system, or the radial distribution  
function of a liquid, and ($b$) {\em
dynamic or non-equilibrium properties}, such as the viscosity of a 
liquid, diffusion processes in membranes, the dynamics of phase 
changes, reaction kinetics, or the dynamics of defects in crystals.  
The choice of technique depends on the question asked and on the 
feasibility of the method to yield reliable results at the present 
state of the art. Ideally, the (relativistic) time-dependent 
\swapindex{Schr{\"o}dinger}{equation} 
describes the properties of molecular systems 
with high accuracy, but anything more complex than the equilibrium 
state of a few atoms cannot be handled at this {\em ab initio} level. 
Thus, approximations are necessary; the higher the complexity of a 
system and the longer the time span of the processes of interest is, 
the more severe the required approximations are. At a certain point 
(reached very much earlier than one would wish), the {\em ab initio} 
approach must be augmented or replaced by {\em empirical} 
parameterization of the model used. Where simulations based on physical 
principles of atomic interactions still fail due to the complexity of the 
system, 
molecular modeling is based entirely on a similarity analysis of known 
structural and chemical data. The \normindex{QSAR} methods (Quantitative 
Structure-Activity Relations) and many homology-based protein structure  
predictions belong to the latter category.

Macroscopic properties are always \swapindex{ensemble}{average}s over a 
representative statistical ensemble (either equilibrium or 
non-equilibrium) of molecular systems. For molecular modeling, this has 
two important consequences:
\begin{itemize}
\item   The knowledge of a single structure, even if it is the structure 
        of the global energy minimum, is not sufficient. It is necessary to 
        generate a representative ensemble at a given temperature, in order to 
        compute macroscopic properties. But this is not enough to compute 
        thermodynamic equilibrium properties that are based on free energies, 
        such as phase equilibria, binding constants, solubilities,  relative 
        stability of molecular conformations, etc. The computation of free 
        energies and thermodynamic potentials requires special extensions of 
        molecular simulation techniques.
\item   While molecular simulations, in principle, provide atomic details 
        of the structures and motions, such details are often not relevant for 
        the macroscopic properties of interest. This opens the way to simplify 
        the description of interactions and average over irrelevant details. 
        The science of \swapindex{statistical}{mechanics} 
        provides the theoretical framework 
        for such simplifications. There is a hierarchy of methods ranging from 
        considering groups of atoms as one unit, describing motion in a 
        reduced 
        number of collective coordinates, averaging over solvent molecules 
        with 
        potentials of mean force combined with 
        \swapindex{stochastic}{dynamics}~\cite{Gunsteren90}, to {\em 
        \swapindex{mesoscopic}{dynamics}} 
        describing densities rather than atoms and fluxes 
        as response to thermodynamic gradients rather than velocities or 
        accelerations as response to forces~\cite{Fraaije93}.
\end{itemize}

For the generation of a representative equilibrium ensemble two methods 
are available: ($a$) {\em Monte Carlo simulations} and ($b$) {\em Molecular 
Dynamics simulations}. For the generation of non-equilibrium ensembles 
and for the analysis of dynamic events, only the second method is 
appropriate. While Monte Carlo simulations are more simple than MD (they 
do not require the computation of forces), they do not yield 
significantly better statistics than MD in a given amount of computer time. 
Therefore, MD is the more universal technique. If a starting 
configuration is very far from equilibrium, the forces may be 
excessively large and the MD simulation may fail. In those cases, a 
robust {\em energy minimization} is required. Another reason to perform 
an energy minimization is the removal of all kinetic energy from the 
system: if several ``snapshots'' from dynamic simulations must be compared, 
energy minimization reduces the thermal noise in the structures and  
potential energies so that they can be compared better.

\section{Molecular Dynamics Simulations}
\label{sec:MDsimulations}
MD simulations solve Newton's \swapindex{equations of}{motion} 
for a system of $N$ interacting atoms:
\beq
  m_i \frac{\partial^2 \ve{r}_i}{\partial t^2}  = \ve{F}_i, \;i=1 \ldots N.
\eeq
The forces are the negative derivatives of a potential function $V(\ve{r}_1, 
\ve{r}_2, \ldots, \ve{r}_N)$:
\beq
  \ve{F}_i = - \frac{\partial V}{\partial \ve{r}_i}
\eeq
The equations are solved simultaneously in small time steps. The
system is followed for some time, taking care that the temperature and
pressure remain at the required values, and the coordinates are
written to an output file at regular intervals. The coordinates as a
function of time represent a {\em trajectory} of the system. After
initial changes, the system will usually reach an {\em equilibrium
state}. By averaging over an equilibrium trajectory, many macroscopic
properties can be extracted from the output file.

It is useful at this point to consider the \normindex{limitations} of MD
simulations. The user should be aware of those limitations and always
perform checks on known experimental properties to assess the accuracy
of the simulation. We list the approximations below.

\begin{description}
\item[{\bf The simulations are classical}]\mbox{}\\
Using Newton's equation of motion automatically implies the use of
{\em classical mechanics} to describe the motion of atoms. This is
all right for most atoms at normal temperatures, but there are
exceptions. Hydrogen atoms are quite light and the motion of protons
is sometimes of essential quantum mechanical character. For example, a
proton may {\em tunnel} through a potential barrier in the course of a
transfer over a hydrogen bond. Such processes cannot be properly
treated by classical dynamics! Helium liquid at low temperature is
another example where classical mechanics breaks down. While helium
may not deeply concern us, the high frequency vibrations of covalent
bonds should make us worry! The statistical mechanics of a classical
harmonic oscillator differs appreciably from that of a real quantum
oscillator when the resonance frequency $\nu$ approximates or exceeds
$k_BT/h$. Now at room temperature the wavenumber $\sigma = 1/\lambda =
\nu/c$ at which $h
\nu = k_BT$ is approximately 200 cm$^{-1}$. Thus, all frequencies
higher than, say, 100 cm$^{-1}$ may misbehave in
classical simulations. This means that practically all bond and
bond-angle vibrations are suspect, and even hydrogen-bonded motions as
translational or librational H-bond vibrations are beyond the
classical limit (see \tabref{vibrations}). What can we do?

\begin{table}
\begin{center} 
\begin{tabular}{|l|l|r@{--}rl|}
\dline
                & \mcc{1}{type of}   & \mcc{3}{wavenumber}  \\
type of bond    & \mcc{1}{vibration} & \mcc{3}{(cm$^{-1}$)} \\
\hline
C-H, O-H, N-H   & stretch       & 3000  & 3500  & \\
C=C, C=O        & stretch       & 1700  & 2000  & \\
HOH             & bending       & \mcl{2}{1600} & \\
C-C             & stretch       & 1400  & 1600  & \\
H$_2$CX         & sciss, rock   & 1000  & 1500  & \\ 
CCC             & bending       &  800  & 1000  & \\
O-H$\cdots$O    & libration     &  400  & 700   & \\
O-H$\cdots$O    & stretch       &   50  & 200   & \\
\dline
\end{tabular} 
\end{center} 
\caption[Typical vibrational frequencies.]{Typical vibrational
frequencies (wavenumbers) in molecules and hydrogen-bonded
liquids. Compare $kT/h = 200$ cm$^{-1}$ at 300~K.}
\label{tab:vibrations}
\end{table}

Well, apart from real quantum-dynamical simulations, we can do one
of two things: \\ (a) If we perform MD simulations using harmonic
oscillators for bonds, we should make corrections to the total
internal energy $U = E_{kin} + E_{pot}$ and specific heat $C_V$ (and
to entropy $S$ and free energy $A$ or $G$ if those are
calculated). The corrections to the energy and specific heat of a
one-dimensional oscillator with frequency $\nu$
are:~\cite{McQuarrie76}
\beq
 U^{QM} = U^{cl} +kT \left( \half x - 1 + \frac{x}{e^x-1} \right)
\eeq
\beq
 C_V^{QM} = C_V^{cl} + k \left( \frac{x^2e^x}{(e^x-1)^2} - 1 \right), 
\eeq
where $x=h\nu /kT$. The classical oscillator absorbs too much energy
($kT$), while the high-frequency quantum oscillator is in its ground
state at the zero-point energy level of $\frac{1}{2} h\nu$. \\ (b) We
can treat the bonds (and bond angles) as {\em \normindex{constraint}s}
in the equations of motion. The rationale behind this is that a quantum
oscillator in its ground state resembles a constrained bond more
closely than a classical oscillator. A good practical reason for this
choice is that the algorithm can use larger time steps when the
highest frequencies are removed. In practice the time step can be made
four times as large when bonds are constrained than when they are
oscillators~\cite{Gunsteren77}. {\gromacs} has this option for the
bonds and bond angles.  The flexibility of the latter is
rather essential to allow for the realistic motion and coverage of
configurational space~\cite{Gunsteren82}.

\item[{\bf Electrons are in the ground state}]\mbox{}\\
In MD we use a {\em conservative} force field that is a
function of the positions of atoms only.  This means that the
electronic motions are not considered: the electrons are supposed to
adjust their dynamics instantly when the atomic positions change
(the {\em \normindex{Born-Oppenheimer}} approximation), and remain in
their ground state. This is really all right, almost always. But of
course, electron transfer processes and electronically excited states
can not be treated. Neither can chemical reactions be treated
properly, but there are other reasons to shy away from reactions for
the time being.

\item[{\bf Force fields are approximate}]\mbox{}\\
Force fields \index{force field} provide the forces.
They are not really a part of the
simulation method and their parameters can be modified by the user as the
need arises or knowledge improves. But the form of the forces that can
be used in a particular program is subject to limitations. The force
field that is incorporated in {\gromacs} is described in Chapter 4. In
the present version the force field is pair-additive (apart from
long-range Coulomb forces), it cannot incorporate
polarizabilities, and it does not contain fine-tuning of bonded
interactions. This urges the inclusion of some limitations in this
list below.  For the rest it is quite useful and fairly reliable for
biologically-relevant macromolecules in aqueous solution!

\item[{\bf The force field is pair-additive}]\mbox{}\\
This means that all {\em non-bonded} forces result from the sum of
non-bonded pair interactions. Non pair-additive interactions, the most
important example of which is interaction through atomic
polarizability, are represented by {\em effective pair
potentials}. Only average non pair-additive contributions are
incorporated. This also means that the pair interactions are not pure,
{\ie}, they are not valid for isolated pairs or for situations
that differ appreciably from the test systems on which the models were
parameterized. In fact, the effective pair potentials are not that bad
in practice. But the omission of polarizability also means that
electrons in atoms do not provide a dielectric constant as they
should. For example, real liquid alkanes have a dielectric constant of
slightly more than 2, which reduce the long-range electrostatic
interaction between (partial) charges. Thus, the simulations will
exaggerate the long-range Coulomb terms. Luckily, the next item
compensates this effect a bit.

\item[{\bf Long-range interactions are cut off}]\mbox{}\\
In this version, {\gromacs} always uses a \normindex{cut-off} radius for the
Lennard-Jones interactions and sometimes for the Coulomb interactions
as well. The ``minimum-image convention'' used by {\gromacs} requires that only one image of each
particle in the periodic boundary conditions is considered for a pair
interaction, so the cut-off radius cannot exceed half the box size. That
is still pretty big for large systems, and trouble is only expected
for systems containing charged particles. But then truly bad things can
happen, like accumulation of charges at the cut-off boundary or very
wrong energies! For such systems, you should consider using one of the
implemented long-range electrostatic algorithms, such as 
particle-mesh Ewald~\cite{Darden93,Essmann95}.

\item[{\bf Boundary conditions are unnatural}]\mbox{}\\
Since system size is small (even 10,000 particles is small), a cluster
of particles will have a lot of unwanted boundary with its environment
(vacuum). We must avoid this condition if we wish to simulate a bulk system.
As such, we use periodic boundary conditions to avoid real phase
boundaries. Since liquids are not crystals, something unnatural
remains. This item is mentioned last because it is the
least of the evils. For large systems, the errors are small, but for
small systems with a lot of internal spatial correlation, the periodic
boundaries may enhance internal correlation. In that case, beware of, and
test, the influence of system size. This is especially important when
using lattice sums for long-range electrostatics, since these are known
to sometimes introduce extra ordering.
\end{description}

\section{Energy Minimization and Search Methods}

As mentioned in \secref{Compchem}, in many cases energy minimization
is required. {\gromacs} provides a number of methods for local energy
minimization, as detailed in \secref{EM}.

The potential energy function of a (macro)molecular system is a very
complex landscape (or {\em hypersurface}) in a large number of
dimensions. It has one deepest point, the {\em global minimum} and a
very large number of {\em local minima}, where all derivatives of the
potential energy function with respect to the coordinates are zero and
all second derivatives are non-negative. The matrix of second
derivatives, which is called the {\em Hessian matrix}, has non-negative
eigenvalues; only the collective coordinates that correspond to
translation and rotation (for an isolated molecule) have zero
eigenvalues. In between the local minima there are {\em saddle
points}, where the Hessian matrix has only one negative
eigenvalue. These points are the mountain passes through which the
system can migrate from one local minimum to another.

Knowledge of all local minima, including the global one, and of all
saddle points would enable us to describe the relevant structures and
conformations and their free energies, as well as the dynamics of
structural transitions. Unfortunately, the dimensionality of the
configurational space and the number of local minima is so high that
it is impossible to sample the space at a sufficient number of points
to obtain a complete survey. In particular, no minimization method
exists that guarantees the determination of the global minimum in any
practical amount of time. Impractical methods exist, some much faster
than others~\cite{Geman84}.  However, given a starting configuration,
it is possible to find the {\em nearest local minimum}. ``Nearest'' in
this context does not always imply ``nearest'' in a geometrical sense
({\ie}, the least sum of square coordinate differences), but means the
minimum that can be reached by systematically moving down the steepest
local gradient. Finding this nearest local minimum is all that
{\gromacs} can do for you, sorry! If you want to find other minima and
hope to discover the global minimum in the process, the best advice is
to experiment with temperature-coupled MD: run your system at a high
temperature for a while and then quench it slowly down to the required
temperature; do this repeatedly!  If something as a melting or glass
transition temperature exists, it is wise to stay for some time
slightly below that temperature and cool down slowly according to some
clever scheme, a process called {\em simulated annealing}. Since no
physical truth is required, you can use your imagination to speed up this
process. One trick that often works is to make hydrogen atoms heavier
(mass 10 or so): although that will slow down the otherwise very rapid
motions of hydrogen atoms, it will hardly influence the slower motions
in the system, while enabling you to increase the time step by a factor
of 3 or 4. You can also modify the potential energy function during
the search procedure, {\eg} by removing barriers (remove dihedral
angle functions or replace repulsive potentials by {\em soft-core}
potentials~\cite{Nilges88}), but always take care to restore the
correct functions slowly. The best search method that allows rather
drastic structural changes is to allow excursions into
four-dimensional space~\cite{Schaik93}, but this requires some extra
programming beyond the standard capabilities of {\gromacs}.

Three possible energy minimization methods are:
\begin{itemize}
\item   Those that require only function evaluations. Examples are the 
        simplex method and its variants. A step is made on the basis of the 
        results of previous evaluations. If derivative information is 
        available, such methods are inferior to those that use 
        this information.
\item   Those that use derivative information. Since the partial 
        derivatives of the potential energy with respect to all 
        coordinates are known in MD programs (these are equal to minus 
        the forces) this class of methods is very suitable as modification 
        of MD programs.
\item   Those that use second derivative information as well. These methods 
        are superior in their convergence properties near the minimum: a 
        quadratic potential function is minimized in one step! The problem 
        is that for $N$ particles a $3N\times 3N$ matrix must be computed, 
        stored, and inverted. Apart from the extra programming to obtain 
        second derivatives, for most systems of interest this is beyond the 
        available capacity. There are intermediate methods that build up the 
        Hessian matrix on the fly, but they also suffer from excessive 
        storage requirements. So {\gromacs} will shy away from this class 
        of methods.
\end{itemize}


The {\em steepest descent} method, available in {\gromacs}, is of the
second class. It simply takes a step in the direction of the negative
gradient (hence in the direction of the force), without any
consideration of the history built up in previous steps. The step size
is adjusted such that the search is fast, but the motion is always
downhill. This is a simple and sturdy, but somewhat stupid, method:
its convergence can be quite slow, especially in the vicinity of the
local minimum! The faster-converging {\em conjugate gradient method}
(see {\eg} \cite{Zimmerman91}) uses gradient information from previous
steps. In general, steepest descents will bring you close to the
nearest local minimum very quickly, while conjugate gradients brings
you {\em very} close to the local minimum, but performs worse far away
from the minimum. {\gromacs} also supports the L-BFGS minimizer, which
is mostly comparable to {\em conjugate gradient method}, but in some 
cases converges faster.

% LocalWords:  Schr dinger initio parameterization QSAR equilibria solubilities
% LocalWords:  mesoscopic BT wavenumber rl HOH CX sciss CCC libration kT minima
% LocalWords:  wavenumbers polarizabilities polarizability parameterized BFGS
% LocalWords:  alkanes Compchem librational configurational Ewald
% LocalWords:  hypersurface
