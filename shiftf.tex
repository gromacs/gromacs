\subsection{Shift or Switch functions.}
In the {\gromacs} force field the coulomb force can be modified by
a shift function.
There is {\em no} fundamental difference between a switch function 
and a shift function, the switch function is a special case of the 
shift function, which can generally be written as:
\bea
F_s(r)~=&~F(r)		& r < r_1		\\
F_s(r)~=&~F(r)+S(r)	& r_1 \le r < r_c	\\
F_s(r)~=&~0		& r_c \le r	
\eea
When $r_1=0$ this is a traditional shift function, otherwise it acts as a 
switch function. The effective coulomb potential then reads:
\beq
V_s(r)~=~\int_{\infty}^r~F_s(x) dx
\eeq

The {\gromacs} shift function should be smooth at the boundaries, therefore
the following boundary conditions are imposed on the shift function:
\bea
S(r_1)		&=&0		\\
S'(r_1)		&=&0		\\
S(r_c)		&=&-F(r_c)	\\
S'(r_c)		&=&-F'(r_c)
\eea
Because there are four boundary conditions we need to have a function of
four variables, so a 3$^{rd}$ degree polynomial will do:
\beq
S(r)~=~\sum_{n=0}^3 a_i(r - r_1)^n
\eeq
Because of the boundary conditions at $r_1$ the first two terms must vanish,
therefore
\beq
S(r)	~=~	A(r-r_1)^2+B(r-r_1)^3
\eeq
without loss of generality. The constants A and B are easily derived by 
substituting the boundary condition at $r_c$:
\bea
A	&~=~&	-\frac{5r_c~-~2r_1} {r_c^3~(r_c-r_1)^2} \\
B	&~=~&	\frac{4r_c~-~2r_1}{r_c^3~(r_c-r_1)^3}
\eea
When $r_1$ = 0, the total force is
\beq
{\bf F}_s(r)	~=~	\frac{1}{r^2} - 
			\frac{5 r^2}{r_c^4} +
			\frac{4 r^3}{r_c^5}
\eeq
see also fig~\ref{fig:shift}.
\begin{figure}
\centerline{\psfig {figure=shiftf.eps,angle=270,width=8cm}}
\caption {The Shift Function $S(r)$}
\label{fig:shift}
\end {figure}

