% Process from LaTeX via XML to XHTML with
% latexml --destination installguide.xml --xml installguide.tex
% latexmlpost --destination installguide.xhtml --format=xhtml installguide.xml
%
% Crude hack to remove ugly symbols:
% sed -e 's/§//g' -i installguide.xhtml
%
% Strip off header for pasting into the website at
% http://www.gromacs.org/Documentation/Installation_Instructions:
%
% grep -A 99999 "class=\"main\"" installguide.xhtml > installguide_web.xhtml

\documentclass{article}[12pt,a4paper,twoside]
\usepackage{hyperref}
% haven't made these work with LaTeXML yet...
%\usepackage[strings]{underscore}
%\usepackage[english]{babel}

\title{GROMACS installation guide}

% macros to keep style uniform
\newcommand{\gromacs}{GROMACS}
\newcommand{\nvidia}{NVIDIA}
\newcommand{\cuda}{CUDA}
\newcommand{\fftw}{FFTW}
\newcommand{\mkl}{MKL}
\newcommand{\mpi}{MPI}
\newcommand{\threadmpi}{ThreadMPI}
\newcommand{\openmpi}{OpenMPI}
\newcommand{\openmp}{OpenMP}
\newcommand{\openmm}{OpenMM}
\newcommand{\lammpi}{LAM/MPI}
\newcommand{\mpich}{MPICH}
\newcommand{\cmake}{CMake}
\newcommand{\sse}{SSE}
\newcommand{\ssetwo}{SSE2}
\newcommand{\avx}{AVX}
\newcommand{\fft}{FFT}
\newcommand{\blas}{BLAS}
\newcommand{\lapack}{LAPACK}
\newcommand{\vmd}{VMD}
\newcommand{\pymol}{PyMOL}
\newcommand{\grace}{Grace}
%\newcommand{\}{}
%\newcommand{\}{}

% later, make CMake keep this version current for us
\newcommand{\fftwversion}{3.3.2}
\newcommand{\cmakeversion}{2.8.0}
\newcommand{\cudaversion}{3.2}

\begin{document}
\section{Building GROMACS}
These instructions pertain to building \gromacs{} 4.6 beta releases
and newer. For installations instructions for old \gromacs{} versions,
see here
\url{http://www.gromacs.org/Documentation/Installation_Instructions_4.5}.

\section{Prerequisites}

\subsection{Platform}

\gromacs{} can be compiled for any distribution of Linux, Mac OS X,
Windows (native, Cygwin or MinGW), BlueGene, Cray and probably others.
Technically, it can be compiled on any platform with an ANSI C
compiler and supporting libraries, such as the GNU C library. It can
even compile on an iPhone! Later, there will be a detailed list of
hardware, platform and compilers upon which we do build and regression
testing.

\subsection{Compiler}

\gromacs{} requires an ANSI C compiler that complies with the C89
standard. For best performance, the \gromacs{} team strongly
recommends you get the most recent version of your preferred compiler
for your platform (e.g. GCC 4.7 or Intel 12.0 or newer on x86
hardware). There is a large amount of \gromacs{} code introduced in
version 4.6 that depends on effective compiler optimization to get
high performance - the old assembly-language routines have all
gone. For other platforms, use the vendor's compiler, and check for
specialized information below.

\subsubsection{Running in parallel}

\gromacs{} can run in parallel on multiple cores of a single
workstation using its built-in \threadmpi. No user action is required
in order to enable this.

If you wish to use the excellent new native GPU support in \gromacs,
\nvidia{}'s \cuda{}
\url{http://www.nvidia.com/object/cuda_home_new.html} version
\cudaversion{} software development kit is required, and the latest
version is encouraged. \nvidia{} GPUs with at least \nvidia{} compute
capability 2.0 are required, e.g. Fermi or Kepler cards.

The GPU support from \gromacs{} version 4.5 using \openmm{}
\url{https://simtk.org/home/openmm} is still available, also requires
\cuda{}, and remains the only hardware-based acceleration available
for implicit solvent simulations in \gromacs{}. This parallelization
path may not be maintained in the future.

If you wish to run in parallel on multiple machines across a network,
you will need to have
\begin{itemize}
\item an \mpi{} library installed that supports the \mpi{} 1.3
  standard, and
\item wrapper compilers that will compile code using that library.
\end{itemize}
The \gromacs{} team recommends \openmpi{}
\url{http://www.open-mpi.org/} version 1.4.1 (or higher), \mpich{}
\url{http://www.mpich.org/} version 1.4.1 (or higher), or your
hardware vendor's \mpi{} installation. The most recent version of
either of this is likely to be the best. \lammpi{}
\url{http://www.lam-mpi.org/} may work, but since it has been
deprecated for years, it is not supported.

In some cases, \openmp{} parallelism is an advantage for \gromacs{},
but support for this is generally built into your compiler and you
need to make no advance preparation for this. The performance gain you
might achieve can vary with the compiler.

It is important to examine how you will use \gromacs{} and upon what
hardware and with what compilers in deciding which parallelization
paths to make available. Testing the performance of different options
is unfortunately mandatory. The days of being able to just build and
run '\verb+mdrun+' and get decent performance by default on your
hardware are long gone. \gromacs{} will always run correctly, and does
a decent job of trying to maximize your performance, but if you want
to approach close to the optimum, you will need to do some work for
it!

\subsection{CMake}

From version 4.6, \gromacs{} has moved to use the build system
\cmake{}. The previous build system that used \texttt{configure} from
the GNU autotools package has been removed permanently. \cmake{}
permits the \gromacs{} team to support a very wide range of hardware,
compilers and build configurations while continuing to provide the
portability, robustness and performance for which \gromacs{} is known.

\gromacs{} requires \cmake{} version \cmakeversion{} or higher. Lower
versions will not work. You can check whether \cmake{} is installed,
and what version it is, with \texttt{cmake --version}. If you need to
install \cmake{}, then first check whether your platform's package
management system provides a suitable version, or visit
\url{http://www.cmake.org/cmake/help/install.html} for pre-compiled
binaries, source code and installation instructions. The \gromacs{}
team recommends you install the most recent version of \cmake{} you
can.

\subsection{Fast Fourier Transform library}

Many simulations in \gromacs{} make extensive use of Fourier transforms,
and a software library to perform these is always required. We
recommend \fftw{} \url{http://www.fftw.org/} (version 3 or higher
only) or Intel's \mkl{}
\url{http://software.intel.com/en-us/intel-mkl}.

\subsubsection{\fftw{}}

\fftw{} is likely to be available for your platform via its package
management system, but there can be compatibility and significant
performance issues associated with these packages. In particular,
\gromacs{} simulations are normally run in single floating-point
precision whereas the default \fftw{} package is normally in double
precision, and good compiler options to use for \fftw{} when linked to
\gromacs{} may not have been used. Accordingly, the \gromacs{} team
recommends either
\begin{itemize}
\item that you permit the \gromacs{} installation to download and
  build \fftw{} \fftwversion{} from source automatically
  for you, or
\item that you build \fftw{} from the source code.
\end{itemize}

If you build \fftw{} from source yourself, get the most recent version
and follow its installation guide
(e.g. \url{http://www.fftw.org/fftw3_doc/Installation-and-Customization.html}). Choose
the precision (i.e. single or float vs double) to match what you will
later require for \gromacs{}. There is no need to compile with
threading or \mpi{} support, but it does no harm. On x86 hardware,
compile \emph{only} with \texttt{--enable-sse2} (regardless of
precision) even if your processors can take advantage of \avx{}
extensions to \sse{}. The way \gromacs{} uses Fourier transforms
cannot take advantage of this feature in \fftw{} because of memory
system performance limitations, it can degrade performance by around
20\%, and there is no way for \gromacs{} to require the use of the
\ssetwo{} at run time if \avx{} support has been compiled into \fftw{}.

\subsubsection{\mkl{}}

Using \mkl{} requires a set of linker flags that \gromacs{} is not
able to detect for you, so setting up optimal linking is tricky at the
moment. Need better documentation later.

\subsection{Optional build components}

\begin{itemize}
\item A hardware-optimized \blas{} or \lapack{} library is useful for
  some of the \gromacs{} utilities, but is not needed for running
  simulations.
\item The built-in \gromacs{} trajectory viewer \texttt{ngmx} requires
  X11 and Motif/Lesstif libraries and header files. Generally, the
  \gromacs{} team recommends you use third-party software for
  visualization, such as \vmd{}
  \url{http://www.ks.uiuc.edu/Research/vmd/} or \pymol{}
  \url{http://www.pymol.org/}.
\end{itemize}

\section{Doing a build of \gromacs}

This section will cover a general build of \gromacs{} with \cmake{},
but it is not an exhaustive discussion of how to use \cmake{}. There
are many resources available on the web, which we suggest you search
for when you encounter problems not covered here. The material below
applies specifically to builds on Unix-like systems, including Linux,
Mac OS X, MinGW and Cygwin. For other platforms, see the specialist
instructions below.

\subsection{Configuring with \cmake{}}

\cmake{} will run many tests on your system and do its best to work
out how to build \gromacs{} for you. If you are building \gromacs{} on
hardware that is identical to that where you will run \texttt{mdrun},
then you can be sure that the defaults will be pretty good. Howver, if
you want to control aspects of the build, there's plenty of things you
can set, too.

The best way to use \cmake{} to configure \gromacs{} is to do an
``out-of-source'' build, by making another directory from which you
will run \cmake{}. This can be a subdirectory or not, it doesn't
matter. It also means you can never corrupt your source code by trying
to build it! So, the only required argument on the \cmake{} command
line is the name of the directory containing the
\texttt{CMakeLists.txt} file of the code you want to build. For
example, download the source tarball and use
% TODO: keep up to date as we approach real releases!
\begin{verbatim}
$ tar xfz gromacs-4.6-beta1-src.tgz
$ cd gromacs-4.6-beta1
$ mkdir build-cmake
$ cd build-cmake
$ cmake ..
\end{verbatim}

You will see \texttt{cmake} report the results of a large number of
tests on your system made by \cmake{} and by \gromacs{}. These are
written to the \cmake{} cache, kept in \texttt{CMakeCache.txt}. You
can edit this file by hand, but this is not recommended because it is
easy to reach an inconsistent state. You should not attempt to move or
copy this file to do another build, because the paths are hard-coded
within it. If you mess things up, just delete this file and start
again with '\verb+cmake+'.

If there's a serious problem detected at this stage, then you will see
a fatal error and some suggestions for how to overcome it. If you're
not sure how to deal with that, please start by searching on the web
(most computer problems already have known solutions!) and then
consult the \texttt{gmx-users} mailing list. There are also
informational warnings that you might like to take on board or
not. Piping the output of \texttt{cmake} through \texttt{less} or
\texttt{tee} can be useful, too.

\cmake{} works in an iterative fashion, re-running each time a setting
is changed to try to make sure other things are consistent. Once
things seem consistent, the iterations stop. Once \texttt{cmake}
returns, you can see all the settings that were chosen and information
about them by using 
\begin{verbatim}
$ ccmake ..
\end{verbatim}
Check out \url{http://www.cmake.org/cmake/help/runningcmake.html} for
general advice on what you are seeing and how to navigate and change
things. The settings you might normally want to change are already
presented. If you make any changes, then \texttt{ccmake} will notice
that and require that you re-configure (using '\verb+c+'), so that it
gets a chance to make changes that depend on yours and perform more
checking. This might require several configuration stages when you are
using \texttt{ccmake} - when you are using \texttt{cmake} the
iteration is done behind the scenes.

A key thing to consider here is the setting of
\texttt{GMX\_INSTALL\_PREFIX}. You will need to be able to write to this
directory in order to install \gromacs{} later, and if you change your
mind later, changing it in the cache triggers a full re-build,
unfortunately. So if you do not have super-user privileges on your
machine, then you will need to choose a sensible location within your
home directory for your \gromacs{} installation.

When \texttt{cmake} or \texttt{ccmake} have completed iterating, the
cache is stable and a build tree can be generated, with '\verb+g+' in
\texttt{ccmake} or automatically with \texttt{cmake}.

\subsection{Using CMake command-line options}
Once you become comfortable with setting and changing options, you
may know in advance how you will configure GROMACS. If so, you can
speed things up by invoking \texttt{cmake} with a command like:
\begin{verbatim}
$ cmake .. -DGMX_GPU=ON -DGMX_MPI=ON -DGMX_INSTALL_PREFIX=/home/marydoe/programs
\end{verbatim}
to build with GPUs, MPI and install in a custom location. You can even
save that in a shell script to make it even easier next time. You can
also do this kind of thing with \texttt{ccmake}, but you should avoid
this, because the options set with '\verb+-D+' will not be able to be
changed interactively in that run of \texttt{ccmake}.

\subsection{CMake advanced options}
The options that can be seen with \texttt{ccmake} are ones that we
think a reasonable number of users might want to consider
changing. There are a lot more options available, which you can see by
toggling the advanced mode in \texttt{ccmake} on and off with
'\verb+t+'. Even there, most of the variables that you might want to
change have a '\verb+CMAKE_+' or '\verb+GMX_+' prefix.

\subsection{Helping CMake find the right libraries/headers/programs}

If libraries are installed in non-default locations their location can
be specified using the following environment variables:
\begin{itemize}
\item \texttt{CMAKE\_INCLUDE\_PATH} for header files
\item \texttt{CMAKE\_LIBRARY\_PATH} for libraries
\item \texttt{CMAKE\_PREFIX\_PATH} for header, libraries and binaries
  (e.g. '\verb+/usr/local+').
\end{itemize}
The respective '\verb+include+', '\verb+lib+', or '\verb+bin+' is
appended to the path. For each of these variables, a list of paths can
be specified (on Unix seperated with ":"). Note that these are
enviroment variables (and not \cmake{} command-line arguments) and in
a '\verb+bash+' shell are used like:
\begin{verbatim}
$ CMAKE_PREFIX_PATH=/opt/fftw:/opt/cuda cmake ..
\end{verbatim}

The \texttt{CC} and \texttt{CXX} environment variables are also useful
for indicating to \cmake{} which compilers to use, which can be very
important for maximising \gromacs{} performance. Similarly,
\texttt{CFLAGS}/\texttt{CXXFLAGS} can be used to pass compiler
options, but note that these will be appended to those set by
\gromacs{} for your build platform and build type. You can customize
some of this with advanced options such as \texttt{CMAKE\_C\_FLAGS}
and its relatives.

See also: \url{http://cmake.org/Wiki/CMake_Useful_Variables#Environment_Variables}

\subsection{CMake advice during the GROMACS 4.6 beta phase}
We'd like users to have the ability to change any setting and still
have the \cmake{} cache stable; ie. not have things you set
mysteriously change, or (worse) the whole thing breaks. We're not
there yet. If you know in advance you will want to use a particular
setting, set that on the initial \texttt{cmake} command line. If you
have to change compilers, do that there, or immediately afterwards in
\texttt{ccmake}. Gross changes like GPU or shared libraries on/off are
more likely to work if you do them on the initial command line,
because that's how we've been doing it while developing and
testing. If you do make a mess of things, there's a great thing about
an out-of-source build - you can just do '\verb+rm -rf *+' and start
again. Easy!

We are interested in learning how you managed to break things. If you
can reproducibly reach a state where \cmake{} can't proceed, or
subsequent compilation/linking/running fails, then we need to know so
we can fix it!

\subsection{Native GPU acceleration}
If you have the \cuda{} SDK installed, you can use \cmake{}
with:
\begin{verbatim}
cmake .. -DGMX_GPU=ON -DCUDA_TOOLKIT_ROOT_DIR=/usr/local/cuda
\end{verbatim}
(or whichever path has your installation). Note that this will require
a working C++ compiler, and in some cases you might need to handle
this manually, e.g. with the advanced option
\texttt{CUDA\_NVCC\_HOST\_COMPILER}.

More documentation needed here - particular discussion of fiddly
details on Windows, Linux and Mac required. Not all compilers on all
systems can be made to work.

\subsection{Static linking}
Dynamic linking of the \gromacs{} executables will lead to a smaller
disk footprint when installed, and so is the default. However, on some
hardware or under some circumstances you might need to do static
linking. To link \gromacs{} binaries statically against the internal
\gromacs{} libraries, set \texttt{BUILD\_SHARED\_LIBS=OFF}. To link
statically against external libraries as well, the
\texttt{GMX\_PREFER\_STATIC\_LIBS=ON} option can be used. Note, that
in general \cmake{} picks up whatever is available, so this option
only instructs \cmake{} to prefer static libraries when both static
and shared are available. If no static version of an external library
is available, even when the aforementioned option is ON, the shared
library will be used. Also note, that the resulting binaries will
still be dynamically linked against system libraries if that is all
that is available.

\subsection{Suffixes for binaries and libraries}
It is sometimes convenient to have different versions of the same
\gromacs{} libraries installed. The most common use cases have been
single and double precision, and with and without \mpi{}. By default,
\gromacs{} will suffix binaries and libraries for such builds with
'\verb+_d+' for double precision and/or '\verb+_mpi+' for \mpi{} (and
nothing otherwise). This can be controlled manually with
\texttt{GMX\_DEFAULT\_SUFFIX}, \texttt{GMX\_BINARY\_SUFFIX} and
\texttt{GMX\_LIBRARY\_SUFFIX}.

\subsection{Building \gromacs{}}

Once you have a stable cache, you can build \gromacs{}. If you're not
sure the cache is stable, you can re-run \verb+cmake ..+ or
\verb+ccmake ..+' to see. Then you can run \texttt{make} to start the
compilation. Before actual compilation starts, \texttt{make} checks
that the cache is stable, so if it isn't you will see \cmake{} run
again.

So long as any changes you've made to the configuration are sensible,
it is expected that the \texttt{make} procedure will always complete
successfully. The tests \gromacs{} makes on the settings you choose
are pretty extensive, but there are probably a few cases we haven't
thought of yet. Search the web first for solutions to problems,
but if you need help, ask on \texttt{gmx-users}, being sure to provide
as much information as possible about what you did, the system you are
building on, and what went wrong.

If you have a multi-core or multi-CPU machine with \texttt{N}
processors, then using
\begin{verbatim}
$ make -j N
\end{verbatim}
will generally speed things up by quite a bit.

\subsection{Installing \gromacs{}}

Finally, \texttt{make install} will install \gromacs{} in the
directory given in \texttt{GMX\_INSTALL\_PREFIX}. If this is an system
directory, then you will need permission to write there, and you
should use super-user privileges only for \texttt{make install} and
not the whole procedure.

\subsection{Getting access to \gromacs{} after installation}

\gromacs{} installs the script \texttt{GMXRC} in the \texttt{bin}
subdirectory of the installation directory
(e.g. \texttt{/usr/local/gromacs/bin/GMXRC}), which you should source
from your shell:
\begin{verbatim}
$ source your-installation-prefix-here/bin/GMXRC
\end{verbatim}

It will detect what kind of shell you are running and
set up your environment for using \gromacs{}. You may wish to arrange
for your login scripts to do this automatically; please search the web
for instructions on how to do this for your shell.

\subsection{Testing \gromacs{} for correctness}
TODO install and use regression set

\subsection{Testing \gromacs{} for performance}
TODO benchmarks

\subsection{Having difficulty?}

You're not alone, this can be a complex task. If you encounter a
problem with installing \gromacs{}, then there are a number of
locations where you can find assistance. It is recommended that you
follow these steps to find the solution:

\begin{enumerate}
\item Read the installation instructions again, taking note that you
  have followed each and every step correctly.
\item Search the \gromacs{} website and users emailing list for
  information on the error.
\item Search the internet using a search engine such as Google.
\item Post to the \gromacs{} users emailing list \texttt{gmx-users}
  for assistance. Be sure to give a full description of what you have
  done and why you think it didn't work. Give details about the system
  on which you are installing. Copy and paste your command line and as
  much of the output as you think might be relevant - certainly from
  the first indication of a problem. Describe the machine and
  operating system you are running on. People who might volunteer to
  help you do not have time to ask you interactive detailed follow-up
  questions, so you will get an answer faster if you provide as much
  information as you think could possibly help.
\end{enumerate}

\section{Special instructions for some platforms}

\subsection{Building on Windows}

Building on Cygwin/MinGW/etc. works just like Unix. Please see the
instructions above.

Building on Windows using native compilers is rather similar to
building on Unix, so please start by reading the above. Then, download
and unpack the GROMACS source archive. The UNIX-standard
\texttt{.tar.gz} format can be managed on Windows, but you may prefer
to browse \url{ftp://ftp.gromacs.org/pub/gromacs} to obtain a
\texttt{.zip} format file, which doesn't need any external tools to
unzip on recent Windows systems. Make a folder in which to do the
out-of-source build of \gromacs{}. For example, make it within the
folder unpacked from the source archive, and call it ``build-cmake''.

Next, you need to open a command shell. If you do this from within
your IDE (e.g. Microsoft Visual Studio), it will configure the
environment for you. If you use a normal Windows command shell, then
you will need to either set up the environment to find your compilers
and libraries yourself, or run the \texttt{vcvarsall.bat} batch script
provided by MSVC (just like sourcing a bash script under
Unix). Presumably Intel's IDE has a similar functionality.

Within that command shell, change to the folder you created above. Run
\verb+cmake ..+, where the folder you point \cmake{} towards is the
folder created by the \gromacs{} installer. Resolve issues as
above. You will probably make your life easier and faster by using the
new facility to download and install \fftw{} automatically. After the
initial run of \verb+cmake+, you may wish to use \verb+cmake+,
\verb+ccmake+ or the GUI version of \cmake{} until your configuration
is complete.

To compile \gromacs{}, you then use \verb+cmake --build .+ so the
right tools get used.

\subsection{Building on Cray}

Probably you need to build static libraries only? Volunteer needed.

\subsection{Building on BlueGene}

\subsubsection{BlueGene/P}

Mark to write later. There is currently no native acceleration on this
platform, but the default plain C kernels will work. There will be
accelerated Verlet kernels in 4.6 release.

\subsubsection{BlueGene/Q}

There is currently no native acceleration on this platform, but the
default plain C kernels will work.

Only static linking with XL compilers is supported by \gromacs{}. Dynamic
linking would be supported by the architecture and \gromacs{}, but has no
advantages other than disk space, and is generally discouraged on
BlueGene for performance reasons.

Computation on BlueGene floating-point units is always done in
double-precision. However, single-precision builds of \gromacs{} are
still normal and encouraged. The BlueGene hardware automatically
converts values stored in single precision in memory to double
precision in registers for computation, converts the results back to
single precision correctly, and does so for no additional cost. As
with other platforms, doing the whole computation in double precision
normally shows no improvement in accuracy and costs twice as much time
moving memory around.

You need to arrange for FFTW to be installed correctly, following the
above instructions.

mpicc is used for compiling and linking. This can make it awkward to
attempt to use IBM's optimized BLAS/LAPACK called ESSL. Since mdrun is
the only part of \gromacs{} that should normally run on the compute
nodes, and there is nearly no need for linear algebra support for
mdrun, it is recommended to use the \gromacs{} built-in linear algebra
routines.

cmake .. -DCMAKE_TOOLCHAIN_FILE=BlueGeneQ-static-XL-C -DCMAKE_PREFIX_PATH=/your/fftw/installation/prefix
make mdrun
make install-mdrun

It is possible to configure and make the remaining \gromacs{} tools
with the compute node toolchain, but as none of those tools are
\mpi{}-aware, this would not normally be useful. Instead, these should
be planned to run on the login node, and a seperate \gromacs{}
installation performed for that using the login node's toolchain.

\section{Tested platforms}

While it is our best belief that \gromacs{} will build and run pretty
much everywhere, it's important that we tell you where we really know
it works because we've tested it. We do test on Linux, Windows, and
Mac with a range of compilers and libraries for a range of our
configuration options. Every commit in our \texttt{git} source code
repository is tested on ... We test irregularly on...

Contributions to this section are welcome.

Later we might set up the ability for users to contribute test results
to Jenkins.

\section{Other issues}

The \gromacs{} utility programs often write data files in formats
suitable for the \grace{} plotting tool, but it is straightforward to
use these files in other plotting programs, too.

\end{document}
