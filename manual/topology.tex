%
% This file is part of the GROMACS molecular simulation package.
%
% Copyright (c) 2013, by the GROMACS development team, led by
% Mark Abraham, David van der Spoel, Berk Hess, and Erik Lindahl,
% and including many others, as listed in the AUTHORS file in the
% top-level source directory and at http://www.gromacs.org.
%
% GROMACS is free software; you can redistribute it and/or
% modify it under the terms of the GNU Lesser General Public License
% as published by the Free Software Foundation; either version 2.1
% of the License, or (at your option) any later version.
%
% GROMACS is distributed in the hope that it will be useful,
% but WITHOUT ANY WARRANTY; without even the implied warranty of
% MERCHANTABILITY or FITNESS FOR A PARTICULAR PURPOSE.  See the GNU
% Lesser General Public License for more details.
%
% You should have received a copy of the GNU Lesser General Public
% License along with GROMACS; if not, see
% http://www.gnu.org/licenses, or write to the Free Software Foundation,
% Inc., 51 Franklin Street, Fifth Floor, Boston, MA  02110-1301  USA.
%
% If you want to redistribute modifications to GROMACS, please
% consider that scientific software is very special. Version
% control is crucial - bugs must be traceable. We will be happy to
% consider code for inclusion in the official distribution, but
% derived work must not be called official GROMACS. Details are found
% in the README & COPYING files - if they are missing, get the
% official version at http://www.gromacs.org.
%
% To help us fund GROMACS development, we humbly ask that you cite
% the research papers on the package. Check out http://www.gromacs.org.

\chapter{Topologies}
\label{ch:top}
\section{Introduction}
{\gromacs} must know on which atoms and combinations of atoms the
various contributions to the potential functions (see
\chref{ff}) must act. It must
also know what \normindex{parameter}s must be applied to the various
functions. All this is described in the {\em \normindex{topology}} file
{\tt *.top}, which lists the {\em constant attributes} of each atom.
There are many more atom types than elements, but only atom types
present in biological systems are parameterized in the force field,
plus some metals, ions and silicon. The bonded and special
interactions are determined by fixed lists that are included in the
topology file. Certain non-bonded interactions must be excluded (first
and second neighbors), as these are already treated in bonded
interactions.  In addition, there are {\em dynamic attributes} of
atoms - their positions, velocities and forces. These do not
strictly belong to the molecular topology, and are stored in the
coordinate file {\tt *.gro} (positions and velocities), or trajectory
file {\tt *.trr} (positions, velocities, forces).

This chapter describes the setup of the topology file, the
{\tt *.top} file and the database files: what the parameters
stand for and how/where to change them if needed.
First, all file formats are explained.
Section \ssecref{fffiles} describes the organization of
the files in each force field.

{\bf Note:} if you construct your own topologies, we encourage you
to upload them to our topology archive at {\wwwpage}! Just imagine
how thankful you'd have been if your topology had been available
there before you started. The same goes for new force fields or
modified versions of the standard force fields - contribute them
to the force field archive!

\section{Particle type}
\label{sec:parttype}

In {\gromacs}, there are three types of \normindex{particle}s, see
\tabref{ptype}. Only regular atoms and virtual interaction sites are used
in {\gromacs}; shells are necessary for
polarizable models like the Shell-Water models~\cite{Maaren2001a}.

\begin{table}
\centerline{
\begin{tabular}{|l|c|}
\dline
Particle        		& Symbol        \\
\hline
\seeindex{atom}{particle}s      & A   \\
\seeindex{shell}{particle}s     & S   \\
\normindex{virtual interaction sites}	& V (or D)   \\
\dline
\end{tabular}
}
\caption{Particle types in {\gromacs}}
\label{tab:ptype}
\end{table}

\subsection{Atom types}
\label{subsec:atomtype}

Each force field defines a set of \swapindex{atom}{type}s,
which have a characteristic name or number, and mass (in
a.m.u.). These listings are found in the {\tt atomtypes.atp}
file (.atp = {\bf a}tom {\bf t}ype {\bf p}arameter file).
Therefore, it is in this file that you can begin to change
and/or add an atom type. A sample from the deprecated
{\tt gmx.ff} force field is listed below.

{\small
\begin{verbatim}
    O  15.99940 ;     carbonyl oxygen (C=O)
   OM  15.99940 ;     carboxyl oxygen (CO-)
   OA  15.99940 ;     hydroxyl oxygen (OH)
   OW  15.99940 ;     water oxygen
    N  14.00670 ;     peptide nitrogen (N or NH)
   NT  14.00670 ;     terminal nitrogen (NH2)
   NL  14.00670 ;     terminal nitrogen (NH3)
  NR5  14.00670 ;     aromatic N (5-ring,2 bonds)
 NR5*  14.00670 ;     aromatic N (5-ring,3 bonds)
   NP  14.00670 ;     porphyrin nitrogen
    C  12.01100 ;     bare carbon (peptide,C=O,C-N)
  CH1  13.01900 ;     aliphatic CH-group
  CH2  14.02700 ;     aliphatic CH2-group
  CH3  15.03500 ;     aliphatic CH3-group
\end{verbatim}}

{\bf Note:} {\gromacs} makes use of the atom types as a name, {\em
not} as a number (as {\eg} in {\gromos}).

%Atomic detail is used except for hydrogen atoms bound to (aliphatic)
%carbon atoms, which are treated as {\em \swapindex{united}{atom}s}. No
%special \normindex{hydrogen-bond} term is included. {\bf Note} that other force field
%like OPLS/AA, CHARMM, and AMBER use all atoms.

%\subsection{Nucleus}
%{\em Necessary for \normindex{polarisability}, not implemented yet.}
%
%\subsection{Shell}
%{\em Necessary for polarisability, not implemented yet.}
%
%\subsection{Bond shell}
%{\em Necessary for polarisability, not implemented yet.}

\subsection{Virtual sites}
\label{sec:vsitetop}
Some force fields use \normindex{virtual interaction sites}
(interaction sites that are constructed from other particle positions)
on which certain interactions are located
({\eg} on benzene rings, to reproduce the correct
\normindex{quadrupole}). This is described in~\secref{virtual_sites}.

To make virtual sites in your system, you should include a section
{\tt [~virtual_sites?~]} (for backward compatibility the old name
{\tt [~dummies?~]} can also be used) in your topology file,
where the `{\tt ?}' stands
for the number constructing particles for the virtual site. This will be
`{\tt 2}' for type 2, `{\tt 3}' for types 3, 3fd, 3fad and 3out and
`{\tt 4}' for type 4fdn. The last of these replace an older 4fd type (with the `type' value 1) 
that could occasionally be unstable; while it is still supported internally
in the code, the old 4fd type should not be used in new input files.
 The different types are explained
in~\secref{virtual_sites}.

Parameters for type 2 should look like this:
{\small
\begin{verbatim}
[ virtual_sites2 ]
; Site  from        funct  a
5       1     2     1      0.7439756
\end{verbatim}}

for type 3 like this:
{\small
\begin{verbatim}
[ virtual_sites3 ]
; Site  from               funct   a          b
5       1     2     3      1       0.7439756  0.128012
\end{verbatim}}

for type 3fd like this:
{\small
\begin{verbatim}
[ virtual_sites3 ]
; Site  from               funct   a          d
5       1     2     3      2       0.5        -0.105
\end{verbatim}}

for type 3fad like this:
{\small
\begin{verbatim}
[ virtual_sites3 ]
; Site  from               funct   theta      d
5       1     2     3      3       120        0.5
\end{verbatim}}

for type 3out like this:
{\small
\begin{verbatim}
[ virtual_sites3 ]
; Site  from               funct   a          b          c
5       1     2     3      4       -0.4       -0.4       6.9281
\end{verbatim}}

for type 4fdn like this:
{\small
\begin{verbatim}
[ virtual_sites4 ]
; Site  from                      funct   a          b          c
5       1     2     3     4       2       1.0        0.9       0.105
\end{verbatim}}

This will result in the construction of a virtual site, number 5
(first column `{\tt Site}'), based on the positions of the atoms
whose indices are 1 and 2 or 1, 2 and 3 or 1, 2, 3 and 4 (next two,
three or four columns `{\tt from}') following the rules determined by the function number
(next column `{\tt funct}') with the parameters specified (last one,
two or three columns `{\tt a b} . .'). Obviously, the atom numbers
(including virtual site number) depend
on the molecule. It may be instructive to study the topologies for
TIP4P or TIP5P water models that are included with the {\gromacs} distribution.

{\bf Note} that if any constant bonded interactions are defined between
virtual sites and/or normal atoms, they will be removed by {\tt grompp}
(unless the option {tt -normvsbds} is used).
This removal of bonded interactions is done after generating exclusions,
as the generation of exclusions is based on ``chemically'' bonded interactions.

Virtual sites can be constructed in a more generic way using basic geometric
parameters.  The directive that can be used is {\tt [ virtual_sitesn ]}.  Required
parameters are listed in~\tabref{topfile2}.  An example entry for defining a virtual
site at the center of geometry of a given set of atoms might be:

{\small
\begin{verbatim}
[ virtual_sitesn ]
; Site   funct    from
5        1        1     2     3     4
\end{verbatim}}

\section{Parameter files}

\label{sec:paramfiles}

\subsection{Atoms}
The {\em static} properties (see \tabref{statprop} assigned to the
atom types are assigned based on data in several places.
The mass is listed in {\tt atomtypes.atp}
(see~\ssecref{atomtype}), whereas the charge is listed in {\tt *.rtp}
(.rtp = {\bf r}esidue {\bf t}opology {\bf p}arameter file,
see~\ssecref{rtp}).  This implies that the charges are only defined in
the \normindex{building block}s of amino acids, nucleic acids or
otherwise, as defined by the user. When generating a topology
({\tt *.top}) using the {\tt \normindex{pdb2gmx}} program, the
information from these files is combined.
 
\begin{table}
\centerline{
\begin{tabular}{|l|c|c|}
\dline
Property        & Symbol        & Unit          \\
\hline
Type            & -             & -             \\
Mass            & m             & a.m.u.        \\
Charge          & q             & electron      \\
epsilon         & $\epsilon$    & kJ/mol        \\
sigma           & $\sigma$      & nm            \\
\dline
\end{tabular}
}
\caption{Static atom type properties in {\gromacs}}
\label{tab:statprop}
\end{table}

%The following {\em dynamic} quantities are associated with an atom
%\begin{itemize}
%\item   Position {\bf x}
%\item   Velocity {\bf v}
%\end{itemize}
%These quantities are listed in the coordinate file, {\tt *.gro}
%(see section~\ssecref{grofile}).

\subsection{Non-bonded parameters}
\label{subsec:nbpar}
The \swapindex{non-bonded}{parameter}s consist of the van der Waals 
parameters V ({\tt c6} or $\sigma$, depending on the combination rule) and W 
({\tt c12} or $\epsilon$), as listed in the file {\tt ffnonbonded.itp}, where 
{\tt ptype} is the particle type (see \tabref{ptype}). As with the bonded parameters, entries in {\tt [~*type~]} directives 
are applied to their counterparts in the topology file.  Missing parameters 
generate warnings, except as noted below in section~\ssecref{pairinteractions}.

{\small
\begin{verbatim}
[ atomtypes ]
;name   at.num      mass      charge   ptype         V(c6)        W(c12)
    O        8  15.99940       0.000       A   0.22617E-02   0.74158E-06
   OM        8  15.99940       0.000       A   0.22617E-02   0.74158E-06
   .....

[ nonbond_params ]
  ; i    j func       V(c6)        W(c12)
    O    O    1 0.22617E-02   0.74158E-06
    O   OA    1 0.22617E-02   0.13807E-05
    .....
\end{verbatim}}

{\bf Note} that most of the included force fields also include the {\tt at.num.} column, 
but this same information is implied in the OPLS-AA {\tt bond_type} column.
The interpretation of the parameters V and W depends on the combination rule 
that was chosen in the {\tt [~defaults~]} section of the topology file 
(see~\ssecref{topfile}):
\begin{eqnarray}
\mbox{for combination rule 1}: & &
\begin{array}{llllll}
  \mbox{V}_{ii} & = & C^{(6)}_{i}  & = & 4\,\epsilon_i\sigma_i^{6} &
  \mbox{[ kJ mol$^{-1}$ nm$^{6}$ ]}\\
  \mbox{W}_{ii} & = & C^{(12)}_{i} & = & 4\,\epsilon_i\sigma_i^{12} &
  \mbox{[ kJ mol$^{-1}$ nm$^{12}$ ]}\\
\end{array}
\\
\mbox{for combination rules 2 and 3}: & &
\begin{array}{llll}
  \mbox{V}_{ii} & = & \sigma_i   & \mbox{[ nm ]} \\
  \mbox{W}_{ii} & = & \epsilon_i & \mbox{[ kJ mol$^{-1}$ ]}
\end{array}
\end{eqnarray}
Some or all combinations for different atom types can be given in the 
{\tt [~nonbond_params~]} section, again with parameters V and W as defined 
above. Any combination that is not given will be computed from the parameters 
for the corresponding atom types, according to the \normindex{combination rule}:
\begin{eqnarray}
\mbox{for combination rules 1 and 3}: & &
\begin{array}{lll}
  C^{(6)}_{ij}  & = & \left(C^{(6)}_i\,C^{(6)}_j\right)^{\frac{1}{2}} \\
  C^{(12)}_{ij} & = & \left(C^{(12)}_i\,C^{(12)}_j\right)^{\frac{1}{2}}
\end{array}
\\
\mbox{for combination rule 2}: & &
\begin{array}{lll}
  \sigma_{ij}   & = & \frac{1}{2}(\sigma_i+\sigma_j) \\
  \epsilon_{ij} & = & \sqrt{\epsilon_i\,\epsilon_j}
\end{array}
\end{eqnarray}
When $\sigma$ and $\epsilon$ need to be supplied (rules 2 and 3),
it would seem it is impossible to have a non-zero $C^{12}$ combined
with a zero $C^6$ parameter. However, providing a negative $\sigma$
will do exactly that, such that $C^6$ is set to zero and $C^{12}$ is
calculated normally. This situation represents a special case in reading
the value of $\sigma$, and nothing more.

There is only one set of \normindex{combination rule}s
for Buckingham potentials:
\beq
\begin{array}{rcl}
A_{ij}   &=& \left(A_{ii} \, A_{jj}\right)^{1/2}    \\
B_{ij}   &=& 2 / \left(\frac{1}{B_{ii}} + \frac{1}{B_{jj}}\right)        \\
C_{ij}   &=& \left(C_{ii} \, C_{jj}\right)^{1/2}
\end{array}
\eeq

\subsection{Bonded parameters}
\label{subsec:bondparam}
The \swapindex{bonded}{parameter}s ({\ie} bonds, bond angles, improper and proper
dihedrals) are listed in {\tt ffbonded.itp}.~
% The term {\tt func} is 1 for
% harmonic and 2 for \gromosv{96} bond and angle potentials.
% For the dihedral, this is explained after this listing.
The entries in this database describe, respectively, the atom types
in the interactions, the type of the interaction, and the parameters
associated with that interaction. These parameters are then read
by {\tt \normindex{grompp}} when processing a topology and applied
to the relevant bonded parameters, {\ie} {\tt bondtypes} are applied to
entries in the {\tt [~bonds~]} directive, etc. Any bonded parameter that is
missing from the relevant {\tt [~*type~]} directive generates a fatal error.
The types of interactions are listed in \tabref{topfile2}.
Example excerpts from such files follow:

{\small 
\begin{verbatim}
[ bondtypes ]
  ; i    j func        b0          kb
    C    O    1   0.12300     502080.
    C   OM    1   0.12500     418400.
    ......

[ angletypes ]
  ; i    j    k func       th0         cth
   HO   OA    C    1   109.500     397.480
   HO   OA  CH1    1   109.500     397.480
   ......

[ dihedraltypes ]
  ; i    l func        q0          cq
 NR5*  NR5    2     0.000     167.360
 NR5* NR5*    2     0.000     167.360
 ......

[ dihedraltypes ]
  ; j    k func      phi0          cp   mult
    C   OA    1   180.000      16.736      2
    C    N    1   180.000      33.472      2
    ......

[ dihedraltypes ]
;
; Ryckaert-Bellemans Dihedrals
;
; aj    ak      funct
CP2     CP2     3       9.2789  12.156  -13.120 -3.0597 26.240  -31.495
\end{verbatim}}

%Also in this file are the
%\normindex{Ryckaert-Bellemans}~\cite{Ryckaert78} parameters for the
%CP2-CP2 dihedrals in alkanes or alkane tails with the following
%constants:

%\begin{center}
%(kJ/mol)\\
%\begin{tabular}{llrllrllr}
%$C_0$ & $=$ & $~ 9.28$ & $C_2$ & $=$ & $-13.12$ & $C_4$ & $=$ & $ 26.24$ \\
%$C_1$ & $=$ & $ 12.16$ & $C_3$ & $=$ & $~-3.06$ & $C_5$ & $=$ & $-31.5 $ \\
%\end{tabular}
%\end{center}

%({\bf Note:} The use of this potential implies the exclusion of LJ interactions
%between the first and the last atom of the dihedral, and $\psi$ is defined
%according to the ``\swapindex{polymer}{convention}'' ($\psi_{trans}=0$)).

%So there are three types of dihedrals in the {\gromacs} force field:
%\begin{itemize}
%\item \swapindex{proper}{dihedral} : funct = 1, with mult = multiplicity, so the
%                                   number of possible angles
%\item \swapindex{improper}{dihedral} : funct = 2
%\item Ryckaert-Bellemans dihedral : funct = 3
%\end{itemize}

In the {\tt ffbonded.itp} file, you can add bonded parameters. If you
want to include parameters for new atom types, make sure you define
them in {\tt atomtypes.atp} as well.



\subsection{Intramolecular pair interactions\index{intramolecular pair interaction}}
\label{subsec:pairinteractions}
Extra Lennard-Jones and electrostatic interactions between pairs
of atoms in a molecule can be added in the {\tt [~pairs~]} section of
a molecule definition. The parameters for these interactions can
be set independently from the non-bonded interaction parameters.
In the {\gromos} force fields, pairs are only used
to modify the \normindex{1-4 interaction}s (interactions of atoms
separated by three bonds). In these force fields the 1-4 interactions
are excluded from the non-bonded interactions (see \secref{excl}).

{\small
\begin{verbatim}

[ pairtypes ]
  ; i    j func         cs6          cs12 ; THESE ARE 1-4 INTERACTIONS
    O    O    1 0.22617E-02   0.74158E-06
    O   OM    1 0.22617E-02   0.74158E-06
    .....
\end{verbatim}}

The pair interaction parameters for the atom types
in {\tt ffnonbonded.itp} are listed in the {\tt [~pairtypes~]} section.
The {\gromos} force fields list all these interaction parameters
explicitly, but this section might be empty for force fields like
OPLS that calculate the \normindex{1-4 interaction}s by uniformly scaling the parameters.
Pair parameters that are not present in the {\tt [~pairtypes~]} section
are only generated when {\tt gen-pairs} is set to ``yes'' in the {\tt [~defaults~]}
directive of {\tt forcefield.itp} (see \ssecref{topfile}). 
When {\tt gen-pairs} is set to ``no,'' {\tt \normindex{grompp}}
will give a warning for each pair type for which no parameters are given.

The normal pair interactions, intended for \normindex{1-4 interaction}s,
have function type 1. Function type 2 and the {\tt [~pairs_nb~]} are intended
for free-energy simulations. When determining hydration
free energies, the solute needs to be decoupled from the solvent.
This can be done by adding a B-state topology (see \secref{fecalc})
that uses zero for all solute non-bonded parameters, {\ie} charges and LJ parameters.
However, the free energy difference between the A and
B states is not the total hydration free energy.  One has to
add the free energy for reintroducing the internal Coulomb and 
LJ interactions in the solute when in vacuum. This second step can be combined with
the first step when the Coulomb and LJ interactions within
the solute are not modified. For this purpose, there is a pairs
function type 2, which is identical to function type 1, except
that the B-state parameters are always identical to the A-state
parameters. For searching the parameters in the {\tt [~pairtypes~]} section,
no distinction is made between function type 1 and 2.
The pairs section {\tt [~pairs_nb~]} is intended to replace the non-bonded interaction.
It uses the unscaled charges and the non-bonded LJ parameters;
it also only uses the A-state parameters. {\bf Note} that
one should add exclusions for all atom pairs listed in {\tt [~pairs_nb~]},
otherwise such pairs will also end up in the normal neighbor lists.

Alternatively, this same behavior can be achieved without ever
touching the topology, by using the {\tt couple-moltype}, {\tt
  couple-lambda0}, {\tt couple-lambda1}, and {\tt couple-intramol}
keywords.  See sections \secref{fecalc} and \secref{dgimplement} for
more information.

All three pair types always use plain Coulomb interactions,
even when Reaction-field, PME, Ewald or shifted Coulomb interactions
are selected for the non-bonded interactions.
Energies for types 1 and 2 are written to the energy and log file
in separate ``LJ-14'' and ``Coulomb-14'' entries per energy group pair.
Energies for {\tt [~pairs_nb~]} are added to the ``LJ-(SR)'' and ``Coulomb-(SR)'' terms.

\subsection{Implicit solvation parameters\index{implicit solvation parameters}}
Starting with {\gromacs} 4.5, implicit solvent is supported. A section in the
topology has been introduced to list those parameters:

{\small
\begin{verbatim}
[ implicit_genborn_params ]
; Atomtype  sar     st   pi      gbr      hct
NH1         0.155   1    1.028   0.17063  0.79 ; N
N           0.155   1    1       0.155    0.79 ; Proline backbone N
H           0.1     1    1       0.115    0.85 ; H
CT1         0.180   1    1.276   0.190    0.72 ; C
\end{verbatim}}

In this example the atom type is listed first, followed by five
numbers, and a comment (following a semicolon).

Values in columns 1-3 are not currently used. They pertain to more
elaborate surface area algorithms, the one from Qiu {\em et al.}~\cite{Still97} in
particular.  Column 4 contains the atomic van der Waals radii, which are used
in computing the Born radii. The dielectric offset is specified in
the {\tt *.mdp} file, and gets subtracted from the input van der Waals radii for the different
Born radii methods, as described by Onufriev {\em et al.}~\cite{Case04}.  Column 5 is the 
scale factor for the HCT and OBC models. The values are taken from the Tinker implementation of 
the HCT pairwise scaling method~\cite{Truhlar96}.  This method has been modified such that the
scaling factors have been adjusted to minimize differences between analytical surface areas and
GB using the HCT algorithm.  The scaling is further modified in that it is not applied pairwise
as proposed by Hawkins {\em et al.}~\cite{Truhlar96}, but on a per-atom (rather than a per-pair) 
basis.



\section{Exclusions}
\label{sec:excl}
The \normindex{exclusions} for non-bonded interactions are generated by {\tt
grompp} for neighboring atoms up to a certain number of bonds away, as
defined in the {\tt [~moleculetype~]} section in the topology file
(see \ssecref{topfile}). Particles are considered bonded when they are
connected by ``chemical'' bonds ({\tt [~bonds~]} types 1 to 5, 7 or 8)
or constraints ({\tt [~constraints~]} type 1).
Type 5 {\tt [~bonds~]} can be used to create a \normindex{connection}
between two atoms without creating an interaction.
There is a \normindex{harmonic interaction}
({\tt [~bonds~]} type 6) that does not connect the atoms by a chemical bond.
There is also a second constraint type ({\tt [~constraints~]} type 2)
that fixes the distance, but does not connect
the atoms by a chemical bond.
For a complete list of all these interactions, see \tabref{topfile2}.

Extra exclusions within a molecule can be added manually
in a {\tt [~exclusions~]} section. Each line should start with one
atom index, followed by one or more atom indices. All non-bonded
interactions between the first atom and the other atoms will be excluded.

When all non-bonded interactions within or between groups of atoms need
to be excluded, is it more convenient and much more efficient to use
energy monitor group exclusions (see \secref{groupconcept}).

\section{Constraint algorithms\index{constraint algorithms}}
\label{sec:constraints}
Constraints are defined in the {\tt [~constraints~]} section.
The format is two atom numbers followed by the function type,
which can be 1 or 2, and the constraint distance.
The only difference between the two types is that type 1 is used
for generating exclusions and type 2 is not (see \secref{excl}).
The distances are constrained using the LINCS or the SHAKE algorithm,
which can be selected in the {\tt *.mdp} file.
Both types of constraints can be perturbed in free-energy calculations
by adding a second constraint distance (see \ssecref{constraintforce}).
Several types of bonds and angles (see \tabref{topfile2}) can
be converted automatically to constraints by {\tt grompp}.
There are several options for this in the {\tt *.mdp} file.

We have also implemented the \normindex{SETTLE} algorithm~\cite{Miyamoto92},
which is an analytical solution of SHAKE, specifically for water. 
SETTLE can be selected in the topology file. See, for instance, the
SPC molecule definition:

{\small
\begin{verbatim}
[ moleculetype ]
; molname       nrexcl
SOL             1

[ atoms ]
; nr    at type res nr  ren nm  at nm   cg nr   charge
1       OW      1       SOL     OW1     1       -0.82
2       HW      1       SOL     HW2     1        0.41
3       HW      1       SOL     HW3     1        0.41

[ settles ]
; OW    funct   doh     dhh
1       1       0.1     0.16333

[ exclusions ]
1       2       3
2       1       3
3       1       2
\end{verbatim}}

The {\tt [~settles~]} directive defines the first atom of the water molecule.
The settle funct is always 1, and the distance between O-H and H-H distances
must be given. {\bf Note} that the algorithm can also be used
for TIP3P and TIP4P~\cite{Jorgensen83}.
TIP3P just has another geometry. TIP4P has a virtual site, but since 
that is generated it does not need to be shaken (nor stirred).

\section{\normindex{pdb2gmx} input files}
\label{sec:pdb2gmxfiles}
The {\gromacs} program {\tt pdb2gmx} generates a topology for
the input coordinate file. Several formats are supported for
that coordinate file, but {\tt *.pdb} is the most commonly-used format
(hence the name {\tt pdb2gmx}).
{\tt pdb2gmx} searches for force fields in sub-directories of the {\gromacs} {\tt share/top}
directory and your working directory. Force fields are recognized from
the file {\tt forcefield.itp} in a directory with the extension {\tt .ff}.
The file {\tt forcefield.doc} may be present, and if so, its first line
will be used by {\tt pdb2gmx} to present a short description to the
user to help in choosing a force field. Otherwise, the user can
choose a force field with the {\tt -ff xxx} command-line argument
to {\tt pdb2gmx}, which indicates that a force field in a
{\tt xxx.ff} directory is desired. {\tt pdb2gmx} will search first in the
working directory, then in the {\gromacs} {\tt share/top} directory, and
use the first matching {\tt xxx.ff} directory found.

Two general files are read by {\tt pdb2gmx}: an atom type file
(extension {\tt .atp}, see~\ssecref{atomtype}) from the force field directory,
and a file called {\tt residuetypes.dat} from either the working directory, or
the {\gromacs} {\tt share/top} directory. {\tt residuetypes.dat}
determines which residue names are considered protein, DNA, RNA,
water, and ions.

{\tt pdb2gmx} can read one or multiple databases with topological information
for different types of molecules. A set of files belonging to one database
should have the same basename, preferably telling something about the type
of molecules ({\eg} aminoacids, rna, dna). The possible files are:
\begin{itemize}
\item {\tt <basename>.rtp}
\item {\tt <basename>.r2b} (optional)
\item {\tt <basename>.arn} (optional)
\item {\tt <basename>.hdb} (optional)
\item {\tt <basename>.n.tdb} (optional)
\item {\tt <basename>.c.tdb} (optional)
\end{itemize}
Only the {\tt .rtp} file, which contains the topologies of the building
blocks, is mandatory. Information from other files will only be used 
for building blocks that come from an {\tt .rtp} file with the same base name.
The user can add building blocks to a force field by having additional
files with the same base name in their working directory. By default, only
extra building blocks can be defined, but calling {\tt pdb2gmx} with
the {\tt -rtpo} option will allow building blocks in a local file
to replace the default ones in the force field.


\subsection{Residue database}
\label{subsec:rtp}
The files holding the residue databases have the extension {\tt .rtp}.
Originally this file contained building blocks (amino acids) for proteins,
and is the {\gromacs} interpretation of the {\tt rt37c4.dat} file of {\gromos}.
So the residue database file contains information (bonds, charges, charge groups,
and improper dihedrals) for a frequently-used building block. It is
better {\em not} to change this file because it is standard input for
{\tt pdb2gmx}, but if changes are needed make them in the
{\tt *.top} file (see~\ssecref{topfile}), or in a {\tt .rtp} file
in the working directory as explained in \secref{pdb2gmxfiles}.
Defining topologies of new small molecules is probably easier
by writing an include topology file {\tt *.itp} directly.
This will be discussed in section~\ssecref{molitp}.
When adding a new protein residue to the database, don't forget to
add the residue name to the {\tt \normindex{residuetypes.dat}} file,
so that {\tt grompp}, {\tt make_ndx} and analysis tools can recognize
the residue as a protein residue (see \ssecref{defaultgroups}).

The {\tt .rtp} files are only used by {\tt pdb2gmx}.
As mentioned before, the only extra information this
program needs from the {\tt .rtp} database is bonds, charges of atoms,
charge groups, and improper dihedrals, because the rest is read from
the coordinate input file.
Some proteins contain residues that are not standard, but are
listed in the coordinate file. You have to construct a building block
for this ``strange'' residue, otherwise you will not obtain a
{\tt *.top} file. This also holds for molecules in the
coordinate file such as ligands, polyatomic ions, crystallization co-solvents, etc.
The residue database is constructed in the following way:

{\small
\begin{verbatim}
[ bondedtypes ]  ; mandatory
; bonds  angles  dihedrals  impropers
     1       1          1          2  ; mandatory

[ GLY ]  ; mandatory

 [ atoms ]  ; mandatory 
; name  type  charge  chargegroup 
     N     N  -0.280     0
     H     H   0.280     0
    CA   CH2   0.000     1
     C     C   0.380     2
     O     O  -0.380     2

 [ bonds ]  ; optional
;atom1 atom2      b0      kb
     N     H
     N    CA
    CA     C
     C     O
    -C     N

 [ exclusions ]  ; optional
;atom1 atom2

 [ angles ]  ; optional
;atom1 atom2 atom3    th0    cth

 [ dihedrals ]  ; optional
;atom1 atom2 atom3 atom4   phi0     cp   mult

 [ impropers ]  ; optional
;atom1 atom2 atom3 atom4     q0     cq
     N    -C    CA     H
    -C   -CA     N    -O

[ ZN ]

 [ atoms ]
    ZN    ZN   2.000     0
\end{verbatim}}

The file is free format; the only restriction is that there can be at
most one entry on a line.  The first field in the file is the
{\tt [~bondedtypes~]} field, which is followed by four numbers,
indicating the interaction type for bonds, angles, dihedrals, and
improper dihedrals.  The file contains residue entries, which consist
of atoms and (optionally) bonds, angles, dihedrals, and impropers.  The
charge group codes denote the charge group numbers. Atoms in the same
charge group should always be ordered consecutively. When using the
hydrogen database with {\tt pdb2gmx} for adding missing hydrogens
(see~\ssecref{hdb}), the atom names defined in the {\tt .rtp} entry
should correspond exactly to the naming convention used in the
hydrogen database. The atom names in the bonded interaction can be
preceded by a minus or a plus, indicating that the atom is in the
preceding or following residue respectively.  Explicit parameters added
to bonds, angles, dihedrals, and impropers override
the standard parameters in the {\tt .itp} files.  This should only be
used in special cases. Instead of parameters, a string can be added
for each bonded interaction.  This is used in \gromosv{96} {\tt .rtp}
files. These strings are copied to the topology file and can be
replaced by force field parameters by the C-preprocessor in {\tt grompp}
using {\tt \#define} statements.

{\tt pdb2gmx} automatically generates all angles. This means that for the
{\tt gmx.ff} force field,
the {\tt [~angles~]} field is only useful for overriding {\tt .itp}
parameters. For the \gromosv{96} force field the interaction number
of all angles need to be specified.

{\tt pdb2gmx} automatically generates one proper dihedral for every rotatable
bond, preferably on heavy atoms. When the {\tt [~dihedrals~]} field is used,
no other dihedrals will be generated for the bonds corresponding to the
specified  dihedrals. It is possible to put more than one dihedral
function on a rotatable bond. 

{\tt pdb2gmx} sets the number of exclusions to 3, which
means that interactions between atoms connected by at most 3 bonds are
excluded. Pair interactions are generated for all pairs of atoms that are
separated by 3 bonds (except pairs of hydrogens).
When more interactions need to be excluded, or some pair interactions should
not be generated, an {\tt [~exclusions~]} field can be added, followed by
pairs of atom names on separate lines. All non-bonded and pair interactions
between these atoms will be excluded.

\subsection{Residue to building block database}
Each force field has its own naming convention for residues.
Most residues have consistent naming, but some, especially those
with different protonation states, can have many different names.
The {\tt .r2b} files are used to convert standard residue names to
the force field build block names. If no {\tt .r2b} is present
in the force field directory or a residue is not listed, the building
block name is assumed to be identical to the residue name.
The {\tt .r2b} can contain 2 or 5 columns. The 2-column format
has the residue name in the first column and the building block name
in the second. The 5-column format has 3 additional columns with
the building block for the residue occurring in the N-terminus, C-terminus
and both termini at the same time (single residue molecule).
This is useful for, for instance, the AMBER force fields.
If one or more of the terminal versions are not present, a dash should be entered
in the corresponding column.

There is a {\gromacs} naming convention for residues which is only
apparent (except for the {\tt pdb2gmx} code) through the {\tt .r2b} file
and {\tt specbond.dat} files.
This convention is only of importance when you are adding residue types
to an {\tt .rtp} file. The convention is listed in \tabref{r2b}.
For special bonds with, for instance, a heme group, the {\gromacs} naming
convention is introduced through {\tt specbond.dat} (see~\ssecref{specbond}), which can
subsequently be translated by the {\tt .r2b} file, if required.

\begin{table}
\centerline{
\begin{tabular}{|ll|}
\dline
ARG  & protonated arginine \\
ARGN & neutral arginine \\
ASP  & negatively charged aspartic acid \\
ASPH & neutral aspartic acid \\
CYS  & neutral cysteine \\
CYS2 & cysteine with sulfur bound to another cysteine or a heme \\
GLU  & negatively charged glutamic acid \\
GLUH & neutral glutamic acid \\
HISD & neutral histidine with N$_\delta$ protonated \\
HISE & neutral histidine with N$_\epsilon$ protonated \\
HISH & positive histidine with both N$_\delta$ and N$_\epsilon$ protonated \\
HIS1 & histidine bound to a heme \\
LYSN & neutral lysine \\
LYS  & protonated lysine \\
HEME & heme \\
\dline
\end{tabular}
}
\caption{Internal {\gromacs} residue naming convention.}
\label{tab:r2b}
\end{table}

\subsection{Atom renaming database}
Force fields often use atom names that do not follow IUPAC or PDB convention.
The {\tt .arn} database is used to translate the atom names in the coordinate
file to the force field names. Atoms that are not listed keep their names.
The file has three columns: the building block name,
the old atom name, and the new atom name, respectively. The residue name
supports question-mark wildcards that match a single character.

An additional general atom renaming file called {\tt xlateat.dat} is present
in the {\tt share/top} directory, which translates common non-standard
atom names in the coordinate file to IUPAC/PDB convention. Thus, when writing
force field files, you can assume standard atom names and no further
atom name translation is required, except for translating from standard atom names
to the force field ones.

\subsection{Hydrogen database}
\label{subsec:hdb}
The \swapindex{hydrogen}{database} is stored in {\tt .hdb} files. It
contains information for the {\tt pdb2gmx} program on how to connect
hydrogen atoms to existing atoms. In versions of the database before
{\gromacs} 3.3, hydrogen atoms were named after the atom they are
connected to: the first letter of the atom name was replaced by an
`H.' In the versions from 3.3 onwards, the H atom has to be listed explicitly,
because the old behavior was protein-specific and hence could not
be generalized to other molecules.
If more than one hydrogen atom is connected to the same atom, a
number will be added to the end of the hydrogen atom name. For
example, adding two hydrogen atoms to \texttt{ND2} (in asparagine), the
hydrogen atoms will be named \texttt{HD21} and \texttt{HD22}. This is
important since atom naming in the \texttt{.rtp} file (see~\ssecref{rtp})
must be the same. The format of the hydrogen database is as follows:

{\small
\begin{verbatim}
; res   # additions
        # H add type    H       i       j       k
ALA     1
        1       1       H       N       -C      CA
ARG     4
        1       2       H       N       CA      C
        1       1       HE      NE      CD      CZ
        2       3       HH1     NH1     CZ      NE
        2       3       HH2     NH2     CZ      NE
\end{verbatim}}

On the first line we see the residue name (ALA or ARG) and the number
of kinds of hydrogen atoms that may be added to this residue by the
hydrogen database. After that follows one line for each addition, on which
we see:
\begin{itemize}
\item The number of H atoms added
\item The method for adding H atoms, which can be any of:
\begin{enumerate}
\item[1]{\em one planar hydrogen, {\eg} rings or peptide bond}\\
One hydrogen atom (n) is generated, lying in the plane of atoms
(i,j,k) on the plane bisecting angle (j-i-k) at a distance of 0.1 nm
from atom i, such that the angles (n-i-j) and (n-i-k) are $>$ 90$^{\rm o}$.

\item[2]{\em one single hydrogen, {\eg} hydroxyl}\\
One hydrogen atom (n) is generated at a distance of 0.1 nm from atom
i, such that angle (n-i-j)=109.5 degrees and dihedral (n-i-j-k)=trans.

\item[3]{\em two planar hydrogens, {\eg} ethylene -C=CH{$_2$}, or amide -C(=O)NH{$_2$}}\\
Two hydrogens (n1,n2) are generated at a distance of 0.1 nm from atom
i, such that angle (n1-i-j)=(n2-i-j)=120 degrees and dihedral
(n1-i-j-k)=cis and (n2-i-j-k)=trans, such that names are according to
IUPAC standards~\cite{iupac70}.

\item[4]{\em two or three tetrahedral hydrogens, {\eg} -CH{$_3$}}\\
Three (n1,n2,n3) or two (n1,n2) hydrogens are generated at a distance
of 0.1 nm from atom i, such that angle
(n1-i-j)=(n2-i-j)=(n3-i-j)=109.47$^{\rm o}$, dihedral (n1-i-j-k)=trans,
(n2-i-j-k)=trans+120 and (n3-i-j-k)=trans+240$^{\rm o}$.

\item[5]{\em one tetrahedral hydrogen, {\eg} C{$_3$}CH}\\
One hydrogen atom (n$^\prime$) is generated at a distance of 0.1 nm from atom
i in tetrahedral conformation such that angle
(n$^\prime$-i-j)=(n$^\prime$-i-k)=(n$^\prime$-i-l)=109.47$^{\rm o}$.

\item[6]{\em two tetrahedral hydrogens, {\eg} C-CH{$_2$}-C}\\
Two hydrogen atoms (n1,n2) are generated at a distance of 0.1 nm from
atom i in tetrahedral conformation on the plane bisecting angle j-i-k
with angle (n1-i-n2)=(n1-i-j)=(n1-i-k)=109.47$^{\rm o}$.

\item[7]{\em two water hydrogens}\\
Two hydrogens are generated around atom i according to
SPC~\cite{Berendsen81} water geometry. The symmetry axis will
alternate between three coordinate axes in both directions.

\item[10]{\em three water ``hydrogens''}\\
Two hydrogens are generated around atom i according to
SPC~\cite{Berendsen81} water geometry. The symmetry axis will
alternate between three coordinate axes in both directions. In addition,
an extra particle is generated on the position of the oxygen with
the first letter of the name replaced by `M'. This is for
use with four-atom water models such as TIP4P~\cite{Jorgensen83}.

\item[11]{\em four water ``hydrogens''}\\
Same as above, except that two additional
particles are generated on the position of the oxygen, with names
`LP1' and `LP2.' This is for
use with five-atom water models such as TIP5P~\cite{Mahoney2000a}.
\end{enumerate}

\item
The name of the new H atom (or its prefix, {\eg} {\tt HD2} for
the asparagine example given earlier).

\item
Three or four control atoms (i,j,k,l), where the first always is the
atom to which the H atoms are connected. The other two or three depend
on the code selected. For water, there is only one control atom.
\end{itemize}

Some more exotic cases can be approximately constructed from the above tools,
and with suitable use of energy minimization are good enough for beginning
MD simulations. For example secondary amine hydrogen, nitrenyl hydrogen
(C\nolinebreak[4]=\nolinebreak[4]NH) and even ethynyl hydrogen could be
approximately constructed using method 2 above for hydroxyl hydrogen.

\subsection{Termini database}
\label{subsec:tdb}
The \swapindex{termini}{database}s are stored in {\tt aminoacids.n.tdb} and
{\tt aminoacids.c.tdb} for the N- and C-termini respectively. They contain
information for the {\tt pdb2gmx} program on how to connect new atoms
to existing ones, which atoms should be removed or changed, and which
bonded interactions should be added. The format of the is as follows
(from {\tt gmx.ff/aminoacids.c.tdb}):

{\small
\begin{verbatim}
[ COO- ]

[ replace ]
C       C       C       12.011  0.27

[ add ]
2       8       O       C       CA      N
        OM      15.9994 -0.635

[ delete ]
O

[ impropers ]
C       O1      O2      CA

[ None ]
\end{verbatim}}

The file is organized in blocks, each with a header specifying the
name of the block. These blocks correspond to different types of
termini that can be added to a molecule. In this example {\tt [~COO-~]}
is the first block, corresponding to changing the terminal carbon
atom into a deprotonated carboxyl group. {\tt [~None~]} is the
second terminus type, corresponding to a terminus that leaves
the molecule as it is. Block names cannot be any of the following:
{\tt replace}, {\tt add}, {\tt delete}, {\tt bonds}, {\tt angles},
{\tt dihedrals}, {\tt impropers}.  Doing so would interfere with
the parameters of the block, and would probably also be very confusing
to human readers.

For each block the following options are present:
\begin{itemize}
\item {\tt [~replace~]} \\
Replace an existing atom by one with a different atom type, atom name,
charge, and/or mass. This entry can be used to replace an atom that is
present both in the input coordinates and in the {\tt .rtp} database,
but also to only rename an atom in the input coordinates such that
it matches the name in the force field. In the latter case, there
should also be a corresponding {\tt [~add~]} section present that
gives instructions to add the same atom, such that the position in the sequence
and the bonding is known. Such an atom can be present in the input
coordinates and kept, or not present and constructed by {\tt pdb2gmx}.
For each atom to be replaced on line should be
entered with the following fields:
\begin{itemize}
\item name of the atom to be replaced
\item new atom name (optional)
\item new atom type
\item new mass
\item new charge
\end{itemize}
\item {\tt [~add~]} \\
Add new atoms. For each (group of) added atom(s), a two-line entry is
necessary. The first line contains the same fields as an entry in the
hydrogen database (name of the new atom, 
number of atoms, type of addition, control atoms,
see~\ssecref{hdb}), but the possible types of addition are extended
by two more, specifically for C-terminal additions:
\begin{enumerate}
\item[8]{\em two carboxyl oxygens, -COO{$^-$}}\\
Two oxygens (n1,n2) are generated according to rule 3, at a distance
of 0.136 nm from atom i and an angle (n1-i-j)=(n2-i-j)=117 degrees
\item[9]{\em carboxyl oxygens and hydrogen, -COOH}\\
Two oxygens (n1,n2) are generated according to rule 3, at distances of
0.123 nm and 0.125 nm from atom i for n1 and n2, respectively, and angles
(n1-i-j)=121 and (n2-i-j)=115 degrees. One hydrogen (n$^\prime$) is generated
around n2 according to rule 2, where n-i-j and n-i-j-k should be read
as n$^\prime$-n2-i and n$^\prime$-n2-i-j, respectively.
\end{enumerate}
After this line, another line follows that specifies the details of
the added atom(s), in the same way as for replacing atoms, {\ie}: 
\begin{itemize}
\item atom type
\item mass
\item charge
\item charge group (optional)
\end{itemize}
Like in the hydrogen database (see~\ssecref{rtp}), when more than
one atom is connected to an existing one, a number will be appended to
the end of the atom name. {\bf Note} that, like in the hydrogen database, the
atom name is now on the same line as the control atoms, whereas it was
at the beginning of the second line prior to {\gromacs} version 3.3.
When the charge group field is left out, the added atom will have
the same charge group number as the atom that it is bonded to.
\item {\tt [~delete~]}\\
Delete existing atoms. One atom name per line.
\item {\tt [~bonds~]}, {\tt [~angles~]}, {\tt [~dihedrals~]} and {\tt [~impropers~]}\\
Add additional bonded parameters. The format is identical to that used
in the {\tt *.rtp} file, see~\ssecref{rtp}.
\end{itemize}

\subsection{Virtual site database}
Since we cannot rely on the positions of hydrogens in input files, we need a special
input file to decide the geometries and parameters with which to add virtual site
hydrogens. For more complex virtual site constructs ({\eg} when entire aromatic side chains
are made rigid) we also need information about the equilibrium bond lengths and angles
for all atoms in the side chain. This information is specified in the {\tt .vsd} file for each force 
field. Just as for the termini, there is one such file for each class of residues in 
the {\tt .rtp} file. 

The virtual site database is not really a very simple list of information. The first couple of sections
specify which mass centers (typically called MCH$_3$/MNH$_3$) to use for CH$_3$, NH$_3$, 
and NH$_2$ groups. Depending on the 
equilibrium bond lengths and angles between the hydrogens and heavy atoms we need to apply
slightly different constraint distances between these mass centers. {\bf Note} that we do {\em not} have to
specify the actual parameters (that is automatic), just the type of mass center to use. To accomplish this,
there are three sections names \verb+[ CH3 ]+, \verb+[ NH3 ]+, and \verb+[ NH2 ]+. For each of these we
expect three columns. The first column is the atom type bound to the 2/3 hydrogens, the second column
is the next heavy atom type which this is bound, and the third column the type of mass center to use.
As a special case, in the  \verb+[ NH2 ]+ section it is also possible to specify \verb+planar+ in the second
column, which will use a different construction without mass center. There are currently different opinions
in some force fields whether an NH$_2$ group should be planar or not, but we try hard to stick to the
default equilibrium parameters of the force field.

The second part of the virtual site database contains explicit equilibrium bond lengths and angles
for pairs/triplets of atoms in aromatic side chains. These entries are currently read by specific routines
in the virtual site generation code, so if you would like to extend it {\eg} to nucleic acids you would also
need to write new code there. These sections are named after the short amino acid names
(\verb+[ PHE ]+, \verb+[ TYR ]+, \verb+[ TRP ]+, \verb+[ HID ]+, \verb+[ HIE ]+, \verb+[ HIP ]+), and simply
contain 2 or 3 columns with atom names, followed by a number specifying the bond length (in nm) or angle
(in degrees). {\bf Note} that these are approximations of the equilibrated geometry for the entire molecule, 
which might not be identical to the equilibrium value for a single bond/angle if the molecule is strained.

\subsection{Special bonds}
\label{subsec:specbond}
The primary mechanism used by {\tt \normindex{pdb2gmx}} to generate
inter-residue bonds relies on head-to-tail linking of backbone atoms
in different residues to build a macromolecule. In some cases ({\eg}
\normindex{disulfide bonds}, a \normindex{heme group},
\normindex{branched polymers}), it is necessary to create
inter-residue bonds that do not lie on the backbone. The file {\tt
  \normindex{specbond.dat}} takes care of this function. It is
necessary that the residues belong to the same {\tt [~moleculetype~]}.
The {\tt -merge} and {\tt -chainsep} functions of {\tt pdb2gmx} can be
useful when managing special inter-residue bonds between different
chains.

The first line of {\tt specbond.dat} indicates the number of entries that are in the file. If you
add a new entry, be sure to increment this number. The remaining lines in the file provide the
specifications for creating bonds. The format of the lines is as follows:

{\tt resA  atomA  nbondsA  resB  atomB  nbondsB  length  newresA  newresB }

The columns indicate:
\begin{enumerate}
\item {\tt resA} The name of residue A that participates in the bond.
\item {\tt atomA} The name of the atom in residue A that forms the bond.
\item {\tt nbondsA} The total number of bonds {\tt atomA} can form.
\item {\tt resB} The name of residue B that participates in the bond.
\item {\tt atomB} The name of the atom in residue B that forms the bond.
\item {\tt nbondsB} The total number of bonds {\tt atomB} can form.
\item {\tt length} The reference length for the bond. If {\tt atomA} and {\tt atomB} are not within
{\tt length} $\pm$ 10\% in the coordinate file supplied to {\tt pdb2gmx}, no bond will be formed.
\item {\tt newresA} The new name of residue A, if necessary. Some force fields use {\eg} CYS2 for 
a cysteine in a disulfide or heme linkage.
\item {\tt newresB} The new name of residue B, likewise.
\end{enumerate}


\section{File formats}
\subsection{Topology file\swapindexquiet{topology}{file}}
\label{subsec:topfile}
The topology file is built following the {\gromacs} specification for a
molecular topology.  A {\tt *.top} file can be generated by
{\tt pdb2gmx}.
All possible entries in the topology file are listed in
Tables \ref{tab:topfile1} and \ref{tab:topfile2}.
Also tabulated are: all the units
of the parameters, which interactions can be perturbed for free energy
calculations, which bonded interactions are used by {\tt grompp}
for generating exclusions, and which bonded interactions can be converted
to constraints by {\tt grompp}.

%\renewcommand\floatpagefraction{.2}

\newcommand{\tts}{\tt \small}

% move these figures so they end up on facing pages 
% (first figure on even page)
\newcommand{\kJmol}{kJ~mol$^{-1}$}
\newcommand{\kJmolnm}[1]{\kJmol~nm$^{#1}$}
\newcommand{\kJmolrad}[1]{\kJmol~rad$^{#1}$}
\newcommand{\kJmoldeg}[1]{\kJmol~deg$^{#1}$}

\begin{table}[p]
\centering{
\begin{tabular}{|l|llllc|}
\multicolumn{6}{c}{\bf \large Parameters} \\
\dline
interaction 	& directive   	      & \#  & f. & parameters 				& F. E. \\
type	&		      	      & at. & tp &					& 	\\
\dline
{\em mandatory} & {\tts defaults}	& & &	non-bonded function type; & \\
		&			& & &	combination rule$^{(cr)}$; &\\
		&			& & &   generate pairs (no/yes); & \\
		&			& & &	fudge LJ (); fudge QQ () & \\
\hline
{\em mandatory} & {\tts atomtypes}	&   & 	& atom type; m (u); q (e); particle type; & \\
		&			&   &	& V$^{(cr)}$; W$^{(cr)}$ & \\
%\hline
		& {\tts bondtypes}	& \multicolumn{3}{l}{(see \tabref{topfile2}, directive {\tts bonds})}		& \\
		& {\tts pairtypes}	& \multicolumn{3}{l}{(see \tabref{topfile2}, directive {\tts pairs})}		& \\
		& {\tts angletypes}	& \multicolumn{3}{l}{(see \tabref{topfile2}, directive {\tts angles})}		& \\
		& {\tts dihedraltypes}$^{(*)}$ & \multicolumn{3}{l}{(see \tabref{topfile2}, directive {\tts dihedrals})}& \\
		& {\tts constrainttypes}& \multicolumn{3}{l}{(see \tabref{topfile2}, directive {\tts constraints})}	& \\
LJ 		& {\tts nonbond_params}	& 2 & 1	& $V^{(cr)}$; $W^{(cr)}$ & \\
Buckingham    	& {\tts nonbond_params}	& 2 & 2	& $a$ (\kJmol); $b$ (nm$^{-1})$;  & \\
 & & & & $c_6$ (\kJmolnm{6}) & \\
\dline
\multicolumn{6}{c}{~} \\
\multicolumn{6}{c}{\bf \large Molecule definition(s)} \\
\dline
{\em mandatory} & {\tts moleculetype}	& & & 	molecule name; $n_{ex}^{(nrexcl)}$  &	\\
\hline
{\em mandatory} & {\tts atoms}		& 1 & 	& atom type; residue number; 	& type	\\
		&			&   &	& residue name; atom name; 	& 	\\
		&			&   &	& charge group number; $q$ (e); $m$ (u) 	& $q,m$ \\
\hline
\multicolumn{6}{|c|}{} \\
\multicolumn{6}{|c|}{intra-molecular interaction and geometry definitions as described
in \tabref{topfile2}} \\
\multicolumn{6}{|c|}{} \\
\dline
\multicolumn{6}{c}{~} \\
\multicolumn{6}{c}{\bf \large System} \\
\dline
{\em mandatory} & {\tts system}		& & &	system name	&	\\
\hline
{\em mandatory} & {\tts molecules}	& & &	\multicolumn{2}{l|}{molecule name; number of molecules}	\\
\dline
\multicolumn{6}{c}{~} \\
\multicolumn{6}{l}{`\# at' is the required number of atom type indices for this directive} \\
\multicolumn{6}{l}{`f. tp' is the value used to select this function type} \\
\multicolumn{6}{l}{`F. E.' indicates which of the parameters for this interaction can be} \\
\multicolumn{6}{l}{\phantom{`F. E.'} interpolated during free energy calculations} \\
\multicolumn{6}{l}{~$^{(cr)}$ the combination rule determines the type of LJ parameters, see~\ssecref{nbpar}}\\
\multicolumn{6}{l}{~$^{(*)}$ for {\tts dihedraltypes} one can specify 4 atoms or the inner (outer for improper) 2 atoms}\\
\multicolumn{6}{l}{~$^{(nrexcl)}$ exclude neighbors $n_{ex}$ bonds away for non-bonded interactions}\\
\multicolumn{6}{l}{For free energy calculations, type, $q$ and $m$  or no parameters should be added}\\
\multicolumn{6}{l}{for topology `B' ($\lambda = 1$) on the same line, after the normal parameters.}
\end{tabular}
}
\caption{The topology ({\tts *.top}) file.}
\label{tab:topfile1}
\end{table}

\newcommand{\fnm}[1]{\footnotemark[#1]}
\renewcommand{\thefootnote}{\fnsymbol{footnote}}
%\renewcommand{\tts}{\tt \small}
\newcommand{\ttss}{\tt \scriptsize}

\begin{landscape}
\begin{longtable}{|l|lcc>{\raggedright}p{2.5in}cc|}
\caption{Details of {\tt [~moleculetype~]} directives}\\
\dline
Name of interaction              & Topology file directive          & num.  & func. & Order of parameters and their units                   & use in     & Cross- \\
                                 &                                  & atoms\fnm{1} & type\fnm{2} &                                          & F.E.?\fnm{3} & references \\
\dline
\endhead
\dline
\endfoot
\label{tab:topfile2}\footnotetext[1]{The required number of atom indices for this directive}\footnotetext[2]{The index to use to select this function type}\footnotetext[3]{Indicates which of the parameters for this interaction can be interpolated during free energy calculations}\footnotetext[4]{This interaction type will be used by {{\tts grompp}} for generating exclusions}\footnotetext[5]{This interaction type can be converted to constraints by {{\tts grompp}}}\footnotetext[7]{The combination rule determines the type of LJ parameters, see~\ssecref{nbpar}}\footnotetext[6]{No connection, and so no exclusions, are generated for this interaction}bond
                                   & {\tts bonds}\fnm{4},\fnm{5}    & 2     & 1     & $b_0$ (nm); $k_b$ (\kJmolnm{-2})                      & all        & \ssecref{harmonicbond} \\
G96 bond                           & {\tts bonds}\fnm{4},\fnm{5}    & 2     & 2     & $b_0$ (nm); $k_b$ (\kJmolnm{-4})                      & all        & \ssecref{G96bond} \\
Morse                              & {\tts bonds}\fnm{4},\fnm{5}    & 2     & 3     & $b_0$ (nm); $D$ (\kJmol); $\beta$ (nm$^{-1}$)         & all        & \ssecref{Morsebond} \\
cubic bond                         & {\tts bonds}\fnm{4},\fnm{5}    & 2     & 4     & $b_0$ (nm); $C_{i=2,3}$ (\kJmolnm{-i})                &            & \ssecref{cubicbond} \\
connection                         & {\tts bonds}\fnm{4}            & 2     & 5     &                                                       &            & \tsecref{excl} \\
harmonic potential                 & {\tts bonds}                   & 2     & 6     & $b_0$ (nm); $k_b$ (\kJmolnm{-2})                      & all        & \ssecref{harmonicbond},\tsecref{excl} \\
FENE bond                          & {\tts bonds}\fnm{4}            & 2     & 7     & $b_m$ (nm); $k_b$ (\kJmolnm{-2})                      &            & \ssecref{FENEbond} \\
tabulated bond                     & {\tts bonds}\fnm{4}            & 2     & 8     & table number ($\geq 0$); $k$ (\kJmol)                 & $k$        & \ssecref{tabulatedinteraction} \\
tabulated bond\fnm{6}              & {\tts bonds}                   & 2     & 9     & table number ($\geq 0$); $k$ (\kJmol)                 & $k$        & \ssecref{tabulatedinteraction},\tsecref{excl} \\
restraint potential                & {\tts bonds}                   & 2     & 10    & low, up$_1$, up$_2$ (nm); $k_{dr}$ (\kJmolnm{-2})     & all        & \ssecref{harmonicrestraint} \\
extra LJ or Coulomb                & {\tts pairs}                   & 2     & 1     & $V$\fnm{7}; $W$\fnm{7}                                & all        & \ssecref{pairinteractions} \\
extra LJ or Coulomb                & {\tts pairs}                   & 2     & 2     & fudge QQ (); $q_i$, $q_j$ (e), $V$\fnm{7}; $W$\fnm{7} &            & \ssecref{pairinteractions} \\
extra LJ or Coulomb                & {\tts pairs_nb}                & 2     & 1     & $q_i$, $q_j$ (e); $V$\fnm{7}; $W$\fnm{7}              &            & \ssecref{pairinteractions} \\
angle                              & {\tts angles}\fnm{5}           & 3     & 1     & $\theta_0$ (deg); $k_\theta$ (\kJmolrad{-2})          & all        & \ssecref{harmonicangle} \\
G96 angle                          & {\tts angles}\fnm{5}           & 3     & 2     & $\theta_0$ (deg); $k_\theta$ (\kJmol)                 & all        & \ssecref{G96angle} \\
cross bond-bond                    & {\tts angles}                  & 3     & 3     & $r_{1e}$, $r_{2e}$ (nm); $k_{rr'}$ (\kJmolnm{-2})     &            & \ssecref{bondbondcross} \\
cross bond-angle                   & {\tts angles}                  & 3     & 4     & $r_{1e}$, $r_{2e}$ $r_{3e}$ (nm); $k_{r\theta}$ (\kJmolnm{-2}) &   & \ssecref{bondanglecross} \\
Urey-Bradley                       & {\tts angles}\fnm{5}           & 3     & 5     & $\theta_0$ (deg); $k_\theta$ (\kJmolrad{-2}); $r_{13}$ (nm); $k_{UB}$ (\kJmolnm{-2}) & all & \ssecref{Urey-Bradley} \\
quartic angle                      & {\tts angles}\fnm{5}           & 3     & 6     & $\theta_0$ (deg); $C_{i=0,1,2,3,4}$ (\kJmolrad{-i})   &            & \ssecref{quarticangle} \\
tabulated angle                    & {\tts angles}                  & 3     & 8     & table number ($\geq 0$); $k$ (\kJmol)                 & $k$        & \ssecref{tabulatedinteraction} \\
proper dihedral                    & {\tts dihedrals}               & 4     & 1     & $\phi_s$ (deg); $k_\phi$ (\kJmol); multiplicity       & $\phi,k$   & \ssecref{properdihedral} \\
improper dihedral                  & {\tts dihedrals}               & 4     & 2     & $\xi_0$ (deg); $k_\xi$ (\kJmolrad{-2})                & all        & \ssecref{harmonicimproperdihedral} \\
Ryckaert-Bellemans dihedral        & {\tts dihedrals}               & 4     & 3     & $C_0$, $C_1$, $C_2$, $C_3$, $C_4$, $C_5$ (\kJmol)     & all        & \ssecref{RBdihedral} \\
periodic improper dihedral         & {\tts dihedrals}               & 4     & 4     & $\phi_s$ (deg); $k_\phi$ (\kJmol); multiplicity       & $\phi,k$   & \ssecref{periodicimproperdihedral} \\
Fourier dihedral                   & {\tts dihedrals}               & 4     & 5     & $C_1$, $C_2$, $C_3$, $C_4$ (\kJmol)                   & all        & \ssecref{Fourierdihedral} \\
tabulated dihedral                 & {\tts dihedrals}               & 4     & 8     & table number ($\geq 0$); $k$ (\kJmol)                 & $k$        & \ssecref{tabulatedinteraction} \\
proper dihedral (multiple)         & {\tts dihedrals}               & 4     & 9     & $\phi_s$ (deg); $k_\phi$ (\kJmol); multiplicity       & $\phi,k$   & \ssecref{properdihedral} \\
exclusions                         & {\tts exclusions}              & 1     &       & one or more atom indices                              &            & \tsecref{excl} \\
constraint                         & {\tts constraints}\fnm{4}      & 2     & 1     & $b_0$ (nm)                                            & all        & \sssecref{constraints},\tsecref{constraints} \\
constraint\fnm{6}                  & {\tts constraints}             & 2     & 2     & $b_0$ (nm)                                            & all        & \sssecref{constraints},\tsecref{constraints},\tsecref{excl} \\
SETTLE                             & {\tts settles}                 & 1     & 1     & $d_{\mbox{\sc oh}}$, $d_{\mbox{\sc hh}}$ (nm)         &            & \ssecref{SETTLE},\tsecref{constraints} \\
2-body virtual site                & {\tts virtual_sites2}          & 3     & 1     & $a$ ()                                                &            & \ssecref{vsite2} \\
3-body virtual site                & {\tts virtual_sites3}          & 4     & 1     & $a$, $b$ ()                                           &            & \ssecref{vsite3} \\
3-body virtual site (fd)           & {\tts virtual_sites3}          & 4     & 2     & $a$ (); $d$ (nm)                                      &            & \ssecref{vsite3fd} \\
3-body virtual site (fad)          & {\tts virtual_sites3}          & 4     & 3     & $\theta$ (deg); $d$ (nm)                              &            & \ssecref{vsite3fad} \\
3-body virtual site (out)          & {\tts virtual_sites3}          & 4     & 4     & $a$, $b$ (); $c$ (nm$^{-1}$)                          &            & \ssecref{vsite3out} \\
4-body virtual site (fdn)          & {\tts virtual_sites4}          & 5     & 2     & $a$, $b$ (); $c$ (nm)                                 &            & \ssecref{vsite4fdn} \\
N-body virtual site (COG)          & {\tts virtual_sitesn}          & 1     & 1     & one or more constructing atom indices                 &            & \ssecref{vsiteN} \\
N-body virtual site (COM)          & {\tts virtual_sitesn}          & 1     & 2     & one or more constructing atom indices                 &            & \ssecref{vsiteN} \\
N-body virtual site (COW)          & {\tts virtual_sitesn}          & 1     & 3     & one or more pairs consisting of constructing atom index and weight & & \ssecref{vsiteN} \\
position restraint                 & {\ttss position_restraints}    & 1     & 1     & $k_{x}$, $k_{y}$, $k_{z}$ (\kJmolnm{-2}) & all                     & \ssecref{positionrestraint} \\
flat-bottomed position restraint   & {\ttss position_restraints}    & 1     & 2     & $g$, $r$ (nm), $k$ (\kJmolnm{-2})                     &            & \ssecref{fbpositionrestraint} \\
%restraint potential                & {\tts bonds}                   & 2     & 10    & low, up$_1$, up$_2$ (nm); $k_{dr}$ (\kJmolnm{-2})     &            & \ssecref{} \\
distance restraint                 & {\ttss distance_restraints}    & 2     & 1     & type; label; low, up$_1$, up$_2$ (nm); weight ()      &            & \ssecref{distancerestraint} \\
dihedral restraint                 & {\ttss dihedral_restraints}    & 4     & 1     & $\phi_0$ (deg); $\Delta\phi$ (deg); &    all   & \ssecref{dihedralrestraint} \\
orientation restraint              & {\ttss orientation_restraints} & 2     & 1     & exp.; label; $\alpha$; $c$ (U nm$^\alpha$); obs. (U); weight (U$^{-1}$) & & \ssecref{orientationrestraint} \\
angle restraint                    & {\ttss angle_restraints}       & 4     & 1     & $\theta_0$ (deg); $k_c$ (\kJmol); multiplicity        & $\theta,k$ & \ssecref{anglerestraint} \\
angle restraint (z)                & {\ttss angle_restraints_z}     & 2     & 1     & $\theta_0$ (deg); $k_c$ (\kJmol); multiplicity        & $\theta,k$ & \ssecref{anglerestraint} \\
\end{longtable}
\end{landscape}

\renewcommand{\thefootnote}{\arabic{footnote}}

%\renewcommand\floatpagefraction{.5}


Description of the file layout:
\begin{itemize}
\item Semicolon (;) and newline characters surround comments
\item On a line ending with $\backslash$ the newline character is ignored.
\item Directives are surrounded by {\tt [} and {\tt ]}
\item The topology hierarchy (which must be followed) consists of three levels:
\begin{itemize}
\item the parameter level, which defines certain force field specifications 
      (see~\tabref{topfile1})
\item the molecule level, which should contain one or more molecule
      definitions (see~\tabref{topfile2})
\item the system level, containing only system-specific information 
      ({\tt [~system~]} and {\tt [~molecules~]})
\end{itemize}
\item Items should be separated by spaces or tabs, not commas
\item Atoms in molecules should be numbered consecutively starting at 1
\item Atoms in the same charge group must be listed consecutively
\item The file is parsed only once, which implies that no forward
      references can be treated: items must be defined before they
      can be used
\item Exclusions can be generated from the bonds or
      overridden manually
\item The bonded force types can be generated from the atom types or
      overridden per bond
\item It is possible to apply multiple bonded interactions of the same type
      on the same atoms
\item Descriptive comment lines and empty lines are highly recommended
\item Starting with {\gromacs} version 3.1.3, all directives at the
      parameter level can be used multiple times and there are no
      restrictions on the order, except that an atom type needs to be
      defined before it can be used in other parameter definitions
\item If parameters for a certain interaction are defined multiple times
      for the same combination of atom types the last definition is used;
      starting with {\gromacs} version 3.1.3 {\tt grompp} generates a
      warning for parameter redefinitions with different values
\item Using one of the {\tt [~atoms~]}, {\tt [~bonds~]},
      {\tt [~pairs~]}, {\tt [~angles~]}, etc. without having used
      {\tt [~moleculetype~]}
      before is meaningless and generates a warning
\item Using {\tt [~molecules~]} without having used
      {\tt [~system~]} before is meaningless and generates a warning.
\item After {\tt [~system~]} the only allowed directive is {\tt [~molecules~]}
\item Using an unknown string in {\tt [ ]} causes all the data until
      the next directive to be ignored and generates a warning
\end{itemize}

Here is an example of a topology file, {\tt urea.top}:

{\small
\begin{verbatim}
;
;       Example topology file
;
; The force field files to be included
#include "gmx.ff/forcefield.itp"

[ moleculetype ]
; name  nrexcl
Urea         3

[ atoms ]
;   nr    type   resnr  residu    atom    cgnr  charge
     1       C       1    UREA      C1       1   0.683
     2       O       1    UREA      O2       1  -0.683
     3      NT       1    UREA      N3       2  -0.622
     4       H       1    UREA      H4       2   0.346
     5       H       1    UREA      H5       2   0.276
     6      NT       1    UREA      N6       3  -0.622
     7       H       1    UREA      H7       3   0.346
     8       H       1    UREA      H8       3   0.276

[ bonds ]
;  ai    aj funct           b0           kb
    3     4     1 1.000000e-01 3.744680e+05 
    3     5     1 1.000000e-01 3.744680e+05 
    6     7     1 1.000000e-01 3.744680e+05 
    6     8     1 1.000000e-01 3.744680e+05 
    1     2     1 1.230000e-01 5.020800e+05 
    1     3     1 1.330000e-01 3.765600e+05 
    1     6     1 1.330000e-01 3.765600e+05 

[ pairs ]
;  ai    aj funct           c6          c12
    2     4     1 0.000000e+00 0.000000e+00 
    2     5     1 0.000000e+00 0.000000e+00 
    2     7     1 0.000000e+00 0.000000e+00 
    2     8     1 0.000000e+00 0.000000e+00 
    3     7     1 0.000000e+00 0.000000e+00 
    3     8     1 0.000000e+00 0.000000e+00 
    4     6     1 0.000000e+00 0.000000e+00 
    5     6     1 0.000000e+00 0.000000e+00 

[ angles ]
;  ai    aj    ak funct          th0          cth
    1     3     4     1 1.200000e+02 2.928800e+02 
    1     3     5     1 1.200000e+02 2.928800e+02 
    4     3     5     1 1.200000e+02 3.347200e+02 
    1     6     7     1 1.200000e+02 2.928800e+02 
    1     6     8     1 1.200000e+02 2.928800e+02 
    7     6     8     1 1.200000e+02 3.347200e+02 
    2     1     3     1 1.215000e+02 5.020800e+02 
    2     1     6     1 1.215000e+02 5.020800e+02 
    3     1     6     1 1.170000e+02 5.020800e+02 

[ dihedrals ]
;  ai    aj    ak    al funct          phi           cp         mult
    2     1     3     4     1 1.800000e+02 3.347200e+01 2.000000e+00 
    6     1     3     4     1 1.800000e+02 3.347200e+01 2.000000e+00 
    2     1     3     5     1 1.800000e+02 3.347200e+01 2.000000e+00 
    6     1     3     5     1 1.800000e+02 3.347200e+01 2.000000e+00 
    2     1     6     7     1 1.800000e+02 3.347200e+01 2.000000e+00 
    3     1     6     7     1 1.800000e+02 3.347200e+01 2.000000e+00 
    2     1     6     8     1 1.800000e+02 3.347200e+01 2.000000e+00 
    3     1     6     8     1 1.800000e+02 3.347200e+01 2.000000e+00 

[ dihedrals ]
;  ai    aj    ak    al funct           q0           cq
    3     4     5     1     2 0.000000e+00 1.673600e+02 
    6     7     8     1     2 0.000000e+00 1.673600e+02 
    1     3     6     2     2 0.000000e+00 1.673600e+02 

[ position_restraints ]
; you wouldn't normally use this for a molecule like Urea,
; but we include it here for didactic purposes
; ai   funct    fc
   1     1     1000    1000    1000 ; Restrain to a point
   2     1     1000       0    1000 ; Restrain to a line (Y-axis)
   3     1     1000       0       0 ; Restrain to a plane (Y-Z-plane)

; Include SPC water topology
#include "spc.itp"

[ system ]
Urea in Water

[ molecules ]
;molecule name   nr.
Urea             1
SOL              1000
\end{verbatim}}

Here follows the explanatory text.

{\bf {\tt [~defaults~]} :}
\begin{itemize}
\item {\tt nbfunc} is the non-bonded function type. Use 1 (Lennard-Jones) or 2 (Buckingham)
\item {\tt comb-rule} is the number of the \normindex{combination rule} (see \ssecref{nbpar}).
\item {\tt gen-pairs} is for pair generation. The default is `no', {\ie} 
get 1-4 parameters from the pairtypes list. When parameters
are not present in the list, stop with a fatal error.
Setting `yes' generates 1-4 parameters that are not present in the pair list
from normal Lennard-Jones parameters using {\tt fudgeLJ}
\item {\tt fudgeLJ} is the factor by which to multiply Lennard-Jones 1-4 interactions, default 1
\item {\tt fudgeQQ} is the factor by which to multiply electrostatic 1-4 interactions, default 1
\item $N$ is the power for the repulsion term in a 6-$N$ potential (with 
nonbonded-type Lennard-Jones only), starting with {\gromacs} version 4.5,
{\tt mdrun} also reads and applies $N$, for values not equal to 12 tabulated
interaction functions are used
(in older version you would have to use user tabulated interactions).
\end{itemize}
{\bf Note} that {\tt gen-pairs}, {\tt fudgeLJ}, {\tt fudgeQQ}, and $N$ are optional.
{\tt fudgeLJ} is only used when generate pairs is set to `yes', and
{\tt fudgeQQ} is always used. However, if you
want to specify $N$ you need to give a value for the other parameters as well.

% move these figures so they end up on facing pages 
% (first figure on even page)
%\input{topolfig}

{\bf {\tt \#include "gmx.ff/forcefield.itp"} :} this includes the bonded and
non-bonded force field parameters, the {\tt gmx} in {\tt gmx.ff} will be
replaced by the name of the force field you are actually using.

{\bf {\tt [~moleculetype~]} :} defines the name of your molecule in
this {\tt *.top} and nrexcl = 3 stands for excluding non-bonded
interactions between atoms that are no further than 3 bonds away.

{\bf {\tt [~atoms~]} :} defines the molecule, where {\tt nr} and
{\tt type} are fixed, the rest is user defined. So {\tt atom} can be named
as you like, {\tt cgnr} made larger or smaller (if possible, the total
charge of a charge group should be zero), and charges can be changed
here too.

{\bf {\tt [~bonds~]} :} no comment.

{\bf {\tt [~pairs~]} :} LJ and Coulomb 1-4 interactions

{\bf {\tt [~angles~]} :} no comment

{\bf {\tt [~dihedrals~]} :} in this case there are 9 proper dihedrals
(funct = 1), 3 improper (funct = 2) and no Ryckaert-Bellemans type
dihedrals. If you want to include Ryckaert-Bellemans type dihedrals
in a topology, do the following (in case of {\eg} decane):
\begin{verbatim}
[ dihedrals ]
;  ai    aj    ak    al funct       c0       c1       c2
    1    2     3     4     3 
    2    3     4     5     3
\end{verbatim}
In the original implementation of the potential for
alkanes~\cite{Ryckaert78} no 1-4 interactions were used, which means
that in order to implement that particular force field you need to remove the 1-4
interactions from the {\tt [~pairs~]} section of your topology. In
most modern force fields, like OPLS/AA or Amber the rules are
different, and the Ryckaert-Bellemans potential is used as a cosine
series in combination with 1-4 interactions.

{\bf {\tt [~position_restraints~]} :} harmonically restrain the selected particles
to reference positions (\ssecref{positionrestraint}). 
The reference positions are read from a 
separate coordinate file by {\tt \normindex{grompp}}.

{\bf {\tt \#include "spc.itp"} :} includes a topology file that was already
constructed (see section~\ssecref{molitp}).

{\bf {\tt [~system~]} :} title of your system, user-defined

{\bf {\tt [~molecules~]} :} this defines the total number of (sub)molecules
in your system that are defined in this {\tt *.top}. In this
example file, it stands for 1 urea molecule dissolved in 1000 water
molecules. The molecule type SOL is defined in the {\tt spc.itp} file.
Each name here must correspond to a name given with {\tt [~moleculetype~]}
earlier in the topology. The order of the blocks of molecule types and
the numbers of such molecules must match the coordinate file that
accompanies the topology when supplied to {\tt \normindex{grompp}}.
The blocks of molecules do not need to be contiguous, but some
tools (e.g. {\tt \normindex{genion}}) may act only on the first or
last such block of a particular molecule type. Also, these blocks
have nothing to do with the definition of \normindex{groups}
(see \secref{groupconcept} and \secref{usinggroups}).

\subsection{Molecule.itp file}
\label{subsec:molitp}
If you construct a topology file you will use frequently (like the water
molecule, {\tt spc.itp}, which is already constructed for you) it is
good to make a {\tt molecule.itp} file. This only lists the
information of one particular molecule and allows you to re-use the
{\tt [ moleculetype ]} in multiple systems without re-invoking
{\tt pdb2gmx} or manually copying and pasting. An example follows: 

{\small
\begin{verbatim}
[ moleculetype ]
; name  nrexcl
Urea       3

[ atoms ]
;   nr    type   resnr  residu    atom    cgnr  charge
     1       C       1    UREA      C1       1   0.683  
     .................
     .................
     8       H       1    UREA      H8       3   0.276

[ bonds ]
;  ai    aj funct           c0           c1
    3     4     1 1.000000e-01 3.744680e+05 
     .................
     .................
    1     6     1 1.330000e-01 3.765600e+05 

[ pairs ]
;  ai    aj funct           c0           c1
    2     4     1 0.000000e+00 0.000000e+00 
     .................
     .................
    5     6     1 0.000000e+00 0.000000e+00 

[ angles ]
;  ai    aj    ak funct           c0           c1
    1     3     4     1 1.200000e+02 2.928800e+02 
     .................
     .................
    3     1     6     1 1.170000e+02 5.020800e+02 

[ dihedrals ]
;  ai    aj    ak    al funct           c0           c1           c2
    2     1     3     4     1 1.800000e+02 3.347200e+01 2.000000e+00 
     .................
     .................
    3     1     6     8     1 1.800000e+02 3.347200e+01 2.000000e+00 

[ dihedrals ]
;  ai    aj    ak    al funct           c0           c1
    3     4     5     1     2 0.000000e+00 1.673600e+02 
    6     7     8     1     2 0.000000e+00 1.673600e+02 
    1     3     6     2     2 0.000000e+00 1.673600e+02 
\end{verbatim}}

Using {\tt *.itp} files results in a very short {\tt *.top} file:

{\small
\begin{verbatim}
; The force field files to be included
#include "gmx.ff/forcefield.itp"
        
; Include urea topology
#include "urea.itp"

; Include SPC water topology
#include "spc.itp"

[ system ]
Urea in Water

[ molecules ]
;molecule name  number
Urea              1
SOL               1000
\end{verbatim}}

\subsection{Ifdef statements}
\label{subsec:ifdef}
A very powerful feature in {\gromacs} is the use of {\tt \#ifdef}
statements in your {\tt *.top} file. By making use of this statement,
different parameters for one molecule can be used in the same
{\tt *.top} file. An example is given for TFE, where there is an option to
use different charges on the atoms: charges derived by De Loof
{\etal}~\cite{Loof92} or by Van Buuren and
Berendsen~\cite{Buuren93a}. In fact, you can use much of the functionality of the
C preprocessor, {\tt cpp}, because {\tt grompp} contains similar pre-processing
functions to scan the file.  The
way to make use of the {\tt \#ifdef} option is as follows:
\begin{itemize}
\item either use the option {\tt define = -DDeLoof} in the
      {\tt *.mdp} file (containing {\tt grompp} input
      parameters), or use the line {\tt \#define DeLoof}
      early in your {\tt *.top} or {\tt *.itp} file; and
\item put the {\tt \#ifdef} statements in your {\tt *.top}, as
      shown below: 
\end{itemize}

{\small
\begin{verbatim}
...



[ atoms ]
; nr     type     resnr    residu     atom      cgnr      charge        mass
#ifdef DeLoof
; Use Charges from DeLoof
   1        C        1        TFE        C         1        0.74        
   2        F        1        TFE        F         1       -0.25        
   3        F        1        TFE        F         1       -0.25        
   4        F        1        TFE        F         1       -0.25        
   5      CH2        1        TFE      CH2         1        0.25        
   6       OA        1        TFE       OA         1       -0.65        
   7       HO        1        TFE       HO         1        0.41        
#else
; Use Charges from VanBuuren
   1        C        1        TFE        C         1        0.59        
   2        F        1        TFE        F         1       -0.2         
   3        F        1        TFE        F         1       -0.2         
   4        F        1        TFE        F         1       -0.2         
   5      CH2        1        TFE      CH2         1        0.26        
   6       OA        1        TFE       OA         1       -0.55        
   7       HO        1        TFE       HO         1        0.3         
#endif

[ bonds ]
;  ai    aj funct           c0           c1
    6     7     1 1.000000e-01 3.138000e+05 
    1     2     1 1.360000e-01 4.184000e+05 
    1     3     1 1.360000e-01 4.184000e+05 
    1     4     1 1.360000e-01 4.184000e+05 
    1     5     1 1.530000e-01 3.347000e+05 
    5     6     1 1.430000e-01 3.347000e+05 
...
\end{verbatim}}

This mechanism is used by {\tt pdb2gmx} to implement optional position
restraints (\ssecref{positionrestraint}) by {\tt \#include}-ing an {\tt .itp} file whose contents
will be meaningful only if a particular {\tt \#define} is set (and spelled
correctly!)

\subsection{Topologies for free energy calculations}
\index{free energy topologies}
Free energy differences between two systems, A and B, can be calculated as
described in \secref{fecalc}.
Systems A and B are described by topologies
consisting of the same number of molecules with the same number of
atoms. Masses and non-bonded interactions can be perturbed by adding B
parameters under the {\tt [~atoms~]} directive. Bonded interactions can be 
perturbed by adding B parameters to the bonded types or the bonded
interactions. The parameters that can be perturbed are listed in  
Tables \ref{tab:topfile1} and \ref{tab:topfile2}.
The $\lambda$-dependence of the interactions is described
in section \secref{feia}.
The bonded parameters that are used (on the line of the bonded
interaction definition, or the ones looked up on atom types
in the bonded type lists) is explained in \tabref{topfe}.
In most cases, things should work intuitively.
When the A and B atom types in a bonded interaction
are not all identical and parameters are not present for the B-state,
either on the line or in the bonded types,
{\tt grompp} uses the A-state parameters and issues a warning.
For free energy calculations, all or no parameters for topology B
($\lambda = 1$) should be added on the same line, after the normal
parameters, in the same order as the normal parameters.
From {\gromacs} 4.6 onward, if $\lambda$ is treated as a vector, then
the {\tt bonded-lambdas} component controls all bonded terms that are
not explicitly labeled as restraints.  Restrain terms are controlled
by the {\tt restraint-lambdas} component.

\begin{table}
\centerline{
\begin{tabular}{|c|cc|cc|cc|c|}
\dline
B-state atom types & \multicolumn{2}{c|}{parameters} & \multicolumn{4}{c|}{parameters in bonded types} & \\
all identical to      & \multicolumn{2}{c|}{on line} & \multicolumn{2}{c|}{A atom types} & \multicolumn{2}{c|}{B atom types} & message \\
A-state atom types & A & B & A & B & A & B & \\
\dline
    & +AB & $-$ &  x  &  x  &     &     & \\
    & +A  & +B  &  x  &  x  &     &     & \\
yes & $-$ & $-$ & $-$ & $-$ &     &     & error \\
    & $-$ & $-$ & +AB & $-$ &     &     & \\
    & $-$ & $-$ & +A  & +B  &     &     & \\
\hline
    & +AB & $-$ &  x  &  x  &  x  &  x  & warning \\
    & +A  & +B  &  x  &  x  &  x  &  x  & \\
    & $-$ & $-$ & $-$ & $-$ &  x  &  x  & error \\
no  & $-$ & $-$ & +AB & $-$ & $-$ & $-$ & warning \\
    & $-$ & $-$ & +A  & +B  & $-$ & $-$ & warning \\
    & $-$ & $-$ & +A  &  x  & +B  & $-$ & \\
    & $-$ & $-$ & +A  &  x  &  +  & +B  & \\
\dline
\end{tabular}
}
\caption{The bonded parameters that are used for free energy topologies,
on the line of the bonded interaction definition or looked up
in the bond types section based on atom types. A and B indicate the
parameters used for state A and B respectively, + and $-$ indicate
the (non-)presence of parameters in the topology, x indicates that
the presence has no influence.}
\label{tab:topfe}
\end{table}

Below is an example of a topology which changes from 200 propanols to
200 pentanes using the \gromosv{96} force field.\\

{\small
\begin{verbatim}
 
; Include force field parameters
#include "gromos43a1.ff/forcefield.itp"

[ moleculetype ]
; Name            nrexcl
PropPent          3

[ atoms ]
; nr type resnr residue atom cgnr  charge    mass  typeB chargeB  massB
  1    H    1     PROP    PH    1   0.398    1.008  CH3     0.0  15.035
  2   OA    1     PROP    PO    1  -0.548  15.9994  CH2     0.0  14.027
  3  CH2    1     PROP   PC1    1   0.150   14.027  CH2     0.0  14.027
  4  CH2    1     PROP   PC2    2   0.000   14.027
  5  CH3    1     PROP   PC3    2   0.000   15.035

[ bonds ]
;  ai    aj funct    par_A  par_B 
    1     2     2    gb_1   gb_26
    2     3     2    gb_17  gb_26
    3     4     2    gb_26  gb_26
    4     5     2    gb_26

[ pairs ]
;  ai    aj funct
    1     4     1
    2     5     1

[ angles ]
;  ai    aj    ak funct    par_A   par_B
    1     2     3     2    ga_11   ga_14
    2     3     4     2    ga_14   ga_14
    3     4     5     2    ga_14   ga_14

[ dihedrals ]
;  ai    aj    ak    al funct    par_A   par_B
    1     2     3     4     1    gd_12   gd_17
    2     3     4     5     1    gd_17   gd_17

[ system ]
; Name
Propanol to Pentane

[ molecules ]
; Compound        #mols
PropPent          200
\end{verbatim}}

Atoms that are not perturbed, {\tt PC2} and {\tt PC3}, do not need B-state parameter
specifications, since the B parameters will be copied from the A parameters.
Bonded interactions between atoms that are not perturbed do not need B
parameter specifications, as is the case for the last bond in the example topology.
Topologies using the OPLS/AA force field need no bonded parameters at all,
since both the A and B parameters are determined by the atom types.
Non-bonded interactions involving one or two perturbed atoms use the 
free-energy perturbation functional forms.
Non-bonded interactions between two non-perturbed atoms use the normal
functional forms.
This means that when, for instance, only the charge of a particle is
perturbed, its Lennard-Jones interactions will also be affected when
lambda is not equal to zero or one.

{\bf Note} that this topology uses the \gromosv{96} force field, in which the bonded
interactions are not determined by the atom types. The bonded interaction
strings are converted by the C-preprocessor. The force field parameter
files contain lines like:

{\small
\begin{verbatim}
#define gb_26       0.1530  7.1500e+06

#define gd_17     0.000       5.86          3
\end{verbatim}}

\subsection{Constraint forces\index{constraint force}}
\label{subsec:constraintforce}
The constraint force between two atoms in one molecule can be calculated
with the free energy perturbation code by adding a constraint between the
two atoms, with a different length in the A and B topology. When the B length
is 1 nm longer than the A length and lambda is kept constant at zero,
the derivative of the Hamiltonian with respect to lambda is the constraint
force. For constraints between molecules, the pull code can be used,
see \secref{pull}.
Below is an example for calculating the constraint force at 0.7 nm
between two methanes in water, by combining the two methanes into one ``molecule.''
{\bf Note} that the definition of a ``molecule'' in {\gromacs} does not necessarily
correspond to the chemical definition of a molecule.  In {\gromacs}, a ``molecule''
can be defined as any group of atoms that one wishes to consider simultaneously.
The added constraint is of function type 2, which means that it is not
used for generating exclusions (see~\secref{excl}). 
Note that the constraint free energy term is included in the derivative term, and is
specifically included in the {\tt bonded-lambdas} component. However, the free
energy for changing constraints is {\em not} included in the potential energy
differences used for BAR and MBAR, as this requires reevaluating the energy at
each of the constraint components.  This functionality is planned for later versions.\\

{\small
\begin{verbatim}
; Include force field parameters
#include "gromos43a1.ff/forcefield.itp"

[ moleculetype ]
; Name            nrexcl
Methanes               1

[ atoms ]
; nr   type   resnr  residu   atom    cgnr     charge    mass
   1    CH4     1     CH4      C1       1          0    16.043
   2    CH4     1     CH4      C2       2          0    16.043
[ constraints ]
;  ai    aj funct   length_A  length_B
    1     2     2        0.7       1.7

#include "spc.itp"

[ system ]
; Name
Methanes in Water

[ molecules ]
; Compound        #mols
Methanes              1
SOL                2002
\end{verbatim}}

\subsection{Coordinate file}
\label{subsec:grofile}
Files with the {\tt .gro} file extension contain a molecular structure in 
\gromosv{87} format. A sample piece is included below:

{\small
\begin{verbatim}
MD of 2 waters, reformat step, PA aug-91
    6
    1WATER  OW1    1   0.126   1.624   1.679  0.1227 -0.0580  0.0434
    1WATER  HW2    2   0.190   1.661   1.747  0.8085  0.3191 -0.7791
    1WATER  HW3    3   0.177   1.568   1.613 -0.9045 -2.6469  1.3180
    2WATER  OW1    4   1.275   0.053   0.622  0.2519  0.3140 -0.1734
    2WATER  HW2    5   1.337   0.002   0.680 -1.0641 -1.1349  0.0257
    2WATER  HW3    6   1.326   0.120   0.568  1.9427 -0.8216 -0.0244
   1.82060   1.82060   1.82060
\end{verbatim}}

This format is fixed, {\ie} all columns are in a fixed position. If you
want to read such a file in your own program without using the
{\gromacs} libraries you can use the following formats:

{\bf C-format:} {\tt "\%5i\%5s\%5s\%5i\%8.3f\%8.3f\%8.3f\%8.4f\%8.4f\%8.4f"}

Or to be more precise, with title {\em etc.} it looks like this:

\begin{verbatim}
  "%s\n", Title
  "%5d\n", natoms
  for (i=0; (i<natoms); i++) {
    "%5d%-5s%5s%5d%8.3f%8.3f%8.3f%8.4f%8.4f%8.4f\n",
      residuenr,residuename,atomname,atomnr,x,y,z,vx,vy,vz
  }
  "%10.5f%10.5f%10.5f%10.5f%10.5f%10.5f%10.5f%10.5f%10.5f\n",
    box[X][X],box[Y][Y],box[Z][Z],
    box[X][Y],box[X][Z],box[Y][X],box[Y][Z],box[Z][X],box[Z][Y]
\end{verbatim}

{\bf Fortran format:} {\tt (i5,2a5,i5,3f8.3,3f8.4)}

So {\tt confin.gro} is the {\gromacs} coordinate file and is almost
the same as the \gromosv{87} file (for {\gromos} users: when used with
{\tt ntx=7}).  The only difference is the box for which {\gromacs} uses a
tensor, not a vector.



\section{Force field organization \index{force field organization}}
\label{sec:fforganization}

\subsection{Force field files}
\label{subsec:fffiles}
As of {\gromacs} version 4.5, 14 force fields are available by default.
Force fields are detected by the presence of {\tt <name>.ff} directories
in the {\gromacs} {\tt /share/top} sub-directory and/or the working directory.
The information regarding the location of the force field files is printed
by {\tt pdb2gmx} so you can easily keep track of which version of a force field
is being called, in case you have made modifications in one location or another.
The force fields included with {\gromacs} are:

{\small
\begin{itemize}
 \item AMBER03 force field (Duan et al., J. Comp. Chem. 24, 1999-2012, 2003) 
 \item AMBER94 force field (Cornell et al., JACS 117, 5179-5197, 1995) 
 \item AMBER96 force field (Kollman et al., Acc. Chem. Res. 29, 461-469, 1996) 
 \item AMBER99 force field (Wang et al., J. Comp. Chem. 21, 1049-1074, 2000) 
 \item AMBER99SB force field (Hornak et al., Proteins 65, 712-725, 2006) 
 \item AMBER99SB-ILDN force field (Lindorff-Larsen et al., Proteins 78, 1950-58, 2010) 
 \item AMBERGS force field (Garcia \& Sanbonmatsu, PNAS 99, 2782-2787, 2002) 
 \item CHARMM27 all-atom force field (with CMAP) 
 \item GROMOS96 43A1 force field 
 \item GROMOS96 43A2 force field (improved alkane dihedrals) 
 \item GROMOS96 45A3 force field (Schuler JCC 2001 22 1205) 
 \item GROMOS96 53A5 force field (JCC 2004 vol 25 pag 1656) 
 \item GROMOS96 53A6 force field (JCC 2004 vol 25 pag 1656) 
 \item OPLS-AA/L all-atom force field (2001 aminoacid dihedrals) 
\end{itemize}} 
 
There are also some additional deprecated force fields listed in the selection from
{\tt pdb2gmx}, but we do not currently recommend that you use those for new simulations.
 
A force field is included at the beginning of a topology file with an
{\tt \#include} statement followed by {\tt <name>.ff/forcefield.itp}.
This statement includes the force field file,
which, in turn, may include other force field files. All the force fields
are organized in the same way. As an example, we show the {\tt gmx.ff/forcefield.itp}
file:

{\small
\begin{verbatim}
#define _FF_GROMACS
#define _FF_GROMACS1

[ defaults ]
; nbfunc        comb-rule       gen-pairs       fudgeLJ fudgeQQ
  1             1               no              1.0     1.0

#include "ffnonbonded.itp"
#include "ffbonded.itp"
\end{verbatim}}

The first {\tt \#define} can be used in topologies to parse data which is
specific for all {\gromacs} force fields, the second {\tt \#define} is to parse
data specific to this force field. The {\tt [~defaults~]} section is
explained in \ssecref{topfile}. The included file {\tt ffnonbonded.itp} contains
all atom types and non-bonded parameters. The included file {\tt ffbonded.itp}
contains all bonded parameters.

For each force field, there several files which are only used by {\tt pdb2gmx}.
These are: residue databases ({\tt .rtp}, see~\ssecref{rtp})
the hydrogen database ({\tt .hdb}, see~\ssecref{hdb}), two termini databases
({\tt .n.tdb} and {\tt .c.tdb}, see~\ssecref{tdb}) and
the atom type database ({\tt .atp}, see~\ssecref{atomtype}), which contains only the masses.  Other optional
files are described in~\secref{pdb2gmxfiles}.


\subsection{Changing force field parameters\index{force field}}
If one wants to change the parameters of few bonded interactions in
a molecule, this is most easily accomplished by typing the parameters
behind the definition of the bonded interaction directly in the {\tt *.top} file 
under the {\tt [~moleculetype~]} section (see \ssecref{topfile} for the format
and units).
If one wants to change the parameters for all instances of a certain
interaction one can change them in the force-field file or add a
new {\tt [~???types~]} section after including the force field.
When parameters for a certain interaction are defined multiple times,
the last definition is used. As of {\gromacs} version 3.1.3, a warning is
generated when parameters are redefined with a different value.
Changing the Lennard-Jones parameters of an atom type is not
recommended, because in the {\gromos} force fields
the Lennard-Jones parameters for several combinations of atom types
are not generated according to the standard combination rules.
Such combinations (and possibly others that do follow the
combination rules) are defined in the {\tt [~nonbond_params~]}
section, and changing the Lennard-Jones parameters of an atom type
has no effect on these combinations.

\subsection{Adding atom types\swapindexquiet{adding}{atom types}}
As of {\gromacs} version 3.1.3, atom types can be added in an extra
{\tt [~atomtypes~]} section after the the inclusion of the normal
force field. After the definition of the new atom type(s), additional
non-bonded and pair parameters can be defined.
In pre-3.1.3 versions of {\gromacs}, the new atom types needed to be
added in the {\tt [~atomtypes~]} section of the force field files,
because all non-bonded parameters above the last {\tt [~atomtypes~]}
section would be overwritten using the standard combination rules.

\section{{\tt gmx.ff} documentation}
For backward compatibility we retain here some reference to parameters
present in the {\tt gmx.ff} force field. The last 10 atom types were
not part of the original \gromosv{87} force field~\cite{biomos}, so
if you use them you should refer to one or more of the following
papers:
\begin{itemize}
\item F was taken from ref.~\cite{Buuren93a}, 
\item CP2 and CP3 from ref.~\cite{Buuren93b} and references cited therein, 
\item CR5, CR6 and HCR from ref.~\cite{Spoel96c}
\item OWT3 from ref.~\cite{Jorgensen83}
\item SD, OD and CD from ref.~\cite{Liu95}
\end{itemize}
{\bf Note that we recommend against using these parameters in new projects
since they are not well-tested.}

% LocalWords:  parameterized fffiles ptype polarizable gromacs atp ype arameter
% LocalWords:  lll carboxyl OA hydroxyl NL porphyrin OPLS CP HCR OWT fd funct
% LocalWords:  grompp statprop atomtype rtp esidue opology pdb gmx kJ mol gro
% LocalWords:  grofile dihedrals bon itp func kb th cth cq cp mult Ryckaert aj
% LocalWords:  Bellemans ak alkanes alkane llrllrllr LJ der nb topfile llllll
% LocalWords:  llll nonbond params ij pairtypes fecalc moleculetype indices mdp
% LocalWords:  constraintforce SPC molname nrexcl nr ren HW doh dhh aminoacids
% LocalWords:  dat basename rna dna arn hdb sn rtpo gmxfiles molitp ndx ARG CYS
% LocalWords:  defaultgroups impropers chargegroup bondedtypes hydrogens ARGN
% LocalWords:  preprocessor protonation specbond protonated arginine aspartic
% LocalWords:  ASPH GLU glutamic GLUH HISD histidine HISE HISH LYSN LYS IUPAC
% LocalWords:  wildcards xlateat asparagine HD HH cis deprotonated oxygens COOH
% LocalWords:  llllc tp cr QQ atomtypes bondtypes angletypes dihedraltypes FENE
% LocalWords:  constrainttypes intra nbpar morse dr Coul rr UB dih constr hh ai
% LocalWords:  vsite sitesn construc restr ffgmx resnr residu cgnr al fc spc gb
% LocalWords:  FudgeLJ FudgeQQ nonbonded mdrun decane posre Ifdef ifdef TFE cpp
% LocalWords:  Loof Buuren Berendsen DDeloof DeLoof VanBuuren endif feia topfe
% LocalWords:  propanols pentanes ffG PropPent typeB chargeB massB ga gd mols
% LocalWords:  Propanol Pentane methanes aug natoms residuenr residuename vx vy
% LocalWords:  atomname atomnr vz Fortran confin ntx GROMOS nbfunc GROningen ff
% LocalWords:  fudgeLJ fudgeQQ ffgmxnb ffgmxbon tdb ffbonded ffnonbonded nonbond
% LocalWords:  MAchine BIOSON Groningen Spoel Drunen Comp Phys Comm trr AA fdn
% LocalWords:  aliphatic CHARMM polarisability quadrupole tt normvsbds Waals jj
% LocalWords:  pairinteractions num Buckingham rcl trans Intramolecular Lennard
% LocalWords:  excl gen solute unscaled moltype intramol dgimplement Qiu HCT rt
% LocalWords:  Onufriev OBC LINCS doc xxx residuetypes polyatomic co rotatable
% LocalWords:  heme cysteine lysine CH NH LP amine nitrenyl ethynyl vsd MCH MNH
% LocalWords:  chainsep resA atomA nbondsA resB atomB nbondsB newresA newresB
% LocalWords:  rad deg lcc cc nm intramolecular forcefield PME Ewald
% LocalWords:  solvation et groupconcept PHE TYR TRP equilibrated pre
% LocalWords:  macromolecule disulfide harmonicbond Morsebond vsiteN
% LocalWords:  cubicbond FENEbond tabulatedinteraction harmonicangle
% LocalWords:  harmonicrestraint bondbondcross bondanglecross genion
% LocalWords:  quarticangle properdihedral harmonicimproperdihedral
% LocalWords:  RBdihedral periodicimproperdihedral Fourierdihedral
% LocalWords:  positionrestraint distancerestraint dihedralrestraint
% LocalWords:  orientationrestraint anglerestraint usinggroups ing
% LocalWords:  DDeLoof MBAR Duan JACS Kollman Acc Hornak ILDN AMBERGS
% LocalWords:  Lindorff Sanbonmatsu PNAS CMAP Schuler JCC pag
% LocalWords:  aminoacid
