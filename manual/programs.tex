%
% 
%       This source code is part of
% 
%        G   R   O   M   A   C   S
% 
% GROningen MAchine for Chemical Simulations
% 
%               VERSION 2.0
% 
% Copyright (c) 1991-1999
% BIOSON Research Institute, Dept. of Biophysical Chemistry
% University of Groningen, The Netherlands
% 
% Please refer to:
% GROMACS: A message-passing parallel molecular dynamics implementation
% H.J.C. Berendsen, D. van der Spoel and R. van Drunen
% Comp. Phys. Comm. 91, 43-56 (1995)
% 
% Also check out our WWW page:
% http://md.chem.rug.nl/~gmx
% or e-mail to:
% gromacs@chem.rug.nl
% 
% And Hey:
% Giving Russians Opium May Alter Current Situation
%

\chapter{Run parameters and Programs}
\label{ch:programs}

\section{On-line and HTML manuals}
\index{online manual}
\index{html manual}
All the information in this chapter can also be found in HTML
format in your GROMACS data directory. The path depends on
where your files are installed, but the default location is \\
\centerline{\tt /usr/local/gromacs/share/html/online.html}
Or, if you installed from Linux packages it can be found as\\
\centerline{\tt /usr/local/share/gromacs/html/online.html}
You can also use the online from our web site,\\
\centerline{\href{http://manual.gromacs.org/current/}{http://manual.gromacs.org/current/}}

In addition, we install standard UNIX manuals for all the programs. If
you have sourced the {\tt GMXRC} script in the GROMACS binary directory for
your host they should already be present in your {\tt \$MANPATH}, and you
should be able to type {\eg} {\tt man grompp}.

The program manual pages can also be found in
\appref{progman} in this manual.

\section{File types\swapindexquiet{file}{type}}
\label{sec:fileformats}
\tabref{form} lists the file types used by {\gromacs} along with
a short description, and you can find a more detail description for
each file in your HTML reference, or in our online version.

{\gromacs} files written in \normindex{XDR} format can be read on any
architecture with {\gromacs} version 1.6 or later if the configuration
script found the XDR libraries on your system. They should always be
present on UNIX since they are necessary for NFS support.

\begin{table}
\begin{tabularx}{\linewidth}{|r@{\tt.}lccX|}
\dline
\mc{2}{|c}{Default} &      & Default &  \\[-0.1ex]
\mc{1}{|c}{Name} & \mc{1}{c}{Ext.} & Type &  Option & Description \\[-0.1ex]
\hline
\tt   atomtp & \tt atp & Asc & \tt    & Atomtype file used by {\tt pdb2gmx} \\[-0.1ex]
\tt    eiwit & \tt brk & Asc & \tt -f & Brookhaven data bank file \\[-0.1ex]
\tt    state & \tt cpt & xdr & \tt    & Checkpoint file \\[-0.1ex]
\tt   nnnice & \tt dat & Asc & \tt    & Generic data file \\[-0.1ex]
\tt     user & \tt dlg & Asc & \tt    & Dialog Box data for {\tt ngmx} \\[-0.1ex]
\tt      sam & \tt edi & Asc & \tt    & ED sampling input \\[-0.1ex]
\tt      sam & \tt edo & Asc & \tt    & ED sampling output \\[-0.1ex]
\tt     ener & \tt edr &     & \tt    & Generic energy: \tt edr ene \\[-0.1ex]
\tt     ener & \tt edr & xdr & \tt    & Energy file in portable xdr format \\[-0.1ex]
\tt     ener & \tt ene & Bin & \tt    & Energy file \\[-0.1ex]
\tt    eiwit & \tt ent & Asc & \tt -f & Entry in the protein date bank \\[-0.1ex]
\tt     plot & \tt eps & Asc & \tt    & Encapsulated PostScript (tm) file \\[-0.1ex]
\tt     conf & \tt esp & Asc & \tt -c & Coordinate file in ESPResSo format \\[-0.1ex]
\tt    gtraj & \tt g87 & Asc & \tt    & Gromos-87 ASCII trajectory format \\[-0.1ex]
\tt     conf & \tt g96 & Asc & \tt -c & Coordinate file in Gromos-96 format \\[-0.1ex]
\tt     conf & \tt gro & Asc & \tt -c & Coordinate file in Gromos-87 format \\[-0.1ex]
\tt     conf & \tt gro &     & \tt -c & Structure: \tt gro g96 pdb esp tpr tpb tpa \\[-0.1ex]
\tt      out & \tt gro &     & \tt -o & Structure: \tt gro g96 pdb esp \\[-0.1ex]
\tt    polar & \tt hdb & Asc & \tt    & Hydrogen data base \\[-0.1ex]
\tt   topinc & \tt itp & Asc & \tt    & Include file for topology \\[-0.1ex]
\tt      run & \tt log & Asc & \tt -l & Log file \\[-0.1ex]
\tt       ps & \tt m2p & Asc & \tt    & Input file for mat2ps \\[-0.1ex]
\tt       ss & \tt map & Asc & \tt    & File that maps matrix data to colors \\[-0.1ex]
\tt       ss & \tt mat & Asc & \tt    & Matrix Data file \\[-0.1ex]
\tt   grompp & \tt mdp & Asc & \tt -f & {\tt grompp} input file with MD parameters \\[-0.1ex]
\tt  hessian & \tt mtx & Bin & \tt -m & Hessian matrix \\[-0.1ex]
\tt    index & \tt ndx & Asc & \tt -n & Index file \\[-0.1ex]
\tt    hello & \tt out & Asc & \tt -o & Generic output file \\[-0.1ex]
\tt    eiwit & \tt pdb & Asc & \tt -f & Protein data bank file \\[-0.1ex]
\tt  residue & \tt rtp & Asc & \tt    & Residue Type file used by {\tt pdb2gmx} \\[-0.1ex]
\tt      doc & \tt tex & Asc & \tt -o & LaTeX file \\[-0.1ex]
\tt    topol & \tt top & Asc & \tt -p & Topology file \\[-0.1ex]
\tt    topol & \tt tpb & Bin & \tt -s & Binary run input file \\[-0.1ex]
\tt    topol & \tt tpr &     & \tt -s & Generic run input: \tt tpr tpb tpa \\[-0.1ex]
\tt    topol & \tt tpr &     & \tt -s & Structure+mass(db): \tt tpr tpb tpa gro g96 pdb \\[-0.1ex]
\tt    topol & \tt tpr & xdr & \tt -s & Portable xdr run input file \\[-0.1ex]
\tt     traj & \tt trj & Bin & \tt    & Trajectory file (architecture specific) \\[-0.1ex]
\tt     traj & \tt trr &     & \tt    & Full precision trajectory: \tt trr trj cpt \\[-0.1ex]
\tt     traj & \tt trr & xdr & \tt    & Trajectory in portable xdr format \\[-0.1ex]
\tt     root & \tt xpm & Asc & \tt    & X PixMap compatible matrix file \\[-0.1ex]
\tt     traj & \tt xtc &     & \tt -f & Trajec., input: \tt xtc trr trj cpt gro g96 pdb \\[-0.1ex]
\tt     traj & \tt xtc &     & \tt -f & Trajectory, output: \tt xtc trr trj gro g96 pdb \\[-0.1ex]
\tt     traj & \tt xtc & xdr & \tt    & Compressed trajectory (portable xdr format) \\[-0.1ex]
\tt    graph & \tt xvg & Asc & \tt -o & xvgr/xmgr file \\[-0.1ex]
\dline
\end{tabularx}
\caption{The {\gromacs} file types.}
\label{tab:form}
\end{table}

% LocalWords:  lccX atomtp atp Asc Atomtype pdb gmx eiwit brk Brookhaven cpt tm
% LocalWords:  xdr nnnice dat dlg ngmx sam edi edo ener edr ene ent eps conf ss
% LocalWords:  PostScript ESPResSo gtraj Gromos gro tpr tpb tpa hdb topinc itp
% LocalWords:  grompp mdp mtx ndx rtp tex LaTeX topol traj trj trr xpm PixMap
% LocalWords:  xtc Trajec xvg xvgr xmgr


\section{Run Parameters\swapindexquiet{run}{parameter}}
\subsection{General}

Default values are given in parentheses. The first option is
always the default option. Units are given in square brackets The
difference between a dash and an underscore is ignored. 

A sample {\tt .mdp} file is
available. This should be appropriate to start a normal
simulation. Edit it to suit your specific needs and desires. 

\subsection{Preprocessing}
\begin{description}
\item[{\bf title:}]\mbox{}\\
this is redundant, so you can type anything you want
\item[{\bf cpp: }(/lib/cpp)]\mbox{}\\
your preprocessor
\item[{\bf include:}]\mbox{}\\
directories to include in your topology. format: 
\\{\tt-I/home/john/my\_lib -I../more\_lib}\\
\item[{\bf define: }()]\mbox{}\\
defines to pass to the preprocessor, default is no defines. You can use
any defines to control options in your customized topology files. Options
that are already available by default are:
\vspace{-2ex}\begin{description}
\item[{\bf -DFLEX\_SPC}]\mbox{}\\
Will tell grompp to include FLEX\_SPC in stead of SPC into your
topology, this is necessary to make 
{\bf conjugate gradient} work and will allow 
{\bf steepest descent} to minimize further.
\item[{\bf -DPOSRE}]\mbox{}\\
Will tell grompp to include posre.itp into your topology, used for
\normindex{position restraints}.
\end{description}
\end{description}

\subsection{Run control}
\begin{description}
\item[{\bf integrator:}]\mbox{}\\
\vspace{-2ex}\begin{description}
\item[{\bf md} ]\mbox{}\\
A \normindex{leap-frog} algorithm for integrating Newton's
equations.
\item[{\bf steep}]\mbox{}\\
A \normindex{steepest descent} algorithm for energy
minimization. The maximum step size is {\bf emstep}
[nm], the tolerance is {\bf emtol} [kJ
mol$^{-1}$ nm$^{-1}$].
\item[{\bf cg}]\mbox{}\\
 A \normindex{conjugate gradient} algorithm for energy
minimization, the tolerance is {\bf emtol} [kJ mol$^{-1}$
nm$^{-1}$].  CG is more efficient when a steepest descent step
is done every once in a while, this is determined by 
{\bf nstcgsteep}.
\item[{\bf ld}]\mbox{}\\
 An Euler integrator for position Langevin dynamics, the
velocity is the force divided by a friction coefficient 
({\bf ld\_fric} [amu ps$^{-1}$])
plus random thermal noise ({\bf ld\_temp} [K]). 
The random generator is initialized with {\bf ld\_seed}
\end{description}
\item[{\bf tinit: }(0) {[ps]}]\mbox{}\\
starting time for your run (only makes sense for integrators {\bf md} 
and {\bf ld})
\item[{\bf dt: }(0.001) {[ps]}]\mbox{}\\
time step for integration (only makes sense for integrators {\bf md} 
and {\bf ld})
\item[{\bf nsteps: }(1)]\mbox{}\\
maximum number of steps to integrate
\item[{\bf nstcomm: }(1) {[steps]}]\mbox{}\\
if positive: frequency for center of mass motion removal
if negative: frequency for center of mass motion and rotational 
motion removal (should only be used for vacuum simulations)
\end{description}

\subsection{\normindex{Langevin dynamics}}
\begin{description}
\item[{\bf ld\_temp: }(300) {[K]}]\mbox{}\\
temperature in ld run (controls thermal noise level)
\item[{\bf ld\_fric: }(0) {[amu ps$^{-1}$]}]\mbox{}\\
ld friction coefficient
\item[{\bf ld\_seed: }(1993) {[integer]}]\mbox{}\\
used to initialize random generator for thermal noise
when {\bf ld\_seed} is set to -1, the seed is calculated as
{\tt (time() + getpid()) \% 65536}
\end{description}

\subsection{\normindex{Energy minimization}}
\begin{description}
\item[{\bf emtol: }(100.0) {[kJ mol$^{-1}$ nm$^{-1}$]}]\mbox{}\\
the minimization is converged when the maximum force is smaller than 
this value
\item[{\bf emstep: }(0.01) {[nm]}]\mbox{}\\
initial step-size
\item[{\bf nstcgsteep: }(1000) {[steps]}]\mbox{}\\
frequency of performing 1 steepest descent step while doing
conjugate gradient energy minimization.
\end{description}

\subsection{Output control}
\begin{description}
\item[{\bf nstxout: }(100) {[steps]}]\mbox{}\\
frequency to write coordinates to output 
\normindex{trajectory file}, the last coordinates are always written
\item[{\bf nstvout: }(100) {[steps]}]\mbox{}\\
frequency  to write velocities to output trajectory,
the last velocities are always written
\item[{\bf nstfout: }(0) {[steps]}]\mbox{}\\
frequency to write forces to output trajectory.
\item[{\bf nstlog: }(100) {[steps]}]\mbox{}\\
frequency to write energies to \normindex{log file},
the last energies are always written
\item[{\bf nstenergy: }(100) {[steps]}]\mbox{}\\
frequency to write energies to energy file,
the last energies are always written
\item[{\bf nstxtcout: }(0) {[steps]}]\mbox{}\\
frequency to write coordinates to xtc trajectory
\item[{\bf xtc\_precision: }(1000) {[real]}]\mbox{}\\
precision to write to xtc trajectory
\item[{\bf xtc\_grps:}]\mbox{}\\
group(s) to write to xtc trajectory, default the whole system is written
(if {\bf nstxtcout} is larger than zero)  
\item[{\bf energygrps:}]\mbox{}\\
group(s) to write to energy file
\end{description}

\subsection{\normindex{Neighbor searching}}
\begin{description}
\item[{\bf nstlist: }(10) {[steps]}]\mbox{}\\
frequency to update \normindex{neighborlist}
\item[{\bf ns\_type:}]\mbox{}\\
\vspace{-2ex}\begin{description}
\item[{\bf grid}]\mbox{}\\
Make a grid in the box and only check atoms in neighboring grid
cells when constructing a new neighbor list every {\bf nstlist} steps.
The number of grid cells per Coulomb cut-off
length is set with {\bf deltagrid}, this number should be 2 for
optimal performance.  In large systems grid search is much faster than
simple search.
\item[{\bf simple}]\mbox{}\\
Check every atom in the box when constructing a new neighbor list
every {\bf nstlist} steps.
\end{description}
\item[{\bf deltagrid: }(2)]\mbox{}\\
number of grid cells per Coulomb cut-off length
\item[{\bf box:}]\mbox{}\\
\vspace{-2ex}\begin{description}
\item[{\bf rectangular}]\mbox{}\\
Selects a rectangular box shape.
\item[{\bf none}]\mbox{}\\
Selects no box, for use in vacuum simulations.
\end{description}
\item[{\bf rlist: }(1) {[nm]}]\mbox{}\\
cut-off distance for making the neighbor list
\end{description}

\subsection{\normindex{Electrostatics} and VdW}
\begin{description}
\item[{\bf coulombtype:}]\mbox{}\\
\vspace{-2ex}\begin{description}
\item[{\bf Cut-off}]\mbox{}\\
Twin range cut-off's with neighborlist cut-off {\bf rlist} and 
Coulomb cut-off {\bf rcoulomb},
where {\bf rlist} {\tt $<$} {\bf rvdw} {\tt $<$} {\bf rcoulomb}.
The dielectric constant is set with {\bf epsilon\_r}.
\item[{\bf Ewald}]\mbox{}\\
Classical \normindex{Ewald sum} electrostatics.
Use {\eg} {\bf rlist}=0.9,
{\bf rvdw}=0.9, {\bf rcoulomb}=0.9. The highest magnitude of
wave vectors used in reciprocal space is controlled by {\bf fourierspacing}.
The relative accuracy of direct/reciprocal space
is controlled by {\bf ewald\_rtol}. NOTE: Ewald scales as O(N$^{3/2}$)) and
is thus extremely slow for large systems. It is included mainly for
reference - in most cases PME will perform much better.
\item[\normindex{PME} ]\mbox{}\\
Fast Particle-Mesh Ewald electrostatics. Direct space is similar
to the Ewald sum, while the reciprocal part is performed with
FFTs. Grid dimensions are controlled with {\bf fourierspacing} and the
interpolation order with {\bf pme\_order}. With a grid spacing of 0.1
nm and cubic interpolation the electrostatic forces have an accuracy
of 2-3e-4. Since the error from the vdw-cutoff is larger than this you
might try 0.15 nm. When running in parallel the interpolation
parallelizes better than the FFT, so try decreasing grid dimensions
while increasing interpolation.
\item[\normindex{PPPM}]\mbox{}\\
Particle-Particle Particle-Mesh algorithm for long range
electrostatic interactions.
Use for example {\bf rlist}{\tt =1.0}, {\bf rcoulomb\_switch}{\tt =0.0},
{\bf rcoulomb}{\tt =0.85}, {\bf rvdw\_switch}{\tt =1.0}
and {\bf rvdw}{\tt =1.0}. The grid
dimensions are controlled by {\bf fourierspacing}.
Reasonable grid spacing for PPPM is 0.05-0.1 nm.
See {\tt Shift} for the details of the particle-particle potential.
NOTE: the pressure in incorrect when using PPPM.
\item[\normindex{Reaction-Field}]\mbox{}\\
Reaction field with Coulomb cut-off {\bf rcoulomb},
where {\bf rcoulomb} {\tt $>$} {\bf rvdw} {\tt $>$} {\bf rlist}.
The dielectric constant beyond the cut-off is {\bf epsilon\_r}.
The dielectric constant can be set to infinity by setting {\bf epsilon\_r}=0.
\item[{\bf Generalized-Reaction-Field}]\mbox{}\\
Generalized reaction field with Coulomb cut-off {\bf rcoulomb},
where {\bf rcoulomb} {\tt $>$} {\bf rvdw} {\tt $>$} {\bf rlist}.
The dielectric constant beyond the cut-off is {\bf epsilon\_r}.
The ionic strength is computed from the number of charged 
({\ie} with non zero charge) \normindex{charge group}s.
The temperature for the GRF potential is set with 
{\bf ref\_t} [K].
\item[{\bf Shift}]\mbox{}\\
The Coulomb
potential is decreased over the whole range and the forces decay smoothly
to zero between {\bf rcoulomb\_switch} and {\bf rcoulomb}.
The neighbor search cut-off {\bf rlist} should be 0.1 to 0.3 nm larger than
{\bf rcoulomb} to accommodate for the size of charge groups and diffusion
between neighbor list updates.
\item[{\bf User}]\mbox{}\\
Specify {\bf rshort} and {\bf rlong} to the same value, {\tt mdrun}
will now expect to find a file {\tt ctab.xvg} with user-defined functions.
This files should contain 5 columns:
the {\tt x} value, and the function value with its 1$^{st}$
to 3$^{rd}$ derivative. The {\tt x} should run from 0 [nm] to
{\bf rlist}{\tt +0.5} [nm], with a spacing of {\tt 0.002}
[nm] when you run in single precision, or {\tt 0.0005} [nm] when
you run in double precision. The function value at {\tt x=0} is not
important.
\end{description}
\item[{\bf rcoulomb\_switch: }(0) {[nm]}]\mbox{}\\
where to start switching the Coulomb potential
\item[{\bf rcoulomb: }(1) {[nm]}]\mbox{}\\
distance for the Coulomb \normindex{cut-off}
\item[{\bf epsilon\_r: }(1)]\mbox{}\\
\normindex{dielectric constant}
\item[{\bf vdwtype:}]\mbox{}\\
\vspace{-2ex}\begin{description}
\item[{\bf Cut-off}]\mbox{}\\
Twin range cut-off's with neighbor list cut-off {\bf rlist} and 
VdW cut-off {\bf rvdw},
where {\bf rvdw} {\tt $>$} {\bf rlist}.
\item[{\bf Shift}]\mbox{}\\
The LJ (not Buckingham) potential is decreased over the whole
range and the forces decay smoothly to zero between {\bf rvdw\_switch}
and {\bf rvdw}.  The neighbor search cut-off {\bf rlist} should be
0.1 to 0.3 nm larger than {\bf rvdw} to accommodate for the size of
charge groups and diffusion between neighbor list
updates.
\item[{\bf User}]\mbox{}\\
{\tt mdrun} will now expect to find two files with user-defined
functions: {\tt rtab.xvg} for Repulsion, {\tt dtab.xvg} for 
Dispersion. These files should contain 5 columns:
the {\tt x} value, and the function value with its 1$^{st}$
to 3$^{rd}$ derivative. The {\tt x} should run from 0 [nm] to
{\bf rvdw}{\tt +0.5} [nm], with a spacing of {\tt 0.002}
[nm] when you run in single precision, or {\tt 0.0005} [nm] when
you run in double precision. The function value at {\tt x=0} is not
important. When you want to use LJ correction, make sure that {\bf rvdw}
corresponds to the cut-off in the user-defined function.
\end{description}
\item[{\bf rvdw\_switch: }(0) {[nm]}]\mbox{}\\
where to start switching the LJ potential
\item[{\bf rvdw: }(1) {[nm]}]\mbox{}\\
distance for the LJ or Buckingham \normindex{cut-off}
\item[{\bf bDispCorr:}]\mbox{}\\
\vspace{-2ex}\begin{description}
\item[{\bf no}]\mbox{}\\
don't apply any correction
\item[{\bf yes}]\mbox{}\\
apply long range \normindex{dispersion correction}s for Energy
and Pressure
\end{description}
\item[{\bf fourierspacing: }(0.12) {[nm]}]\mbox{}\\
The maximum grid spacing for the FFT grid when using PPPM or PME.
For ordinary Ewald the spacing times the box dimensions determines the
highest magnitude to use in each direction. In all cases
each direction can be overridden by entering a non-zero value for
{\bf fourier\_n*}. 
\item[{\bf fourier\_nx }(0){\bf  ; fourier\_ny }(0){\bf  ; fourier\_nz: }(0)]\mbox{}\\
Highest magnitude of wave vectors in reciprocal space when using Ewald.
Grid size when using PPPM or PME. These values override
{\bf fourierspacing} per direction. The best choice is powers of
2, 3, 5 and 7. Avoid large primes.
\item[{\bf pme\_order }(4)]\mbox{}\\
Interpolation order for PME. 4 equals cubic interpolation. You might try
6/8/10 when running in parallel and simultaneously decrease grid dimension.
\item[{\bf ewald\_rtol }(1e-5)]\mbox{}\\
The relative strength of the Ewald-shifted direct potential at the cutoff
is given by {\bf ewald\_rtol}. Decreasing this will give a more accurate
direct sum, but then you need more wave vectors for the reciprocal sum.
\item[{\bf optimize\_fft:}]\mbox{}\\
\vspace{-2ex}\begin{description}
\item[{\bf no}]\mbox{}\\
Don't calculate the optimal FFT plan for the grid at startup.
\item[{\bf yes}]\mbox{}\\
Calculate the optimal FFT plan for the grid at startup. This saves a
few percent for long simulations, but takes a couple of minutes
at start.
\end{description}
\end{description}

\subsection{\normindex{Temperature coupling}}
\begin{description}
\item[{\bf tcoupl:}]\mbox{}\\
\vspace{-2ex}\begin{description}
\item[{\bf no}]\mbox{}\\
No temperature coupling. 
\item[{\bf yes}]\mbox{}\\
Temperature coupling with a Berendsen-thermostat to a bath with
temperature {\bf ref\_t} [K], with time constant {\bf tau\_t} [ps].
Several groups can be coupled separately, these are specified in the
{\bf tc\_grps} field separated by spaces.
\end{description}
\item[{\bf tc\_grps:}]\mbox{}\\
groups to couple separately to temperature bath
\item[{\bf tau\_t: }[ps]]\mbox{}\\
time constant for coupling (one for each group in tc\_grps)
\item[{\bf ref\_t: }[K]]\mbox{}\\
reference temperature for coupling (one for each group in tc\_grps)
\end{description}

\subsection{\normindex{Pressure coupling}}
\begin{description}
\item[{\bf pcoupl:}]\mbox{}\\
\vspace{-2ex}\begin{description}
\item[{\bf no}]\mbox{}\\
No pressure coupling. This means a fixed box size.
\item[{\bf isotropic}]\mbox{}\\
Pressure coupling with time constant {\bf tau\_p} [ps].
The compressibility and reference pressure are set with
{\bf compressibility} [bar$^{-1}$] and {\bf ref\_p} [bar], one
value is needed.
\item[{\bf semiisotropic}]\mbox{}\\
Pressure coupling which is isotropic in the x and y direction,
but different in the z direction.
This can be useful for membrane simulations.
2 values are needed for x/y and z directions respectively.
\item[{\bf anisotropic}]\mbox{}\\
Idem, but 3 values are needed for x, y and z directions respectively.
Beware that isotropic scaling can lead to extreme deformation
of the simulation box.
\item[{\bf surface-tension}]\mbox{}\\
Surface tension coupling for surfaces parallel to the xy-plane.
Uses normal pressure coupling for the z-direction, while the surface tension
is coupled to the x/y dimensions of the box.
The first {\bf ref\_p} value is the reference surface tension times
the number of surfaces [bar nm], 
the second value is the reference z-pressure [bar].
The two {\bf compressibility} [bar$^{-1}$] values are the compressibility
in the x/y and z direction respectively.
The value for the z-compressibility should be reasonably accurate since it
influences the converge of the surface-tension, it can also be set to zero
to have a box with constant height.
\item[{\bf triclinic}]\mbox{}\\
Not supported yet.
\end{description}
\item[{\bf tau\_p: }(1) {[ps]}]\mbox{}\\
time constant for coupling
\item[{\bf compressibility: }[bar$^{-1}$]]\mbox{}\\
compressibility (NOTE: this is now really in bar$^{-1}$)
For water at 1 atm and 300 K the compressibility is 4.5e-5 [bar$^{-1}$].
\item[{\bf ref\_p: }[bar]]\mbox{}\\
reference pressure for coupling
\end{description}

\subsection{\normindex{Simulated annealing}}
\begin{description}
\item[{\bf annealing:}]\mbox{}\\
\vspace{-2ex}\begin{description}
\item[{\bf no}]\mbox{}\\
No simulated annealing. 
\item[{\bf yes}]\mbox{}\\
Simulated annealing to 0 [K] at time {\bf zero\_temp\_time} (ps).
Reference temperature for the Berendsen-thermostat is
{\bf ref\_t} x (1 - time / {\bf zero\_temp\_time}),
time constant is {\bf tau\_t} [ps]. Note that the reference temperature
will not go below 0 [K], {\ie} after {\bf zero\_temp\_time} (if it is positive) 
the reference temperature will be 0 [K]. Negative {\bf zero\_temp\_time} 
results in heating, which will go on indefinitely.
\end{description}
\item[{\bf zero\_temp\_time: }(0) {[ps]}]\mbox{}\\
time at which temperature will be zero (can be negative). Temperature
during the run can be seen as a straight line going through 
T={\bf ref\_t} [K] at t=0 [ps], and 
T=0 [K] at t={\bf zero\_temp\_time} [ps]. Look in our 
FAQ for a schematic 
graph of temperature versus time.
\end{description}

\subsection{Velocity generation}
\begin{description}
\item[{\bf gen\_vel:}]\mbox{}\\
\vspace{-2ex}\begin{description}
\item[{\bf no}]\mbox{}\\
 Do not generate velocities at startup. The velocities are set to zero
when there are no velocities in the input structure file.
\item[{\bf yes}]\mbox{}\\
Generate velocities according to a Maxwell distribution at
temperature {\bf gen\_temp} [K], with random seed {\bf gen\_seed}. 
This is only meaningful with integrator {\bf md}.
\end{description}
\item[{\bf gen\_temp: }(300) {[K]}]\mbox{}\\
temperature for Maxwell distribution
\item[{\bf gen\_seed: }(173529) {[integer]}]\mbox{}\\
used to initialize random generator for random velocities
\end{description}

\subsection{Solvent optimization}
\begin{description}
\item[{\bf solvent\_optimization:}]\mbox{}\\
\vspace{-2ex}\begin{description}
\item[{\bf $<$empty$>$}]\mbox{}\\
Do not use water specific non-bonded optimizations
\item[{\bf $<$solvent molecule name$>$}]\mbox{}\\
Use water specific non-bonded optimizations. This string should match the
solvent molecule name in your topology. Check your run time to see 
if it is faster. 
\end{description}
\end{description}

\subsection{Bonds}
\begin{description}
\item[{\bf \normindex{constraints}:}]\mbox{}\\
\vspace{-2ex}\begin{description}
\item[{\bf none}]\mbox{}\\
No constraints, {\ie} bonds are represented by a harmonic or a
Morse potential (depending on the setting of {\bf morse}) and angles
by a harmonic potential.
\item[{\bf hbonds}]\mbox{}\\
Only constrain the bonds with H-atoms.
\item[{\bf all-bonds}]\mbox{}\\
Constrain all bonds.
\item[{\bf h-angles}]\mbox{}\\
Constrain all bonds and constrain the angles that involve H-atoms
by adding bond-constraints.
\item[{\bf all-angles}]\mbox{}\\
Constrain all bonds and constrain all angles by adding bond-constraints.
\end{description}
\item[{\bf constraint\_alg:}]\mbox{}\\
\vspace{-2ex}\begin{description}
\item[\normindex{lincs}]\mbox{}\\
LINear Constraint Solver. The accuracy in set with
{\bf lincs\_order}, which sets the number of matrices in the expansion
for the matrix inversion, 4 is enough for a "normal" MD simulation, 8 is
needed for LD with large time-steps. If a bond rotates more than
{\bf lincs\_warnangle} [degrees] in one step, 
a warning will be printed both to the log file and to {\tt stderr}. 
Lincs should not be used with coupled angle constraints.
\item[\normindex{shake}]\mbox{}\\
Shake is slower and less stable than Lincs, but does work with 
angle constraints. 
The relative tolerance is set with {\bf shake\_tol}, 0.0001 is a good value
for "normal" MD. 
\end{description}
\item[{\bf unconstrained\_start:}]\mbox{}\\
\vspace{-2ex}\begin{description}
\item[{\bf no}]\mbox{}\\
apply constraints to the start configuration
\item[{\bf yes}]\mbox{}\\
do not apply constraints to the start configuration
\end{description}
\item[{\bf shake\_tol: }(0.0001)]\mbox{}\\
relative tolerance for shake
\item[{\bf lincs\_order: }(4)]\mbox{}\\
Highest order in the expansion of the constraint coupling matrix.
{\bf lincs\_order} is also used for the number of Lincs iterations
during energy minimization, only one iteration is used in MD.
\item[{\bf lincs\_warnangle: }(30) {[degrees]}]\mbox{}\\
maximum angle that a bond can rotate before Lincs will complain
\item[{\bf nstlincsout: }(1000) {[steps]}]\mbox{}\\
frequency to output constraint accuracy in log file
\item[{\bf morse:}]\mbox{}\\
\vspace{-2ex}\begin{description}
\item[{\bf no}]\mbox{}\\
bonds are represented by a harmonic potential
\item[{\bf yes}]\mbox{}\\
bonds are represented by a Morse potential
\end{description}
\end{description}

\subsection{\normindex{NMR refinement}}
\begin{description}
\item[{\bf disre:}]\mbox{}\\
\vspace{-2ex}\begin{description}
\item[{\bf none}]\mbox{}\\
no \normindex{distance restraints} (ignore distance
restraints information in topology file)
\item[{\bf simple}]\mbox{}\\
simple (per-molecule) distance restraints
\item[{\bf ensemble}]\mbox{}\\
distance restraints over an ensemble of molecules
\end{description}
\item[{\bf disre\_weighting:}]\mbox{}\\
\vspace{-2ex}\begin{description}
\item[{\bf equal}]\mbox{}\\
divide the restraint force equally over all atom pairs in the restraint
\item[{\bf conservative}]\mbox{}\\
the forces are the derivative of the restraint potential,
this results in an r$^{-7}$ weighting of the atom pairs
\end{description}
\item[{\bf disre\_mixed:}]\mbox{}\\
\vspace{-2ex}\begin{description}
\item[{\bf no}]\mbox{}\\
the violation used in the calculation of the restraint force is the
time averaged violation 
\item[{\bf yes}]\mbox{}\\
the violation used in the calculation of the restraint force is the
square root of the time averaged violation times the instantaneous violation 
\end{description}
\item[{\bf disre\_fc: }(1000) {[kJ mol$^{-1}$ nm$^{-2}$]}]\mbox{}\\
force constant for distance restraints, which is multiplied by a
(possibly) different factor for each restraint
\item[{\bf disre\_tau: }(10) {[ps]}]\mbox{}\\
time constant for distance restraints running average
\item[{\bf nstdisreout: }(100) {[steps]}]\mbox{}\\
frequency to write the running time averaged and instantaneous distances
of all atom pairs involved in restraints to the energy file
(can make the energy file very large)
\end{description}

\subsection{\normindex{Free Energy Perturbation}}
\begin{description}
\item[{\bf free\_energy:}]\mbox{}\\
\vspace{-2ex}\begin{description}
\item[{\bf no}]\mbox{}\\
Only use topology A. 
\item[{\bf yes}]\mbox{}\\
Change the system from topology A (lambda=0) to topology B (lambda=1)
and calculate the free energy difference.
The starting value of lambda is {\bf init\_lambda} the increase
per time step is {\bf delta\_lambda}.
\end{description}
\item[{\bf init\_lambda: }(0)]\mbox{}\\
starting value for lambda
\item[{\bf delta\_lambda: }(0)]\mbox{}\\
increase per time step for lambda
\end{description}

\subsection{\normindex{Non-equilibrium MD}}
\begin{description}
\item[{\bf acc\_grps: }]\mbox{}\\
groups for constant acceleration ({\eg}: {\tt Protein Sol})
all atoms in groups Protein and Sol will experience constant acceleration
as specified in the {\bf accelerate} line
\item[{\bf accelerate: }(0) {[nm ps$^{-2}$]}]\mbox{}\\
acceleration for {\bf acc\_grps}; x, y and z for each group
({\eg} {\tt 0.1 0.0 0.0 -0.1 0.0 0.0} means that first group has constant 
acceleration of 0.1 nm ps$^{-2}$ in X direction, second group the 
opposite).
\item[{\bf freezegrps: }]\mbox{}\\
Groups that are to be frozen ({\ie} their X, Y, and/or Z position will
not be updated; {\eg} {\tt Lipid SOL}). {\bf freezedim} specifies for
which dimension the freezing applies.
\item[{\bf freezedim: }]\mbox{}\\
dimensions for which groups in {\bf freezegrps} should be frozen, 
specify {\tt Y} or {\tt N} for X, Y and Z and for each group
({\eg} {\tt Y Y N N N N} means that particles in the first group 
can move only in Z direction. The particles in the second group can 
move in any direction).
\end{description}

\subsection{\normindex{Electric field}s}
\begin{description}
\item[{\bf E\_x ; E\_y ; E\_z:}]\mbox{}\\
If you want to use an electric field in a direction, enter 3 numbers
after the appropriate E\_*, the first number: the number of cosines,
only 1 is implemented (with frequency 0) so enter 1,
the second number: the strength of the electric field in
V nm$^{-1}$,
the third number: the phase of the cosine, you can enter any number here
since a cosine of frequency zero has no phase.
\item[{\bf E\_xt }{\bf  ;  E\_yt }{\bf  ;  E\_zt: }]\mbox{}\\
not implemented yet
\end{description}

\subsection{User defined thingies}
\begin{description}
\item[{\bf user1\_grps }{\bf  ; user2\_grps }{\bf  ; user3\_grps: }]\mbox{}\\
\item[{\bf userint1 }(0){\bf  ; userint2 }(0){\bf  ; userint3 }(0){\bf  ; userint4: }(0)]\mbox{}\\
\item[{\bf userreal1 }(0){\bf  ; userreal2 }(0){\bf  ; userreal3 }(0){\bf  ; userreal4: }(0)]\mbox{}\\
These you can use if you hack out code. You can pass integers and
reals to your subroutine. Check the inputrec definition in
{\tt src/include/types/inputrec.h}
\end{description}



\section{Programs by topic\index{programs by topic}}
\begin{description}
\item {\large\bf Generating topogies and coordinates}
\begin{tabbing}
{\bf mk\_angndx} \= \kill
{\bf pdb2gmx} \> converts pdb files to topology and coordinate files  \\
{\bf editconf} \> edits the box and writes subgroups  \\
{\bf genbox} \> solvates a system \\
{\bf genion} \> generates monoatomic ions on energetically favorable positions \\
{\bf genconf} \> multiplies a conformation in 'random' orientations \\
{\bf genpr} \> generates position restraints for index groups \\
{\bf protonate} \> protonates structures \\
\end{tabbing}

\item {\large\bf Running a simulation}
\begin{tabbing}
{\bf mk\_angndx} \= \kill
{\bf grompp} \> makes a run input file \\
{\bf tpbconv} \> makes a run input file for restarting a crashed run \\
{\bf mdrun} \> performs a simulation \\
\end{tabbing}

\item {\large\bf Viewing trajectories}
\begin{tabbing}
{\bf mk\_angndx} \= \kill
{\bf ngmx} \> displays a trajectory \\
{\bf trjconv} \> converts trajectories to e.g. pdb which can be viewed with e.g. rasmol \\
\end{tabbing}

\item {\large\bf Processing energies}
\begin{tabbing}
{\bf mk\_angndx} \= \kill
{\bf g\_energy} \> writes energies to xvg files and displays averages \\
{\bf g\_enemat} \> extracts an energy matrix from an energy file \\
{\bf mdrun} \> with -rerun (re)calculates energies for trajectory frames \\
\end{tabbing}

\item {\large\bf Converting files}
\begin{tabbing}
{\bf mk\_angndx} \= \kill
{\bf editconf} \> converts and manipulates structure files \\
{\bf trjconv} \> converts and manipulates trajectory files \\
{\bf eneconv} \> converts energy files \\
{\bf xmp2ps} \> converts XPM matrices to encapsulated postscript (or XPM) \\
\end{tabbing}

\item {\large\bf Tools}
\begin{tabbing}
{\bf mk\_angndx} \= \kill
{\bf make\_ndx} \> makes index files \\
{\bf mk\_angndx} \> generates index files for g\_angle \\
{\bf gmxcheck} \> checks and compares files \\
{\bf gmxdump} \> makes binary files human readable \\
{\bf g\_analyze} \> analyzes data sets \\
\end{tabbing}

\item {\large\bf Distances between structures}
\begin{tabbing}
{\bf mk\_angndx} \= \kill
{\bf g\_rms} \> calculates rmsd's with a reference structure and rmsd matrices \\
{\bf g\_confrms} \> fits two structures and calculates the rmsd  \\
{\bf g\_cluster} \> clusters structures \\
{\bf g\_rmsf} \> calculates atomic fluctuations \\
\end{tabbing}

\item {\large\bf Distances in structures over time}
\begin{tabbing}
{\bf mk\_angndx} \= \kill
{\bf g\_mindist} \> calculates the minimum distance between two groups \\
{\bf g\_dist} \> calculates the distances between the centers of mass of two groups \\
{\bf g\_mdmat} \> calculates residue contact maps \\
{\bf g\_rmsdist} \> calculates atom pair distances averaged with power 2, -3 or -6 \\
\end{tabbing}

\item {\large\bf Mass distribution properties over time}
\begin{tabbing}
{\bf mk\_angndx} \= \kill
{\bf g\_com} \> calculates the center of mass \\
{\bf g\_gyrate} \> calculates the radius of gyration \\
{\bf g\_msd} \> calcates mean square displacements \\
{\bf g\_rotacf} \> calculates the rotational correlation function for molecules \\
{\bf g\_rdf} \> calculates RDF's \\
{\bf g\_rdens} \> calculates radial densities \\
\end{tabbing}

\item {\large\bf Analyzing bonded interactions}
\begin{tabbing}
{\bf mk\_angndx} \= \kill
{\bf g\_bond} \> calculates bond length distributions \\
{\bf mk\_angndx} \> generates index files for g\_angle \\
{\bf g\_angle} \> calculates distributions and correlations for angles and dihedrals \\
{\bf g\_dih} \> analyzes dihedral transitions \\
\end{tabbing}

\item {\large\bf Structural properties}
\begin{tabbing}
{\bf mk\_angndx} \= \kill
{\bf g\_hbond} \> computes and analyzes hydrogen bonds \\
{\bf g\_saltbr} \> computes salt bridges \\
{\bf g\_sas} \> computes solvent accessible surface area \\
{\bf g\_order} \> computes the order parameter per atom for carbon tails \\
{\bf g\_sgangle} \> computes the angle and distance between two groups \\
{\bf g\_disre} \> analyzes distance restraints \\
\end{tabbing}

\item {\large\bf Kinetic properties}
\begin{tabbing}
{\bf mk\_angndx} \= \kill
{\bf g\_velacc} \> calculates velocity autocorrelation functions \\
\end{tabbing}

\item {\large\bf Electrostatic properties}
\begin{tabbing}
{\bf mk\_angndx} \= \kill
{\bf genion} \> generates monoatomic ions on energetically favorable positions \\
{\bf g\_potential} \> calculates the electrostatic potential across the box \\
{\bf g\_dipoles} \> computes the total dipole plus fluctuations \\
{\bf g\_dielectric} \> calculates frequency dependent dielectric constants \\
\end{tabbing}

\item {\large\bf Protein specific analysis}
\begin{tabbing}
{\bf mk\_angndx} \= \kill
{\bf do\_dssp} \> assigns secondary structure and calculates solvent accessible surface area \\
{\bf g\_chi} \> calculates everything you want to know about chi and other dihedrals \\
{\bf g\_helix} \> calculates everything you want to know about helices \\
{\bf g\_rama} \> computes Ramachandran plots \\
{\bf xrama} \> shows animated Ramachandran plots \\
{\bf wheel} \> plots helical wheels \\
\end{tabbing}

\item {\large\bf Interfaces}
\begin{tabbing}
{\bf mk\_angndx} \= \kill
{\bf g\_potential} \> calculates the electrostatic potential across the box \\
{\bf g\_density} \> calculates the density of the system \\
{\bf g\_order} \> computes the order parameter per atom for carbon tails \\
{\bf g\_h2order} \> computes the orientation of water molecules \\
\end{tabbing}

\item {\large\bf Covariance analysis}
\begin{tabbing}
{\bf mk\_angndx} \= \kill
{\bf g\_covar} \> calculates and diagonalizes the covariance matrix \\
{\bf g\_anaeig} \> analyzes the eigenvectors \\
\end{tabbing}

\item {\large\bf Normal modes}
\begin{tabbing}
{\bf mk\_angndx} \= \kill
{\bf grompp} \> makes a run input file \\
{\bf mdrun} \> finds a potential energy minimum \\
{\bf nmrun} \> calculates the Hessian \\
{\bf g\_nmeig} \> diagonalizes the Hessian  \\
{\bf g\_anaeig} \> analyzes the normal modes \\
{\bf g\_nmens} \> generates an ensemble of structures from the normal modes \\
\end{tabbing}

\end{description}


% LocalWords:  online html GMXRC MANPATH grompp progman xdr NFS
