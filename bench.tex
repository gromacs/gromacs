\section{Benchmarks}
A few selected benchmarks have been run on the computers in our lab.
The benchmarks all represent ``real-life'' examples, i.e. they
were taken from ongoing research projects in the lab. To appreciate
the numbers one should consider the following points:
\begin{enumerate}
\item	Hardware: is the processor scalar or vector. The {\gromacs}
	software is optimized for scalar and short vector machines,
	i.e. for processors that can process multiple instructions
	simultaneously, like the {\intel} or the R8000. The software
	does not perform very well on old-fashioned vector computers
	like Cray etc. Another important point is the amount
	of cache memory your processor has, usually these are on the
	order of 1 Mb. If you have a small system that fits in the cache
	memory, the performance will be much higher than for larger
	systems.
\item	Compilers. Most compilers are lousy in optimizing code (although
	vendors never admit it). What makes it even worse: most compilers
	do not give good instructions for optimizing your code.
	In the {\gromacs} software we have implemented part of the code
	in FORTRAN as well as C, because on one of our local machines
	(SGI PowerChallenge) the innerloop for nonbonded interactions
	is three times as fast in FORTRAN than in C.
\item	Software. {\gromacs} has optimizations for water, because that is
	an abundant molecule in biological systems. If your system has more
	water, you will have higher performance/
\end{enumerate}

In Table~\ref{tab:benchdef} the definition of the benchmarks is given, in
Table~\ref{tab:benchtime} the results are printed. It can be seen that
scaling on the parallel machine (Samba) is very much dependent on system size.
The Micel simulation which takes 50\% more time than the HIV simulation
on our PowerChallenge, are equally fast on the Samba.

\begin{table}[ht]
\begin{tabularx}{\linewidth}{lXrrrrr}
\dline
Test	& Description			& \# Atoms & \# steps & cut-off	& nstlist & SHAKE	\\
\hline
SOL1 	& 848 SPC in rectang. box	& 2544	& 100	& 0.9/0.9	& 10	& All Bonds	\\
SOL2 	& 1728 SPC in cubic box		& 5184	& 100	& 1.0/1.0	& 10	& All Bonds	\\
PEP	& 32 residue pept. from LDH	& 8652	& 100	& 1.0/1.7	& 10	& All Bonds	\\
MICEL 	& 54 DPC lipids			& 16956 & 100	& 1.0/1.7	& 10	& Water		\\
HIV	& HIV-Protease dimer 		& 19290 & 100	& 1.0/1.0	& 10	& H-Bonds	\\
\dline
\end{tabularx}
\caption{Definition of the benchmarks. All test were performed in explicit solvent. nstlist is the update-frequency of the neighbourlist in steps, cut-off is the short/long range part of the cut-off, where long range means Coulomb interactions only.}
\label{tab:benchdef}
\end{table}
	
\begin{table}[ht]
\centerline{
\begin{tabular}{|ll|rrrrr|}
\dline
Machine	& CPU			&SOL1	&SOL2	& PEP	& MICEL	& HIV	\\
\hline	
rugmd2	& MIPS-R4400/250 MHz	& 52.2	& 	& 227.6	& 709.5	& 535.3	\\
rugmd17	& Sparc-5		& 111	& 	& 463	& 1600	& 1039	\\
bioson01& MIPS-R8000/75 MHz	& 16.1	& 41.3	& 123	& 269	& 181	\\
%bioson01& 4x id. (MPI)		& 10	& 16	& 41	& 86	& 68	\\
%bioson01& 4x id. + shuffle	&	&	&	& 79	& 63(x)	\\
msc-hpc	& MIPS-R10k/190 Mhz	& 14.6	& 40.3 	& 93.7	& 190	& 155	\\
rugch5	& HP-735		& 19.2	& 52.3	& 162	& 350	& 244	\\
Cray	& C-90 (1 CPU)		&	&	&	&	& 235	\\
Samba	& 4 x Intel i860 40 Mhz	& 43	& 99	& 277	& 485	& (*)	\\
Samba	& 28 x id.		& 	& 	& 	& 135	& 135 	\\
Samba	& 28 x id. + shuffle	& 	& 	& 	& 133	& 	\\
\hline
Convex	& 1 HP-735 CPU		&	& 57	& 175	& 368	& 263	\\
	& 2 HP-735 CPU		& 12	& 29.5	& 95.3	& 190	& 136	\\
	& 3 HP-735 CPU		& 9.9	& 23.3	& 76.3	& 145	& 100	\\
	& 4 HP-735 CPU		& 8.5	& 18.4	& 55.6	& 115	& 78.7	\\
	& 5 HP-735 CPU		& 7.8	& 16.2	& 45.7	& 99.0	& 68.7	\\
	& 6 HP-735 CPU		& 7.3	& 14.6	& 41.4	& 88.0	& 61.3	\\
\dline
\end{tabular}
}
\caption{Time in seconds for each test. (*) indicates not enough memory. 
(x) indicates manual shuffling.
Please note that the Samba is our local parallel computer.}
\label{tab:benchtime}
\end{table}


