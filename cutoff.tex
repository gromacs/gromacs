%
% $Id$
% 
%       This source code is part of
% 
%        G   R   O   M   A   C   S
% 
% GROningen MAchine for Chemical Simulations
% 
%               VERSION 2.0
% 
% Copyright (c) 1991-1999
% BIOSON Research Institute, Dept. of Biophysical Chemistry
% University of Groningen, The Netherlands
% 
% Please refer to:
% GROMACS: A message-passing parallel molecular dynamics implementation
% H.J.C. Berendsen, D. van der Spoel and R. van Drunen
% Comp. Phys. Comm. 91, 43-56 (1995)
% 
% Also check out our WWW page:
% http://md.chem.rug.nl/~gmx
% or e-mail to:
% gromacs@chem.rug.nl
% 
% And Hey:
% Gnomes, ROck Monsters And Chili Sauce
%

\subsection{Treatment of cut-offs}
\newcommand{\rs}{$r_{short}$}
\newcommand{\rl}{$r_{long}$}
{\gromacs} is quite flexible in treating cut-offs, which implies
that there are quite a number of parameters to set. The parameters are
set in the input file for grompp. One should distinguish two parts of
the parameters: firstly the parameters that describe the function
(Coulomb / VDW,
\tabref{funcparm}) and
secondly the parameters that describe neighbor searching.

In summary, for both Coulomb and VdW there are a type selector
({\tt vdwtype} resp. {\tt coulombtype}) and two parameters,
for a total of six parameters. See \secref{mdpopt} for a complete
description of these parameters.

The neighbor searching (NS) maybe done using a single-range, or a twin-range 
approach. Since the former is merely a special case of the latter we will 
discuss the more general twin-range. In this case NS is described by two
radii {\tt rlist} and max({\tt rcoulomb},{\tt rvdw}).
Usually one builds the neighbor list every 10 time steps
or every 20 fs (parameter {\tt nstlist}). In the neighbor list all interaction 
pairs that  fall within {\tt rlist} are stored. Furthermore, the 
interactions between pairs that do not
fall within {\tt rlist} but do fall within and max({\tt rcoulomb},{\tt rvdw})
are computed during NS, and the
forces and energy are stored separately, and added to short-range forces
at every time step between successive NS. If {\tt rlist} = 
max({\tt rcoulomb},{\tt rvdw}) no forces
are evaluated during neighbor list generation.

Except for the plain cut-off,
all of the interaction functions in \tabref{funcparm}
require that neighbor searching is done with a larger radius than the $r_c$
specified for the functional form, because of the use of charge groups.
The extra radius is typically of the order of 0.25 nm (roughly the 
largest distance between two atoms in a charge group plus the distance a 
charge group can diffuse within neighbor list updates).

%If your charge groups are very large it may be interesting to turn off charge
%groups, by setting the option 
%{\tt bAtomList = yes} in your {\tt grompp.mdp} file.
%In this case only a small extra radius to account for diffusion needs to be 
%added (0.1 nm). Do not however use this together with the plain cut-off
%method, as it will generate large artifacts (\secref{cg}).
%In summary, there are four parameters that describe NS behavior:
%{\tt nstlist} (update frequency in number of time steps),
%{\tt bAtomList} (whether or not charge groups are used to generate neighbor list, the default is to use charge groups, so {\tt bAtomList = no}),
%{\tt rshort} and {\tt rlong} which are the two radii {\rs} and {\rl}
%described above.

\begin{table}[h]
\centering
\begin{tabular}{|ll|l|}
\dline
\multicolumn{2}{|c|}{Type}		& Parameters		\\
\hline
Coulomb&Plain cut-off	& $r_c$, $\epsr$	\\
&Reaction field		& $r_c$, $\epsrf$	\\
&Shift function		& $r_1$, $r_c$, $\epsr$		\\
&Switch function 	& $r_1$, $r_c$, $\epsr$		\\
\hline
VdW&Plain cut-off	& $r_c$ 	\\
&Shift function	   	& $r_1$, $r_c$ 		\\
&Switch function   	& $r_1$, $r_c$ 		\\
\dline
\end{tabular}
\caption[Parameters for the different functional forms of the
non-bonded interactions.]{Parameters for the different functional
forms of the non-bonded interactions.}
\label{tab:funcparm}
\end{table}

